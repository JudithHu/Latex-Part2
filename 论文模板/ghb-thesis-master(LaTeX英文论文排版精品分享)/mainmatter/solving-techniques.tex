% Copyright (c) 2017-2018, Gabriel Hjort Blindell <ghb@kth.se>
%
% This work is licensed under a Creative Commons Attribution-NoDerivatives 4.0
% International License (see LICENSE file or visit
% <http://creativecommons.org/licenses/by-nc-nd/4.0/> for details).

\chapter{Solving Techniques}
\labelChapter{solving-techniques}

This chapter introduces the techniques applied for improving solving of the
\gls{constraint model} introduced in the previous chapter.
%
We begin in \refSectionList{st-refining-define-before-use-constraint,
  st-refined-objective-function} with refining the \gls{constraint model} to
strengthen \gls{propagation}.
%
We continue to augment the \glsshort{constraint model} in
\refSectionList{st-impl-constraints, st-dom-breaking-constraints} by adding
\gls{implied.c}, \gls{symmetry breaking.c}, and \gls{dominance breaking.c}
\glspl{constraint}.
%
We then describe techniques for tightening the cost bounds in
\refSection{st-cost-bounds} and describe \glspl{branching strategy} in
\refSection{st-branching-strategies}.
%
In \refSection{st-presolving}, we describe \gls{presolving} techniques for
removing \glspl{match} that, for one reason or another, can be removed before
instantiating the \glsshort{constraint model}.
%
With all solving techniques in place, we then evaluate the impact of these
techniques, both individually and as groups, in
\refSection{st-experimental-evaluation}.
%
Lastly, a summary is given in \refSection{st-summary}.


\section{Refining the Define-Before-Use Constraint}
\labelSection{st-refining-define-before-use-constraint}

In \refChapter{constraint-model} on \refPageOfEquation{naive-dom-alt}, a simple
but naive implementation is used for implementing the \gls{constraint} that all
\glspl{datum} must be \gls{define.d}[d] before \gls{use.d}[d]
(\refEquation{naive-dom-alt}).
%
To begin with, it requires the use of set~\glspl{variable},\!%
%
\footnote{%
  This is because the $\mVar{dplace}$~\glspl{variable} need to be members of the
  $\mDom$~\gls{function}, which is implemented as an array into which the
  $\mVar{oplace}$~\glspl{variable} are used as indices.%
}
%
which are expensive to use and not necessarily supported by all
\glspl{constraint solver}.
%
This is for example the case of the \gls{constraint solver} used in the
experiments, thus preventing us from evaluating the impact of the naive
implementation.
%
In addition, if we know in which \glspl{block} a \gls{datum} is \gls{use.d}[d],
then many \gls{implied.c} \glspl{constraint} can be applied to strengthen
\gls{propagation}.

We first eliminate the set~\glspl{variable} by capturing the information in the
$\mDom$~\gls{function} as a dominance relation matrix
%
\begin{equation}
  \begin{aMatrix}{@{\:}c|c@{\:}}
      \mTuple{b_1, b_2}
    & b_1, b_2 \in \mBlockSet, b_1 \in \mDom(b_2)
  \end{aMatrix}
  \labelEquation{dominance-matrix}
\end{equation}
%
where each row denotes the fact that a \gls{block}~$b_1$ is \gls{dominate.b}[d]
by another \gls{block}~$b_2$.
%
Next, we introduce the \glspl{variable} needed for capturing \gls{use.d}[s] of
\glspl{datum}.


\paragraph{Variables}

The set of \glspl{variable} \mbox{$\mVar{uplace}[p] \in \mBlockSet$} models in
which \gls{block} the \gls{datum} connected to \gls{operand}~$p$ is used.
%
Note that unlike the $\mVar{dplace}$~\glspl{variable}, which are indexed using a
\gls{datum}, the $\mVar{uplace}$~\glspl{variable} are indexed using an
\gls{operand}.


\paragraph{Constraints}

Obviously, every \gls{use.d} of \glspl{datum} must be \gls{dominate.b}[d] by its
definition.
%
This \gls{constraint} is modeled as
%
\begin{equation}
  \forall p \in \mPhiOperandCompSet :
  \mTable(
    \mTuple{\mVar{uplace}[p], \mVar{dplace}[\mVar{alt}[p]]} \hspace{-1pt},
    \mDomMatrix
  ),
  \labelEquation{refined-dom}
\end{equation}
%
where \mbox{$\mPhiOperandCompSet \subseteq \mOperandSet$} denotes the set of
\glspl{operand} not appearing in any \gls{phi-match} and $\mDomMatrix$ is the
\glsshort{dominate.b}[ance] relation matrix computed using
\refEquation{dominance-matrix}.
%
We exclude such \glspl{operand} for the same reasons \glspl{phi-match} are
excluded in \refEquation{naive-dom}.

Next, if a \gls{match}~$m$ is selected and placed in
\gls{block}~\mbox{$b$\hspace{-1pt},} then all \gls{use.d}[s] of \glspl{datum}
made by $m$ must also occur in~\mbox{$b$\hspace{-1pt}.}
%
This is modeled as
%
\begin{equation}
  \forall m \in \mPhiMatchCompSet,
  \forall o \in \mCovers(m),
  \forall p \in \mUses(m) :
  \mVar{sel}[m] \mImp \mVar{oplace}[o] = \mVar{uplace}[p].
  \labelEquation{refined-dom-selected}
\end{equation}
%
An example illustrating the interaction between \refEquation{refined-dom} and
\refEquation{refined-dom-selected} is shown in \refFigure{refined-dom-example}.

\begin{figure}
  \centering%
  % Copyright (c) 2017-2018, Gabriel Hjort Blindell <ghb@kth.se>
%
% This work is licensed under a Creative Commons Attribution-NoDerivatives 4.0
% International License (see LICENSE file or visit
% <http://creativecommons.org/licenses/by-nc-nd/4.0/> for details).
%
\begingroup%
\setlength{\nodeDist}{12pt}%
\figureFont\figureFontSize%
\pgfdeclarelayer{background}%
\pgfsetlayers{background,main}%
\adjustbox{valign=M}{%
  \begin{tikzpicture}[
      outer match node/.style={
        match node,
        draw=none,
        inner sep=3pt,
      },
      match label/.append style={
        outer sep=0,
      },
    ]

    \node [computation node] (o2) {$o_2$};
    \node [value node, position=135 degrees from o2] (c) {$d$};
    \node [value node, position= 45 degrees from o2] (d) {};
    \node [value node, below=of o2] (e) {};
    \node [computation node, above=of c] (o1) {$o_1$};
    \node [value node, position=135 degrees from o1] (a) {};
    \node [value node, position= 45 degrees from o1] (b) {};

    \begin{scope}[data flow]
      \draw (a) -- (o1);
      \draw (b) -- (o1);
      \draw (o1) -- (c);
      \draw (c) -- (o2);
      \draw (d) -- (o2);
      \draw (o2) -- (e);
    \end{scope}

    \begin{pgfonlayer}{background}
      % m1
      \begin{scope}[outer match node/.append style={inner sep=5pt}]
        \node [outer match node, fit=(a)] (m1a) {};
        \node [outer match node, fit=(b)] (m1b) {};
        \node [outer match node, fit=(c)] (m1c) {};
      \end{scope}

      \def\pathMI{
        [rounded corners=8pt]
        (m1a.north west)
        --
        (m1b.north east)
        --
        (m1b.south east)
        --
        (m1c.south east)
        --
        (m1c.south west)
        -- coordinate (m1)
        (m1a.south west)
        --
        cycle
      }
      \path [fill=shade1] \pathMI;

      % m2
      \begin{scope}[outer match node/.append style={inner sep=5pt}]
        \node [outer match node, fit=(c)] (m2a) {};
        \node [outer match node, fit=(d)] (m2b) {};
        \node [outer match node, fit=(e)] (m2c) {};
      \end{scope}

      \def\pathMII{
        [rounded corners=8pt]
        (m2a.north west)
        --
        (m2b.north east)
        --
        (m2b.south east)
        --
        (m2c.south east)
        --
        (m2c.south west)
        -- coordinate (m2)
        (m2a.south west)
        --
        cycle
      }
      \path [fill=shade1] \pathMII;

      \begin{scope}
        \path [clip] \pathMI;
        \path [fill=shade2] \pathMII;
      \end{scope}

      \begin{scope}[match line]
        \draw \pathMI;
        \draw \pathMII;
      \end{scope}
    \end{pgfonlayer}

    % Match labels
    \foreach \i in {1, 2} {
      \node [match label, position=-155 degrees from m\i] (m\i l) {$m_\i$};
      \draw [match attachment line] (m\i) -- (m\i l);
    }
  \end{tikzpicture}%
}%
\hspace{1cm}%
\setlength{\abovedisplayskip}{0pt}%
\setlength{\belowdisplayskip}{0pt}%
\adjustbox{valign=M}{%
  \begin{tikzpicture}[
      block node/.append style={
        minimum width=1cm,
        minimum height=5mm,
      },
      dominance/.style={
        control flow,
        dashed,
      },
      math/.style={
        nothing,
        node distance=4mm,
      },
    ]

    \node [block node] (b1) {$b_1$};
    \node [block node, below=15mm of b1] (b2) {$b_2$};

    \begin{pgfonlayer}{background}
      \draw [dominance]
            (b1)
            -- node [auto] {\smaller\rotatebox{-90}{dominates}}
            (b2);
    \end{pgfonlayer}

    \node [math, right=of b1] {%
      \begin{minipage}{30mm}%
        \begin{displaymath}
          m_1
          \left\{
          \begin{array}{@{\:}l@{}}
            \mVar{oplace}[o_1] = b_1 \\
            \mVar{dplace}[d] = b_1
          \end{array}
          \right.
        \end{displaymath}%
      \end{minipage}%
    };

    \node [math, right=of b2] {%
      \begin{minipage}{30mm}%
        \begin{displaymath}
          m_2
          \left\{
          \begin{array}{@{\:}l@{}}
            \mVar{alt}[p] = d \\
            \mVar{uplace}[p] = b_2 \\
            \mVar{oplace}[o_2] = b_2
          \end{array}
          \right.
        \end{displaymath}%
      \end{minipage}%
    };
  \end{tikzpicture}%
}%
\endgroup%
%

  \caption[Example illustrating the refined define-before-use constraint]%
          {%
            Example illustrating the refined define-before-use constraint.
            %
            The assignments on the right-hand side show how these
            variables should be set if the two matches~$m_1$ and~$m_2$ are
            placed in blocks~$b_1$ respectively $b_2$, where $b_1$ is assumed to
            dominate~$b_2$%
          }
  \labelFigure{refined-dom-example}
\end{figure}

Due to \refEquation{refined-dom-selected}, the assignment to the
$\mVar{uplace}$~\glspl{variable} for non-selected \glspl{match} does not matter
as long as \refEquation{refined-dom} is satisfied.
%
This gives rise to symmetric \glspl{solution}.
%
To break these symmetries, we fix the assignments in such cases to the
\gls{block} wherein the \gls{datum} is \gls{define.d}[d] as a \gls{block} always
\gls{dominate.b}[s] itself.
%
This is modeled as
%
\begin{equation}
  \forall m \in \mPhiMatchCompSet,
  \forall p \in \mUses(m) :
  \mNot\mVar{sel}[m] \mImp \mVar{uplace}[p] = \mVar{dplace}[\mVar{alt}[p]],
  \labelEquation{refined-dom-not-selected}
\end{equation}
%
where \mbox{$\mPhiMatchCompSet \subseteq \mMatchSet$} denotes the set of
\glspl{match} without the \glspl{phi-match}.

Since neither of the above \glspl{constraint} apply to \glspl{operand} belonging
to \glspl{phi-match}, the assignment to their $\mVar{uplace}$~\glspl{variable}
does not matter, again giving rise to symmetric \glspl{solution}.
%
We therefore fix the assignment in such cases, which is modeled as
%
\begin{equation}
  \forall p \in \mPhiOperandSet :
  \mVar{uplace}[p] = \mMin(\mBlockSet).
  \labelEquation{refined-dom-phi-operands}
\end{equation}


\section{Refining the Objective Function}
\labelSection{st-refined-objective-function}

The straightforward implementation of the \gls{objective function}
(\refEquation{naive-objective-function}) is naive because it fails to reason on
how cost is distributed across the \glspl{operation} that need to be covered.
%
This in turn results in poor \gls{propagation}.
%
See for example \refFigure{cost-example}.
%
\begin{figure}
  \mbox{}%
  \hfill%
  \subcaptionbox{UF graph\labelFigure{cost-example-graph}}%
                {%
                  % Copyright (c) 2017-2018, Gabriel Hjort Blindell <ghb@kth.se>
%
% This work is licensed under a Creative Commons Attribution-NoDerivatives 4.0
% International License (see LICENSE file or visit
% <http://creativecommons.org/licenses/by-nc-nd/4.0/> for details).
%
\begingroup%
\setlength{\nodeDist}{12pt}%
\figureFont\figureFontSize%
\pgfdeclarelayer{background}%
\pgfsetlayers{background,main}%
\begin{tikzpicture}[
    outer match node/.style={
      match node,
      draw=none,
      inner sep=3pt,
    },
    match label/.append style={
      outer sep=0,
    },
  ]

  \node [computation node] (o2) {$o_2$};
  \node [value node, position=135 degrees from o2] (c) {};
  \node [value node, position= 45 degrees from o2] (d) {};
  \node [value node, below=of o2] (e) {};
  \node [computation node, above=of c] (o1) {$o_1$};
  \node [value node, position=135 degrees from o1] (a) {};
  \node [value node, position= 45 degrees from o1] (b) {};

  \begin{scope}[data flow]
    \draw (a) -- (o1);
    \draw (b) -- (o1);
    \draw (o1) -- (c);
    \draw (c) -- (o2);
    \draw (d) -- (o2);
    \draw (o2) -- (e);
  \end{scope}

  \begin{pgfonlayer}{background}
    % m3
    \begin{scope}[outer match node/.append style={inner sep=8pt}]
      \node [outer match node, fit=(a)] (m3a) {};
      \node [outer match node, fit=(b)] (m3b) {};
      \node [outer match node, fit=(d)] (m3c) {};
      \node [outer match node, fit=(e)] (m3d) {};
    \end{scope}

    \def\pathMIII{
      [rounded corners=8pt]
      (m3a.north west)
      --
      (m3b.north east)
      --
      (m3c.north east)
      -- coordinate (m3)
      (m3c.south east)
      --
      (m3d.south east)
      --
      (m3d.south west)
      --
      (m3a.south west)
      --
      cycle
    }
    \path [fill=shade3] \pathMIII;

    % m1
    \begin{scope}[outer match node/.append style={inner sep=5pt}]
      \node [outer match node, fit=(a)] (m1a) {};
      \node [outer match node, fit=(b)] (m1b) {};
      \node [outer match node, fit=(c)] (m1c) {};
    \end{scope}

    \def\pathMI{
      [rounded corners=8pt]
      (m1a.north west)
      --
      (m1b.north east)
      --
      (m1b.south east)
      --
      (m1c.south east)
      --
      (m1c.south west)
      -- coordinate (m1)
      (m1a.south west)
      --
      cycle
    }
    \path [fill=shade1] \pathMI;

    % m2
    \begin{scope}[outer match node/.append style={inner sep=5pt}]
      \node [outer match node, fit=(c)] (m2a) {};
      \node [outer match node, fit=(d)] (m2b) {};
      \node [outer match node, fit=(e)] (m2c) {};
    \end{scope}

    \def\pathMII{
      [rounded corners=8pt]
      (m2a.north west)
      --
      (m2b.north east)
      --
      (m2b.south east)
      --
      (m2c.south east)
      --
      (m2c.south west)
      -- coordinate (m2)
      (m2a.south west)
      --
      cycle
    }
    \path [fill=shade1] \pathMII;

    \begin{scope}
      \path [clip] \pathMI;
      \path [fill=shade2] \pathMII;
    \end{scope}

    \begin{scope}[match line]
      \draw \pathMI;
      \draw \pathMII;
      \draw \pathMIII;
    \end{scope}
  \end{pgfonlayer}

  % Match labels
    \foreach \i in {1, 2} {
      \node [match label, position=-155 degrees from m\i] (m\i l) {$m_\i$};
      \draw [match attachment line] (m\i) -- (m\i l);
    }
  \node [match label, right=of m3, outer sep=1pt] (m3l) {$\strut m_3$};

  \foreach \i in {1, 2, 3} {
    \draw [match attachment line] (m\i) -- (m\i l);
  }
\end{tikzpicture}%
\endgroup%
%
                }%
  \hfill%
  \subcaptionbox{Cost matrix\labelFigure{cost-example-matrix}}%
                {%
                  \figureFontSize%
                  \begin{minipage}{50mm}%
                    \centering%
                    \def\Sstackgap{0pt}%
                    \begin{displaymath}
                      \begin{aMatrix}{cccr@{\;\times\;}r@{\;=\;}r}
                          \Shortstack{%
                            \raisebox{1mm}[0pt][0pt]{\tabhead op}
                            $\mathclap{\phantom{b}}% Needed to horizontally
                                                   % align labels
                             o_1$%
                          }
                        & \Shortstack{%
                            \raisebox{1mm}[0pt][0pt]{\tabhead match}
                            $\mathclap{\phantom{b}}% Needed to horizontally
                                                   % align labels
                             m_1$%
                          }
                        & \Shortstack{%
                            \raisebox{1mm}[0pt][0pt]{\tabhead block}
                            $b_1$%
                          }
                        & 4
                        & \Shortstack{%
                            \raisebox{1mm}[0pt][0pt]{%
                              \hspace{3pt}% Needed to vertically align label
                              \clap{\tabhead opcost}%
                            }
                            $\mathclap{\phantom{b}}% Needed to horizontally
                                                   % align labels
                             10$%
                          }
                        & 40 \\
                        o_1 & m_1 & b_2 & 4 &  1 &  4 \\
                        o_1 & m_3 & b_1 & 3 & 10 & 30 \\
                        o_1 & m_3 & b_2 & 3 &  1 &  3 \\
                        o_2 & m_2 & b_1 & 1 & 10 & 10 \\
                        o_2 & m_2 & b_2 & 1 &  1 &  1 \\
                        o_2 & m_3 & b_1 & 2 & 10 & 20 \\
                        o_2 & m_3 & b_2 & 2 &  1 &  2
                      \end{aMatrix}
                    \end{displaymath}
                  \end{minipage}%
                }%
  \hfill%
  \mbox{}

  \caption[Example of match costs distributed over operations]%
          {%
            Example of match costs distributed over operations.
            %
            It is assumed that matches~$m_1$, $m_2$, and~$m_3$ have
            costs~\num{4}, \num{1}, and~\num{5}, respectively, and that they can
            be placed in one of two blocks, $b_1$ and $b_2$, with execution
            frequencies~\num{10} and~\num{1}, respectively.
            %
            The cost of $m_3$ distributed over $o_1$ and $o_2$ is~\num{3}
            and~\num{2}, respectively%
          }
  \labelFigure{cost-example}
\end{figure}
%
Assume that a \gls{UF graph} can be covered by three \glspl{match}~$m_1$, $m_2$,
and~$m_3$ with costs~\num{4}, \num{1}, and~\num{5}, respectively
(\refFigure{cost-example-graph}).
%
Assume further that these \glspl{match} can be placed in one of two blocks,
$b_1$ and $b_2$, with execution frequencies~\num{10} and~\num{1}, respectively.
%
Because \refEquation{naive-objective-function} is modeled as a summation, it can
only \glsshort{propagation}[e] the bounds of the \gls{cost variable}.
%
Consequently, any \gls{match}~$m$ for which $\mVar{sel}[m]$ is still undecided
incurs a cost between zero (if not selected) and \mbox{$\mCost(m) \times
  \mMax(\cup_{b \,\in\, \mBlockSet} \mFreq(b))$} (if selected and placed in the
\gls{block} with highest execution frequency).
%
In the example above, this means the \gls{cost variable} is initially bounded as
\mbox{$0 \leq \mVar{cost} \leq 100$}.
%
These are very weak bounds as we know that either both $m_1$ and $m_2$ are
selected and placed in $b_2$, or $m_3$ is selected and placed in $b_2$,
resulting in a cost of at least~\num{5}.
%
Also, in the worst case all selected \glspl{match} are placed in $b_1$,
resulting in a cost of at most~\num{50}.

Instead of reasoning about the cost incurred by the \glspl{match} -- which may
or may not be selected -- a better approach is to deduce the cost incurred on
the \glspl{operation} -- which must always be covered.
%
The idea is as follows.
%
First, for each \gls{match}~\mbox{$m$\hspace{-.8pt},} evenly divide the cost of
$m$ over each \gls{operation}~$o$ covered by~\mbox{$m$\hspace{-.8pt}.}
%
If a strict partial order $<$ exists over the set of \glspl{operation}, and
$\mCovers(m)$ returns an ordered list that can be indexed starting from~\num{1},
then the cost can be computed as
%
\begin{equation}
  \mOpCost(m \hspace{-.8pt}, o) =
  \left\{
  \begin{array}{ll}
    q + 1 & \text{if $o < \mCovers(m)[r+1]$}, \\
    q     & \text{otherwise}, \\
  \end{array}
  \right.
  \labelEquation{div-mul-op-cost-function}
\end{equation}
%
where \mbox{$q = \lfloor \mCost(m) / \mCard{\!\mCovers(m)} \rfloor$} and
\mbox{$r = \mCost(m) \! \mod \mCard{\!\mCovers(m)}$}.
%
Consequently, for any \gls{match}~$m$, \mbox{$\mCost(m) = \sum_{o \,\in\,
    \mCovers(m)} \mCost(m \hspace{-.8pt}, o)$}.
%
Then, for each \gls{block}~$b$ we multiply \mbox{$\mOpCost(m\hspace{-.8pt}, o)$}
with the execution frequency of~\mbox{$b$\hspace{-1pt}.}
%
This information can be represented as a cost matrix
%
\begin{equation}
  \begin{aMatrix}{@{\:}c|c@{\:}}
      o \hspace{-1pt},
      m \hspace{-.8pt},
      b \hspace{-1pt},
      \big( \! \mOpCost(m, o) \times \mFreq(b) \big)
    & m \in \mMatchSet, o \in \mCovers(m), b \in \mBlockSet
  \end{aMatrix}
  \labelEquation{div-mul-cost-matrix}
\end{equation}
%
where each row denotes the cost of an \gls{operation}~$o$ if covered by a
\gls{match}~$m$ and placed in a \gls{block}~\mbox{$b$\hspace{-1pt}.}
%
From the cost matrix given in \refFigure{cost-example-matrix}, we can deduce
that the cost of covering \glspl{operation}~$o_1$ and~$o_2$ is between~\num{3}
and~\num{40} and between~\num{1} and~\num{20}, respectively.
%
Hence the total cost can be bounded as \mbox{$4 \leq \mVar{cost} \leq 60$},
which is a much tighter bound compared to that achieved using the naive
\gls{objective function}.

An alternative method for computing the cost is to \emph{first} multiply the
\gls{match} cost with the execution frequency and \emph{then} evenly divide the
product over the covered \glspl{operation}.
%
We call the first method the \gls!{divide-then-multiply method}, and the second
method the \gls!{multiply-then-divide method}.
%
For the latter, the cost matrix is computed as
%
\begin{equation}
  \begin{aMatrix}{@{\:}c|c@{\:}}
      \mTuple{
        o \hspace{-1pt},
        m \hspace{-.8pt},
        b \hspace{-1pt},
        \mOpCost(m \hspace{-.8pt}, o \hspace{-1pt}, b)
      }
    & m \in \mMatchSet, o \in \mCovers(m), b \in \mBlockSet
  \end{aMatrix}
  \labelEquation{mul-div-cost-matrix}
\end{equation}
%
where
%
\begin{equation}
  \mOpCost(m \hspace{-.8pt}, o \hspace{-1pt}, b) =
  \left\{
  \begin{array}{ll}
    q + 1 & \text{if $o < \mCovers(m)[r+1]$}, \\
    q     & \text{otherwise}, \\
  \end{array}
  \right.
  \labelEquation{mul-div-op-cost-function}
\end{equation}
%
where \mbox{$q = \lfloor d / \mCard{\!\mCovers(m)} \rfloor$}, \mbox{$r =
  d \hspace{-2pt} \mod \mCard{\!\mCovers(m)}$}, and \mbox{$d = \mCost(m) \times
  \mFreq(b)$}.
%
Consequently, for any \gls{match}~$m$ and \gls{block}~\mbox{$b$\hspace{-1pt},}
\mbox{$\mCost(m\hspace{-.8pt}, o) \times \mFreq(b) = \sum_{o \,\in\,
    \mCovers(m)} \mCost(m \hspace{-.8pt}, o \hspace{-1pt}, b)$.}

At first glace this design decision would appear to make no difference, but they
unexpectedly exhibit significantly different solving time characteristics, as
will be seen later in the experimental evaluation.


\paragraph{Variables}

The set of \glspl{variable} \mbox{$\mVar{ocost}[o] \in \mNatNumSet$} models the
cost incurred by covering \gls{operation}~\mbox{$o$\hspace{-.8pt}.}
%
It is assumed the \gls{domain} is the same as for the \gls{cost variable}.


\paragraph{Constraints}

For each \gls{operation}~\mbox{$o$\hspace{-.8pt},} the combination
\mbox{$o\hspace{-1pt}, \mVar{omatch}[o], \mVar{oplace}[o], \mVar{ocost}[o]$}
must appear as a row in the cost matrix.
%
Given a cost matrix~$\mCostMatrix$ computed using either
\refEquation{div-mul-cost-matrix} or~\refEquation*{mul-div-cost-matrix}, this
\gls{constraint} is modeled as
%
\begin{equation}
  \forall o \in \mOpSet :
  \mTable(
    \mTuple{
      o\hspace{-1pt},
      \mVar{omatch}[o],
      \mVar{oplace}[o],
      \mVar{ocost}[o]
    } \hspace{-1pt},
    \mCostMatrix
  ).
  \labelEquation{omatch-oplace-ocost-connection}
\end{equation}
%
The total cost is then modeled as
%
\begin{equation}
  \mVar{cost} = \sum_{\mathclap{o \in \mOpSet}} \mVar{ocost}[o].
  \labelEquation{total-cost}
\end{equation}

Note that the cost computed by \refEquation{total-cost} is exactly the same as
that computed by \refEquation{naive-objective-function}.
%
The proof is as follows.
%
Because every \gls{operation} must be covered by exactly one selected
\gls{match}, there is a one-to-one correspondence between a selected
\gls{match}~$m$ and the set~$Q$ of \glspl{operation} covered
by~\mbox{$m$\hspace{-.8pt}.}
%
Consequently, the total cost incurred by $m$ -- which is \mbox{$\mCost(m) \times
  \mFreq(b)$}, where $b$ is the \gls{block} wherein $m$ is placed -- should be
exactly the same as that incurred by the \glspl{operation} in $Q$.
%
Due to \refEquation{total-cost}, the total cost incurred by the
\glspl{operation} in $Q$ is \mbox{$\sum_{o \,\in\, Q} \mVar{ocost}[o]$}.
%
Due to \refEquationList{div-mul-cost-matrix, omatch-oplace-ocost-connection}, we
know that for each \mbox{$o \in Q$}, \mbox{$\mVar{ocost}[o] =
  \mCost(m \hspace{-.8pt}, o) \times \mFreq{b}$.}
%
Since \mbox{$\mCost(m) = \sum_{o \,\in\, \mCovers(m)} \mCost(m \hspace{-.8pt},
  o)$}, the total cost incurred by \glspl{operation} in $Q$ is \mbox{$\mCost(m)
  \times \mFreq(b)$.}
%
\hfill\qedsymbol


\section{Implied Constraints}
\labelSection{st-impl-constraints}

As explained in \refChapter{constraint-programming}, \gls{implied.c}
\glspl{constraint} are \glspl{constraint} that strengthen the \gls{propagation}
while preserving all \glspl{solution}.
%
Stronger \gls{propagation} leads to less \gls{search}, which in turn leads to
shorter solving times.
%
In this section, we introduce such \glspl{constraint} that are relevant for the
\glsshort{constraint model}.


\subsection{Implied Operation and Data Placements}

Due to \refEquationList{naive-dom, spanning}, if a selected \gls{match}
\gls{define.d}[s] some \gls{datum}~$d_1$ and \gls{use.d}[s] some other
\gls{datum}~$d_2$, then the \gls{block} wherein $d_2$ is \gls{define.d}[d] must
\gls{dominate.b} the \gls{block} wherein $d_1$ is \gls{define.d}[d].
%
From this observation, we can infer that if all \glspl{match} covering a
non-\gls{phi-node} \gls{operation}~$o$ do not \gls{span.b} any \glspl{block},
\gls{define.d} some \gls{datum}~$d_1$, and \gls{use.d} some \gls{datum}~$d_2$,
then the \gls{block} wherein $d_2$ is \gls{define.d}[d] must \gls{dominate.b}
the \gls{block} wherein $d_1$ is \gls{define.d}[d].
%
In addition, $o$ must be placed in the same \gls{block} wherein $d_1$ is
\gls{define.d}[d].
%
This is modeled as
%
\begin{equation}
  \begin{array}{@{}c@{}}
    \forall o \in
      \mSetBuilder{o'}%
                  {%
                    o' \in \mOpCompSet{\mPhi},
                    m \in \mMatchSet[o'] \hspace{-1pt}
                    \text{ \st }
                    \mConsumes(m) = \mEmptySet
                  }, \\
    \forall d_1 \in
      \mSetBuilder{d}%
                  {%
                    d \in \mDataOf(o \hspace{-1pt}, \mDefines),
                    m \in \mMatchSet[o] \hspace{-1pt},
                    \exists p \in \mDefines(m) :
                    \mDataSet[p] = \mSet{d}
                  }, \\
    \forall d_2 \in
      \mSetBuilder{d}%
                  {
                    d \in \mDataOf(o \hspace{-1pt}, \mUses),
                    m \in \mMatchSet[o] \hspace{-1pt},
                    \exists p \in \mUses(m) :
                    \mDataSet[p] = \mSet{d}
                  } : \\
    \mTable(
      \mTuple{\mVar{dplace}[d_1], \mVar{dplace}[d_2]} \hspace{-1pt},
      \mDomMatrix
    )
    \mAnd
    \mVar{oplace}[o] = \mVar{dplace}[d_1].
  \end{array}
  \labelEquation{impl-cons-defs-dominate-defs}
\end{equation}
%
where
%
\begin{equation}
  \mDataOf(o \hspace{-1pt}, f)
  \equiv
  \bigcup_{
    \mathclap{
      \substack{
        m \, \in \, \mMatchSet[o] \hspace{-.8pt}, \:
        p \, \in \, f\hspace{-1pt}(m) \text{ \st} \\
        \mCovers(m) \, = \, \mSet{o}
      }
    }
  }
  \mDataSet[p]
  \labelEquation{data-of-function}
\end{equation}

From the same observation, we can also infer that if all \glspl{match} covering
the same non-\gls{phi-node} \gls{operation} \gls{span.b} a set~$S$ of
\glspl{block} and \gls{define.d} some \gls{datum}~\mbox{$d$\hspace{-1pt},} then
$d$ must be \gls{define.d}[d] in one of the \glspl{block}
in~\mbox{$S$\hspace{-.8pt}.}
%
This is modeled as
%
\begin{equation}
  \begin{array}{@{}c@{}}
    \forall S \in \mPowerset{\mBlockSet} \hspace{-2pt},
    \forall d \in \mDataSet \!,
    \forall o \in
      \mSetBuilder*{o'}%
                   {
                     \begin{array}{@{}l@{}}
                       o' \in \mOpCompSet{\mPhi},
                       m \in \mMatchSet[o'] \hspace{-1pt},
                       \exists p \in \mDefines(m) : \\
                       \mSpans(m) = S \mAnd \mDataSet[p] = \mSet{d}
                     \end{array}
                   } : \\
    \mVar{dplace}[d] \in S.
  \end{array}
  \labelEquation{impl-cons-defs-in-spanned-blocks}
\end{equation}

From \refEquation{preventing-control-flow-op-moves}, we can infer that if all
non-\glspl{phi-match} covering \gls{operation}~$o$ have \gls{entry
  block}~\mbox{$b$\hspace{-1pt},} then $o$ must for sure be placed
in~\mbox{$b$\hspace{-1pt}.}
%
This is modeled as
%
\begin{equation}
  \begin{array}{@{}c@{}}
    \forall b \in \mBlockSet \hspace{-1pt},
    \forall o \in
      \mSetBuilder{o'}%
                  {
                    o' \in \mOpSet,
                    m \in \mMatchSet[o'] \setminus \mPhiMatchSet
                    \text{ \st }
                    \mEntry(m) = \mSet{b}
                  } : \\
    \mVar{oplace}[o] = b \hspace{-1pt}.
  \end{array}
  \labelEquation{impl-cons-identical-entry-blocks}
\end{equation}

From the same \gls{constraint}, we can also infer that if the \glspl{match}
covering the same non-\gls{phi-node} \gls{operation} all have identical
\glspl{entry block}, say~\mbox{$b$\hspace{-1pt},} and make \gls{use.d} of some
\gls{datum}~\mbox{$d$\hspace{-1pt},} then the \gls{block} wherein $d$ is
\gls{define.d}[d] must dominate~\mbox{$b$\hspace{-1pt}.}
%
This is modeled as
%
\begin{equation}
  \begin{array}{@{}c@{}}
    \forall b \in \mBlockSet \hspace{-1pt},
    \forall d \in
      \mSetBuilder*{d'}%
                   {%
                     \begin{array}{@{}l@{}}
                       o' \in \mOpCompSet{\mPhi},
                       m \in \mMatchSet[d'] \hspace{-1pt},
                       \exists p \in \mUses(m) : \\
                       \mEntry(m) = \mSet{b} \mAnd \mDataSet[p] = \mSet{d}
                     \end{array}
                   } : \\
    \mTable(
      \mTuple{b \hspace{-1pt}, \mVar{dplace}[d]} \hspace{-1pt},
      \mDomMatrix
    ).
  \end{array}
  \labelEquation{impl-cons-defs-dominate-entry-blocks}
\end{equation}

From \refEquation{def-edges}, we can infer that if a \gls{datum}~$d$ appears in
a \gls{definition edge}~$\mEdge{d}{b}$ and is \gls{define.d}[d] by
\glspl{phi-match} only, then the \gls{operation} covered by these \glspl{match}
must be placed~\mbox{$b$\hspace{-1pt}.}
%
This is modeled as
%
\begin{equation}
  \forall \mEdge{d}{b} \in \mFunctionDefEdgeSet,
  \forall o \in
    \mSetBuilder{o'}%
                {%
                  \begin{array}{@{}l@{}}
                    m \in \mMatchSet[d] \cap \mPhiMatchSet,
                    o' \in \mCovers(m)
                  \end{array}
                } :
  \mVar{oplace}[o] = b \hspace{-1pt}.
  \labelEquation{impl-cons-place-phi-ops-same-as-def-edges}
\end{equation}
%
It is assumed that the \glspl{edge} in $\mFunctionDefEdgeSet$ have been
reoriented such that all \glspl{source} are either \glsshort{state node} or
\glspl{value node} and all \glspl{target} are \glspl{block node}.


\subsection{Implied Constraints Due to the Define-Before-Use Refinement}

From \refEquationRange{refined-dom}{refined-dom-phi-operands}, we can infer the
following \gls{implied.c} \glspl{constraint}.

If a non-\gls{phi-match}~$m$ \gls{span.b}[ning] no \glspl{block} is selected,
then all \glspl{datum} \gls{use.d}[d] and \gls{define.d}[d] by $m$ must take
place in the same \gls{block}.
%
This is modeled as
%
\begin{equation}
  \begin{array}{@{}c@{}}
    \forall m \in
      \mSetBuilder{m'}%
                  {%
                    m \in \mPhiMatchCompSet \hspace{-1pt},
                    \mSpans(m) = \mEmptySet
                  },
    \forall p_1, p_2 \in \mUses(m) \text{ \st } p_1 < p_2 : \\
    \mVar{sel}[m] \mImp \mVar{uplace}[p_1] = \mVar{uplace}[p_2],
  \end{array}
  \labelEquation{impl-cons-no-span-uses}
\end{equation}
%
\begin{equation}
  \begin{array}{c}
    \forall m \in
      \mSetBuilder{m'}%
                  {%
                    m \in \mPhiMatchCompSet \hspace{-1pt},
                    \mSpans(m) = \mEmptySet
                  },
    \forall p_1, p_2 \in \mDefines(m) \text{ \st } p_1 < p_2 : \\
    \mVar{sel}[m]
    \mImp
    \mVar{dplace}[\mVar{alt}[p_1]] = \mVar{dplace}[\mVar{alt}[p_2]],
  \end{array}
  \labelEquation{impl-cons-no-span-defs}
\end{equation}
%
\begin{equation}
  \begin{array}{@{}c@{}}
    \forall m \in
      \mSetBuilder{m'}%
                  {%
                    m \in \mPhiMatchCompSet \hspace{-1pt},
                    \mSpans(m) = \mEmptySet
                  },
    \forall p_1 \in \mUses(m) \setminus \mDefines(m), \\
    \forall p_2 \in \mDefines(m) :
    \mVar{sel}[m]
    \mImp
    \mVar{uplace}[p_1] = \mVar{dplace}[\mVar{alt}[p_2]].
  \end{array}
  \labelEquation{impl-cons-no-span-use-defs}
\end{equation}

In addition, if a non-\gls{phi-match} \gls{span.b}[ning] some \glspl{block} is
selected, then all \gls{use.d}[s] of its input \glspl{datum} must occur in the
same \gls{block}.
%
This is modeled as
%
\begin{equation}
  \begin{array}{@{}c@{}}
    \forall m \in
      \mSetBuilder{m'}%
                  {%
                    m \in \mPhiMatchCompSet \hspace{-1pt},
                    \mSpans(m) \neq \mEmptySet
                  }, \\
    \forall p_1, p_2 \in \mUses(m) \setminus \mDefines(m)
    \text{ \st } p_1 < p_2 : \\
    \mVar{sel}[m] \mImp \mVar{uplace}[p_1] = \mVar{uplace}[p_2].
  \end{array}
  \labelEquation{impl-cons-spanned-input}
\end{equation}


\subsection{Implied Data Locations}

From \refEquation{compatible-locations}, we can infer several \gls{implied.c}
\glspl{constraint}.

If all non-\glspl{kill match} covering some \gls{operation} require some
non-\glsshort{state node} \gls{datum}~$d$ as input, then $d$ cannot be an
intermediate value nor be \gls{killed.d}.
%
Such \glspl{datum} is said to be \gls!{available.d}, meaning they cannot be
located in either $\mIntLocation$ or $\mKilledLocation$.
%
If the input can be one of several values (due to \glspl{alternative value}),
then at least one of those values must be made \gls{available.d}.
%
This is modeled as
%
\begin{equation}
  \begin{array}{@{}c@{}}
    \forall S \in \mPowerset{\mStateDataCompSet} \hspace{-2pt},
    \forall o \in
      \mSetBuilder{o'}%
                  {
                    o' \in \mOpSet \!,
                    m \in \mMatchSet[o'] \hspace{-1pt},
                    \exists p \in \mUses(m) \setminus \mDefines(m) :
                    \mDataSet[p] = S
                  }, \\
    \exists d \in S :
    \mVar{loc}[d] \notin \mSet{\mIntLocation, \mKilledLocation},
  \end{array}
  \labelEquation{impl-cons-used-data-must-be-available}
\end{equation}
%
where \mbox{$\mStateDataCompSet \subseteq \mDataSet$} denotes the set of
\glspl{datum} without the \glspl{state node}.

If all non-\glspl{kill match} defining a non-\glsshort{state node}
\gls{datum}~$d$ have $d$ as an \gls{exterior value}, then $d$ must be made
\gls{available.d}.
%
This is modeled as
%
\begin{equation}
  \begin{array}{@{}c@{}}
    \forall d \in
      \mSetBuilder*{d'}%
                   {%
                     \begin{array}{@{}l@{}}
                       d' \in \mStateDataCompSet,
                       m \in \mMatchSet[d'] \setminus \mKillMatchSet,
                       \exists p \in \mDefines(m) : \\
                       \mDataSet[p] = \mSet{d'} \mAnd \mIsExt(m, p)
                     \end{array}
                   } \!, \\
    \mVar{loc}[d] \notin \mSet{\mIntLocation, \mKilledLocation},
  \end{array}
  \labelEquation{impl-cons-exterior-data-must-be-available}
\end{equation}
%
where \mbox{$\mIsExt(m, p)$} denotes whether an \gls{operand}~$p$ in a
\gls{match}~$m$ represents an \gls{exterior value}.

We can always constrain the \glspl{location} of a non-\glsshort{state node}
\gls{datum}~$d$ to those \glspl{location} where the definers can
put~\mbox{$d$\hspace{-.8pt}.}
%
The intuition here is to take the union of all those \glspl{location}, which is
modeled as
%
\begin{equation}
  \begin{array}{@{}c@{}}
    \forall d \in \mStateDataCompSet,
    \forall S \in
      \mPowerset{\mLocationSet
      \, \cup \,
      \mSet{\mIntLocation, \mKilledLocation}} \text{ \st} \\
    S = \mSetBuilder{l}%
                    {%
                      m \in \mDataSet[d] \setminus \mKillMatchSet,
                      p \in \mDefines(m),
                      l \in \mStores(m, p)
                      \text{ \st }
                      d \in \mDataSet[p]
                    } : \\
    \mVar{loc}[d] \in S.
  \end{array}
  \labelEquation{impl-cons-locs-of-uses}
\end{equation}

Likewise, we can always constrain the \glspl{location} of a non-\glsshort{state
  node} \gls{datum}~$d$ to those \glspl{location} where the users can
access~\mbox{$d$\hspace{-.8pt}.}
%
Assuming there is always at least one \gls{match} making use of $d$), this is
similarly modeled as
%
\begin{equation}
  \begin{array}{@{}c@{}}
    \forall d \in \mStateDataCompSet,
    \forall S \in
      \mPowerset{\mLocationSet
      \, \cup \,
      \mSet{\mIntLocation, \mKilledLocation}} \text{ \st} \\
    S = \mSetBuilder{l}%
                    {%
                      m \in \mKillMatchCompSet,
                      p \in \mUses(m),
                      l \in \mStores(m, p)
                      \text{ \st }
                      d \in \mDataSet[p]
                    }
    \mAnd
    S \neq \mEmptySet : \\
    \mVar{loc}[d] \in S.
  \end{array}
  \labelEquation{impl-cons-locs-of-defs}
\end{equation}


\subsection{Implied Fall-Throughs}

Due to \refEquation{fall-through}, if for any two \glspl{block}~$b_1$ and~$b_2$
there exists a \gls{match} requiring $b_2$ to follow $b_1$ but there are no
\glspl{match} requiring any other \gls{block} to follow $b_1$ nor requiring
$b_2$ to follow any other \gls{block}, then it is always safe to force $b_2$ to
follow $b_1$.
%
This is modeled as
%
\begin{equation}
  \begin{array}{@{}c@{}}
    \forall b_1, b_2 \in \mBlockSet
    \text{ \st }
    \mSetBuilder{\mEntry(m)}{\mPair{m}{b_2} \in \mFallThroughSet} = \mSet{b_1}
    \mAnd \mbox{} \\
    \mSetBuilder{b}%
                {%
                  \mPair{m}{b} \in \mFallThroughSet
                  \text{ \st }
                  \mEntry(m) = \mSet{b_1}%
                } = \mSet{b_2} :
    \mVar{succ}[b_1] = b_2.
  \end{array}
  \labelEquation{impl-cons-fix-fall-throughs}
\end{equation}


\section{Symmetry and Dominance Breaking Constraints}
\labelSection{st-dom-breaking-constraints}

As explained in \refChapter{constraint-programming}, \glsshort{symmetry
  breaking.c} and \gls{dominance breaking.c} \glspl{constraint} are
\glspl{constraint} that remove \glspl{solution} from the \gls{search space} that
are either symmetric to one another or dominated by some other \gls{solution}.
%
Since this leads to a smaller \gls{search space}, the solving time is reduced.
%
In this section, we introduce such \glspl{constraint} that are relevant for the
\glsshort{constraint model}.


\subsection{Location of State Nodes}

Since \glspl{datum} also includes the \glspl{state node}, a
$\mVar{loc}$~\gls{variable} will be introduced for every \gls{state node}.
%
However, since \glspl{state node} are abstract entities used only to capture
implicit dependencies between certain \gls{operation}, the assignment to these
\glspl{variable} has no impact on code quality.
%
Hence many symmetric \glspl{solution} arise.
%
We remove these symmetries by always fixing the \gls{location} for each
\gls{state node}, which is modeled as
%
\begin{equation}
  \forall d \in \mStateDataSet :
  \mVar{loc}[d] = \mIntLocation,
  \labelEquation{dom-cons-locs-of-states}
\end{equation}
%
where \mbox{$\mStateDataSet \subseteq \mDataSet$} denotes the set of
\glspl{state node}.


\subsection{Operands of Non-Selected Matches}

The $\mVar{alt}$~\glspl{variable} of \glspl{match} that are not selected still
need to be assigned a value.
%
Since this assignment has no impact on code quality, it gives rise to many
symmetric \glspl{solution}.
%
We therefore fix the $\mVar{alt}$ assignments in such cases, which is modeled as
%
\begin{equation}
  \forall m \in \mMatchSet,
  \forall p \in \mDefines(m) \cup \mUses(m) :
  \mNot\mVar{sel}[m] \mImp \mVar{alt}[p] = \mMin(\mDataSet[\hspace{-1pt}p]).
  \labelEquation{dom-cons-operands-of-non-selected-matches}
\end{equation}

The \gls{symmetry breaking.c} \gls{constraint} above also implies that if an
\gls{operand} representing input with multiple \glspl{datum} does not take its
minimum value, then the corresponding \gls{match} must be selected.
%
In addition, the corresponding \gls{datum} must be made \gls{available.d}.
%
This is modeled as
%
\begin{equation}
  \begin{array}{@{}c@{}}
    \forall m \in \mMatchSet,
    \forall p \in \mUses(m) \setminus \mDefines(m)
    \text{ \st }
    \mCard{\mDataSet[p]} > 1 : \\
    \mVar{alt}[p] \neq \mMin(\mDataSet[p])
    \mImp
    \mVar{alt}[p] \notin \mSet{\mIntLocation, \mKilledLocation}.
  \end{array}
  \labelEquation{impl-cons-input-operands-not-taking-min-value}
\end{equation}


\subsection{Interchangeable Data}

As described in \refChapter{constraint-model} on
\refPageOfSection{modeling-value-reuse}, \glspl{datum} in the \gls{UF graph}
that are copies of the same value are \gls{copy-related.d} and therefore
interchangeable.
%
This is another source for symmetric \glspl{solution}, which is illustrated in
\refFigure{interchangeable-data-example}.

\begin{figure}
  \begin{minipage}[b]{58mm}%
    \centering%
    \subcaptionbox{%
                    UF graph, where the values \irVar*{v}[1] and \irVar*{v}[2]
                    constitute a chain of interchangeable data%
                    \labelFigure{interchangeable-data-example-annotated-graph}%
                  }%
                  [\textwidth]%
                  {%
                    \input{%
                      figures/solving-techniques/%
                      interchangeable-data-example-annotated-graph%
                    }%
                  }

    \vspace{\betweensubfigures}

    \subcaptionbox{%
                    Symmetries due to how data can be connected to operands%
                    \labelFigure{interchangeable-data-example-alt-solutions}%
                  }%
                  [\textwidth]%
                  {%
                    \input{%
                      figures/solving-techniques/%
                      interchangeable-data-example-alt-1%
                    }%
                    \hspace{8mm}%
                    \input{%
                      figures/solving-techniques/%
                      interchangeable-data-example-alt-2%
                    }%
                  }%
  \end{minipage}%
  \hfill%
  \begin{minipage}[b]{58mm}%
    \centering%
    \subcaptionbox{%
                    Symmetries due to how null-copy matches can be selected%
                    \labelFigure{interchangeable-data-example-null-solutions}%
                  }%
                  [\textwidth]%
                  {%
                    \input{%
                      figures/solving-techniques/%
                      interchangeable-data-example-null-1%
                    }%
                    \hspace{8mm}%
                    \input{%
                      figures/solving-techniques/%
                      interchangeable-data-example-null-2%
                    }%
                  }

    \vspace{\betweensubfigures}

    \subcaptionbox{%
                    Symmetries due to how kill matches can be selected%
                    \labelFigure{interchangeable-data-example-kill-solutions}%
                  }%
                  [\textwidth]%
                  {%
                    \input{%
                      figures/solving-techniques/%
                      interchangeable-data-example-kill-1%
                    }%
                    \hspace{8mm}%
                    \input{%
                      figures/solving-techniques/%
                      interchangeable-data-example-kill-2%
                    }%
                  }%
  \end{minipage}

  \caption[Example of interchangeable data]%
          {%
            Example of interchangeable data and how these give rise to
            symmetries%
          }
  \labelFigure{interchangeable-data-example}
\end{figure}

Assume that a \gls{UF graph} contains two \gls{copy-related.d} values,
\irVar*{v}[1] and~\irVar*{v}[2], which may both be connected to two
\glspl{operand}~$p_1$ and~$p_2$
(\refFigure{interchangeable-data-example-annotated-graph}).
%
We say that a set of values constitute a chain of \gls!{interchangeable.d}[
  \glspl{datum}] if they can be swapped in a \gls{solution} without affecting
the \gls{program} semantics.
%
This is the case if the values are all \gls{copy-related.d} and none are both
\gls{define.d}[d] and \gls{use.d}[d] by some \gls{match}.
%
In the above example, \irVar*{v}[1] and~\irVar*{v}[2] constitute such a chain
and can therefore be swapped for $p_1$ and~$p_2$, giving rise to unwanted
symmetric \glspl{solution}
(\refFigure{interchangeable-data-example-alt-solutions}).
%
The intuition here is to prevent \glspl{solution} containing ``cross-over''
connections between the values in a chain and the $\mVar{alt}$~\glspl{variable}.
%
As a precaution, however, we exclude \glspl{operand} used by \glspl{phi-match}
which may require such cross-over connections due to the \glspl{definition
  edge}.

If we assume that there exists a partial order $\leq$ for~$\mDataSet$, then we
can remove these symmetries by enforcing an order of the values assigned to the
$\mVar{alt}$~\glspl{variable}.
%
To this end, we use the \gls{value-precede-chain constraint} introduced in
\refChapter{constraint-programming} on \refPageOfSection{cp-vpc}.
%
Let $\mInterchDataSet$ denote the set of chains of \gls{interchangeable.d}
\glspl{datum} and \mbox{$\mOperandCompSet[\mPhi] \subseteq \mOperandSet$} denote
the set of \glspl{operand} not used by \glspl{phi-match}.
%
With these definitions, this \gls{constraint} is modeled as
%
\begin{equation}
  \begin{array}{@{}c@{}}
    \forall c \in \mInterchDataSet,
    \forall p_1, \ldots, p_k \in \mOperandCompSet[\mPhi]
    \text{ \st }
    p_1 \neq \cdots \neq p_k
    \mAnd
    (\forall 1 \leq i \leq k : \mDataSet[p_i] \! = c) : \\
    \mValuePrecChain(c, \mVar{alt}[p_1], \ldots, \mVar{alt}[p_k]).
  \end{array}
  \labelEquation{dom-cons-interch-data-chains}
\end{equation}

Additional symmetries may appear due to \glspl{null-copy match}.
%
Returning to the example above, if one of the two \glspl{copy node} needs to be
covered using a \gls{copy match} derived from actual copy \gls{instruction},
then we are free to decide which.
%
Intuitively, we want to prevent \glspl{solution} where selected \glspl{null-copy
  match} ``appear to the left'' of a non-\gls{null-copy match}.
%
Let $\mInterchDataSet[\mCopy]$ denote the set of chains of \glspl{datum} that
can only be defined by \glspl{copy match}, \mbox{$\mNullCopyMatchSet \subseteq
  \mMatchSet$} denote the set of \glspl{null-copy match}, and
\mbox{$\mMatchSet[d] \subseteq \mMatchSet$} denote the set of \glspl{match} that
can define a \gls{datum}~\mbox{$d$\hspace{-.8pt}.}
%
Assuming there exists exactly one \gls{null-copy match} to cover each \gls{copy
  node}, this is modeled as
%
\begin{equation}
  \begin{array}{@{}c@{}}
    \forall c \in \mInterchDataSet[\mCopy],
    \forall 1 \leq i < k,
    \exists m_i \in \mMatchSet[c[i]] \cap \mNullCopyMatchSet : \\
    \mIncreasing(\mVar{sel}[m_1], \ldots, \mVar{sel}[m_k]),
  \end{array}
  \labelEquation{dom-cons-null-copy-match-selection}
\end{equation}
%
where
%
\begin{equation}
  \mIncreasing(\mVar{x}_1, \ldots, \mVar{x}_k)
  \equiv
  \mBigAnd_{\mathclap{1 \,\leq\, i \,<\, k}}
  \mVar{x}_{i} \leq \mVar{x}_{i + 1}
  \labelEquation{increasing-function}
\end{equation}

Similarly to \glspl{null-copy match}, symmetries can also arise due to
\glspl{kill match}.
%
Intuitively, we want to prevent \glspl{solution} where \gls{killed.d}
\glspl{datum} ``appear to the right'' of non-\gls{killed.d} \glspl{datum}.
%
Assuming there exists exactly one \gls{kill match} to cover each \gls{copy
  node}, this is modeled as
%
\begin{equation}
  \begin{array}{@{}c@{}}
    \forall c \in \mInterchDataSet[\mCopy],
    \forall 1 \leq i < k,
    \exists m_i \in \mMatchSet[c[i]] \cap \mKillMatchSet : \\
    \mIncreasing(\mVar{sel}[m_1], \ldots, \mVar{sel}[m_k]),
  \end{array}
  \labelEquation{dom-cons-kill-match-selection}
\end{equation}
%
where $\mKillMatchSet$ denotes the set of \glspl{kill match}.


\section{Tightening the Cost Bounds}
\labelSection{st-cost-bounds}

As explained in \refChapter{constraint-programming}, in \gls{CP} optimization
problems are solved using \gls{branch and bound}.
%
In other words, when a \gls{solution} is found a \gls{constraint} is added to
the \glsshort{constraint model}, forcing any subsequently found \glspl{solution}
to be strictly better.
%
For this to be effective, however, the cost bounds must be tight.
%
A tight upper bound enables the \gls{constraint solver} to prune away parts of
the \gls{search space} that only contains inferior \glspl{solution}.
%
This is most useful for proving whether a particular \gls{solution} is optimal.
%
Likewise, a tight lower bound enables the \gls{constraint solver} to prune away
parts of the \gls{search space} that contains no \glspl{solution}.
%
This is partially achieved by the refined \gls{objective function} introduced in
\refSection{st-refined-objective-function}, but the bounds can be further
tightened using complementary mechanisms.

The upper bound can be further tightened by solving the same problem using a
greedy but fast heuristic.
%
To this end, any modern \gls{compiler} can be used.
%
The lower bound can be further tightened by solving a relaxed -- and thereby
simpler -- version of the \gls{constraint model}.
%
In this context, a relaxed version corresponds to a \glsshort{constraint model}
that only integrates \gls{global.is} \gls{instruction selection} and \gls{block
  ordering}.
%
Hence the relaxed \glsshort{constraint model} consists of only the
$\mVar{omatch}$, $\mVar{opcosts}$, $\mVar{sel}$, $\mVar{succ}$, and
$\mVar{cost}$ variables, \refEquation{operation-coverage}, relaxed versions of
\refEquationList{block-order, fall-through} that allow fall-throughs via
non-\gls{empty.b} \glspl{block}, and modified versions of
\refEquationList{div-mul-op-cost-function, div-mul-cost-matrix, total-cost} that
are optimistic about the execution frequencies.

If $\mRelaxedCost$ and $\mHeuristicCost$ denote the cost computed from the
relaxation and by the heuristic, respectively, then the \gls{cost variable} is
bounded as
%
\begin{equation}
  \mRelaxedCost \leq \mVar{cost} < \mHeuristicCost.
  \labelEquation{cost-bounds}
\end{equation}


\section{Branching Strategies}
\labelSection{st-branching-strategies}

As explained in \refChapter{constraint-programming}, the \gls{branching
  strategy} decides how to explore the \gls{search space}.
%
Following the \gls{first-fail principle} (see \refSection{cp-search} on
\refPageOfSection{cp-search}), we first branch on the
$\mVar{ocost}$~\glspl{variable}.
%
When branching on these \glspl{variable}, we select the \gls{variable}~$v$ with
largest difference between the two smallest values in its \gls{domain} -- this
is typically called the maximum \gls!{regret}~\cite{Sen:2010} -- and the
smallest value in the \gls{domain} of~\mbox{$v$\hspace{-1pt}.}
%
The intuition here is that, because the \gls{objective function} strives to
minimize the total cost, we wish to minimize the cost incurred per
\gls{operation}.
%
Hence we try to cover the \glspl{operation} for which the cost of bad decisions
is largest, and we try to do so at least cost.\!%
%
\footnote{%
  In the field of decision theory, this is known as the \gls!{minimax
    approach}~\cite{Savage:1951}.
}
%
Note that this \gls{branching strategy} is only possible due to the refined
\gls{objective function} introduced in
\refSection{st-refined-objective-function}.
%
Remaining decisions are left to the \gls{constraint solver}['s] discretion;
%
in the experiments, we use \gls!{free search} which alternates between
user-specified and activity-based \gls{search} when \gls{search} is restarted
(set to~\num{100}).

To improve solving, we make sure to arrange the \glspl{match} in order of
increasing cost.
%
If there is a tie between two \glspl{match}, and either of them is a \gls{kill
  match}, then the \gls{kill match} comes first.
%
Otherwise, the \gls{match} covering more \glspl{operation} comes first (hence
mimicking the scheme of \gls{maximum munch}).
%
This is because the \gls{constraint solver} will most likely attempt to select
\glspl{match} in the order given to the \glsshort{constraint model}.
%
In such a setting, it is generally a good approach to first try a \gls{kill
  match} -- which incurs no cost and encourages value reuse -- and then try the
\gls{match} incurring the least cost.


\section{Presolving}
\labelSection{st-presolving}

As explained in \refChapter{constraint-programming}, \gls{presolving} is the
process of applying problem-specific algorithms to reduce the number of
\glspl{variable} or to shrink the \gls{variable} \glspl{domain} before
solving.\!%
%
\footnote{%
  In this sense, the bound tightening technique described in
  \refSection{st-cost-bounds} is a form of \gls{presolving}.%
}
%
In this section, \gls{presolving} is used to remove \glspl{match} which can be
safely removed without compromising code quality.
%
This directly translates to fewer $\mVar{alt}$, $\mVar{sel}$, and
$\mVar{uplace}$~\glspl{variable}, smaller $\mVar{dplace}$ and
$\mVar{oplace}$~\glspl{domain}, as well as fewer \glspl{constraint} that need to
be managed by the \gls{constraint solver}.


\subsection{Dominated Matches}
\labelSection{st-pre-dom-matches}

If two \glspl{match} are equal in all respects except cost, then the \gls{match}
with greater cost is \gls{dominate.m}[d] and can thus safely be removed from the
\gls{match set}.
%
A \gls{match}~$m_1$ is \gls!{dominate.m}[d] if there exists another
\gls{match}~$m_2$ such that
%
\begin{itemize}
  \item $m_1$ has greater than or equal cost to $m_2$,
  \item both cover the same \glspl{operation},
  \item both have the same \glspl{entry block} (if any),
  \item both \gls{span.b} the same \glspl{block} (if any),
  \item both have the same \glspl{definition edge} (if any),
  \item $m_1$ has at least as strong \gls{location} requirements on its
    \glspl{datum} as $m_2$ -- that is, $\forall p_1 \in \mUses(m_1) \cup
    \mDefines(m_1) : \exists p_2 \in \mUses(m_2) \cup \mDefines(m_2) :
    \mDataSet[p_1] \subseteq \mDataSet[p_2] \mAnd \mStores(m_1, p_1) \subseteq
    \mStores(m_2, p_2)$ -- and
  \item both apply the same additional \glspl{constraint} (if any) when
    selected.
\end{itemize}
%
As a precaution, we assume that \glspl{null match}, \glspl{phi-match}, and
\glspl{match} with \gls{fall-through} conditions can never be
\gls{dominate.m}[d].

The method above can be generalized to letting combinations of \glspl{match} to
be jointly \gls{dominate.m}[d] by another \gls{match}.
%
Intuitively, if the combination of \glspl{match} can be selected, then the
\gls{solution} can always be improved by replacing them with the single
\gls{match}.
%
The idea is to combine the \glspl{match} into a single
\gls{match}~\mbox{$m$\hspace{-.8pt},} and then check whether the above
conditions for \glsshort{dominate.m}[ance] apply with the additional check that
none of intermediate values of $m$ are used by other \glspl{match}.
%
An example is shown in \refFigure{dominated-matches-example}.

\begin{figure}
  \centering%
  % Copyright (c) 2017-2018, Gabriel Hjort Blindell <ghb@kth.se>
%
% This work is licensed under a Creative Commons Attribution-NoDerivatives 4.0
% International License (see LICENSE file or visit
% <http://creativecommons.org/licenses/by-nc-nd/4.0/> for details).
%
\begingroup%
\setlength{\nodeDist}{12pt}%
\figureFont\figureFontSize%
\pgfdeclarelayer{background}%
\pgfsetlayers{background,main}%
\adjustbox{valign=M}{%
  \begin{tikzpicture}[
      outer match node/.style={
        match node,
        draw=none,
        inner sep=3pt,
      },
      match label/.append style={
        outer sep=0,
      },
    ]

    \node [computation node] (o2) {$o_2$};
    \node [value node, position=135 degrees from o2] (c) {\nVar{v}[3]};
    \node [value node, position= 45 degrees from o2] (d) {\nVar{v}[4]};
    \node [value node, below=of o2] (e) {\nVar{v}[5]};
    \node [computation node, above=of c] (o1) {$o_1$};
    \node [value node, position=135 degrees from o1] (a) {\nVar{v}[1]};
    \node [value node, position= 45 degrees from o1] (b) {\nVar{v}[2]};

    \begin{scope}[data flow]
      \draw (a) -- (o1);
      \draw (b) -- (o1);
      \draw (o1) -- (c);
      \draw (c) -- (o2);
      \draw (d) -- (o2);
      \draw (o2) -- (e);
    \end{scope}

    \begin{pgfonlayer}{background}
      % m3
      \begin{scope}[outer match node/.append style={inner sep=8pt}]
        \node [outer match node, fit=(a)] (m3a) {};
        \node [outer match node, fit=(b)] (m3b) {};
        \node [outer match node, fit=(d)] (m3c) {};
        \node [outer match node, fit=(e)] (m3d) {};
      \end{scope}

      \def\pathMIII{
        [rounded corners=8pt]
        (m3a.north west)
        --
        (m3b.north east)
        --
        (m3c.north east)
        --
        (m3c.south east)
        --
        (m3d.south east)
        --
        (m3d.south west)
        -- coordinate (m3)
        (m3a.south west)
        --
        cycle
      }
      \path [fill=shade3] \pathMIII;

      % m1
      \begin{scope}[outer match node/.append style={inner sep=5pt}]
        \node [outer match node, fit=(a)] (m1a) {};
        \node [outer match node, fit=(b)] (m1b) {};
        \node [outer match node, fit=(c)] (m1c) {};
      \end{scope}

      \def\pathMI{
        [rounded corners=8pt]
        (m1a.north west)
        --
        (m1b.north east)
        --
        (m1b.south east)
        --
        (m1c.south east)
        --
        (m1c.south west)
        -- coordinate (m1)
        (m1a.south west)
        --
        cycle
      }
      \path [fill=shade1] \pathMI;

      % m2
      \begin{scope}[outer match node/.append style={inner sep=5pt}]
        \node [outer match node, fit=(c)] (m2a) {};
        \node [outer match node, fit=(d)] (m2b) {};
        \node [outer match node, fit=(e)] (m2c) {};
      \end{scope}

      \def\pathMII{
        [rounded corners=8pt]
        (m2a.north west)
        --
        (m2b.north east)
        --
        (m2b.south east)
        --
        (m2c.south east)
        --
        (m2c.south west)
        -- coordinate (m2)
        (m2a.south west)
        --
        cycle
      }
      \path [fill=shade1] \pathMII;

      \begin{scope}
        \path [clip] \pathMI;
        \path [fill=shade2] \pathMII;
      \end{scope}

      \begin{scope}[match line]
        \draw \pathMI;
        \draw \pathMII;
        \draw \pathMIII;
      \end{scope}
    \end{pgfonlayer}

    % Match labels
    \foreach \i in {1, 2, 3} {
      \node [match label, position=-155 degrees from m\i] (m\i l) {$m_\i$};
      \draw [match attachment line] (m\i) -- (m\i l);
    }
  \end{tikzpicture}%
}%
\hfill%
\begin{tabular}{cccccc}
  \toprule
      \tabhead match
    & \tabhead \irVar{v}[1]
    & \tabhead \irVar{v}[2]
    & \tabhead \irVar{v}[3]
    & \tabhead \irVar{v}[4]
    & \tabhead \irVar{v}[5] \\
  \midrule
      $m_1$
    & $l_1 \ldots l_4$
    & $l_1 \ldots l_4$
    & $l_1 \ldots l_4$
    &
    & \\
      $m_2$
    &
    &
    & $l_3 \ldots l_4$
    & $l_3 \ldots l_4$
    & $l_3 \ldots l_6$ \\
      $m_3$
    & $l_1 \ldots l_6$
    & $l_1 \ldots l_6$
    & $\mIntLocation$
    & $l_1 \ldots l_6$
    & $l_1 \ldots l_6$ \\
  \bottomrule
\end{tabular}%
\endgroup%


  \caption[Example of dominated matches]%
          {%
            Example where matches~$m_1$ and $m_2$ are jointly dominated by
            match~$m_3$ and can therefore safely be removed (provided that no
            other match uses value~\irVar*{v}[3]).
            %
            The table contains the location restrictions enforced by each match%
          }
  \labelFigure{dominated-matches-example}
\end{figure}


\subsection{Illegal Matches}

Depending on the \gls{instruction set}, the \gls{match set} may contain
\glspl{match} that will -- for one reason or another -- never participate in any
\gls{solution}.
%
Such \glspl{match} are said to be \gls!{illegal.m}.

One set of \gls{illegal.m} \glspl{match} are those which would leave some
\gls{operation} uncoverable if selected.
%
In \refFigure{uncovered-operations-example}, for example, selecting
\gls{match}~$m_1$ would leave \gls{operation}~$o_2$ uncovered since it can only
be covered by match~$m_2$.
%
However, selection of $m_2$ is inhibited if $m_1$ is selected.
%
\begin{figure}
  \centering%
  % Copyright (c) 2017-2018, Gabriel Hjort Blindell <ghb@kth.se>
%
% This work is licensed under a Creative Commons Attribution-NoDerivatives 4.0
% International License (see LICENSE file or visit
% <http://creativecommons.org/licenses/by-nc-nd/4.0/> for details).
%
\begingroup%
\setlength{\nodeDist}{12pt}%
\figureFont\figureFontSize%
\pgfdeclarelayer{background}%
\pgfsetlayers{background,main}%
\adjustbox{valign=M}{%
  \begin{tikzpicture}[
      outer match node/.style={
        match node,
        draw=none,
        inner sep=3pt,
      },
    ]

    \node [computation node] (o1) {$o_1$};
    \node [value node, position=135 degrees from o1] (v1) {};
    \node [value node, position= 45 degrees from o1] (v2) {};
    \node [value node, below=of o1] (v3) {};
    \node [computation node, below=of v3] (o2) {$o_2$};
    \node [value node, below=of o2] (v4) {};

    \begin{scope}[data flow]
      \draw (v1) -- (o1);
      \draw (v2) -- (o1);
      \draw (o1) -- (v3);
      \draw (v3) -- (o2);
      \draw (o2) -- (v4);
    \end{scope}

    \begin{pgfonlayer}{background}
      % m2
      \begin{scope}[outer match node/.append style={inner sep=8pt}]
        \node [outer match node, fit=(v1)] (m2a) {};
        \node [outer match node, fit=(v2)] (m2b) {};
        \node [outer match node, fit=(v4), inner sep=5pt] (m2c) {};
      \end{scope}

      \def\pathMII{
        [rounded corners=8pt]
        (m2a.north west)
        --
        (m2b.north east)
        --
        (m2b.south east)
        --
        (m2c.south east)
        --
        (m2c.south west)
        -- coordinate (m2)
        (m2a.south west)
        --
        cycle
      }
      \draw [match line, fill=shade2] \pathMII;

      % m1
      \begin{scope}[outer match node/.append style={inner sep=5pt}]
        \node [outer match node, fit=(v1)] (m1a) {};
        \node [outer match node, fit=(v2)] (m1b) {};
        \node [outer match node, fit=(v3)] (m1c) {};
      \end{scope}

      \def\pathMI{
        [rounded corners=8pt]
        (m1a.north west)
        --
        (m1b.north east)
        --
        (m1b.south east)
        --
        (m1c.south east)
        --
        (m1c.south west)
        --
        (m1a.south west)
        -- coordinate (m1)
        cycle
      }
      \draw [match line, fill=shade1] \pathMI;
    \end{pgfonlayer}

    % Match labels
    \node [match label, left=of m1] (m1l) {$\strut m_1$};
    \node [match label, position=-165 degrees from m2, outer sep=0]
          (m2l) {$m_2$};
    \foreach \i in {1, 2} {
      \draw [match attachment line] (m\i) -- (m\i l);
    }
  \end{tikzpicture}%
}%
\hspace{6mm}%
\begin{tabular}{@{}l@{}}
  $\mMatchSet[o_1] = \mSet{m_1, m_2}$ \\
  $\mMatchSet[o_2] = \mSet{m_2}$
\end{tabular}%
\endgroup%


  \caption[Example of an illegal match]%
          {%
            Example of an illegal match, where selecting match~$m_1$ would leave
            operation~$o_2$ uncovered%
          }
  \labelFigure{uncovered-operations-example}
\end{figure}
%
Hence this set of \gls{illegal.m} \glspl{match} is computed as
%
\begin{equation}
  \mSetBuilder{m}%
              {
                m \in \mMatchSet,
                o_1, o_2 \in \mOpSet
                \text{ \st }
                \mMatchSet[o_1] \subset \mMatchSet[o_2]
                \mAnd
                m \in \mMatchSet[o_2]
              }.
  \labelEquation{illegal-matches-uncovered-ops}
\end{equation}

Likewise, a \gls{match} is \gls{illegal.m} if selecting it would
leave some \gls{datum} un\gls{define.d}[d].
%
With similar reasoning, this set of \gls{illegal.m} \glspl{match} is computed as
%
\begin{equation}
  \mSetBuilder{m}%
              {
                m \in \mMatchSet,
                d_1, d_2 \in \mDataSet
                \text{ \st }
                \mMatchSet[d_1] \subset \mMatchSet[d_2]
                \mAnd
                m \in \mMatchSet[d_2]
              }.
  \labelEquation{illegal-matches-undefined-data}
\end{equation}

If a \gls{kill match}~$m$ \gls{define.d}[s] a \gls{datum}~$d$ and every
\gls{match} using $d$ has no \glsshort{alternative value}[s]
but~\mbox{$d$\hspace{-.8pt},} then $m$ is \gls{illegal.m} as $d$ must be
\gls{define.d}[d] by a non-\gls{kill match}.
%
This set of \gls{illegal.m} \glspl{match} is computed as
%
\begin{equation}
  \mSetBuilder*{m_1}%
               {%
                 \begin{array}{@{}l@{}}
                   m_1 \in \mKillMatchSet,
                   p_1 \in \mDefines(m_1),
                   d \in \mDataSet[p_1], \\
                   m_2 \in \mKillMatchCompSet,
                   p_2 \in \mUses(m_2) \text{ \st }
                   d \in \mDataSet[p_2] \mImp \mDataSet[p_2] = \mSet{d}
                 \end{array}
               }\!\!.
  \labelEquation{illegal-matches-kills}
\end{equation}

If a \gls{match}~$m$ is not a \gls{kill match} and \gls{define.d}[s] a
\gls{datum}~$d$ in a \gls{location} that cannot be accessed by any of the
\glspl{match} \glsshort{use.d}[ing] of~\mbox{$d$\hspace{-.8pt},} then $m$ is
\gls{illegal.m}.
%
This set of \gls{illegal.m} \glspl{match} is computed as
%
\begin{equation}
  \mSetBuilder*{m}%
               {%
                 \begin{array}{@{}l@{}}
                   m \in \mKillMatchCompSet,
                   p \in \mDefines(m),
                   d \in \mDataSet[p] \hspace{-1pt} \text{ \st} \\
                   \mIsExt(m, p)
                   \mAnd
                   \mCupUseLocsOf(d) \neq \mEmptySet
                   \mAnd \mbox{} \\
                   \mStores(m, p) \cap \mCupUseLocsOf(d) = \mEmptySet
                 \end{array}
               }\!\!,
  \labelEquation{illegal-matches-def-locs}
\end{equation}
%
where
%
\begin{equation}
  \mCupUseLocsOf(d)
  \equiv
  \hspace{-2.5em}
  \bigcup_{
    \substack{
      m \, \in \, \mMatchSet[d] \setminus \mKillMatchSet, \\
      p \, \in \, \mUses(m)
      \text{ \st } d \, \in \, \mDataSet[p]
    }
  }
    \hspace{-2.5em}
    \mStores(m, p).
  \labelEquation{cup-use-locs-of-function}
\end{equation}
%
Note that if \mbox{$\mCupUseLocsOf(d) = \mEmptySet$} holds, then the
\glspl{match} \glsshort{use.d}[ing] \gls{datum}~$d$ have themselves conflicting
\gls{location} requirements.
%
In such cases, we cannot infer whether a \gls{match} \glsshort{define.d}[ing]
$d$ is \gls{illegal.m}.

Similarly, if a \gls{match}~$m$ is not a \gls{kill match} and \gls{use.d}[s] a
\gls{datum}~$d$ from a \gls{location} that cannot be written to by any of the
\glspl{match} \glsshort{define.d}[ing]~\mbox{$d$\hspace{-.8pt},} then $m$ can
never be selected and is thus \gls{illegal.m}.
%
This set of \gls{illegal.m} \glspl{match} is computed as
%
\begin{equation}
  \mSetBuilder*{m}%
               {%
                 \begin{array}{@{}l@{}}
                   m \in \mKillMatchCompSet,
                   p \in \mUses(m) \setminus \mDefines(m),
                   d \in \mDataSet[p] \hspace{-1pt} \text{ \st} \\
                   \mCupDefLocsOf(d) \neq \mEmptySet
                   \mAnd
                   \mStores(m, p) \cap \mCupDefLocsOf(d) = \mEmptySet
                 \end{array}
               }\!\!,
  \labelEquation{illegal-matches-use-locs}
\end{equation}
%
where
%
\begin{equation}
  \mCupDefLocsOf(d)
  \equiv
  \bigcup_{
    \mathclap{
      \substack{
        m \, \in \, \mMatchSet[d] \setminus \mKillMatchSet, \\
        p \, \in \, \mDefines(m)
        \text{ \st } d \, \in \, \mDataSet[p]
      }
    }
  }
  \mStores(m, p).
  \labelEquation{cup-def-locs-of-function}
\end{equation}


\subsection{Redundant Matches}

In certain circumstances a \gls{match} can safely be removed without
compromising code quality.
%
Such \glspl{match} are said to be \gls!{redundant.m}.

For example, if there exists a \gls{null-copy match} to cover a \gls{copy
  node}~$c$, then the \gls{kill match} covering $c$ is \gls{redundant.m} since
it is always safe to select the \gls{null-copy match} over the \gls{kill match}.
%
Consequently, all \glspl{kill match} covering \glspl{copy node} that take a
non-constant value as input -- which can never be covered using a \gls{null-copy
  match} -- are \gls{redundant.m}.
%
This makes sense as the \glspl{kill match} are added to the \gls{match set} as a
consequence of \glspl{alternative value}, which are used to handle cases where
loaded constants could be reused among \glspl{match}.
%
Hence this set of \gls{redundant.m} \glspl{match} is computed as
%
\begin{equation}
  \mSetBuilder{m}%
              {%
                m \in \mKillMatchSet,
                o \in \mCovers(m)
                \text{ \st }
                \mMatchSet[o] \cap \mNullCopyMatchSet \neq \mEmptySet
              }.
  \labelEquation{redun-kills}
\end{equation}

Another case concerns non-\glspl{null match} that cover a \gls{copy node}.
%
Assume a copy chain \mbox{$\mEdge{v_1}{\mEdge{c}{v_2}}$}, where $c$ is a
\gls{copy node} and $v_1$ and $v_2$ are \glspl{value node}.
%
If every \gls{match} \glsshort{define.d}[ing] $v_1$ writes the value to a
\gls{location} that can be used by all \glspl{match}
\glsshort{use.d}[ing]~$v_2$, then all non-\glspl{null match} covering $c$ are
\gls{redundant.m} since a \gls{null-copy match} can always be selected to cover
$c$.
%
We exclude, however, \glspl{copy node} that take a constant value as input since
such \glspl{node} can never be covered by a \gls{null-copy match}.
%
We also exclude \glspl{copy node} whose \gls{define.d}[d] \gls{datum} is
\gls{use.d}[d] by some \gls{phi-match} since an actual copy may be needed to
satisfy \refEquation{phi-match-locations}.
%
Let \mbox{$\mCopyMatchSet \subseteq \mMatchSet$} denote the set of \glspl{copy
  match} and \mbox{$\mConstDataSet \subseteq \mDataSet$} denote the set of
\glspl{datum} representing constant values.
%
Let also \mbox{$\mCapUseLocsOf(d) \subseteq \mLocationSet$} and
\mbox{$\mCapDefLocsOf(d) \subseteq \mLocationSet$} denote the intersection of
all \glspl{location} for all \gls{match} where a \gls{datum}~$d$ is
\gls{use.d}[d] respectively \gls{define.d}[d].
%
Using these definitions, this set of \gls{redundant.m} \glspl{match} is computed as
%
\begin{equation}
  \mSetBuilder*{m}%
               {%
                 \begin{array}{@{}l@{}}
                   m \in \mCopyMatchSet \setminus \mNullMatchSet,
                   d_1 \in \mUses(m),
                   d_2 \in \mDefines(m) \\
                   \text{\st }
                   \mDataSet[d_1] \cap \mPhiMatchSet = \mEmptySet
                   \mAnd
                   \mDataSet[d_2] \cap \mPhiMatchSet = \mEmptySet
                   \mAnd
                   d_1 \notin \mConstDataSet \\
                   \mbox{}\hspace{-2pt} \mAnd
                   \mCapUseLocsOf(d_1) \cap \mCapDefLocsOf(d_2) \neq \mEmptySet
                 \end{array}
               }\!\!.
  \labelEquation{redun-non-null-copy-matches}
\end{equation}
%
As can be deduced from their name, $\mCapUseLocsOf$ and $\mCapDefLocsOf$ are
defined similarly to $\mCupUseLocsOf$ and $\mCupDefLocsOf$ (see
\refEquationList{cup-use-locs-of-function, cup-def-locs-of-function}), with the
exception that \glspl{location} known to violate
\refEquation{phi-match-locations} are removed from these sets.


\subsection{Canonical Locations}
\labelSection{st-canonical-locations}

For most architectures, its \glspl{instruction} read from and write to the same
set of \glspl{register}.
%
This gives rise to many symmetric \glspl{solution} as the exact \gls{location}
assigned to a value often does not matter.
%
We can remove these symmetries by removing \glspl{location} which are considered
symmetric to one another.

The idea is to select a \gls{location} as a representative for each distinct
intersection made by the storage requirements.
%
See for example \refFigure{canonical-locations-example}.
%
\begin{figure}
  \centering%
  % Copyright (c) 2017-2018, Gabriel Hjort Blindell <ghb@kth.se>
%
% This work is licensed under a Creative Commons Attribution-NoDerivatives 4.0
% International License (see LICENSE file or visit
% <http://creativecommons.org/licenses/by-nc-nd/4.0/> for details).
%
\begingroup%
\figureFont\figureFontSize%
\pgfdeclarelayer{foreground}%
\pgfsetlayers{main,foreground}%
\def\gridYSize{4mm}%
\def\gridXSize{5mm}%
\def\axisWidth{1pt}%
\def\mkReg#1{\instrFont r$_{\text{#1}}$}
\begin{tikzpicture}[%
    y label/.style={
      nothing,
      node distance=6pt,
    },
    x label/.style={
      nothing,
      node distance=3pt,
      font=\strut,
    },
    location range/.style={
      nothing,
      fill=shade1,
      line width=\normalLineWidth,
    },
    location range line/.style={
      draw,
      line width=\normalLineWidth,
    },
    slice line/.style={
      line width=\normalLineWidth,
      dash pattern={on 1.8pt off 2.2pt},
      draw=shade3,
    }
  ]

  % Grid
  \foreach \y in {0, ..., 4} {
    \foreach \x in {0, ..., 10} {
      \pgfmathtruncatemacro{\py}{\y - 1};
      \pgfmathtruncatemacro{\px}{\x - 1};
      \ifnum \x>0
        \coordinate (\y-\x) at ([xshift=\gridXSize] \y-\px);
      \else
        \ifnum \y>0
          \coordinate (\y-\x) at ([yshift=-\gridYSize] \py-\x);
        \else
          \coordinate (\y-\x);
        \fi
      \fi
    }
  }

  % Location ranges
  \node [location range, fit=(0-0) (1-6)] (lr1) {};
  \node [location range, fit=(1-0) (2-4)] (lr2) {};
  \node [location range, fit=(2-5) (3-8)] (lr3) {};
  \node [location range, fit=(3-6) (4-9)] (lr4) {};
  \begin{pgfonlayer}{foreground}
    \begin{scope}[location range line]
      \foreach \i in {1, ..., 4} {
        \draw (lr\i.north west)
              --
              (lr\i.north east)
              --
              (lr\i.south east)
              --
              (lr\i.south west)
              --
              cycle;
      }
    \end{scope}
  \end{pgfonlayer}

  % Labels
  \foreach \i in {0, ..., 9} {
    \pgfmathtruncatemacro{\ni}{\i + 1};
    \node [x label, above=of $(0-\i) !.5! (0-\ni)$] (r\ni)
          {\mkReg{\ni}};
  }
  \foreach \i in {1, 5, 6, 7, 9} {
    \node [x label, below=of lr4.south -| r\i] (cr\i) {\mkReg{\i}};
  }
  \node [y label, left=of r1 -| lr1.west] {\strut locations};
  \node [y label, left=of lr1] {$\textrm{$\mStores$}(m_1, p_1)$};
  \node [y label, left=of lr2] {$\textrm{$\mStores$}(m_1, p_2)$};
  \node [y label, left=of lr3 -| 0-0] {$\textrm{$\mStores$}(m_1, p_3)$};
  \node [y label, left=of lr4 -| 0-0] {$\textrm{$\mStores$}(m_2, p_4)$};
  \node [y label, left=of cr1 -| lr1.west] {\strut canonical locations};
  \foreach \i in {1, ..., 4} {
    \node [y label, right=of 0-10 |- lr\i.east] (t\i) {\i};
  }
  \node[y label, right=of $(t1.east) !.5! (t4.east)$, xshift=-2pt]
       {\rotatebox{90}{tags}};

  % Slices
  \begin{scope}[slice line]
    \foreach \x/\y in {0/0, 4/0, 5/0, 6/0, 8/2} {
      \draw (\y-\x) -- (4-\x);
    }
  \end{scope}

%  \foreach \i in {1, ..., 5} {
%    \node [nothing] at (t\i) {$t_\i$};
%  }
\end{tikzpicture}%
\endgroup%


  \caption[Example of canonical locations]%
          {%
            Example of canonical locations for a location set with ten
            registers%
          }
  \labelFigure{canonical-locations-example}
\end{figure}
%
For sake of discussion, each storage requirement has been labeled with a
\gls!{tag}.
%
Given the \gls{location set} and storage requirements shown in the figure, they
give rise to five intersections with respect to the locations:
%
\def\mReg#1{\text{\instrFont*r$_{\text{#1}}$}}%
%
\mbox{$\mSet{\mReg{1}, \ldots, \mReg{4}}$} due to \glspl{tag}~\num{1}
and~\num{2},
%
\mbox{$\mSet{\mReg{5}}$} due to \gls{tag}~\num{1},
%
$\mSet{\mReg{6}}$ due to \gls{tag}~\num{1} and~\num{3},
%
\mbox{$\mSet{\mReg{7}, \mReg{8}}$} due to \glspl{tag}~\num{3} and~\num{4}, and
%
$\mSet{\mReg{9}}$ due to \gls{tag}~\num{4}.
%
From each of these intersections we select a representative, and the union of
these representative \glspl{location} constitute the set of \gls!{canonical.l}[
  \glspl{location}].
%
An algorithm for computing this set based on the intuition above is shown in
\refAlgorithm{canonical-locs-algorithm}.
%
\begin{algorithm}[t]
  \DeclFunction{CanonicalizeLocs}{location set $L$, match set~$M$}%
  {%
    $T$ \Assign vector with $\mCard{L}$ elements initialized to $\mEmptySet$\;
    $t$ \Assign $1$\;
    \For(\tcp*[f]{assign tags}){$m \in M$}{%
      \For{$p \in \mUses(m) \cup \mDefines(m)$}{%
        \For{$l \in \mStores(m, p)$}{%
          $T[l]$ \Assign $T[l] \cup \mSet{t}$\;
        }
        $t$ \Assign $t$ $+$ $1$\;
      }
    }
    $G$ \Assign $\mSetBuilder{T[l]}{l \in L}$\tcp*{find all groups of tags}
    $L_{\mathsc{c}}$ \Assign $\mEmptySet$\;
    \For{$g \in G$}{%
      $l_{\mathsc{c}}$ \Assign
      $\mMin(\mSetBuilder{l}{l \in L \text{ \st } T[l] = g})$
      \tcp*{find representative for this group of tags}
      $L_{\mathsc{c}}$ \Assign
      $L_{\mathsc{c}} \cup \mSet{l_{\mathsc{c}}}$\;
    }
    \Return $L_{\mathsc{c}}$\;
  }

  \caption[Algorithm for computing the set of canonical locations]%
          {%
            Computes the canonical locations from a given location set.
            %
            If location restrictions for some data are already enforced by
            the function, then these are also tagged and processed accordingly%
          }
  \labelAlgorithm{canonical-locs-algorithm}
\end{algorithm}

Once computed, we substitute all locations appearing in the storage requirements
with their \gls{canonical.l} representative and then replace the original
\gls{location set} with the \gls{canonical.l} set, thus shrinking the
\glspl{domain} of the $\mVar{loc}$~\glspl{variable}.


\section{Experimental Evaluation}
\labelSection{st-experimental-evaluation}

Excluding the refinement described in
\refSection{st-refining-define-before-use-constraint}, whose naive equivalence
cannot be implemented on our experimental setup and therefore not be evaluated,
we now evaluate the impact of the solving techniques introduced in this chapter.

When filtering, we remove all \glspl{function} that have fewer than
\num{50}~\gls{LLVM} \gls{IR} \glspl{instruction} and more than
\num{150}~\glspl{instruction}.
%
Anything smaller will most likely not show the impact of the given solving
technique, and anything larger will lead to unreasonably long experiment
runtimes.
%
This leaves a pool of \num{284}~\glspl{function}, from which we then draw
\num{20}~samples.

To curb experiment runtimes, we apply a time limit of \SI{600}{\s} to the
\gls{constraint solver}.
%
For any given \gls{function}, the last \gls{solution} found is considered
optimal if and only if the \glsshort{constraint solver} has finished its
execution within the time limit.
%
When using an upper cost bound, we take the cost for the \gls{solution} computed
by \mbox{\gls{LLVM} 3.8} for the given \gls{function}.


\subsection{Objective Function Refinements and Cost Bounding}

We first evaluate the different methods for computing the cost matrix.
%
Based on the results, we then use the superior method to evaluate the impact
that different implementations of the \gls{objective function}, coupled with
cost bounding, have on solving time and code quality.


\paragraph{Multiply-then-Divide \versus Divide-then-Multiply}

We evaluate the different methods for computing the cost matrix by
comparing the solving times exhibited by two versions of the \gls{constraint
  model}:
%
\begin{modelList}
  \item \labelModel{obj-fun-mul-div}
    one based on the \gls{multiply-then-divide method}
    (\refEquationList{div-mul-op-cost-function, div-mul-cost-matrix})
  \item \labelModel{obj-fun-div-mul}
    one based on \glspl{divide-then-multiply method}
    (\refEquationList{mul-div-cost-matrix, mul-div-op-cost-function})
\end{modelList}.
%
No hypothesis is attempted on which \glsshort{constraint model} is better.

\RefFigure{new-op-cost-fun-vs-old-plot} shows the normalized solving times
(including \gls{presolving} time) for the two \glspl{constraint model} described
above in the first experiment, with \glsshort{constraint
  model}~\refModel{obj-fun-mul-div} as \gls{baseline} and \glsshort{constraint
  model}~\refModel{obj-fun-div-mul} as \gls{subject}.
%
\begin{figure}
  \centering%
  \maxsizebox{\textwidth}{!}{%
    \trimBarchartPlot{%
      \begin{tikzpicture}[gnuplot]
%% generated with GNUPLOT 5.0p4 (Lua 5.2; terminal rev. 99, script rev. 100)
%% lör  3 feb 2018 16:43:53
\path (0.000,0.000) rectangle (12.500,8.750);
\gpcolor{rgb color={0.753,0.753,0.753}}
\gpsetlinetype{gp lt axes}
\gpsetdashtype{gp dt axes}
\gpsetlinewidth{0.50}
\draw[gp path] (1.012,1.779)--(36.945,1.779);
\gpcolor{color=gp lt color border}
\node[gp node right] at (1.012,1.779) {\plotZCNormTics{0}};
\gpcolor{rgb color={0.753,0.753,0.753}}
\draw[gp path] (1.012,2.513)--(36.945,2.513);
\gpcolor{color=gp lt color border}
\node[gp node right] at (1.012,2.513) {\plotZCNormTics{10}};
\gpcolor{rgb color={0.753,0.753,0.753}}
\draw[gp path] (1.012,3.246)--(36.945,3.246);
\gpcolor{color=gp lt color border}
\node[gp node right] at (1.012,3.246) {\plotZCNormTics{20}};
\gpcolor{rgb color={0.753,0.753,0.753}}
\draw[gp path] (1.012,3.980)--(36.945,3.980);
\gpcolor{color=gp lt color border}
\node[gp node right] at (1.012,3.980) {\plotZCNormTics{30}};
\gpcolor{rgb color={0.753,0.753,0.753}}
\draw[gp path] (1.012,4.713)--(36.945,4.713);
\gpcolor{color=gp lt color border}
\node[gp node right] at (1.012,4.713) {\plotZCNormTics{40}};
\gpcolor{rgb color={0.753,0.753,0.753}}
\draw[gp path] (1.012,5.447)--(36.945,5.447);
\gpcolor{color=gp lt color border}
\node[gp node right] at (1.012,5.447) {\plotZCNormTics{50}};
\gpcolor{rgb color={0.753,0.753,0.753}}
\draw[gp path] (1.012,6.180)--(36.945,6.180);
\gpcolor{color=gp lt color border}
\node[gp node right] at (1.012,6.180) {\plotZCNormTics{60}};
\gpcolor{rgb color={0.753,0.753,0.753}}
\draw[gp path] (1.012,6.914)--(36.945,6.914);
\gpcolor{color=gp lt color border}
\node[gp node right] at (1.012,6.914) {\plotZCNormTics{70}};
\gpcolor{rgb color={0.753,0.753,0.753}}
\draw[gp path] (1.012,7.647)--(36.945,7.647);
\gpcolor{color=gp lt color border}
\node[gp node right] at (1.012,7.647) {\plotZCNormTics{80}};
\gpcolor{rgb color={0.753,0.753,0.753}}
\draw[gp path] (1.012,8.381)--(36.945,8.381);
\gpcolor{color=gp lt color border}
\node[gp node right] at (1.012,8.381) {\plotZCNormTics{90}};
\node[gp node left,rotate=-30] at (2.759,1.534) {\functionName{alloc_name_is_s.}};
\node[gp node left,rotate=-30] at (4.470,1.534) {\functionName{alloc_save_spac.}};
\node[gp node left,rotate=-30] at (6.181,1.534) {\functionName{build_ycc_rgb_t.}};
\node[gp node left,rotate=-30] at (7.892,1.534) {\functionName{checksum}};
\node[gp node left,rotate=-30] at (9.603,1.534) {\functionName{debug_dump_byte.}};
\node[gp node left,rotate=-30] at (11.315,1.534) {\functionName{gl_EnableClient.}};
\node[gp node left,rotate=-30] at (13.026,1.534) {\functionName{gl_init_lists}};
\node[gp node left,rotate=-30] at (14.737,1.534) {\functionName{gl_save_Color4u.}};
\node[gp node left,rotate=-30] at (16.448,1.534) {\functionName{gl_save_EvalPoi.}};
\node[gp node left,rotate=-30] at (18.159,1.534) {\functionName{gl_save_PushMat.}};
\node[gp node left,rotate=-30] at (19.870,1.534) {\functionName{gl_swap4}};
\node[gp node left,rotate=-30] at (21.581,1.534) {\functionName{gl_TexImage3DEX.}};
\node[gp node left,rotate=-30] at (23.292,1.534) {\functionName{gp_enumerate_fi.}};
\node[gp node left,rotate=-30] at (25.003,1.534) {\functionName{gpk_open}};
\node[gp node left,rotate=-30] at (26.714,1.534) {\functionName{gs_interp_init}};
\node[gp node left,rotate=-30] at (28.426,1.534) {\functionName{jinit_forward_d.}};
\node[gp node left,rotate=-30] at (30.137,1.534) {\functionName{jpeg_read_heade.}};
\node[gp node left,rotate=-30] at (31.848,1.534) {\functionName{trueRandAccum}};
\node[gp node left,rotate=-30] at (33.559,1.534) {\functionName{write_file_trai.}};
\node[gp node left,rotate=-30] at (35.270,1.534) {\functionName{zero}};
\gpsetlinetype{gp lt border}
\gpsetdashtype{gp dt solid}
\gpsetlinewidth{1.00}
\draw[gp path] (1.012,8.381)--(1.012,1.779)--(36.945,1.779)--(36.945,8.381)--cycle;
\gpcolor{rgb color={0.000,0.000,0.000}}
\draw[gp path] (1.012,1.779)--(1.375,1.779)--(1.738,1.779)--(2.101,1.779)--(2.464,1.779)%
  --(2.827,1.779)--(3.190,1.779)--(3.553,1.779)--(3.916,1.779)--(4.279,1.779)--(4.642,1.779)%
  --(5.005,1.779)--(5.368,1.779)--(5.730,1.779)--(6.093,1.779)--(6.456,1.779)--(6.819,1.779)%
  --(7.182,1.779)--(7.545,1.779)--(7.908,1.779)--(8.271,1.779)--(8.634,1.779)--(8.997,1.779)%
  --(9.360,1.779)--(9.723,1.779)--(10.086,1.779)--(10.449,1.779)--(10.812,1.779)--(11.175,1.779)%
  --(11.538,1.779)--(11.901,1.779)--(12.264,1.779)--(12.627,1.779)--(12.990,1.779)--(13.353,1.779)%
  --(13.716,1.779)--(14.079,1.779)--(14.442,1.779)--(14.804,1.779)--(15.167,1.779)--(15.530,1.779)%
  --(15.893,1.779)--(16.256,1.779)--(16.619,1.779)--(16.982,1.779)--(17.345,1.779)--(17.708,1.779)%
  --(18.071,1.779)--(18.434,1.779)--(18.797,1.779)--(19.160,1.779)--(19.523,1.779)--(19.886,1.779)%
  --(20.249,1.779)--(20.612,1.779)--(20.975,1.779)--(21.338,1.779)--(21.701,1.779)--(22.064,1.779)%
  --(22.427,1.779)--(22.790,1.779)--(23.153,1.779)--(23.515,1.779)--(23.878,1.779)--(24.241,1.779)%
  --(24.604,1.779)--(24.967,1.779)--(25.330,1.779)--(25.693,1.779)--(26.056,1.779)--(26.419,1.779)%
  --(26.782,1.779)--(27.145,1.779)--(27.508,1.779)--(27.871,1.779)--(28.234,1.779)--(28.597,1.779)%
  --(28.960,1.779)--(29.323,1.779)--(29.686,1.779)--(30.049,1.779)--(30.412,1.779)--(30.775,1.779)%
  --(31.138,1.779)--(31.501,1.779)--(31.864,1.779)--(32.227,1.779)--(32.589,1.779)--(32.952,1.779)%
  --(33.315,1.779)--(33.678,1.779)--(34.041,1.779)--(34.404,1.779)--(34.767,1.779)--(35.130,1.779)%
  --(35.493,1.779)--(35.856,1.779)--(36.219,1.779)--(36.582,1.779)--(36.945,1.779);
\gpfill{rgb color={0.533,0.533,0.533}} (2.652,1.779)--(3.366,1.779)--(3.366,1.918)--(2.652,1.918)--cycle;
\gpcolor{color=gp lt color border}
\draw[gp path] (2.652,1.779)--(2.652,1.917)--(3.365,1.917)--(3.365,1.779)--cycle;
\gpfill{rgb color={0.533,0.533,0.533}} (4.363,1.779)--(5.077,1.779)--(5.077,2.083)--(4.363,2.083)--cycle;
\draw[gp path] (4.363,1.779)--(4.363,2.082)--(5.076,2.082)--(5.076,1.779)--cycle;
\gpfill{rgb color={0.533,0.533,0.533}} (6.074,1.779)--(6.788,1.779)--(6.788,7.959)--(6.074,7.959)--cycle;
\draw[gp path] (6.074,1.779)--(6.074,7.958)--(6.787,7.958)--(6.787,1.779)--cycle;
\gpfill{rgb color={0.533,0.533,0.533}} (7.785,1.779)--(8.499,1.779)--(8.499,2.255)--(7.785,2.255)--cycle;
\draw[gp path] (7.785,1.779)--(7.785,2.254)--(8.498,2.254)--(8.498,1.779)--cycle;
\gpfill{rgb color={0.533,0.533,0.533}} (9.496,1.779)--(10.210,1.779)--(10.210,1.956)--(9.496,1.956)--cycle;
\draw[gp path] (9.496,1.779)--(9.496,1.955)--(10.209,1.955)--(10.209,1.779)--cycle;
\gpfill{rgb color={0.533,0.533,0.533}} (11.207,1.779)--(11.921,1.779)--(11.921,1.784)--(11.207,1.784)--cycle;
\draw[gp path] (11.207,1.779)--(11.207,1.783)--(11.920,1.783)--(11.920,1.779)--cycle;
\gpfill{rgb color={0.533,0.533,0.533}} (12.918,1.779)--(13.632,1.779)--(13.632,1.780)--(12.918,1.780)--cycle;
\draw[gp path] (12.918,1.779)--(13.631,1.779)--cycle;
\gpfill{rgb color={0.533,0.533,0.533}} (14.629,1.779)--(15.343,1.779)--(15.343,1.895)--(14.629,1.895)--cycle;
\draw[gp path] (14.629,1.779)--(14.629,1.894)--(15.342,1.894)--(15.342,1.779)--cycle;
\gpfill{rgb color={0.533,0.533,0.533}} (16.341,1.779)--(17.055,1.779)--(17.055,1.913)--(16.341,1.913)--cycle;
\draw[gp path] (16.341,1.779)--(16.341,1.912)--(17.054,1.912)--(17.054,1.779)--cycle;
\gpfill{rgb color={0.533,0.533,0.533}} (18.052,1.779)--(18.766,1.779)--(18.766,1.808)--(18.052,1.808)--cycle;
\draw[gp path] (18.052,1.779)--(18.052,1.807)--(18.765,1.807)--(18.765,1.779)--cycle;
\gpfill{rgb color={0.533,0.533,0.533}} (19.763,1.779)--(20.477,1.779)--(20.477,2.154)--(19.763,2.154)--cycle;
\draw[gp path] (19.763,1.779)--(19.763,2.153)--(20.476,2.153)--(20.476,1.779)--cycle;
\gpfill{rgb color={0.533,0.533,0.533}} (21.474,1.779)--(22.188,1.779)--(22.188,1.874)--(21.474,1.874)--cycle;
\draw[gp path] (21.474,1.779)--(21.474,1.873)--(22.187,1.873)--(22.187,1.779)--cycle;
\gpfill{rgb color={0.533,0.533,0.533}} (23.185,1.779)--(23.899,1.779)--(23.899,1.785)--(23.185,1.785)--cycle;
\draw[gp path] (23.185,1.779)--(23.185,1.784)--(23.898,1.784)--(23.898,1.779)--cycle;
\gpfill{rgb color={0.533,0.533,0.533}} (24.896,1.779)--(25.610,1.779)--(25.610,1.805)--(24.896,1.805)--cycle;
\draw[gp path] (24.896,1.779)--(24.896,1.804)--(25.609,1.804)--(25.609,1.779)--cycle;
\gpfill{rgb color={0.533,0.533,0.533}} (28.318,1.779)--(29.032,1.779)--(29.032,2.002)--(28.318,2.002)--cycle;
\draw[gp path] (28.318,1.779)--(28.318,2.001)--(29.031,2.001)--(29.031,1.779)--cycle;
\gpfill{rgb color={0.533,0.533,0.533}} (30.029,1.779)--(30.743,1.779)--(30.743,1.781)--(30.029,1.781)--cycle;
\draw[gp path] (30.029,1.779)--(30.029,1.780)--(30.742,1.780)--(30.742,1.779)--cycle;
\gpfill{rgb color={0.533,0.533,0.533}} (31.740,1.779)--(32.454,1.779)--(32.454,4.853)--(31.740,4.853)--cycle;
\draw[gp path] (31.740,1.779)--(31.740,4.852)--(32.453,4.852)--(32.453,1.779)--cycle;
\gpfill{rgb color={0.533,0.533,0.533}} (33.452,1.779)--(34.165,1.779)--(34.165,1.799)--(33.452,1.799)--cycle;
\draw[gp path] (33.452,1.779)--(33.452,1.798)--(34.164,1.798)--(34.164,1.779)--cycle;
\gpfill{rgb color={0.533,0.533,0.533}} (35.163,1.779)--(35.877,1.779)--(35.877,1.815)--(35.163,1.815)--cycle;
\draw[gp path] (35.163,1.779)--(35.163,1.814)--(35.876,1.814)--(35.876,1.779)--cycle;
\node[gp node center] at (2.999,2.101) {\plotBarNormValue{1.881115}};
\node[gp node center] at (4.710,2.266) {\plotBarNormValue{4.124138}};
\node[gp node center] at (6.421,8.142) {\plotBarNormValue{84.229746}};
\node[gp node center] at (8.132,2.438) {\plotBarNormValue{6.479276}};
\node[gp node center] at (9.843,2.139) {\plotBarNormValue{2.404685}};
\node[gp node center] at (11.555,1.967) {\plotBarNormValue{0.056909}};
\node[gp node center] at (13.266,1.963) {\plotBarNormValue{0.000812}};
\node[gp node center] at (14.977,2.078) {\plotBarNormValue{1.568175}};
\node[gp node center] at (16.688,2.096) {\plotBarNormValue{1.807723}};
\node[gp node center] at (18.399,1.991) {\plotBarNormValue{0.383711}};
\node[gp node center] at (20.110,2.337) {\plotBarNormValue{5.098539}};
\node[gp node center] at (21.821,2.057) {\plotBarNormValue{1.275400}};
\node[gp node center] at (23.532,1.968) {\plotBarNormValue{0.064348}};
\node[gp node center] at (25.243,1.988) {\plotBarNormValue{0.347323}};
\node[gp node center] at (26.954,1.963) {\plotBarNormValue{-0.000460}};
\node[gp node center] at (28.666,2.185) {\plotBarNormValue{3.027095}};
\node[gp node center] at (30.377,1.964) {\plotBarNormValue{0.014329}};
\node[gp node center] at (32.088,5.036) {\plotBarNormValue{41.887658}};
\node[gp node center] at (33.799,1.982) {\plotBarNormValue{0.265172}};
\node[gp node center] at (35.510,1.998) {\plotBarNormValue{0.480267}};
\draw[gp path] (1.012,8.381)--(1.012,1.779)--(36.945,1.779)--(36.945,8.381)--cycle;
%% coordinates of the plot area
\gpdefrectangularnode{gp plot 1}{\pgfpoint{1.012cm}{1.779cm}}{\pgfpoint{36.945cm}{8.381cm}}
\end{tikzpicture}
%% gnuplot variables
%
    }%
  }

  \caption[%
            Plot for evaluating the operation cost function's impact on
            solving time%
          ]%
          {%
            Normalized solving times (incl.\ presolving time) for two constraint
            models using different operation cost functions:
            %
            \begin{inlinelist}[itemjoin={, }, itemjoin*={, and}]
              \item one based on the multiply-then-divide method (baseline)
              \item one based on the divide-then-multiply method (subject)
            \end{inlinelist}.
            %
            GMI:~\printGMI{%
              \NewOpCostFunVsOldPrePlusSolvingTimeSpeedupPrePlusSolvingTimeRegularSpeedupGmean%
            },
            CI:~\printGMICI{%
              \NewOpCostFunVsOldPrePlusSolvingTimeSpeedupPrePlusSolvingTimeRegularSpeedupCiMin%
            }{%
              \NewOpCostFunVsOldPrePlusSolvingTimeSpeedupPrePlusSolvingTimeRegularSpeedupCiMax%
            }%
          }
  \labelFigure{new-op-cost-fun-vs-old-plot}
\end{figure}
%
The solving times range from
\printMinSolvingTime{
  \NewOpCostFunVsOldPrePlusSolvingTimeSpeedupPrePlusSolvingTimeAvgMin,
  \NewOpCostFunVsOldPrePlusSolvingTimeSpeedupBaselinePrePlusSolvingTimeAvgMin
} to
\printMaxSolvingTime{
  \NewOpCostFunVsOldPrePlusSolvingTimeSpeedupPrePlusSolvingTimeAvgMax,
  \NewOpCostFunVsOldPrePlusSolvingTimeSpeedupBaselinePrePlusSolvingTimeAvgMax
}%
%
\footnote{%
  The solving time given here greater than the time limit because it also
  includes the \gls{presolving} time while the time limit is only applied on the
  \gls{constraint solver}.%
}
%
with a \gls{CV} of
\numMaxOf{
  \NewOpCostFunVsOldPrePlusSolvingTimeSpeedupPrePlusSolvingTimeCvMax,
  \NewOpCostFunVsOldPrePlusSolvingTimeSpeedupBaselinePrePlusSolvingTimeCvMax
}.
%
The \gls{GMI} is \printGMI{%
  \NewOpCostFunVsOldPrePlusSolvingTimeSpeedupPrePlusSolvingTimeRegularSpeedupGmean%
} with \gls{CI}~\printGMICI{%
  \NewOpCostFunVsOldPrePlusSolvingTimeSpeedupPrePlusSolvingTimeRegularSpeedupCiMin%
}{%
  \NewOpCostFunVsOldPrePlusSolvingTimeSpeedupPrePlusSolvingTimeRegularSpeedupCiMax%
}.

We see clearly that \glsshort{constraint model}~\refModel{obj-fun-div-mul}
results in considerably shorter solving times than \glsshort{constraint
  model}~\refModel{obj-fun-mul-div}, thus underscoring the fact that seemingly
trivial changes to a \gls{constraint model} may have considerable impact on
solving time.
%
For one \gls{function} (\cCode*{build\_ycc\_rgb\_t}) the solving time is
improved by
\printZCNorm{%
  \NewOpCostFunVsOldPrePlusSolvingTimeSpeedupPrePlusSolvingTimeZeroCenteredSpeedupMax%
}, and for no \gls{function} is solving time degraded.
%
Hence the \gls{divide-then-multiply method} is a better design choice over
\gls{multiply-then-divide method} when implementing the refined \gls{objective
  function}.

One possible explanation is that they yield different bounds.
%
For the example given in \refFigure{cost-example} on
\refPageOfFigure{cost-example}, the \gls{multiply-then-divide method} bounds the
total cost as \mbox{$4 \leq \mVar{cost} \leq 65$}, whereas the
\gls{divide-then-multiply method} bounds it to \mbox{$4 \leq \mVar{cost} \leq
  60$}.
%
However, this does not always hold as the \gls{divide-then-multiply method} may
yield a tighter bound than the \gls{multiply-then-divide method} for other
problem instances.
%
Another possible explanation is that the \gls{divide-then-multiply method}
results in \gls{operation} costs that are even multiples of the execution
frequencies.
%
In comparison, the \gls{multiply-then-divide method} could potentially result in
arbitrary cost values that in turn would lead to larger \glspl{domain} for the
\gls{cost variable}.


\paragraph{Impact on Solving Time}

We evaluate the impact that the refined \gls{objective function} together with
cost bounding have on solving by comparing the number of \glspl{function} that
can be solved optimally by six versions of the \gls{constraint model}:
%
\begin{modelList}
  \item \labelModel{obj-fun-naive-no-b}
    one based on the naive implementation of the \gls{objective function}
    (\refEquation{naive-objective-function}), without cost bounds
  \item \labelModel{obj-fun-naive-lb-ub}
    another naive \glsshort{constraint model} but with lower and upper bound
  \item \labelModel{obj-fun-refined-no-b}
    one based on the refined implementation
    (\refEquationList{omatch-oplace-ocost-connection, total-cost}) using the
    \gls{divide-then-multiply method}, without bounds
  \item \labelModel{obj-fun-refined-lb}
    another refined \glsshort{constraint model} but with lower bound
  \item \labelModel{obj-fun-refined-ub}
    another refined \glsshort{constraint model} but with upper bound
  \item \labelModel{obj-fun-refined-lb-ub}
    another refined \glsshort{constraint model} but with both bounds
\end{modelList}.

Since the refined \gls{objective function} enables tighter bounds to be derived
for the \gls{cost variable}, we expect \glsplshort{constraint
  model}~\refModel{obj-fun-refined-no-b}, \refModel{obj-fun-refined-lb},
\refModel{obj-fun-refined-ub}, and~\refModel{obj-fun-refined-lb-ub}{} to find a
greater number of optimal \glspl{solution} compared with \glsplshort{constraint
  model}~\refModel{obj-fun-naive-no-b} and~\refModel{obj-fun-naive-lb-ub}.
%
Due to further tightening of the cost bounds, we expect \glsshort{constraint
  model}~\refModel{obj-fun-naive-lb-ub} to outperform \glsshort{constraint
  model}~\refModel{obj-fun-naive-no-b}, \glsplshort{constraint
  model}~\refModel{obj-fun-refined-lb}, \refModel{obj-fun-refined-ub},
and~\refModel{obj-fun-refined-lb-ub}{} to outperform \glsshort{constraint
  model}~\refModel{obj-fun-refined-no-b}, and \glsshort{constraint
  model}~\refModel{obj-fun-refined-lb-ub} to outperform \glsplshort{constraint
  model}~\refModel{obj-fun-refined-lb} and~\refModel{obj-fun-refined-ub}.

\RefFigure{obj-fun-refined-vs-naive-opt-proofs-over-time-plot} shows the
percentage of optimal \glspl{solution} found over time for the five
\glspl{constraint model} described above in the second experiment.
%
\begin{figure}
  \renewcommand{\plotPercentageFont}{\large}
  \renewcommand{\plotSecondsFont}{\large}
  \centering%
  \maxsizebox{.6\textwidth}{!}{%
    \trimLinechartPlot{%
      \begin{tikzpicture}[gnuplot]
%% generated with GNUPLOT 5.0p4 (Lua 5.2; terminal rev. 99, script rev. 100)
%% tor 25 jan 2018 09:52:27
\path (0.000,0.000) rectangle (12.500,8.750);
\gpcolor{rgb color={0.753,0.753,0.753}}
\gpsetlinetype{gp lt axes}
\gpsetdashtype{gp dt axes}
\gpsetlinewidth{0.50}
\draw[gp path] (1.012,0.616)--(11.947,0.616);
\gpcolor{color=gp lt color border}
\node[gp node right] at (1.012,0.616) {\plotPercentage{0}};
\gpcolor{rgb color={0.753,0.753,0.753}}
\draw[gp path] (1.012,1.479)--(11.947,1.479);
\gpcolor{color=gp lt color border}
\node[gp node right] at (1.012,1.479) {\plotPercentage{10}};
\gpcolor{rgb color={0.753,0.753,0.753}}
\draw[gp path] (1.012,2.342)--(11.947,2.342);
\gpcolor{color=gp lt color border}
\node[gp node right] at (1.012,2.342) {\plotPercentage{20}};
\gpcolor{rgb color={0.753,0.753,0.753}}
\draw[gp path] (1.012,3.204)--(11.947,3.204);
\gpcolor{color=gp lt color border}
\node[gp node right] at (1.012,3.204) {\plotPercentage{30}};
\gpcolor{rgb color={0.753,0.753,0.753}}
\draw[gp path] (1.012,4.067)--(11.947,4.067);
\gpcolor{color=gp lt color border}
\node[gp node right] at (1.012,4.067) {\plotPercentage{40}};
\gpcolor{rgb color={0.753,0.753,0.753}}
\draw[gp path] (1.012,4.930)--(11.947,4.930);
\gpcolor{color=gp lt color border}
\node[gp node right] at (1.012,4.930) {\plotPercentage{50}};
\gpcolor{rgb color={0.753,0.753,0.753}}
\draw[gp path] (1.012,5.793)--(11.947,5.793);
\gpcolor{color=gp lt color border}
\node[gp node right] at (1.012,5.793) {\plotPercentage{60}};
\gpcolor{rgb color={0.753,0.753,0.753}}
\draw[gp path] (1.012,6.655)--(11.947,6.655);
\gpcolor{color=gp lt color border}
\node[gp node right] at (1.012,6.655) {\plotPercentage{70}};
\gpcolor{rgb color={0.753,0.753,0.753}}
\draw[gp path] (1.012,7.518)--(11.947,7.518);
\gpcolor{color=gp lt color border}
\node[gp node right] at (1.012,7.518) {\plotPercentage{80}};
\gpcolor{rgb color={0.753,0.753,0.753}}
\draw[gp path] (1.012,8.381)--(11.947,8.381);
\gpcolor{color=gp lt color border}
\node[gp node right] at (1.012,8.381) {\plotPercentage{90}};
\gpsetlinetype{gp lt border}
\gpsetdashtype{gp dt solid}
\gpsetlinewidth{1.00}
\draw[gp path] (1.012,0.616)--(1.012,0.796);
\draw[gp path] (1.012,8.381)--(1.012,8.201);
\node[gp node center] at (1.012,0.308) {\plotSeconds{1}};
\draw[gp path] (2.109,0.616)--(2.109,0.706);
\draw[gp path] (2.109,8.381)--(2.109,8.291);
\draw[gp path] (2.751,0.616)--(2.751,0.706);
\draw[gp path] (2.751,8.381)--(2.751,8.291);
\draw[gp path] (3.207,0.616)--(3.207,0.706);
\draw[gp path] (3.207,8.381)--(3.207,8.291);
\draw[gp path] (3.560,0.616)--(3.560,0.706);
\draw[gp path] (3.560,8.381)--(3.560,8.291);
\draw[gp path] (3.848,0.616)--(3.848,0.706);
\draw[gp path] (3.848,8.381)--(3.848,8.291);
\draw[gp path] (4.092,0.616)--(4.092,0.706);
\draw[gp path] (4.092,8.381)--(4.092,8.291);
\draw[gp path] (4.304,0.616)--(4.304,0.706);
\draw[gp path] (4.304,8.381)--(4.304,8.291);
\draw[gp path] (4.490,0.616)--(4.490,0.706);
\draw[gp path] (4.490,8.381)--(4.490,8.291);
\draw[gp path] (4.657,0.616)--(4.657,0.796);
\draw[gp path] (4.657,8.381)--(4.657,8.201);
\node[gp node center] at (4.657,0.308) {\plotSeconds{10}};
\draw[gp path] (5.754,0.616)--(5.754,0.706);
\draw[gp path] (5.754,8.381)--(5.754,8.291);
\draw[gp path] (6.396,0.616)--(6.396,0.706);
\draw[gp path] (6.396,8.381)--(6.396,8.291);
\draw[gp path] (6.852,0.616)--(6.852,0.706);
\draw[gp path] (6.852,8.381)--(6.852,8.291);
\draw[gp path] (7.205,0.616)--(7.205,0.706);
\draw[gp path] (7.205,8.381)--(7.205,8.291);
\draw[gp path] (7.493,0.616)--(7.493,0.706);
\draw[gp path] (7.493,8.381)--(7.493,8.291);
\draw[gp path] (7.737,0.616)--(7.737,0.706);
\draw[gp path] (7.737,8.381)--(7.737,8.291);
\draw[gp path] (7.949,0.616)--(7.949,0.706);
\draw[gp path] (7.949,8.381)--(7.949,8.291);
\draw[gp path] (8.135,0.616)--(8.135,0.706);
\draw[gp path] (8.135,8.381)--(8.135,8.291);
\draw[gp path] (8.302,0.616)--(8.302,0.796);
\draw[gp path] (8.302,8.381)--(8.302,8.201);
\node[gp node center] at (8.302,0.308) {\plotSeconds{100}};
\draw[gp path] (9.399,0.616)--(9.399,0.706);
\draw[gp path] (9.399,8.381)--(9.399,8.291);
\draw[gp path] (10.041,0.616)--(10.041,0.706);
\draw[gp path] (10.041,8.381)--(10.041,8.291);
\draw[gp path] (10.497,0.616)--(10.497,0.706);
\draw[gp path] (10.497,8.381)--(10.497,8.291);
\draw[gp path] (10.850,0.616)--(10.850,0.706);
\draw[gp path] (10.850,8.381)--(10.850,8.291);
\draw[gp path] (11.138,0.616)--(11.138,0.706);
\draw[gp path] (11.138,8.381)--(11.138,8.291);
\draw[gp path] (11.382,0.616)--(11.382,0.706);
\draw[gp path] (11.382,8.381)--(11.382,8.291);
\draw[gp path] (11.594,0.616)--(11.594,0.706);
\draw[gp path] (11.594,8.381)--(11.594,8.291);
\draw[gp path] (11.780,0.616)--(11.780,0.706);
\draw[gp path] (11.780,8.381)--(11.780,8.291);
\draw[gp path] (11.947,0.616)--(11.947,0.796);
\draw[gp path] (11.947,8.381)--(11.947,8.201);
\node[gp node center] at (11.947,0.308) {\plotSeconds{1000}};
\draw[gp path] (1.012,8.381)--(1.012,0.616)--(11.947,0.616)--(11.947,8.381)--cycle;
\gpcolor{rgb color={0.000,0.000,0.000}}
\draw[gp path] (1.012,0.616)--(1.114,0.616)--(1.217,0.616)--(1.319,0.616)--(1.421,0.616)%
  --(1.523,0.616)--(1.626,0.616)--(1.728,0.616)--(1.830,0.616)--(1.932,0.616)--(2.035,0.616)%
  --(2.137,0.616)--(2.239,0.616)--(2.341,0.616)--(2.444,0.616)--(2.546,0.616)--(2.648,0.616)%
  --(2.750,0.616)--(2.853,0.616)--(2.955,0.616)--(3.057,0.616)--(3.159,0.616)--(3.262,0.616)%
  --(3.364,0.616)--(3.466,0.616)--(3.568,0.616)--(3.671,0.616)--(3.773,0.616)--(3.875,0.616)%
  --(3.978,0.616)--(4.080,0.616)--(4.182,0.616)--(4.284,0.616)--(4.387,0.616)--(4.489,0.616)%
  --(4.591,0.616)--(4.693,0.616)--(4.796,0.616)--(4.898,0.616)--(5.000,0.616)--(5.102,0.616)%
  --(5.205,0.616)--(5.307,0.616)--(5.409,0.616)--(5.511,0.616)--(5.614,0.616)--(5.716,0.616)%
  --(5.818,0.616)--(5.920,0.616)--(6.023,0.616)--(6.125,0.616)--(6.227,0.616)--(6.330,0.616)%
  --(6.432,0.616)--(6.534,0.616)--(6.636,0.616)--(6.739,0.616)--(6.841,0.616)--(6.943,0.616)%
  --(7.045,0.616)--(7.148,0.616)--(7.250,0.616)--(7.352,0.616)--(7.454,0.616)--(7.557,0.616)%
  --(7.659,0.616)--(7.761,0.616)--(7.863,0.616)--(7.966,0.616)--(8.068,0.616)--(8.170,0.616)%
  --(8.272,0.616)--(8.375,0.616)--(8.477,0.616)--(8.579,0.616)--(8.681,0.616)--(8.784,0.616)%
  --(8.886,0.616)--(8.988,0.616)--(9.091,0.616)--(9.193,0.616)--(9.295,0.616)--(9.397,0.616)%
  --(9.500,0.616)--(9.602,0.616)--(9.704,0.616)--(9.806,0.616)--(9.909,0.616)--(10.011,0.616)%
  --(10.113,0.616)--(10.215,0.616)--(10.318,0.616)--(10.420,0.616)--(10.522,0.616)--(10.624,0.616)%
  --(10.727,0.616)--(10.829,0.616)--(10.931,0.616)--(11.033,0.616)--(11.136,0.616);
\gpsetdashtype{gp dt 4}
\gpsetlinewidth{2.00}
\draw[gp path] (1.012,0.616)--(2.109,0.616)--(2.751,0.616)--(3.207,0.616)--(3.560,0.616)%
  --(3.848,0.616)--(4.092,0.616)--(4.304,0.616)--(4.490,0.616)--(4.657,0.616)--(4.808,0.616)%
  --(4.946,0.616)--(5.072,0.616)--(5.190,0.616)--(5.299,0.616)--(5.401,0.616)--(5.497,0.616)%
  --(5.587,0.616)--(5.673,1.047)--(5.754,1.047)--(5.831,1.047)--(5.905,1.047)--(5.975,1.047)%
  --(6.043,1.047)--(6.107,1.047)--(6.170,1.047)--(6.229,1.047)--(6.287,1.047)--(6.342,1.047)%
  --(6.396,1.047)--(6.448,1.047)--(6.498,1.047)--(6.547,1.047)--(6.594,1.047)--(6.640,1.047)%
  --(6.685,1.047)--(6.728,1.047)--(6.770,1.047)--(6.811,1.047)--(6.852,1.047)--(6.891,1.047)%
  --(6.929,1.479)--(6.966,1.479)--(7.002,1.479)--(7.038,1.479)--(7.073,1.479)--(7.107,1.479)%
  --(7.140,1.479)--(7.173,1.479)--(7.205,1.479)--(7.236,1.479)--(7.267,1.479)--(7.297,1.479)%
  --(7.327,1.479)--(7.356,1.479)--(7.384,1.479)--(7.412,1.479)--(7.440,1.479)--(7.467,1.479)%
  --(7.493,1.479)--(7.520,1.479)--(7.545,1.479)--(7.571,1.479)--(7.596,1.479)--(7.620,1.479)%
  --(7.644,1.479)--(7.668,1.479)--(7.691,1.479)--(7.715,1.479)--(7.737,1.479)--(7.760,1.479)%
  --(7.782,1.479)--(7.804,1.479)--(7.825,1.479)--(7.847,1.479)--(7.868,1.479)--(7.888,1.479)%
  --(7.909,1.479)--(7.929,1.479)--(7.949,1.479)--(7.968,1.479)--(7.988,1.479)--(8.007,1.479)%
  --(8.026,1.479)--(8.045,1.479)--(8.063,1.479)--(8.082,1.479)--(8.100,1.479)--(8.118,1.479)%
  --(8.135,1.479)--(8.153,1.479)--(8.170,1.479)--(8.187,1.479)--(8.204,1.479)--(8.221,1.479)%
  --(8.237,1.479)--(8.254,1.479)--(8.270,1.479)--(8.286,1.479)--(8.302,1.479)--(8.318,1.479)%
  --(8.333,1.479)--(8.349,1.479)--(8.364,1.479)--(8.379,1.479)--(8.394,1.479)--(8.409,1.479)%
  --(8.424,1.479)--(8.438,1.479)--(8.453,1.479)--(8.467,1.479)--(8.481,1.479)--(8.495,1.479)%
  --(8.509,1.479)--(8.523,1.479)--(8.537,1.479)--(8.551,1.479)--(8.564,1.479)--(8.577,1.479)%
  --(8.591,1.479)--(8.604,1.479)--(8.617,1.479)--(8.630,1.479)--(8.643,1.479)--(8.655,1.479)%
  --(8.668,1.479)--(8.680,1.479)--(8.693,1.479)--(8.705,1.479)--(8.717,1.479)--(8.729,1.479)%
  --(8.741,1.479)--(8.753,1.479)--(8.765,1.479)--(8.777,1.479)--(8.789,1.479)--(8.800,1.479)%
  --(8.812,1.479)--(8.823,1.479)--(8.835,1.479)--(8.846,1.479)--(8.857,1.479)--(8.868,1.479)%
  --(8.879,1.479)--(8.890,1.479)--(8.901,1.479)--(8.912,1.479)--(8.923,1.479)--(8.933,1.479)%
  --(8.944,1.479)--(8.954,1.479)--(8.965,1.479)--(8.975,1.479)--(8.986,1.479)--(8.996,1.479)%
  --(9.006,1.479)--(9.016,1.479)--(9.026,1.479)--(9.036,1.479)--(9.046,1.479)--(9.056,1.479)%
  --(9.066,1.479)--(9.075,1.479)--(9.085,1.479)--(9.095,1.479)--(9.104,1.479)--(9.114,1.479)%
  --(9.123,1.479)--(9.133,1.479)--(9.142,1.479)--(9.151,1.479)--(9.161,1.479)--(9.170,1.479)%
  --(9.179,1.479)--(9.188,1.479)--(9.197,1.479)--(9.206,1.479)--(9.215,1.479)--(9.224,1.479)%
  --(9.232,1.479)--(9.241,1.479)--(9.250,1.479)--(9.259,1.479)--(9.267,1.479)--(9.276,1.479)%
  --(9.284,1.479)--(9.293,1.479)--(9.301,1.479)--(9.310,1.479)--(9.318,1.479)--(9.326,1.479)%
  --(9.335,1.479)--(9.343,1.479)--(9.351,1.479)--(9.359,1.479)--(9.367,1.479)--(9.375,1.479)%
  --(9.383,1.479)--(9.391,1.479)--(9.399,1.479)--(9.407,1.479)--(9.415,1.479)--(9.423,1.479)%
  --(9.431,1.479)--(9.438,1.479)--(9.446,1.479)--(9.454,1.479)--(9.461,1.479)--(9.469,1.479)%
  --(9.476,1.479)--(9.484,1.479)--(9.491,1.479)--(9.499,1.479)--(9.506,1.479)--(9.514,1.479)%
  --(9.521,1.479)--(9.528,1.479)--(9.536,1.479)--(9.543,1.479)--(9.550,1.479)--(9.557,1.479)%
  --(9.564,1.479)--(9.572,1.479)--(9.579,1.479)--(9.586,1.479)--(9.593,1.479)--(9.600,1.479)%
  --(9.607,1.479)--(9.614,1.479)--(9.620,1.479)--(9.627,1.479)--(9.634,1.479)--(9.641,1.479)%
  --(9.648,1.479)--(9.655,1.479)--(9.661,1.479)--(9.668,1.479)--(9.675,1.479)--(9.681,1.479)%
  --(9.688,1.479)--(9.694,1.479)--(9.701,1.479)--(9.708,1.479)--(9.714,1.479)--(9.721,1.479)%
  --(9.727,1.479)--(9.733,1.479)--(9.740,1.479)--(9.746,1.479)--(9.752,1.479)--(9.759,1.479)%
  --(9.765,1.479)--(9.771,1.479)--(9.778,1.479)--(9.784,1.479)--(9.790,1.479)--(9.796,1.479)%
  --(9.802,1.479)--(9.808,1.479)--(9.815,1.479)--(9.821,1.479)--(9.827,1.479)--(9.833,1.479)%
  --(9.839,1.479)--(9.845,1.479)--(9.851,1.910)--(9.857,1.910)--(9.863,1.910)--(9.868,1.910)%
  --(9.874,1.910)--(9.880,1.910)--(9.886,1.910)--(9.892,1.910)--(9.898,1.910)--(9.903,1.910)%
  --(9.909,1.910)--(9.915,1.910)--(9.921,1.910)--(9.926,1.910)--(9.932,1.910)--(9.938,1.910)%
  --(9.943,1.910)--(9.949,1.910)--(9.954,1.910)--(9.960,1.910)--(9.965,1.910)--(9.971,1.910)%
  --(9.976,1.910)--(9.982,1.910)--(9.987,1.910)--(9.993,1.910)--(9.998,1.910)--(10.004,1.910)%
  --(10.009,1.910)--(10.015,1.910)--(10.020,1.910)--(10.025,1.910)--(10.031,1.910)--(10.036,1.910)%
  --(10.041,1.910)--(10.046,1.910)--(10.052,1.910)--(10.057,1.910)--(10.062,1.910)--(10.067,1.910)%
  --(10.072,1.910)--(10.078,1.910)--(10.083,1.910)--(10.088,1.910)--(10.093,1.910)--(10.098,1.910)%
  --(10.103,1.910)--(10.108,1.910)--(10.113,1.910)--(10.118,1.910)--(10.123,1.910)--(10.128,1.910)%
  --(10.133,1.910)--(10.138,1.910)--(10.143,1.910)--(10.148,1.910)--(10.153,1.910)--(10.158,1.910)%
  --(10.163,1.910)--(10.168,1.910)--(10.173,1.910)--(10.178,1.910)--(10.182,1.910)--(10.187,1.910)%
  --(10.192,1.910)--(10.197,1.910)--(10.202,1.910)--(10.206,1.910)--(10.211,1.910)--(10.216,1.910)%
  --(10.221,1.910)--(10.225,1.910)--(10.230,1.910)--(10.235,1.910)--(10.239,1.910)--(10.244,1.910)%
  --(10.249,1.910)--(10.253,1.910)--(10.258,1.910)--(10.262,1.910)--(10.267,1.910)--(10.272,1.910)%
  --(10.276,1.910)--(10.281,1.910)--(10.285,1.910)--(10.290,1.910)--(10.294,1.910)--(10.299,1.910)%
  --(10.303,1.910)--(10.308,1.910)--(10.312,1.910)--(10.316,1.910)--(10.321,1.910)--(10.325,1.910)%
  --(10.330,1.910)--(10.334,1.910)--(10.338,1.910)--(10.343,1.910)--(10.347,1.910)--(10.352,1.910)%
  --(10.356,1.910)--(10.360,1.910)--(10.365,1.910)--(10.369,1.910)--(10.373,1.910)--(10.377,1.910)%
  --(10.382,1.910)--(10.386,1.910)--(10.390,1.910)--(10.394,1.910)--(10.399,1.910)--(10.403,1.910)%
  --(10.407,1.910)--(10.411,1.910)--(10.415,1.910)--(10.419,1.910)--(10.424,1.910)--(10.428,1.910)%
  --(10.432,1.910)--(10.436,1.910)--(10.440,1.910)--(10.444,1.910)--(10.448,1.910)--(10.452,1.910)%
  --(10.456,1.910)--(10.460,1.910)--(10.465,1.910)--(10.469,1.910)--(10.473,1.910)--(10.477,1.910)%
  --(10.481,1.910)--(10.485,1.910)--(10.489,1.910)--(10.493,1.910)--(10.497,1.910)--(10.500,1.910)%
  --(10.504,1.910)--(10.508,1.910)--(10.512,1.910)--(10.516,1.910)--(10.520,1.910)--(10.524,1.910)%
  --(10.528,1.910)--(10.532,1.910)--(10.536,1.910)--(10.539,1.910)--(10.543,1.910)--(10.547,1.910)%
  --(10.551,1.910)--(10.555,1.910)--(10.559,1.910)--(10.562,1.910)--(10.566,1.910)--(10.570,1.910)%
  --(10.574,1.910)--(10.578,1.910)--(10.581,1.910)--(10.585,1.910)--(10.589,1.910)--(10.592,1.910)%
  --(10.596,1.910)--(10.600,1.910)--(10.604,1.910)--(10.607,1.910)--(10.611,1.910)--(10.615,1.910)%
  --(10.618,1.910)--(10.622,1.910)--(10.626,1.910)--(10.629,1.910)--(10.633,1.910)--(10.637,1.910)%
  --(10.640,1.910)--(10.644,1.910)--(10.647,1.910)--(10.651,1.910)--(10.655,1.910)--(10.658,1.910)%
  --(10.662,1.910)--(10.665,1.910)--(10.669,1.910)--(10.672,1.910)--(10.676,1.910)--(10.679,1.910)%
  --(10.683,1.910)--(10.686,1.910)--(10.690,1.910)--(10.693,1.910)--(10.697,1.910)--(10.700,1.910)%
  --(10.704,1.910)--(10.707,1.910)--(10.711,1.910)--(10.714,1.910)--(10.718,1.910)--(10.721,1.910)%
  --(10.725,1.910)--(10.728,1.910)--(10.731,1.910)--(10.735,1.910)--(10.738,1.910)--(10.742,1.910)%
  --(10.745,1.910)--(10.748,1.910)--(10.752,1.910)--(10.755,1.910)--(10.759,1.910)--(10.762,1.910)%
  --(10.765,1.910)--(10.769,1.910)--(10.772,1.910)--(10.775,1.910)--(10.779,1.910)--(10.782,1.910)%
  --(10.785,1.910)--(10.788,1.910)--(10.792,1.910)--(10.795,1.910)--(10.798,1.910)--(10.802,1.910)%
  --(10.805,1.910)--(10.808,1.910)--(10.811,1.910)--(10.815,1.910)--(10.818,1.910)--(10.821,1.910)%
  --(10.824,1.910)--(10.827,1.910)--(10.831,1.910)--(10.834,1.910)--(10.837,1.910)--(10.840,1.910)%
  --(10.843,1.910)--(10.847,1.910)--(10.850,1.910)--(10.853,1.910)--(10.856,1.910)--(10.859,1.910)%
  --(10.862,1.910)--(10.865,1.910)--(10.869,1.910)--(10.872,1.910)--(10.875,1.910)--(10.878,1.910)%
  --(10.881,1.910)--(10.884,1.910)--(10.887,1.910)--(10.890,1.910)--(10.893,1.910)--(10.897,1.910)%
  --(10.900,1.910)--(10.903,1.910)--(10.906,1.910)--(10.909,1.910)--(10.912,1.910)--(10.915,1.910)%
  --(10.918,1.910)--(10.921,1.910)--(10.924,1.910)--(10.927,1.910)--(10.930,1.910)--(10.933,1.910)%
  --(10.936,1.910)--(10.939,1.910)--(10.942,1.910)--(10.945,1.910)--(10.948,1.910)--(10.951,1.910)%
  --(10.954,1.910)--(10.957,1.910)--(10.960,1.910)--(10.963,1.910)--(10.966,1.910)--(10.969,1.910)%
  --(10.972,1.910)--(10.975,1.910)--(10.977,1.910)--(10.980,1.910)--(10.983,1.910)--(10.986,1.910)%
  --(10.989,1.910)--(10.992,1.910)--(10.995,1.910)--(10.998,1.910)--(11.001,1.910)--(11.003,1.910)%
  --(11.006,1.910)--(11.009,1.910)--(11.012,1.910)--(11.015,1.910)--(11.018,1.910)--(11.021,1.910)%
  --(11.023,1.910)--(11.026,1.910)--(11.029,1.910)--(11.032,1.910)--(11.035,1.910)--(11.038,1.910)%
  --(11.040,1.910)--(11.043,1.910)--(11.046,1.910)--(11.049,1.910)--(11.052,1.910)--(11.054,1.910)%
  --(11.057,1.910)--(11.060,1.910)--(11.063,1.910)--(11.065,1.910)--(11.068,1.910)--(11.071,1.910)%
  --(11.074,1.910)--(11.076,1.910)--(11.079,1.910)--(11.082,1.910)--(11.085,1.910)--(11.087,1.910)%
  --(11.090,1.910)--(11.093,1.910)--(11.096,1.910)--(11.098,1.910)--(11.101,1.910)--(11.104,1.910)%
  --(11.106,1.910)--(11.109,1.910)--(11.112,1.910)--(11.114,1.910)--(11.117,1.910)--(11.120,1.910)%
  --(11.122,1.910)--(11.125,1.910)--(11.128,1.910)--(11.130,1.910)--(11.133,1.910)--(11.136,1.910);
\gpsetdashtype{gp dt solid}
\draw[gp path] (1.012,0.616)--(2.109,0.616)--(2.751,0.616)--(3.207,1.047)--(3.560,1.479)%
  --(3.848,1.479)--(4.092,1.479)--(4.304,1.479)--(4.490,1.479)--(4.657,1.479)--(4.808,1.479)%
  --(4.946,1.479)--(5.072,1.479)--(5.190,1.479)--(5.299,1.479)--(5.401,1.479)--(5.497,1.479)%
  --(5.587,1.479)--(5.673,1.479)--(5.754,1.479)--(5.831,1.479)--(5.905,1.479)--(5.975,1.479)%
  --(6.043,1.479)--(6.107,1.479)--(6.170,1.479)--(6.229,1.479)--(6.287,1.479)--(6.342,1.479)%
  --(6.396,1.479)--(6.448,1.479)--(6.498,1.479)--(6.547,1.479)--(6.594,1.479)--(6.640,1.479)%
  --(6.685,1.479)--(6.728,1.479)--(6.770,1.479)--(6.811,1.479)--(6.852,1.479)--(6.891,1.479)%
  --(6.929,1.479)--(6.966,1.479)--(7.002,1.479)--(7.038,1.479)--(7.073,1.479)--(7.107,1.479)%
  --(7.140,1.479)--(7.173,1.479)--(7.205,1.479)--(7.236,1.479)--(7.267,1.479)--(7.297,1.479)%
  --(7.327,1.479)--(7.356,1.479)--(7.384,1.479)--(7.412,1.479)--(7.440,1.479)--(7.467,1.479)%
  --(7.493,1.479)--(7.520,1.479)--(7.545,1.479)--(7.571,1.479)--(7.596,1.479)--(7.620,1.479)%
  --(7.644,1.479)--(7.668,1.479)--(7.691,1.479)--(7.715,1.479)--(7.737,1.479)--(7.760,1.479)%
  --(7.782,1.479)--(7.804,1.479)--(7.825,1.479)--(7.847,1.479)--(7.868,1.479)--(7.888,1.479)%
  --(7.909,1.479)--(7.929,1.479)--(7.949,1.479)--(7.968,1.479)--(7.988,1.479)--(8.007,1.910)%
  --(8.026,1.910)--(8.045,1.910)--(8.063,1.910)--(8.082,1.910)--(8.100,1.910)--(8.118,1.910)%
  --(8.135,1.910)--(8.153,1.910)--(8.170,1.910)--(8.187,1.910)--(8.204,1.910)--(8.221,1.910)%
  --(8.237,1.910)--(8.254,1.910)--(8.270,1.910)--(8.286,1.910)--(8.302,1.910)--(8.318,1.910)%
  --(8.333,1.910)--(8.349,1.910)--(8.364,1.910)--(8.379,1.910)--(8.394,1.910)--(8.409,1.910)%
  --(8.424,1.910)--(8.438,1.910)--(8.453,1.910)--(8.467,1.910)--(8.481,1.910)--(8.495,1.910)%
  --(8.509,1.910)--(8.523,1.910)--(8.537,1.910)--(8.551,1.910)--(8.564,1.910)--(8.577,1.910)%
  --(8.591,1.910)--(8.604,1.910)--(8.617,1.910)--(8.630,1.910)--(8.643,1.910)--(8.655,1.910)%
  --(8.668,1.910)--(8.680,1.910)--(8.693,1.910)--(8.705,1.910)--(8.717,1.910)--(8.729,1.910)%
  --(8.741,1.910)--(8.753,1.910)--(8.765,1.910)--(8.777,1.910)--(8.789,1.910)--(8.800,1.910)%
  --(8.812,1.910)--(8.823,1.910)--(8.835,1.910)--(8.846,1.910)--(8.857,1.910)--(8.868,1.910)%
  --(8.879,1.910)--(8.890,1.910)--(8.901,1.910)--(8.912,1.910)--(8.923,1.910)--(8.933,1.910)%
  --(8.944,1.910)--(8.954,1.910)--(8.965,1.910)--(8.975,1.910)--(8.986,1.910)--(8.996,1.910)%
  --(9.006,1.910)--(9.016,1.910)--(9.026,1.910)--(9.036,1.910)--(9.046,1.910)--(9.056,1.910)%
  --(9.066,1.910)--(9.075,1.910)--(9.085,1.910)--(9.095,1.910)--(9.104,1.910)--(9.114,2.342)%
  --(9.123,2.342)--(9.133,2.342)--(9.142,2.342)--(9.151,2.342)--(9.161,2.342)--(9.170,2.342)%
  --(9.179,2.342)--(9.188,2.342)--(9.197,2.342)--(9.206,2.342)--(9.215,2.342)--(9.224,2.342)%
  --(9.232,2.342)--(9.241,2.342)--(9.250,2.342)--(9.259,2.342)--(9.267,2.342)--(9.276,2.342)%
  --(9.284,2.342)--(9.293,2.342)--(9.301,2.342)--(9.310,2.342)--(9.318,2.342)--(9.326,2.342)%
  --(9.335,2.342)--(9.343,2.342)--(9.351,2.342)--(9.359,2.342)--(9.367,2.342)--(9.375,2.342)%
  --(9.383,2.342)--(9.391,2.342)--(9.399,2.342)--(9.407,2.342)--(9.415,2.342)--(9.423,2.342)%
  --(9.431,2.342)--(9.438,2.342)--(9.446,2.342)--(9.454,2.342)--(9.461,2.342)--(9.469,2.342)%
  --(9.476,2.342)--(9.484,2.342)--(9.491,2.342)--(9.499,2.342)--(9.506,2.342)--(9.514,2.342)%
  --(9.521,2.342)--(9.528,2.342)--(9.536,2.342)--(9.543,2.342)--(9.550,2.342)--(9.557,2.342)%
  --(9.564,2.342)--(9.572,2.342)--(9.579,2.342)--(9.586,2.342)--(9.593,2.342)--(9.600,2.342)%
  --(9.607,2.342)--(9.614,2.342)--(9.620,2.342)--(9.627,2.342)--(9.634,2.342)--(9.641,2.342)%
  --(9.648,2.342)--(9.655,2.342)--(9.661,2.342)--(9.668,2.342)--(9.675,2.342)--(9.681,2.342)%
  --(9.688,2.342)--(9.694,2.342)--(9.701,2.342)--(9.708,2.342)--(9.714,2.342)--(9.721,2.342)%
  --(9.727,2.342)--(9.733,2.342)--(9.740,2.342)--(9.746,2.342)--(9.752,2.342)--(9.759,2.342)%
  --(9.765,2.342)--(9.771,2.342)--(9.778,2.342)--(9.784,2.342)--(9.790,2.342)--(9.796,2.342)%
  --(9.802,2.342)--(9.808,2.342)--(9.815,2.342)--(9.821,2.342)--(9.827,2.342)--(9.833,2.342)%
  --(9.839,2.342)--(9.845,2.342)--(9.851,2.342)--(9.857,2.342)--(9.863,2.342)--(9.868,2.342)%
  --(9.874,2.342)--(9.880,2.342)--(9.886,2.342)--(9.892,2.342)--(9.898,2.342)--(9.903,2.342)%
  --(9.909,2.342)--(9.915,2.342)--(9.921,2.342)--(9.926,2.342)--(9.932,2.342)--(9.938,2.342)%
  --(9.943,2.342)--(9.949,2.342)--(9.954,2.342)--(9.960,2.342)--(9.965,2.342)--(9.971,2.342)%
  --(9.976,2.342)--(9.982,2.342)--(9.987,2.342)--(9.993,2.342)--(9.998,2.342)--(10.004,2.342)%
  --(10.009,2.342)--(10.015,2.342)--(10.020,2.342)--(10.025,2.342)--(10.031,2.342)--(10.036,2.342)%
  --(10.041,2.342)--(10.046,2.342)--(10.052,2.342)--(10.057,2.342)--(10.062,2.342)--(10.067,2.342)%
  --(10.072,2.342)--(10.078,2.342)--(10.083,2.342)--(10.088,2.342)--(10.093,2.342)--(10.098,2.342)%
  --(10.103,2.342)--(10.108,2.342)--(10.113,2.342)--(10.118,2.342)--(10.123,2.342)--(10.128,2.342)%
  --(10.133,2.342)--(10.138,2.342)--(10.143,2.342)--(10.148,2.342)--(10.153,2.342)--(10.158,2.342)%
  --(10.163,2.342)--(10.168,2.342)--(10.173,2.342)--(10.178,2.342)--(10.182,2.342)--(10.187,2.342)%
  --(10.192,2.342)--(10.197,2.342)--(10.202,2.342)--(10.206,2.342)--(10.211,2.342)--(10.216,2.342)%
  --(10.221,2.342)--(10.225,2.342)--(10.230,2.342)--(10.235,2.342)--(10.239,2.342)--(10.244,2.342)%
  --(10.249,2.342)--(10.253,2.342)--(10.258,2.342)--(10.262,2.342)--(10.267,2.342)--(10.272,2.342)%
  --(10.276,2.342)--(10.281,2.342)--(10.285,2.342)--(10.290,2.342)--(10.294,2.342)--(10.299,2.342)%
  --(10.303,2.342)--(10.308,2.342)--(10.312,2.342)--(10.316,2.342)--(10.321,2.342)--(10.325,2.342)%
  --(10.330,2.342)--(10.334,2.342)--(10.338,2.342)--(10.343,2.342)--(10.347,2.342)--(10.352,2.342)%
  --(10.356,2.342)--(10.360,2.342)--(10.365,2.342)--(10.369,2.342)--(10.373,2.342)--(10.377,2.342)%
  --(10.382,2.342)--(10.386,2.342)--(10.390,2.342)--(10.394,2.342)--(10.399,2.342)--(10.403,2.342)%
  --(10.407,2.342)--(10.411,2.342)--(10.415,2.342)--(10.419,2.342)--(10.424,2.342)--(10.428,2.342)%
  --(10.432,2.342)--(10.436,2.342)--(10.440,2.342)--(10.444,2.342)--(10.448,2.342)--(10.452,2.342)%
  --(10.456,2.342)--(10.460,2.342)--(10.465,2.342)--(10.469,2.342)--(10.473,2.342)--(10.477,2.342)%
  --(10.481,2.342)--(10.485,2.342)--(10.489,2.342)--(10.493,2.342)--(10.497,2.342)--(10.500,2.342)%
  --(10.504,2.342)--(10.508,2.342)--(10.512,2.342)--(10.516,2.342)--(10.520,2.342)--(10.524,2.342)%
  --(10.528,2.342)--(10.532,2.342)--(10.536,2.342)--(10.539,2.342)--(10.543,2.342)--(10.547,2.342)%
  --(10.551,2.342)--(10.555,2.342)--(10.559,2.342)--(10.562,2.342)--(10.566,2.342)--(10.570,2.342)%
  --(10.574,2.342)--(10.578,2.342)--(10.581,2.342)--(10.585,2.342)--(10.589,2.342)--(10.592,2.342)%
  --(10.596,2.342)--(10.600,2.342)--(10.604,2.342)--(10.607,2.342)--(10.611,2.342)--(10.615,2.342)%
  --(10.618,2.342)--(10.622,2.342)--(10.626,2.342)--(10.629,2.342)--(10.633,2.342)--(10.637,2.342)%
  --(10.640,2.342)--(10.644,2.342)--(10.647,2.342)--(10.651,2.342)--(10.655,2.342)--(10.658,2.342)%
  --(10.662,2.342)--(10.665,2.342)--(10.669,2.342)--(10.672,2.342)--(10.676,2.342)--(10.679,2.342)%
  --(10.683,2.342)--(10.686,2.342)--(10.690,2.342)--(10.693,2.342)--(10.697,2.342)--(10.700,2.342)%
  --(10.704,2.342)--(10.707,2.342)--(10.711,2.342)--(10.714,2.342)--(10.718,2.342)--(10.721,2.342)%
  --(10.725,2.342)--(10.728,2.342)--(10.731,2.342)--(10.735,2.342)--(10.738,2.342)--(10.742,2.342)%
  --(10.745,2.342)--(10.748,2.342)--(10.752,2.342)--(10.755,2.342)--(10.759,2.342)--(10.762,2.342)%
  --(10.765,2.342)--(10.769,2.342)--(10.772,2.342)--(10.775,2.342)--(10.779,2.342)--(10.782,2.342)%
  --(10.785,2.342)--(10.788,2.342)--(10.792,2.342)--(10.795,2.342)--(10.798,2.342)--(10.802,2.342)%
  --(10.805,2.342)--(10.808,2.342)--(10.811,2.342)--(10.815,2.342)--(10.818,2.342)--(10.821,2.342)%
  --(10.824,2.342)--(10.827,2.342)--(10.831,2.342)--(10.834,2.342)--(10.837,2.342)--(10.840,2.342)%
  --(10.843,2.342)--(10.847,2.342)--(10.850,2.342)--(10.853,2.342)--(10.856,2.342)--(10.859,2.342)%
  --(10.862,2.342)--(10.865,2.342)--(10.869,2.342)--(10.872,2.342)--(10.875,2.342)--(10.878,2.342)%
  --(10.881,2.342)--(10.884,2.342)--(10.887,2.342)--(10.890,2.342)--(10.893,2.342)--(10.897,2.342)%
  --(10.900,2.342)--(10.903,2.342)--(10.906,2.342)--(10.909,2.342)--(10.912,2.342)--(10.915,2.342)%
  --(10.918,2.342)--(10.921,2.342)--(10.924,2.342)--(10.927,2.342)--(10.930,2.342)--(10.933,2.342)%
  --(10.936,2.342)--(10.939,2.342)--(10.942,2.342)--(10.945,2.342)--(10.948,2.342)--(10.951,2.342)%
  --(10.954,2.342)--(10.957,2.342)--(10.960,2.342)--(10.963,2.342)--(10.966,2.342)--(10.969,2.342)%
  --(10.972,2.342)--(10.975,2.342)--(10.977,2.342)--(10.980,2.342)--(10.983,2.342)--(10.986,2.342)%
  --(10.989,2.342)--(10.992,2.342)--(10.995,2.342)--(10.998,2.342)--(11.001,2.342)--(11.003,2.342)%
  --(11.006,2.342)--(11.009,2.342)--(11.012,2.342)--(11.015,2.342)--(11.018,2.342)--(11.021,2.342)%
  --(11.023,2.342)--(11.026,2.342)--(11.029,2.342)--(11.032,2.342)--(11.035,2.342)--(11.038,2.342)%
  --(11.040,2.342)--(11.043,2.342)--(11.046,2.342)--(11.049,2.342)--(11.052,2.342)--(11.054,2.342)%
  --(11.057,2.342)--(11.060,2.342)--(11.063,2.342)--(11.065,2.342)--(11.068,2.342)--(11.071,2.342)%
  --(11.074,2.342)--(11.076,2.342)--(11.079,2.342)--(11.082,2.342)--(11.085,2.342)--(11.087,2.342)%
  --(11.090,2.342)--(11.093,2.342)--(11.096,2.342)--(11.098,2.342)--(11.101,2.342)--(11.104,2.342)%
  --(11.106,2.342)--(11.109,2.342)--(11.112,2.342)--(11.114,2.342)--(11.117,2.342)--(11.120,2.342)%
  --(11.122,2.342)--(11.125,2.342)--(11.128,2.342)--(11.130,2.342)--(11.133,2.342)--(11.136,2.342);
\gpsetdashtype{gp dt 4}
\draw[gp path] (1.012,2.342)--(2.109,3.636)--(2.751,4.930)--(3.207,5.361)--(3.560,5.793)%
  --(3.848,5.793)--(4.092,5.793)--(4.304,5.793)--(4.490,5.793)--(4.657,5.793)--(4.808,5.793)%
  --(4.946,5.793)--(5.072,5.793)--(5.190,6.224)--(5.299,6.224)--(5.401,6.224)--(5.497,6.224)%
  --(5.587,6.224)--(5.673,6.224)--(5.754,6.224)--(5.831,6.224)--(5.905,6.224)--(5.975,6.224)%
  --(6.043,6.224)--(6.107,6.224)--(6.170,6.224)--(6.229,6.224)--(6.287,6.224)--(6.342,6.224)%
  --(6.396,6.224)--(6.448,6.224)--(6.498,6.224)--(6.547,6.224)--(6.594,6.224)--(6.640,6.224)%
  --(6.685,6.224)--(6.728,6.224)--(6.770,6.224)--(6.811,6.224)--(6.852,6.224)--(6.891,6.224)%
  --(6.929,6.224)--(6.966,6.224)--(7.002,6.224)--(7.038,6.224)--(7.073,6.224)--(7.107,6.224)%
  --(7.140,6.224)--(7.173,6.224)--(7.205,6.224)--(7.236,6.224)--(7.267,6.224)--(7.297,6.224)%
  --(7.327,6.224)--(7.356,6.224)--(7.384,6.224)--(7.412,6.224)--(7.440,6.224)--(7.467,6.224)%
  --(7.493,6.224)--(7.520,6.224)--(7.545,6.224)--(7.571,6.224)--(7.596,6.224)--(7.620,6.224)%
  --(7.644,6.224)--(7.668,6.224)--(7.691,6.224)--(7.715,6.224)--(7.737,6.224)--(7.760,6.224)%
  --(7.782,6.224)--(7.804,6.224)--(7.825,6.224)--(7.847,6.224)--(7.868,6.224)--(7.888,6.224)%
  --(7.909,6.224)--(7.929,6.224)--(7.949,6.224)--(7.968,6.224)--(7.988,6.224)--(8.007,6.224)%
  --(8.026,6.224)--(8.045,6.224)--(8.063,6.224)--(8.082,6.224)--(8.100,6.224)--(8.118,6.224)%
  --(8.135,6.224)--(8.153,6.224)--(8.170,6.224)--(8.187,6.655)--(8.204,6.655)--(8.221,6.655)%
  --(8.237,6.655)--(8.254,6.655)--(8.270,6.655)--(8.286,6.655)--(8.302,6.655)--(8.318,6.655)%
  --(8.333,6.655)--(8.349,6.655)--(8.364,6.655)--(8.379,6.655)--(8.394,6.655)--(8.409,6.655)%
  --(8.424,6.655)--(8.438,6.655)--(8.453,6.655)--(8.467,6.655)--(8.481,6.655)--(8.495,6.655)%
  --(8.509,6.655)--(8.523,6.655)--(8.537,6.655)--(8.551,7.087)--(8.564,7.087)--(8.577,7.087)%
  --(8.591,7.087)--(8.604,7.087)--(8.617,7.087)--(8.630,7.087)--(8.643,7.087)--(8.655,7.087)%
  --(8.668,7.087)--(8.680,7.087)--(8.693,7.087)--(8.705,7.087)--(8.717,7.087)--(8.729,7.087)%
  --(8.741,7.087)--(8.753,7.087)--(8.765,7.087)--(8.777,7.087)--(8.789,7.087)--(8.800,7.087)%
  --(8.812,7.087)--(8.823,7.087)--(8.835,7.087)--(8.846,7.087)--(8.857,7.087)--(8.868,7.087)%
  --(8.879,7.087)--(8.890,7.087)--(8.901,7.087)--(8.912,7.087)--(8.923,7.087)--(8.933,7.087)%
  --(8.944,7.087)--(8.954,7.087)--(8.965,7.087)--(8.975,7.087)--(8.986,7.087)--(8.996,7.087)%
  --(9.006,7.087)--(9.016,7.087)--(9.026,7.087)--(9.036,7.087)--(9.046,7.087)--(9.056,7.087)%
  --(9.066,7.087)--(9.075,7.087)--(9.085,7.087)--(9.095,7.087)--(9.104,7.087)--(9.114,7.087)%
  --(9.123,7.087)--(9.133,7.087)--(9.142,7.087)--(9.151,7.087)--(9.161,7.087)--(9.170,7.087)%
  --(9.179,7.087)--(9.188,7.087)--(9.197,7.087)--(9.206,7.087)--(9.215,7.087)--(9.224,7.087)%
  --(9.232,7.087)--(9.241,7.087)--(9.250,7.087)--(9.259,7.087)--(9.267,7.087)--(9.276,7.087)%
  --(9.284,7.087)--(9.293,7.087)--(9.301,7.087)--(9.310,7.087)--(9.318,7.087)--(9.326,7.087)%
  --(9.335,7.087)--(9.343,7.087)--(9.351,7.087)--(9.359,7.087)--(9.367,7.087)--(9.375,7.087)%
  --(9.383,7.087)--(9.391,7.087)--(9.399,7.087)--(9.407,7.087)--(9.415,7.087)--(9.423,7.087)%
  --(9.431,7.087)--(9.438,7.087)--(9.446,7.087)--(9.454,7.087)--(9.461,7.087)--(9.469,7.518)%
  --(9.476,7.518)--(9.484,7.518)--(9.491,7.518)--(9.499,7.518)--(9.506,7.518)--(9.514,7.518)%
  --(9.521,7.518)--(9.528,7.518)--(9.536,7.518)--(9.543,7.518)--(9.550,7.518)--(9.557,7.518)%
  --(9.564,7.518)--(9.572,7.518)--(9.579,7.518)--(9.586,7.518)--(9.593,7.518)--(9.600,7.518)%
  --(9.607,7.518)--(9.614,7.518)--(9.620,7.518)--(9.627,7.518)--(9.634,7.518)--(9.641,7.518)%
  --(9.648,7.518)--(9.655,7.518)--(9.661,7.518)--(9.668,7.518)--(9.675,7.518)--(9.681,7.518)%
  --(9.688,7.518)--(9.694,7.518)--(9.701,7.518)--(9.708,7.518)--(9.714,7.518)--(9.721,7.518)%
  --(9.727,7.518)--(9.733,7.518)--(9.740,7.518)--(9.746,7.518)--(9.752,7.518)--(9.759,7.518)%
  --(9.765,7.518)--(9.771,7.518)--(9.778,7.518)--(9.784,7.518)--(9.790,7.518)--(9.796,7.518)%
  --(9.802,7.518)--(9.808,7.518)--(9.815,7.518)--(9.821,7.518)--(9.827,7.518)--(9.833,7.518)%
  --(9.839,7.518)--(9.845,7.518)--(9.851,7.518)--(9.857,7.518)--(9.863,7.518)--(9.868,7.518)%
  --(9.874,7.518)--(9.880,7.518)--(9.886,7.518)--(9.892,7.518)--(9.898,7.518)--(9.903,7.518)%
  --(9.909,7.518)--(9.915,7.518)--(9.921,7.518)--(9.926,7.518)--(9.932,7.518)--(9.938,7.518)%
  --(9.943,7.518)--(9.949,7.518)--(9.954,7.518)--(9.960,7.518)--(9.965,7.518)--(9.971,7.518)%
  --(9.976,7.518)--(9.982,7.518)--(9.987,7.518)--(9.993,7.518)--(9.998,7.518)--(10.004,7.518)%
  --(10.009,7.518)--(10.015,7.518)--(10.020,7.518)--(10.025,7.518)--(10.031,7.518)--(10.036,7.518)%
  --(10.041,7.518)--(10.046,7.518)--(10.052,7.518)--(10.057,7.518)--(10.062,7.518)--(10.067,7.518)%
  --(10.072,7.518)--(10.078,7.518)--(10.083,7.518)--(10.088,7.518)--(10.093,7.518)--(10.098,7.518)%
  --(10.103,7.518)--(10.108,7.518)--(10.113,7.518)--(10.118,7.518)--(10.123,7.518)--(10.128,7.518)%
  --(10.133,7.518)--(10.138,7.518)--(10.143,7.518)--(10.148,7.518)--(10.153,7.518)--(10.158,7.518)%
  --(10.163,7.518)--(10.168,7.518)--(10.173,7.518)--(10.178,7.518)--(10.182,7.518)--(10.187,7.518)%
  --(10.192,7.518)--(10.197,7.518)--(10.202,7.518)--(10.206,7.518)--(10.211,7.518)--(10.216,7.518)%
  --(10.221,7.518)--(10.225,7.518)--(10.230,7.518)--(10.235,7.518)--(10.239,7.518)--(10.244,7.518)%
  --(10.249,7.518)--(10.253,7.518)--(10.258,7.518)--(10.262,7.518)--(10.267,7.518)--(10.272,7.518)%
  --(10.276,7.518)--(10.281,7.518)--(10.285,7.518)--(10.290,7.518)--(10.294,7.518)--(10.299,7.518)%
  --(10.303,7.518)--(10.308,7.518)--(10.312,7.518)--(10.316,7.518)--(10.321,7.518)--(10.325,7.518)%
  --(10.330,7.518)--(10.334,7.518)--(10.338,7.518)--(10.343,7.518)--(10.347,7.518)--(10.352,7.518)%
  --(10.356,7.518)--(10.360,7.518)--(10.365,7.518)--(10.369,7.518)--(10.373,7.518)--(10.377,7.518)%
  --(10.382,7.518)--(10.386,7.518)--(10.390,7.518)--(10.394,7.518)--(10.399,7.518)--(10.403,7.518)%
  --(10.407,7.518)--(10.411,7.518)--(10.415,7.518)--(10.419,7.518)--(10.424,7.518)--(10.428,7.518)%
  --(10.432,7.518)--(10.436,7.518)--(10.440,7.518)--(10.444,7.518)--(10.448,7.518)--(10.452,7.518)%
  --(10.456,7.518)--(10.460,7.518)--(10.465,7.518)--(10.469,7.518)--(10.473,7.518)--(10.477,7.518)%
  --(10.481,7.518)--(10.485,7.518)--(10.489,7.518)--(10.493,7.518)--(10.497,7.518)--(10.500,7.518)%
  --(10.504,7.518)--(10.508,7.518)--(10.512,7.518)--(10.516,7.518)--(10.520,7.518)--(10.524,7.518)%
  --(10.528,7.518)--(10.532,7.518)--(10.536,7.950)--(10.539,7.950)--(10.543,7.950)--(10.547,7.950)%
  --(10.551,7.950)--(10.555,7.950)--(10.559,7.950)--(10.562,7.950)--(10.566,7.950)--(10.570,7.950)%
  --(10.574,7.950)--(10.578,7.950)--(10.581,7.950)--(10.585,7.950)--(10.589,7.950)--(10.592,7.950)%
  --(10.596,7.950)--(10.600,7.950)--(10.604,7.950)--(10.607,7.950)--(10.611,7.950)--(10.615,7.950)%
  --(10.618,7.950)--(10.622,7.950)--(10.626,7.950)--(10.629,7.950)--(10.633,7.950)--(10.637,7.950)%
  --(10.640,7.950)--(10.644,7.950)--(10.647,7.950)--(10.651,7.950)--(10.655,7.950)--(10.658,7.950)%
  --(10.662,7.950)--(10.665,7.950)--(10.669,7.950)--(10.672,7.950)--(10.676,7.950)--(10.679,7.950)%
  --(10.683,8.381)--(10.686,8.381)--(10.690,8.381)--(10.693,8.381)--(10.697,8.381)--(10.700,8.381)%
  --(10.704,8.381)--(10.707,8.381)--(10.711,8.381)--(10.714,8.381)--(10.718,8.381)--(10.721,8.381)%
  --(10.725,8.381)--(10.728,8.381)--(10.731,8.381)--(10.735,8.381)--(10.738,8.381)--(10.742,8.381)%
  --(10.745,8.381)--(10.748,8.381)--(10.752,8.381)--(10.755,8.381)--(10.759,8.381)--(10.762,8.381)%
  --(10.765,8.381)--(10.769,8.381)--(10.772,8.381)--(10.775,8.381)--(10.779,8.381)--(10.782,8.381)%
  --(10.785,8.381)--(10.788,8.381)--(10.792,8.381)--(10.795,8.381)--(10.798,8.381)--(10.802,8.381)%
  --(10.805,8.381)--(10.808,8.381)--(10.811,8.381)--(10.815,8.381)--(10.818,8.381)--(10.821,8.381)%
  --(10.824,8.381)--(10.827,8.381)--(10.831,8.381)--(10.834,8.381)--(10.837,8.381)--(10.840,8.381)%
  --(10.843,8.381)--(10.847,8.381)--(10.850,8.381)--(10.853,8.381)--(10.856,8.381)--(10.859,8.381)%
  --(10.862,8.381)--(10.865,8.381)--(10.869,8.381)--(10.872,8.381)--(10.875,8.381)--(10.878,8.381)%
  --(10.881,8.381)--(10.884,8.381)--(10.887,8.381)--(10.890,8.381)--(10.893,8.381)--(10.897,8.381)%
  --(10.900,8.381)--(10.903,8.381)--(10.906,8.381)--(10.909,8.381)--(10.912,8.381)--(10.915,8.381)%
  --(10.918,8.381)--(10.921,8.381)--(10.924,8.381)--(10.927,8.381)--(10.930,8.381)--(10.933,8.381)%
  --(10.936,8.381)--(10.939,8.381)--(10.942,8.381)--(10.945,8.381)--(10.948,8.381)--(10.951,8.381)%
  --(10.954,8.381)--(10.957,8.381)--(10.960,8.381)--(10.963,8.381)--(10.966,8.381)--(10.969,8.381)%
  --(10.972,8.381)--(10.975,8.381)--(10.977,8.381)--(10.980,8.381)--(10.983,8.381)--(10.986,8.381)%
  --(10.989,8.381)--(10.992,8.381)--(10.995,8.381)--(10.998,8.381)--(11.001,8.381)--(11.003,8.381)%
  --(11.006,8.381)--(11.009,8.381)--(11.012,8.381)--(11.015,8.381)--(11.018,8.381)--(11.021,8.381)%
  --(11.023,8.381)--(11.026,8.381)--(11.029,8.381)--(11.032,8.381)--(11.035,8.381)--(11.038,8.381)%
  --(11.040,8.381)--(11.043,8.381)--(11.046,8.381)--(11.049,8.381)--(11.052,8.381)--(11.054,8.381)%
  --(11.057,8.381)--(11.060,8.381)--(11.063,8.381)--(11.065,8.381)--(11.068,8.381)--(11.071,8.381)%
  --(11.074,8.381)--(11.076,8.381)--(11.079,8.381)--(11.082,8.381)--(11.085,8.381)--(11.087,8.381)%
  --(11.090,8.381)--(11.093,8.381)--(11.096,8.381)--(11.098,8.381)--(11.101,8.381)--(11.104,8.381)%
  --(11.106,8.381)--(11.109,8.381)--(11.112,8.381)--(11.114,8.381)--(11.117,8.381)--(11.120,8.381)%
  --(11.122,8.381)--(11.125,8.381)--(11.128,8.381)--(11.130,8.381)--(11.133,8.381)--(11.136,8.381);
\gpsetdashtype{gp dt 3}
\draw[gp path] (1.012,2.342)--(2.109,3.204)--(2.751,4.067)--(3.207,4.930)--(3.560,4.930)%
  --(3.848,4.930)--(4.092,4.930)--(4.304,4.930)--(4.490,4.930)--(4.657,4.930)--(4.808,5.793)%
  --(4.946,5.793)--(5.072,5.793)--(5.190,5.793)--(5.299,5.793)--(5.401,5.793)--(5.497,5.793)%
  --(5.587,5.793)--(5.673,5.793)--(5.754,5.793)--(5.831,5.793)--(5.905,5.793)--(5.975,5.793)%
  --(6.043,5.793)--(6.107,5.793)--(6.170,5.793)--(6.229,5.793)--(6.287,5.793)--(6.342,5.793)%
  --(6.396,5.793)--(6.448,5.793)--(6.498,5.793)--(6.547,5.793)--(6.594,5.793)--(6.640,5.793)%
  --(6.685,5.793)--(6.728,5.793)--(6.770,5.793)--(6.811,5.793)--(6.852,5.793)--(6.891,5.793)%
  --(6.929,5.793)--(6.966,5.793)--(7.002,5.793)--(7.038,5.793)--(7.073,5.793)--(7.107,5.793)%
  --(7.140,5.793)--(7.173,5.793)--(7.205,5.793)--(7.236,5.793)--(7.267,5.793)--(7.297,5.793)%
  --(7.327,5.793)--(7.356,5.793)--(7.384,5.793)--(7.412,5.793)--(7.440,5.793)--(7.467,5.793)%
  --(7.493,5.793)--(7.520,5.793)--(7.545,5.793)--(7.571,5.793)--(7.596,5.793)--(7.620,5.793)%
  --(7.644,5.793)--(7.668,5.793)--(7.691,5.793)--(7.715,5.793)--(7.737,5.793)--(7.760,5.793)%
  --(7.782,5.793)--(7.804,5.793)--(7.825,5.793)--(7.847,5.793)--(7.868,5.793)--(7.888,5.793)%
  --(7.909,5.793)--(7.929,5.793)--(7.949,5.793)--(7.968,5.793)--(7.988,5.793)--(8.007,5.793)%
  --(8.026,6.224)--(8.045,6.224)--(8.063,6.224)--(8.082,6.224)--(8.100,6.224)--(8.118,6.224)%
  --(8.135,6.224)--(8.153,6.224)--(8.170,6.224)--(8.187,6.224)--(8.204,6.224)--(8.221,6.224)%
  --(8.237,6.224)--(8.254,6.655)--(8.270,6.655)--(8.286,6.655)--(8.302,6.655)--(8.318,6.655)%
  --(8.333,6.655)--(8.349,6.655)--(8.364,6.655)--(8.379,6.655)--(8.394,6.655)--(8.409,6.655)%
  --(8.424,6.655)--(8.438,6.655)--(8.453,6.655)--(8.467,6.655)--(8.481,6.655)--(8.495,6.655)%
  --(8.509,6.655)--(8.523,6.655)--(8.537,6.655)--(8.551,6.655)--(8.564,6.655)--(8.577,6.655)%
  --(8.591,6.655)--(8.604,6.655)--(8.617,6.655)--(8.630,6.655)--(8.643,6.655)--(8.655,6.655)%
  --(8.668,6.655)--(8.680,6.655)--(8.693,6.655)--(8.705,7.087)--(8.717,7.087)--(8.729,7.087)%
  --(8.741,7.087)--(8.753,7.087)--(8.765,7.087)--(8.777,7.087)--(8.789,7.087)--(8.800,7.087)%
  --(8.812,7.087)--(8.823,7.087)--(8.835,7.087)--(8.846,7.087)--(8.857,7.087)--(8.868,7.087)%
  --(8.879,7.087)--(8.890,7.087)--(8.901,7.087)--(8.912,7.087)--(8.923,7.087)--(8.933,7.087)%
  --(8.944,7.087)--(8.954,7.087)--(8.965,7.087)--(8.975,7.087)--(8.986,7.087)--(8.996,7.087)%
  --(9.006,7.087)--(9.016,7.087)--(9.026,7.087)--(9.036,7.087)--(9.046,7.087)--(9.056,7.087)%
  --(9.066,7.087)--(9.075,7.087)--(9.085,7.087)--(9.095,7.087)--(9.104,7.087)--(9.114,7.087)%
  --(9.123,7.087)--(9.133,7.087)--(9.142,7.087)--(9.151,7.087)--(9.161,7.087)--(9.170,7.087)%
  --(9.179,7.087)--(9.188,7.087)--(9.197,7.087)--(9.206,7.087)--(9.215,7.087)--(9.224,7.087)%
  --(9.232,7.087)--(9.241,7.087)--(9.250,7.087)--(9.259,7.087)--(9.267,7.087)--(9.276,7.087)%
  --(9.284,7.087)--(9.293,7.087)--(9.301,7.087)--(9.310,7.087)--(9.318,7.087)--(9.326,7.087)%
  --(9.335,7.087)--(9.343,7.087)--(9.351,7.087)--(9.359,7.087)--(9.367,7.087)--(9.375,7.087)%
  --(9.383,7.087)--(9.391,7.087)--(9.399,7.087)--(9.407,7.087)--(9.415,7.087)--(9.423,7.087)%
  --(9.431,7.087)--(9.438,7.087)--(9.446,7.087)--(9.454,7.087)--(9.461,7.087)--(9.469,7.087)%
  --(9.476,7.087)--(9.484,7.087)--(9.491,7.087)--(9.499,7.087)--(9.506,7.087)--(9.514,7.087)%
  --(9.521,7.087)--(9.528,7.087)--(9.536,7.087)--(9.543,7.087)--(9.550,7.087)--(9.557,7.087)%
  --(9.564,7.087)--(9.572,7.087)--(9.579,7.087)--(9.586,7.087)--(9.593,7.087)--(9.600,7.087)%
  --(9.607,7.087)--(9.614,7.087)--(9.620,7.087)--(9.627,7.087)--(9.634,7.087)--(9.641,7.087)%
  --(9.648,7.087)--(9.655,7.087)--(9.661,7.087)--(9.668,7.087)--(9.675,7.087)--(9.681,7.087)%
  --(9.688,7.087)--(9.694,7.087)--(9.701,7.087)--(9.708,7.087)--(9.714,7.087)--(9.721,7.087)%
  --(9.727,7.087)--(9.733,7.087)--(9.740,7.087)--(9.746,7.087)--(9.752,7.087)--(9.759,7.087)%
  --(9.765,7.087)--(9.771,7.087)--(9.778,7.087)--(9.784,7.087)--(9.790,7.087)--(9.796,7.087)%
  --(9.802,7.087)--(9.808,7.087)--(9.815,7.087)--(9.821,7.087)--(9.827,7.087)--(9.833,7.087)%
  --(9.839,7.087)--(9.845,7.087)--(9.851,7.087)--(9.857,7.087)--(9.863,7.087)--(9.868,7.087)%
  --(9.874,7.087)--(9.880,7.087)--(9.886,7.087)--(9.892,7.087)--(9.898,7.087)--(9.903,7.087)%
  --(9.909,7.087)--(9.915,7.087)--(9.921,7.087)--(9.926,7.087)--(9.932,7.087)--(9.938,7.087)%
  --(9.943,7.087)--(9.949,7.087)--(9.954,7.087)--(9.960,7.087)--(9.965,7.087)--(9.971,7.087)%
  --(9.976,7.087)--(9.982,7.087)--(9.987,7.087)--(9.993,7.087)--(9.998,7.087)--(10.004,7.087)%
  --(10.009,7.087)--(10.015,7.087)--(10.020,7.087)--(10.025,7.087)--(10.031,7.087)--(10.036,7.087)%
  --(10.041,7.087)--(10.046,7.087)--(10.052,7.087)--(10.057,7.087)--(10.062,7.087)--(10.067,7.087)%
  --(10.072,7.087)--(10.078,7.087)--(10.083,7.087)--(10.088,7.087)--(10.093,7.087)--(10.098,7.087)%
  --(10.103,7.087)--(10.108,7.087)--(10.113,7.087)--(10.118,7.087)--(10.123,7.087)--(10.128,7.087)%
  --(10.133,7.087)--(10.138,7.087)--(10.143,7.087)--(10.148,7.087)--(10.153,7.087)--(10.158,7.087)%
  --(10.163,7.087)--(10.168,7.087)--(10.173,7.087)--(10.178,7.087)--(10.182,7.087)--(10.187,7.087)%
  --(10.192,7.087)--(10.197,7.087)--(10.202,7.087)--(10.206,7.087)--(10.211,7.087)--(10.216,7.087)%
  --(10.221,7.087)--(10.225,7.087)--(10.230,7.087)--(10.235,7.087)--(10.239,7.087)--(10.244,7.087)%
  --(10.249,7.087)--(10.253,7.087)--(10.258,7.087)--(10.262,7.087)--(10.267,7.087)--(10.272,7.087)%
  --(10.276,7.087)--(10.281,7.087)--(10.285,7.087)--(10.290,7.087)--(10.294,7.087)--(10.299,7.087)%
  --(10.303,7.087)--(10.308,7.087)--(10.312,7.087)--(10.316,7.087)--(10.321,7.087)--(10.325,7.087)%
  --(10.330,7.087)--(10.334,7.087)--(10.338,7.087)--(10.343,7.087)--(10.347,7.087)--(10.352,7.087)%
  --(10.356,7.087)--(10.360,7.087)--(10.365,7.087)--(10.369,7.087)--(10.373,7.087)--(10.377,7.087)%
  --(10.382,7.087)--(10.386,7.087)--(10.390,7.087)--(10.394,7.087)--(10.399,7.087)--(10.403,7.087)%
  --(10.407,7.087)--(10.411,7.087)--(10.415,7.087)--(10.419,7.087)--(10.424,7.087)--(10.428,7.087)%
  --(10.432,7.087)--(10.436,7.087)--(10.440,7.087)--(10.444,7.087)--(10.448,7.087)--(10.452,7.087)%
  --(10.456,7.087)--(10.460,7.087)--(10.465,7.087)--(10.469,7.087)--(10.473,7.087)--(10.477,7.087)%
  --(10.481,7.087)--(10.485,7.087)--(10.489,7.087)--(10.493,7.087)--(10.497,7.087)--(10.500,7.087)%
  --(10.504,7.087)--(10.508,7.087)--(10.512,7.087)--(10.516,7.087)--(10.520,7.087)--(10.524,7.087)%
  --(10.528,7.087)--(10.532,7.087)--(10.536,7.087)--(10.539,7.087)--(10.543,7.087)--(10.547,7.087)%
  --(10.551,7.087)--(10.555,7.087)--(10.559,7.087)--(10.562,7.087)--(10.566,7.087)--(10.570,7.087)%
  --(10.574,7.087)--(10.578,7.087)--(10.581,7.087)--(10.585,7.087)--(10.589,7.087)--(10.592,7.087)%
  --(10.596,7.087)--(10.600,7.087)--(10.604,7.087)--(10.607,7.087)--(10.611,7.087)--(10.615,7.087)%
  --(10.618,7.087)--(10.622,7.087)--(10.626,7.087)--(10.629,7.087)--(10.633,7.087)--(10.637,7.087)%
  --(10.640,7.087)--(10.644,7.087)--(10.647,7.087)--(10.651,7.087)--(10.655,7.087)--(10.658,7.087)%
  --(10.662,7.087)--(10.665,7.087)--(10.669,7.087)--(10.672,7.087)--(10.676,7.087)--(10.679,7.087)%
  --(10.683,7.087)--(10.686,7.087)--(10.690,7.087)--(10.693,7.087)--(10.697,7.087)--(10.700,7.087)%
  --(10.704,7.087)--(10.707,7.087)--(10.711,7.518)--(10.714,7.518)--(10.718,7.518)--(10.721,7.518)%
  --(10.725,7.518)--(10.728,7.518)--(10.731,7.518)--(10.735,7.518)--(10.738,7.518)--(10.742,7.518)%
  --(10.745,7.518)--(10.748,7.518)--(10.752,7.518)--(10.755,7.518)--(10.759,7.518)--(10.762,7.518)%
  --(10.765,7.518)--(10.769,7.518)--(10.772,7.518)--(10.775,7.518)--(10.779,7.518)--(10.782,7.518)%
  --(10.785,7.518)--(10.788,7.518)--(10.792,7.518)--(10.795,7.518)--(10.798,7.518)--(10.802,7.518)%
  --(10.805,7.518)--(10.808,7.518)--(10.811,7.518)--(10.815,7.518)--(10.818,7.518)--(10.821,7.518)%
  --(10.824,7.518)--(10.827,7.518)--(10.831,7.518)--(10.834,7.518)--(10.837,7.518)--(10.840,7.518)%
  --(10.843,7.518)--(10.847,7.518)--(10.850,7.518)--(10.853,7.518)--(10.856,7.518)--(10.859,7.518)%
  --(10.862,7.518)--(10.865,7.518)--(10.869,7.518)--(10.872,7.518)--(10.875,7.518)--(10.878,7.518)%
  --(10.881,7.518)--(10.884,7.518)--(10.887,7.518)--(10.890,7.518)--(10.893,7.518)--(10.897,7.518)%
  --(10.900,7.518)--(10.903,7.518)--(10.906,7.518)--(10.909,7.518)--(10.912,7.518)--(10.915,7.518)%
  --(10.918,7.518)--(10.921,7.518)--(10.924,7.518)--(10.927,7.518)--(10.930,7.518)--(10.933,7.518)%
  --(10.936,7.518)--(10.939,7.518)--(10.942,7.518)--(10.945,7.518)--(10.948,7.518)--(10.951,7.518)%
  --(10.954,7.518)--(10.957,7.518)--(10.960,7.518)--(10.963,7.518)--(10.966,7.518)--(10.969,7.518)%
  --(10.972,7.518)--(10.975,7.518)--(10.977,7.518)--(10.980,7.518)--(10.983,7.518)--(10.986,7.518)%
  --(10.989,7.518)--(10.992,7.518)--(10.995,7.518)--(10.998,7.518)--(11.001,7.518)--(11.003,7.518)%
  --(11.006,7.518)--(11.009,7.518)--(11.012,7.518)--(11.015,7.518)--(11.018,7.518)--(11.021,7.518)%
  --(11.023,7.518)--(11.026,7.518)--(11.029,7.518)--(11.032,7.518)--(11.035,7.518)--(11.038,7.518)%
  --(11.040,7.518)--(11.043,7.518)--(11.046,7.518)--(11.049,7.518)--(11.052,7.518)--(11.054,7.518)%
  --(11.057,7.518)--(11.060,7.518)--(11.063,7.518)--(11.065,7.518)--(11.068,7.950)--(11.071,7.950)%
  --(11.074,7.950)--(11.076,7.950)--(11.079,7.950)--(11.082,7.950)--(11.085,7.950)--(11.087,7.950)%
  --(11.090,7.950)--(11.093,7.950)--(11.096,7.950)--(11.098,7.950)--(11.101,7.950)--(11.104,7.950)%
  --(11.106,7.950)--(11.109,7.950)--(11.112,7.950)--(11.114,7.950)--(11.117,7.950)--(11.120,7.950)%
  --(11.122,7.950)--(11.125,7.950)--(11.128,7.950)--(11.130,7.950)--(11.133,7.950)--(11.136,7.950);
\gpsetdashtype{gp dt 5}
\draw[gp path] (1.012,2.773)--(2.109,4.499)--(2.751,4.930)--(3.207,5.793)--(3.560,5.793)%
  --(3.848,5.793)--(4.092,5.793)--(4.304,6.224)--(4.490,6.655)--(4.657,6.655)--(4.808,6.655)%
  --(4.946,6.655)--(5.072,6.655)--(5.190,6.655)--(5.299,6.655)--(5.401,6.655)--(5.497,6.655)%
  --(5.587,6.655)--(5.673,6.655)--(5.754,6.655)--(5.831,6.655)--(5.905,6.655)--(5.975,6.655)%
  --(6.043,6.655)--(6.107,6.655)--(6.170,6.655)--(6.229,6.655)--(6.287,6.655)--(6.342,6.655)%
  --(6.396,6.655)--(6.448,6.655)--(6.498,6.655)--(6.547,6.655)--(6.594,6.655)--(6.640,7.087)%
  --(6.685,7.087)--(6.728,7.087)--(6.770,7.087)--(6.811,7.087)--(6.852,7.087)--(6.891,7.087)%
  --(6.929,7.087)--(6.966,7.087)--(7.002,7.087)--(7.038,7.087)--(7.073,7.087)--(7.107,7.087)%
  --(7.140,7.087)--(7.173,7.087)--(7.205,7.087)--(7.236,7.087)--(7.267,7.087)--(7.297,7.087)%
  --(7.327,7.087)--(7.356,7.087)--(7.384,7.087)--(7.412,7.087)--(7.440,7.087)--(7.467,7.087)%
  --(7.493,7.087)--(7.520,7.087)--(7.545,7.087)--(7.571,7.087)--(7.596,7.087)--(7.620,7.087)%
  --(7.644,7.087)--(7.668,7.087)--(7.691,7.087)--(7.715,7.087)--(7.737,7.087)--(7.760,7.087)%
  --(7.782,7.087)--(7.804,7.087)--(7.825,7.087)--(7.847,7.087)--(7.868,7.087)--(7.888,7.087)%
  --(7.909,7.087)--(7.929,7.087)--(7.949,7.087)--(7.968,7.087)--(7.988,7.087)--(8.007,7.087)%
  --(8.026,7.087)--(8.045,7.087)--(8.063,7.087)--(8.082,7.087)--(8.100,7.087)--(8.118,7.087)%
  --(8.135,7.087)--(8.153,7.087)--(8.170,7.087)--(8.187,7.087)--(8.204,7.087)--(8.221,7.087)%
  --(8.237,7.087)--(8.254,7.087)--(8.270,7.087)--(8.286,7.087)--(8.302,7.087)--(8.318,7.087)%
  --(8.333,7.087)--(8.349,7.087)--(8.364,7.087)--(8.379,7.087)--(8.394,7.087)--(8.409,7.087)%
  --(8.424,7.087)--(8.438,7.087)--(8.453,7.087)--(8.467,7.087)--(8.481,7.087)--(8.495,7.087)%
  --(8.509,7.087)--(8.523,7.087)--(8.537,7.087)--(8.551,7.087)--(8.564,7.087)--(8.577,7.087)%
  --(8.591,7.087)--(8.604,7.087)--(8.617,7.087)--(8.630,7.087)--(8.643,7.087)--(8.655,7.087)%
  --(8.668,7.087)--(8.680,7.087)--(8.693,7.087)--(8.705,7.087)--(8.717,7.087)--(8.729,7.087)%
  --(8.741,7.087)--(8.753,7.087)--(8.765,7.087)--(8.777,7.087)--(8.789,7.087)--(8.800,7.087)%
  --(8.812,7.087)--(8.823,7.518)--(8.835,7.518)--(8.846,7.518)--(8.857,7.518)--(8.868,7.518)%
  --(8.879,7.518)--(8.890,7.518)--(8.901,7.518)--(8.912,7.518)--(8.923,7.518)--(8.933,7.518)%
  --(8.944,7.518)--(8.954,7.518)--(8.965,7.518)--(8.975,7.518)--(8.986,7.518)--(8.996,7.518)%
  --(9.006,7.518)--(9.016,7.518)--(9.026,7.518)--(9.036,7.518)--(9.046,7.518)--(9.056,7.518)%
  --(9.066,7.518)--(9.075,7.518)--(9.085,7.518)--(9.095,7.518)--(9.104,7.518)--(9.114,7.518)%
  --(9.123,7.518)--(9.133,7.518)--(9.142,7.518)--(9.151,7.518)--(9.161,7.518)--(9.170,7.518)%
  --(9.179,7.518)--(9.188,7.518)--(9.197,7.518)--(9.206,7.518)--(9.215,7.518)--(9.224,7.518)%
  --(9.232,7.518)--(9.241,7.518)--(9.250,7.518)--(9.259,7.518)--(9.267,7.518)--(9.276,7.518)%
  --(9.284,7.518)--(9.293,7.518)--(9.301,7.518)--(9.310,7.518)--(9.318,7.518)--(9.326,7.518)%
  --(9.335,7.518)--(9.343,7.518)--(9.351,7.518)--(9.359,7.518)--(9.367,7.518)--(9.375,7.518)%
  --(9.383,7.518)--(9.391,7.518)--(9.399,7.518)--(9.407,7.518)--(9.415,7.518)--(9.423,7.518)%
  --(9.431,7.518)--(9.438,7.518)--(9.446,7.518)--(9.454,7.518)--(9.461,7.518)--(9.469,7.518)%
  --(9.476,7.518)--(9.484,7.518)--(9.491,7.518)--(9.499,7.518)--(9.506,7.518)--(9.514,7.518)%
  --(9.521,7.518)--(9.528,7.518)--(9.536,7.518)--(9.543,7.518)--(9.550,7.518)--(9.557,7.518)%
  --(9.564,7.518)--(9.572,7.518)--(9.579,7.518)--(9.586,7.518)--(9.593,7.518)--(9.600,7.518)%
  --(9.607,7.518)--(9.614,7.518)--(9.620,7.518)--(9.627,7.518)--(9.634,7.518)--(9.641,7.518)%
  --(9.648,7.518)--(9.655,7.518)--(9.661,7.518)--(9.668,7.518)--(9.675,7.518)--(9.681,7.518)%
  --(9.688,7.518)--(9.694,7.518)--(9.701,7.518)--(9.708,7.518)--(9.714,7.518)--(9.721,7.518)%
  --(9.727,7.518)--(9.733,7.518)--(9.740,7.518)--(9.746,7.518)--(9.752,7.518)--(9.759,7.518)%
  --(9.765,7.518)--(9.771,7.518)--(9.778,7.518)--(9.784,7.518)--(9.790,7.518)--(9.796,7.518)%
  --(9.802,7.518)--(9.808,7.518)--(9.815,7.518)--(9.821,7.518)--(9.827,7.518)--(9.833,7.518)%
  --(9.839,7.518)--(9.845,7.518)--(9.851,7.518)--(9.857,7.518)--(9.863,7.518)--(9.868,7.518)%
  --(9.874,7.518)--(9.880,7.518)--(9.886,7.518)--(9.892,7.518)--(9.898,7.518)--(9.903,7.518)%
  --(9.909,7.518)--(9.915,7.518)--(9.921,7.518)--(9.926,7.518)--(9.932,7.518)--(9.938,7.518)%
  --(9.943,7.518)--(9.949,7.518)--(9.954,7.518)--(9.960,7.518)--(9.965,7.518)--(9.971,7.518)%
  --(9.976,7.518)--(9.982,7.518)--(9.987,7.518)--(9.993,7.518)--(9.998,7.518)--(10.004,7.518)%
  --(10.009,7.518)--(10.015,7.518)--(10.020,7.518)--(10.025,7.518)--(10.031,7.518)--(10.036,7.518)%
  --(10.041,7.518)--(10.046,7.518)--(10.052,7.518)--(10.057,7.518)--(10.062,7.518)--(10.067,7.518)%
  --(10.072,7.518)--(10.078,7.518)--(10.083,7.518)--(10.088,7.518)--(10.093,7.518)--(10.098,7.518)%
  --(10.103,7.518)--(10.108,7.518)--(10.113,7.518)--(10.118,7.518)--(10.123,7.518)--(10.128,7.518)%
  --(10.133,7.518)--(10.138,7.518)--(10.143,7.518)--(10.148,7.518)--(10.153,7.518)--(10.158,7.518)%
  --(10.163,7.518)--(10.168,7.518)--(10.173,7.518)--(10.178,7.518)--(10.182,7.518)--(10.187,7.518)%
  --(10.192,7.518)--(10.197,7.518)--(10.202,7.518)--(10.206,7.518)--(10.211,7.518)--(10.216,7.518)%
  --(10.221,7.518)--(10.225,7.518)--(10.230,7.518)--(10.235,7.518)--(10.239,7.518)--(10.244,7.518)%
  --(10.249,7.518)--(10.253,7.518)--(10.258,7.518)--(10.262,7.518)--(10.267,7.518)--(10.272,7.518)%
  --(10.276,7.518)--(10.281,7.518)--(10.285,7.518)--(10.290,7.518)--(10.294,7.518)--(10.299,7.518)%
  --(10.303,7.518)--(10.308,7.518)--(10.312,7.518)--(10.316,7.518)--(10.321,7.518)--(10.325,7.518)%
  --(10.330,7.518)--(10.334,7.518)--(10.338,7.518)--(10.343,7.518)--(10.347,7.518)--(10.352,7.518)%
  --(10.356,7.518)--(10.360,7.518)--(10.365,7.518)--(10.369,7.518)--(10.373,7.518)--(10.377,7.518)%
  --(10.382,7.518)--(10.386,7.518)--(10.390,7.518)--(10.394,7.518)--(10.399,7.518)--(10.403,7.518)%
  --(10.407,7.518)--(10.411,7.518)--(10.415,7.518)--(10.419,7.518)--(10.424,7.518)--(10.428,7.518)%
  --(10.432,7.518)--(10.436,7.518)--(10.440,7.518)--(10.444,7.518)--(10.448,7.518)--(10.452,7.518)%
  --(10.456,7.518)--(10.460,7.518)--(10.465,7.518)--(10.469,7.518)--(10.473,7.518)--(10.477,7.518)%
  --(10.481,7.518)--(10.485,7.518)--(10.489,7.518)--(10.493,7.518)--(10.497,7.518)--(10.500,7.518)%
  --(10.504,7.518)--(10.508,7.518)--(10.512,7.518)--(10.516,7.518)--(10.520,7.518)--(10.524,7.518)%
  --(10.528,7.518)--(10.532,7.950)--(10.536,7.950)--(10.539,7.950)--(10.543,7.950)--(10.547,7.950)%
  --(10.551,7.950)--(10.555,7.950)--(10.559,7.950)--(10.562,7.950)--(10.566,7.950)--(10.570,7.950)%
  --(10.574,7.950)--(10.578,7.950)--(10.581,7.950)--(10.585,7.950)--(10.589,7.950)--(10.592,7.950)%
  --(10.596,7.950)--(10.600,7.950)--(10.604,7.950)--(10.607,7.950)--(10.611,7.950)--(10.615,7.950)%
  --(10.618,7.950)--(10.622,7.950)--(10.626,7.950)--(10.629,7.950)--(10.633,7.950)--(10.637,7.950)%
  --(10.640,7.950)--(10.644,7.950)--(10.647,7.950)--(10.651,7.950)--(10.655,7.950)--(10.658,7.950)%
  --(10.662,7.950)--(10.665,7.950)--(10.669,7.950)--(10.672,7.950)--(10.676,7.950)--(10.679,7.950)%
  --(10.683,7.950)--(10.686,7.950)--(10.690,7.950)--(10.693,7.950)--(10.697,7.950)--(10.700,7.950)%
  --(10.704,7.950)--(10.707,7.950)--(10.711,7.950)--(10.714,7.950)--(10.718,7.950)--(10.721,7.950)%
  --(10.725,7.950)--(10.728,7.950)--(10.731,7.950)--(10.735,7.950)--(10.738,7.950)--(10.742,7.950)%
  --(10.745,7.950)--(10.748,7.950)--(10.752,7.950)--(10.755,7.950)--(10.759,7.950)--(10.762,7.950)%
  --(10.765,7.950)--(10.769,7.950)--(10.772,7.950)--(10.775,7.950)--(10.779,7.950)--(10.782,7.950)%
  --(10.785,7.950)--(10.788,7.950)--(10.792,7.950)--(10.795,7.950)--(10.798,7.950)--(10.802,8.381)%
  --(10.805,8.381)--(10.808,8.381)--(10.811,8.381)--(10.815,8.381)--(10.818,8.381)--(10.821,8.381)%
  --(10.824,8.381)--(10.827,8.381)--(10.831,8.381)--(10.834,8.381)--(10.837,8.381)--(10.840,8.381)%
  --(10.843,8.381)--(10.847,8.381)--(10.850,8.381)--(10.853,8.381)--(10.856,8.381)--(10.859,8.381)%
  --(10.862,8.381)--(10.865,8.381)--(10.869,8.381)--(10.872,8.381)--(10.875,8.381)--(10.878,8.381)%
  --(10.881,8.381)--(10.884,8.381)--(10.887,8.381)--(10.890,8.381)--(10.893,8.381)--(10.897,8.381)%
  --(10.900,8.381)--(10.903,8.381)--(10.906,8.381)--(10.909,8.381)--(10.912,8.381)--(10.915,8.381)%
  --(10.918,8.381)--(10.921,8.381)--(10.924,8.381)--(10.927,8.381)--(10.930,8.381)--(10.933,8.381)%
  --(10.936,8.381)--(10.939,8.381)--(10.942,8.381)--(10.945,8.381)--(10.948,8.381)--(10.951,8.381)%
  --(10.954,8.381)--(10.957,8.381)--(10.960,8.381)--(10.963,8.381)--(10.966,8.381)--(10.969,8.381)%
  --(10.972,8.381)--(10.975,8.381)--(10.977,8.381)--(10.980,8.381)--(10.983,8.381)--(10.986,8.381)%
  --(10.989,8.381)--(10.992,8.381)--(10.995,8.381)--(10.998,8.381)--(11.001,8.381)--(11.003,8.381)%
  --(11.006,8.381)--(11.009,8.381)--(11.012,8.381)--(11.015,8.381)--(11.018,8.381)--(11.021,8.381)%
  --(11.023,8.381)--(11.026,8.381)--(11.029,8.381)--(11.032,8.381)--(11.035,8.381)--(11.038,8.381)%
  --(11.040,8.381)--(11.043,8.381)--(11.046,8.381)--(11.049,8.381)--(11.052,8.381)--(11.054,8.381)%
  --(11.057,8.381)--(11.060,8.381)--(11.063,8.381)--(11.065,8.381)--(11.068,8.381)--(11.071,8.381)%
  --(11.074,8.381)--(11.076,8.381)--(11.079,8.381)--(11.082,8.381)--(11.085,8.381)--(11.087,8.381)%
  --(11.090,8.381)--(11.093,8.381)--(11.096,8.381)--(11.098,8.381)--(11.101,8.381)--(11.104,8.381)%
  --(11.106,8.381)--(11.109,8.381)--(11.112,8.381)--(11.114,8.381)--(11.117,8.381)--(11.120,8.381)%
  --(11.122,8.381)--(11.125,8.381)--(11.128,8.381)--(11.130,8.381)--(11.133,8.381)--(11.136,8.381);
\gpsetdashtype{gp dt solid}
\draw[gp path] (1.012,2.773)--(2.109,3.636)--(2.751,4.930)--(3.207,4.930)--(3.560,4.930)%
  --(3.848,5.361)--(4.092,5.361)--(4.304,5.361)--(4.490,5.793)--(4.657,5.793)--(4.808,6.224)%
  --(4.946,6.224)--(5.072,6.224)--(5.190,6.224)--(5.299,6.224)--(5.401,6.224)--(5.497,6.224)%
  --(5.587,6.224)--(5.673,6.224)--(5.754,6.224)--(5.831,6.224)--(5.905,6.224)--(5.975,6.224)%
  --(6.043,6.224)--(6.107,6.224)--(6.170,6.224)--(6.229,6.224)--(6.287,6.224)--(6.342,6.224)%
  --(6.396,6.224)--(6.448,6.224)--(6.498,6.224)--(6.547,6.224)--(6.594,6.224)--(6.640,6.224)%
  --(6.685,6.224)--(6.728,6.224)--(6.770,6.224)--(6.811,6.224)--(6.852,6.224)--(6.891,6.224)%
  --(6.929,6.224)--(6.966,6.224)--(7.002,6.224)--(7.038,6.224)--(7.073,6.224)--(7.107,6.224)%
  --(7.140,6.655)--(7.173,6.655)--(7.205,6.655)--(7.236,6.655)--(7.267,6.655)--(7.297,6.655)%
  --(7.327,6.655)--(7.356,6.655)--(7.384,6.655)--(7.412,6.655)--(7.440,6.655)--(7.467,6.655)%
  --(7.493,6.655)--(7.520,6.655)--(7.545,6.655)--(7.571,6.655)--(7.596,6.655)--(7.620,6.655)%
  --(7.644,6.655)--(7.668,6.655)--(7.691,6.655)--(7.715,6.655)--(7.737,6.655)--(7.760,6.655)%
  --(7.782,6.655)--(7.804,6.655)--(7.825,6.655)--(7.847,6.655)--(7.868,6.655)--(7.888,6.655)%
  --(7.909,6.655)--(7.929,6.655)--(7.949,6.655)--(7.968,6.655)--(7.988,6.655)--(8.007,6.655)%
  --(8.026,6.655)--(8.045,6.655)--(8.063,6.655)--(8.082,6.655)--(8.100,6.655)--(8.118,6.655)%
  --(8.135,6.655)--(8.153,6.655)--(8.170,6.655)--(8.187,6.655)--(8.204,6.655)--(8.221,6.655)%
  --(8.237,6.655)--(8.254,6.655)--(8.270,6.655)--(8.286,6.655)--(8.302,6.655)--(8.318,6.655)%
  --(8.333,6.655)--(8.349,6.655)--(8.364,6.655)--(8.379,6.655)--(8.394,6.655)--(8.409,6.655)%
  --(8.424,6.655)--(8.438,6.655)--(8.453,6.655)--(8.467,6.655)--(8.481,6.655)--(8.495,6.655)%
  --(8.509,6.655)--(8.523,6.655)--(8.537,6.655)--(8.551,6.655)--(8.564,6.655)--(8.577,6.655)%
  --(8.591,6.655)--(8.604,6.655)--(8.617,6.655)--(8.630,6.655)--(8.643,6.655)--(8.655,6.655)%
  --(8.668,6.655)--(8.680,6.655)--(8.693,6.655)--(8.705,6.655)--(8.717,6.655)--(8.729,6.655)%
  --(8.741,6.655)--(8.753,6.655)--(8.765,6.655)--(8.777,6.655)--(8.789,6.655)--(8.800,6.655)%
  --(8.812,6.655)--(8.823,6.655)--(8.835,6.655)--(8.846,6.655)--(8.857,6.655)--(8.868,6.655)%
  --(8.879,6.655)--(8.890,6.655)--(8.901,7.087)--(8.912,7.087)--(8.923,7.087)--(8.933,7.087)%
  --(8.944,7.087)--(8.954,7.087)--(8.965,7.087)--(8.975,7.087)--(8.986,7.087)--(8.996,7.087)%
  --(9.006,7.087)--(9.016,7.087)--(9.026,7.087)--(9.036,7.087)--(9.046,7.087)--(9.056,7.087)%
  --(9.066,7.087)--(9.075,7.087)--(9.085,7.087)--(9.095,7.087)--(9.104,7.087)--(9.114,7.087)%
  --(9.123,7.087)--(9.133,7.087)--(9.142,7.087)--(9.151,7.087)--(9.161,7.087)--(9.170,7.087)%
  --(9.179,7.087)--(9.188,7.087)--(9.197,7.087)--(9.206,7.087)--(9.215,7.087)--(9.224,7.087)%
  --(9.232,7.087)--(9.241,7.087)--(9.250,7.087)--(9.259,7.087)--(9.267,7.087)--(9.276,7.087)%
  --(9.284,7.087)--(9.293,7.087)--(9.301,7.087)--(9.310,7.087)--(9.318,7.087)--(9.326,7.087)%
  --(9.335,7.087)--(9.343,7.087)--(9.351,7.087)--(9.359,7.087)--(9.367,7.087)--(9.375,7.087)%
  --(9.383,7.087)--(9.391,7.087)--(9.399,7.087)--(9.407,7.087)--(9.415,7.087)--(9.423,7.087)%
  --(9.431,7.087)--(9.438,7.087)--(9.446,7.087)--(9.454,7.087)--(9.461,7.087)--(9.469,7.087)%
  --(9.476,7.087)--(9.484,7.087)--(9.491,7.087)--(9.499,7.087)--(9.506,7.087)--(9.514,7.087)%
  --(9.521,7.087)--(9.528,7.087)--(9.536,7.087)--(9.543,7.087)--(9.550,7.087)--(9.557,7.087)%
  --(9.564,7.087)--(9.572,7.087)--(9.579,7.087)--(9.586,7.087)--(9.593,7.087)--(9.600,7.087)%
  --(9.607,7.087)--(9.614,7.087)--(9.620,7.087)--(9.627,7.087)--(9.634,7.087)--(9.641,7.087)%
  --(9.648,7.087)--(9.655,7.087)--(9.661,7.087)--(9.668,7.087)--(9.675,7.087)--(9.681,7.087)%
  --(9.688,7.087)--(9.694,7.087)--(9.701,7.087)--(9.708,7.087)--(9.714,7.087)--(9.721,7.087)%
  --(9.727,7.087)--(9.733,7.087)--(9.740,7.087)--(9.746,7.087)--(9.752,7.087)--(9.759,7.087)%
  --(9.765,7.087)--(9.771,7.087)--(9.778,7.087)--(9.784,7.087)--(9.790,7.087)--(9.796,7.087)%
  --(9.802,7.087)--(9.808,7.087)--(9.815,7.087)--(9.821,7.087)--(9.827,7.087)--(9.833,7.087)%
  --(9.839,7.087)--(9.845,7.087)--(9.851,7.087)--(9.857,7.087)--(9.863,7.087)--(9.868,7.087)%
  --(9.874,7.087)--(9.880,7.087)--(9.886,7.087)--(9.892,7.087)--(9.898,7.087)--(9.903,7.087)%
  --(9.909,7.087)--(9.915,7.087)--(9.921,7.087)--(9.926,7.087)--(9.932,7.087)--(9.938,7.087)%
  --(9.943,7.087)--(9.949,7.087)--(9.954,7.087)--(9.960,7.087)--(9.965,7.087)--(9.971,7.087)%
  --(9.976,7.087)--(9.982,7.087)--(9.987,7.087)--(9.993,7.087)--(9.998,7.087)--(10.004,7.087)%
  --(10.009,7.087)--(10.015,7.087)--(10.020,7.087)--(10.025,7.087)--(10.031,7.087)--(10.036,7.087)%
  --(10.041,7.087)--(10.046,7.087)--(10.052,7.518)--(10.057,7.518)--(10.062,7.518)--(10.067,7.518)%
  --(10.072,7.518)--(10.078,7.518)--(10.083,7.518)--(10.088,7.518)--(10.093,7.518)--(10.098,7.518)%
  --(10.103,7.518)--(10.108,7.518)--(10.113,7.518)--(10.118,7.518)--(10.123,7.518)--(10.128,7.518)%
  --(10.133,7.518)--(10.138,7.518)--(10.143,7.518)--(10.148,7.518)--(10.153,7.518)--(10.158,7.518)%
  --(10.163,7.518)--(10.168,7.518)--(10.173,7.518)--(10.178,7.518)--(10.182,7.518)--(10.187,7.518)%
  --(10.192,7.518)--(10.197,7.518)--(10.202,7.518)--(10.206,7.518)--(10.211,7.518)--(10.216,7.518)%
  --(10.221,7.518)--(10.225,7.518)--(10.230,7.518)--(10.235,7.518)--(10.239,7.518)--(10.244,7.518)%
  --(10.249,7.518)--(10.253,7.518)--(10.258,7.518)--(10.262,7.518)--(10.267,7.518)--(10.272,7.518)%
  --(10.276,7.518)--(10.281,7.518)--(10.285,7.950)--(10.290,7.950)--(10.294,7.950)--(10.299,7.950)%
  --(10.303,7.950)--(10.308,7.950)--(10.312,7.950)--(10.316,7.950)--(10.321,7.950)--(10.325,7.950)%
  --(10.330,7.950)--(10.334,7.950)--(10.338,7.950)--(10.343,7.950)--(10.347,7.950)--(10.352,7.950)%
  --(10.356,7.950)--(10.360,7.950)--(10.365,7.950)--(10.369,7.950)--(10.373,7.950)--(10.377,7.950)%
  --(10.382,7.950)--(10.386,7.950)--(10.390,7.950)--(10.394,7.950)--(10.399,7.950)--(10.403,7.950)%
  --(10.407,7.950)--(10.411,7.950)--(10.415,7.950)--(10.419,7.950)--(10.424,7.950)--(10.428,7.950)%
  --(10.432,7.950)--(10.436,7.950)--(10.440,7.950)--(10.444,7.950)--(10.448,7.950)--(10.452,7.950)%
  --(10.456,7.950)--(10.460,7.950)--(10.465,7.950)--(10.469,7.950)--(10.473,7.950)--(10.477,7.950)%
  --(10.481,7.950)--(10.485,7.950)--(10.489,7.950)--(10.493,7.950)--(10.497,7.950)--(10.500,7.950)%
  --(10.504,7.950)--(10.508,7.950)--(10.512,7.950)--(10.516,7.950)--(10.520,7.950)--(10.524,7.950)%
  --(10.528,7.950)--(10.532,7.950)--(10.536,7.950)--(10.539,7.950)--(10.543,7.950)--(10.547,7.950)%
  --(10.551,7.950)--(10.555,7.950)--(10.559,7.950)--(10.562,7.950)--(10.566,7.950)--(10.570,7.950)%
  --(10.574,7.950)--(10.578,7.950)--(10.581,7.950)--(10.585,7.950)--(10.589,7.950)--(10.592,7.950)%
  --(10.596,7.950)--(10.600,7.950)--(10.604,7.950)--(10.607,7.950)--(10.611,7.950)--(10.615,7.950)%
  --(10.618,7.950)--(10.622,7.950)--(10.626,7.950)--(10.629,7.950)--(10.633,7.950)--(10.637,7.950)%
  --(10.640,7.950)--(10.644,7.950)--(10.647,7.950)--(10.651,7.950)--(10.655,7.950)--(10.658,7.950)%
  --(10.662,7.950)--(10.665,7.950)--(10.669,7.950)--(10.672,7.950)--(10.676,7.950)--(10.679,7.950)%
  --(10.683,7.950)--(10.686,7.950)--(10.690,7.950)--(10.693,7.950)--(10.697,7.950)--(10.700,7.950)%
  --(10.704,7.950)--(10.707,7.950)--(10.711,7.950)--(10.714,7.950)--(10.718,7.950)--(10.721,7.950)%
  --(10.725,7.950)--(10.728,7.950)--(10.731,7.950)--(10.735,7.950)--(10.738,7.950)--(10.742,7.950)%
  --(10.745,7.950)--(10.748,7.950)--(10.752,7.950)--(10.755,7.950)--(10.759,7.950)--(10.762,7.950)%
  --(10.765,7.950)--(10.769,7.950)--(10.772,7.950)--(10.775,7.950)--(10.779,7.950)--(10.782,7.950)%
  --(10.785,7.950)--(10.788,7.950)--(10.792,7.950)--(10.795,7.950)--(10.798,7.950)--(10.802,7.950)%
  --(10.805,7.950)--(10.808,7.950)--(10.811,7.950)--(10.815,7.950)--(10.818,7.950)--(10.821,7.950)%
  --(10.824,7.950)--(10.827,7.950)--(10.831,7.950)--(10.834,7.950)--(10.837,7.950)--(10.840,7.950)%
  --(10.843,7.950)--(10.847,7.950)--(10.850,7.950)--(10.853,7.950)--(10.856,7.950)--(10.859,7.950)%
  --(10.862,7.950)--(10.865,7.950)--(10.869,7.950)--(10.872,7.950)--(10.875,7.950)--(10.878,7.950)%
  --(10.881,7.950)--(10.884,7.950)--(10.887,7.950)--(10.890,7.950)--(10.893,7.950)--(10.897,7.950)%
  --(10.900,7.950)--(10.903,7.950)--(10.906,7.950)--(10.909,7.950)--(10.912,7.950)--(10.915,7.950)%
  --(10.918,7.950)--(10.921,7.950)--(10.924,7.950)--(10.927,7.950)--(10.930,7.950)--(10.933,7.950)%
  --(10.936,7.950)--(10.939,7.950)--(10.942,7.950)--(10.945,7.950)--(10.948,7.950)--(10.951,7.950)%
  --(10.954,7.950)--(10.957,7.950)--(10.960,7.950)--(10.963,7.950)--(10.966,7.950)--(10.969,7.950)%
  --(10.972,7.950)--(10.975,7.950)--(10.977,7.950)--(10.980,7.950)--(10.983,7.950)--(10.986,7.950)%
  --(10.989,7.950)--(10.992,7.950)--(10.995,7.950)--(10.998,7.950)--(11.001,7.950)--(11.003,7.950)%
  --(11.006,7.950)--(11.009,7.950)--(11.012,7.950)--(11.015,7.950)--(11.018,7.950)--(11.021,7.950)%
  --(11.023,7.950)--(11.026,7.950)--(11.029,7.950)--(11.032,7.950)--(11.035,7.950)--(11.038,7.950)%
  --(11.040,7.950)--(11.043,7.950)--(11.046,7.950)--(11.049,7.950)--(11.052,7.950)--(11.054,7.950)%
  --(11.057,7.950)--(11.060,7.950)--(11.063,7.950)--(11.065,7.950)--(11.068,7.950)--(11.071,7.950)%
  --(11.074,7.950)--(11.076,7.950)--(11.079,7.950)--(11.082,7.950)--(11.085,7.950)--(11.087,7.950)%
  --(11.090,7.950)--(11.093,7.950)--(11.096,7.950)--(11.098,7.950)--(11.101,7.950)--(11.104,7.950)%
  --(11.106,7.950)--(11.109,7.950)--(11.112,7.950)--(11.114,7.950)--(11.117,7.950)--(11.120,7.950)%
  --(11.122,7.950)--(11.125,7.950)--(11.128,7.950)--(11.130,7.950)--(11.133,7.950)--(11.136,7.950);
\gpcolor{color=gp lt color border}
\gpsetlinewidth{1.00}
\draw[gp path] (1.012,8.381)--(1.012,0.616)--(11.947,0.616)--(11.947,8.381)--cycle;
\pretocmd{\refModel}{\large}{}{}
\node [nothing] at (9,1.2) {\refModel{obj-fun-naive-no-b}};
\node [nothing] at (10.1,2.6) {\refModel{obj-fun-naive-lb-ub}};
\node [nothing] at (3.6,4.5) (c) {\refModel{obj-fun-refined-no-b}};
\draw [->, shorten <=3pt, line width=.8pt] (c) -- (3.25,5.4);
\node [nothing] at (5.05,5.35) {\refModel{obj-fun-refined-lb}};
\node [nothing] at (4,6.3) {\refModel{obj-fun-refined-ub}};
\node [nothing] at (9.1,6.4) {\refModel{obj-fun-refined-lb-ub}};
%% coordinates of the plot area
\gpdefrectangularnode{gp plot 1}{\pgfpoint{1.012cm}{0.616cm}}{\pgfpoint{11.947cm}{8.381cm}}
\end{tikzpicture}
%% gnuplot variables
%
    }%
  }

  \caption[%
            Plot for evaluating the different objective functions' impact on
            finding optimal solutions%
          ]%
          {%
            Percentage of optimal solutions found over time for six constraint
            models:
            %
            \begin{modelList}
              \item one based on the naive implementation of the \gls{objective
                function} without cost bounds
              \item another naive model but with lower and upper bound
              \item one based on the refined implementation without bounds
              \item another refined model but with lower bound
              \item another refined model but with upper bound
              \item another refined model but with both bounds
            \end{modelList}%
          }
  \labelFigure{obj-fun-refined-vs-naive-opt-proofs-over-time-plot}
\end{figure}
%
The solving times range from
\printMinSolvingTime{
  \ObjFunRefinedVsNaiveOptProofsOverTimeNaiveNoBoundsSolvingTimeAvgMin,
  \ObjFunRefinedVsNaiveOptProofsOverTimeNaiveWLbUbSolvingTimeAvgMin,
  \ObjFunRefinedVsNaiveOptProofsOverTimeRefinedNoBoundsSolvingTimeAvgMin,
  \ObjFunRefinedVsNaiveOptProofsOverTimeRefinedWLbSolvingTimeAvgMin,
  \ObjFunRefinedVsNaiveOptProofsOverTimeRefinedWUbSolvingTimeAvgMin,
  \ObjFunRefinedVsNaiveOptProofsOverTimeRefinedWLbUbSolvingTimeAvgMin
} to
\printMaxSolvingTime{
  \ObjFunRefinedVsNaiveOptProofsOverTimeNaiveNoBoundsSolvingTimeAvgMax,
  \ObjFunRefinedVsNaiveOptProofsOverTimeNaiveWLbUbSolvingTimeAvgMax,
  \ObjFunRefinedVsNaiveOptProofsOverTimeRefinedNoBoundsSolvingTimeAvgMax,
  \ObjFunRefinedVsNaiveOptProofsOverTimeRefinedWLbSolvingTimeAvgMax,
  \ObjFunRefinedVsNaiveOptProofsOverTimeRefinedWUbSolvingTimeAvgMax,
  \ObjFunRefinedVsNaiveOptProofsOverTimeRefinedWLbUbSolvingTimeAvgMax
} with a \gls{CV} of
\numMaxOf{
  \ObjFunRefinedVsNaiveOptProofsOverTimeNaiveNoBoundsSolvingTimeCvMax,
  \ObjFunRefinedVsNaiveOptProofsOverTimeNaiveWLbUbSolvingTimeCvMax,
  \ObjFunRefinedVsNaiveOptProofsOverTimeRefinedNoBoundsSolvingTimeCvMax,
  \ObjFunRefinedVsNaiveOptProofsOverTimeRefinedWLbSolvingTimeCvMax,
  \ObjFunRefinedVsNaiveOptProofsOverTimeRefinedWUbSolvingTimeCvMax,
  \ObjFunRefinedVsNaiveOptProofsOverTimeRefinedWLbUbSolvingTimeCvMax
}.

We see that \glsplshort{constraint model}~\refModel{obj-fun-refined-no-b},
\refModel{obj-fun-refined-lb}, \refModel{obj-fun-refined-ub},
and~\refModel{obj-fun-refined-lb-ub} clearly outperform \glsplshort{constraint
  model}~\refModel{obj-fun-naive-no-b} and~\refModel{obj-fun-naive-lb-ub}.
%
We see also that applying an upper cost bound has a positive effect for both
\glspl{objective function}, although the benefit is greater for the refined
\gls{objective function}.
%
This gain is due to the fact that, for some \glspl{function}, the \gls{solution}
computed by \gls{LLVM} is already optimal with respect to the
\glsshort{constraint model}.
%
Consequently, the \glsshort{constraint solver} need only prove that there exists
no better \gls{solution} instead of exploring the entire \gls{search space}.
%
Hence, in terms of solving time the refined \gls{objective function} coupled
with an upper cost bound is a better design choice over naive \gls{objective
  function}.
%
Moreover, this decision is crucial for scalability.

Applying a lower cost bound, however, does not appear to be equally beneficial.
%
In fact, using a lower bound causes \glsplshort{constraint
  model}~\refModel{obj-fun-refined-lb} and~\refModel{obj-fun-refined-lb-ub} to
fail to find the optimal \gls{solution} for two \glspl{function} whereas
\glsplshort{constraint model}~\refModel{obj-fun-refined-no-b}
and~\refModel{obj-fun-refined-ub} manages to find the optimal \gls{solution} for
all \glspl{function} but one.
%
A possible explanation is that the lower bound computed by the relaxed
\gls{constraint model} is too weak to be lead to any \gls{propagation} and
instead only interferes with \gls{Chuffed}'s \gls{LCG} engine (see
\refChapter{constraint-programming} on
\refPageOfSection{cp-lazy-clause-generation} for a description of \gls{LCG}).


\paragraph{Impact on Code Quality}

We evaluate the impact on code quality by comparing the cost of the best
\gls{solution} found within the time limit by \glsplshort{constraint
  model}~\refModel{obj-fun-naive-lb-ub} (naive \glsshort{constraint model} with
bounds) and~\refModel{obj-fun-refined-lb-ub} (refined \glsshort{constraint
  model} with bounds).
%
For the same reason as in the impact-on-solving-time experiment, we expect the
\glspl{solution} produced by \glsshort{constraint
  model}~\refModel{obj-fun-refined-lb-ub} to be of significantly better quality
compared with those produced by \glsshort{constraint
  model}~\refModel{obj-fun-naive-lb-ub}.

\RefFigure{obj-fun-refined-vs-naive-cycles-speedup-plot} shows the normalized
\gls{solution} costs for the two \glspl{constraint model} described above in the
third experiment, with \glsshort{constraint
  model}~\refModel{obj-fun-naive-lb-ub} as \gls{baseline} and
\glsshort{constraint model}~\refModel{obj-fun-refined-ub} as \gls{subject}.
%
\begin{figure}
  \centering%
  \maxsizebox{\textwidth}{!}{%
    \trimBarchartPlot{%
      \begin{tikzpicture}[gnuplot]
%% generated with GNUPLOT 5.0p4 (Lua 5.2; terminal rev. 99, script rev. 100)
%% lör  3 feb 2018 16:44:03
\path (0.000,0.000) rectangle (12.500,8.750);
\gpcolor{rgb color={0.753,0.753,0.753}}
\gpsetlinetype{gp lt axes}
\gpsetdashtype{gp dt axes}
\gpsetlinewidth{0.50}
\draw[gp path] (1.380,1.779)--(36.945,1.779);
\gpcolor{color=gp lt color border}
\node[gp node right] at (1.380,1.779) {\plotZCNormTics{0}};
\gpcolor{rgb color={0.753,0.753,0.753}}
\draw[gp path] (1.380,3.099)--(36.945,3.099);
\gpcolor{color=gp lt color border}
\node[gp node right] at (1.380,3.099) {\plotZCNormTics{0.05}};
\gpcolor{rgb color={0.753,0.753,0.753}}
\draw[gp path] (1.380,4.420)--(36.945,4.420);
\gpcolor{color=gp lt color border}
\node[gp node right] at (1.380,4.420) {\plotZCNormTics{0.1}};
\gpcolor{rgb color={0.753,0.753,0.753}}
\draw[gp path] (1.380,5.740)--(36.945,5.740);
\gpcolor{color=gp lt color border}
\node[gp node right] at (1.380,5.740) {\plotZCNormTics{0.15}};
\gpcolor{rgb color={0.753,0.753,0.753}}
\draw[gp path] (1.380,7.061)--(36.945,7.061);
\gpcolor{color=gp lt color border}
\node[gp node right] at (1.380,7.061) {\plotZCNormTics{0.2}};
\gpcolor{rgb color={0.753,0.753,0.753}}
\draw[gp path] (1.380,8.381)--(36.945,8.381);
\gpcolor{color=gp lt color border}
\node[gp node right] at (1.380,8.381) {\plotZCNormTics{0.25}};
\node[gp node left,rotate=-30] at (3.110,1.534) {\functionName{alloc_name_is_s.}};
\node[gp node left,rotate=-30] at (4.803,1.534) {\functionName{alloc_save_spac.}};
\node[gp node left,rotate=-30] at (6.497,1.534) {\functionName{build_ycc_rgb_t.}};
\node[gp node left,rotate=-30] at (8.190,1.534) {\functionName{checksum}};
\node[gp node left,rotate=-30] at (9.884,1.534) {\functionName{debug_dump_byte.}};
\node[gp node left,rotate=-30] at (11.577,1.534) {\functionName{gl_EnableClient.}};
\node[gp node left,rotate=-30] at (13.271,1.534) {\functionName{gl_init_lists}};
\node[gp node left,rotate=-30] at (14.965,1.534) {\functionName{gl_save_Color4u.}};
\node[gp node left,rotate=-30] at (16.658,1.534) {\functionName{gl_save_EvalPoi.}};
\node[gp node left,rotate=-30] at (18.352,1.534) {\functionName{gl_save_PushMat.}};
\node[gp node left,rotate=-30] at (20.045,1.534) {\functionName{gl_swap4}};
\node[gp node left,rotate=-30] at (21.739,1.534) {\functionName{gl_TexImage3DEX.}};
\node[gp node left,rotate=-30] at (23.432,1.534) {\functionName{gp_enumerate_fi.}};
\node[gp node left,rotate=-30] at (25.126,1.534) {\functionName{gpk_open}};
\node[gp node left,rotate=-30] at (26.820,1.534) {\functionName{gs_interp_init}};
\node[gp node left,rotate=-30] at (28.513,1.534) {\functionName{jinit_forward_d.}};
\node[gp node left,rotate=-30] at (30.207,1.534) {\functionName{jpeg_read_heade.}};
\node[gp node left,rotate=-30] at (31.900,1.534) {\functionName{trueRandAccum}};
\node[gp node left,rotate=-30] at (33.594,1.534) {\functionName{write_file_trai.}};
\node[gp node left,rotate=-30] at (35.287,1.534) {\functionName{zero}};
\gpsetlinetype{gp lt border}
\gpsetdashtype{gp dt solid}
\gpsetlinewidth{1.00}
\draw[gp path] (1.380,8.381)--(1.380,1.779)--(36.945,1.779)--(36.945,8.381)--cycle;
\gpcolor{rgb color={0.000,0.000,0.000}}
\draw[gp path] (1.380,1.779)--(1.739,1.779)--(2.098,1.779)--(2.458,1.779)--(2.817,1.779)%
  --(3.176,1.779)--(3.535,1.779)--(3.895,1.779)--(4.254,1.779)--(4.613,1.779)--(4.972,1.779)%
  --(5.332,1.779)--(5.691,1.779)--(6.050,1.779)--(6.409,1.779)--(6.769,1.779)--(7.128,1.779)%
  --(7.487,1.779)--(7.846,1.779)--(8.206,1.779)--(8.565,1.779)--(8.924,1.779)--(9.283,1.779)%
  --(9.643,1.779)--(10.002,1.779)--(10.361,1.779)--(10.720,1.779)--(11.080,1.779)--(11.439,1.779)%
  --(11.798,1.779)--(12.157,1.779)--(12.517,1.779)--(12.876,1.779)--(13.235,1.779)--(13.594,1.779)%
  --(13.953,1.779)--(14.313,1.779)--(14.672,1.779)--(15.031,1.779)--(15.390,1.779)--(15.750,1.779)%
  --(16.109,1.779)--(16.468,1.779)--(16.827,1.779)--(17.187,1.779)--(17.546,1.779)--(17.905,1.779)%
  --(18.264,1.779)--(18.624,1.779)--(18.983,1.779)--(19.342,1.779)--(19.701,1.779)--(20.061,1.779)%
  --(20.420,1.779)--(20.779,1.779)--(21.138,1.779)--(21.498,1.779)--(21.857,1.779)--(22.216,1.779)%
  --(22.575,1.779)--(22.935,1.779)--(23.294,1.779)--(23.653,1.779)--(24.012,1.779)--(24.372,1.779)%
  --(24.731,1.779)--(25.090,1.779)--(25.449,1.779)--(25.808,1.779)--(26.168,1.779)--(26.527,1.779)%
  --(26.886,1.779)--(27.245,1.779)--(27.605,1.779)--(27.964,1.779)--(28.323,1.779)--(28.682,1.779)%
  --(29.042,1.779)--(29.401,1.779)--(29.760,1.779)--(30.119,1.779)--(30.479,1.779)--(30.838,1.779)%
  --(31.197,1.779)--(31.556,1.779)--(31.916,1.779)--(32.275,1.779)--(32.634,1.779)--(32.993,1.779)%
  --(33.353,1.779)--(33.712,1.779)--(34.071,1.779)--(34.430,1.779)--(34.790,1.779)--(35.149,1.779)%
  --(35.508,1.779)--(35.867,1.779)--(36.227,1.779)--(36.586,1.779)--(36.945,1.779);
\gpfill{rgb color={0.333,0.333,0.333}} (25.019,1.779)--(25.726,1.779)--(25.726,7.079)--(25.019,7.079)--cycle;
\gpcolor{color=gp lt color border}
\draw[gp path] (25.019,1.779)--(25.019,7.078)--(25.725,7.078)--(25.725,1.779)--cycle;
\gpfill{rgb color={0.333,0.333,0.333}} (31.794,1.779)--(32.500,1.779)--(32.500,2.659)--(31.794,2.659)--cycle;
\draw[gp path] (31.794,1.779)--(31.794,2.658)--(32.499,2.658)--(32.499,1.779)--cycle;
\gpfill{rgb color={0.333,0.333,0.333}} (35.181,1.779)--(35.888,1.779)--(35.888,1.843)--(35.181,1.843)--cycle;
\draw[gp path] (35.181,1.779)--(35.181,1.842)--(35.887,1.842)--(35.887,1.779)--cycle;
\node[gp node center] at (3.350,1.963) {\plotBarNormValueNoSolution};
\node[gp node center] at (5.043,1.963) {\plotBarNormValueNoBaselineSolution};
\node[gp node center] at (6.737,1.963) {\plotBarNormValueNoSolution};
\node[gp node center] at (8.430,1.963) {\plotBarNormValue{-0.000000}};
\node[gp node center] at (10.124,1.963) {\plotBarNormValue{-0.000000}};
\node[gp node center] at (11.817,1.963) {\plotBarNormValueNoSolution};
\node[gp node center] at (13.511,1.963) {\plotBarNormValueNoSolution};
\node[gp node center] at (15.205,1.963) {\plotBarNormValueNoBaselineSolution};
\node[gp node center] at (16.898,1.963) {\plotBarNormValueNoSolution};
\node[gp node center] at (18.592,1.963) {\plotBarNormValueNoSolution};
\node[gp node center] at (20.285,1.963) {\plotBarNormValue{-0.000000}};
\node[gp node center] at (21.979,1.963) {\plotBarNormValueNoBaselineSolution};
\node[gp node center] at (23.672,1.963) {\plotBarNormValueNoSolution};
\node[gp node center] at (25.366,7.262) {\plotBarNormValue{0.200660}};
\node[gp node center] at (27.060,1.963) {\plotBarNormValueNoBaselineSolution};
\node[gp node center] at (28.753,1.963) {\plotBarNormValueNoSolution};
\node[gp node center] at (30.447,1.963) {\plotBarNormValue{-0.000000}};
\node[gp node center] at (32.140,2.842) {\plotBarNormValue{0.033270}};
\node[gp node center] at (33.834,1.963) {\plotBarNormValueNoSolution};
\node[gp node center] at (35.527,2.026) {\plotBarNormValue{0.002381}};
\draw[gp path] (1.380,8.381)--(1.380,1.779)--(36.945,1.779)--(36.945,8.381)--cycle;
%% coordinates of the plot area
\gpdefrectangularnode{gp plot 1}{\pgfpoint{1.380cm}{1.779cm}}{\pgfpoint{36.945cm}{8.381cm}}
\end{tikzpicture}
%% gnuplot variables
%
    }%
  }

  \caption[%
            Plot for evaluating the different objective functions' impact on
            code quality%
          ]%
          {%
            Normalized best solution costs for two constraint models:
            %
            \begin{inlinelist}[itemjoin={, }, itemjoin*={, and}]
              \item one implementing the naive objective function (baseline)
              \item one implementing the refined objective function (subject)
            \end{inlinelist}.
            %
            GMI:~\printGMI{%
              \ObjFunRefinedVsNaiveCyclesSpeedupRefinedVsNaiveWLbUbCyclesRegularSpeedupGmean%
            },
            CI:~\printGMICI[round-precision=4]{%
              \ObjFunRefinedVsNaiveCyclesSpeedupRefinedVsNaiveWLbUbCyclesRegularSpeedupCiMin%
            }{%
              \ObjFunRefinedVsNaiveCyclesSpeedupRefinedVsNaiveWLbUbCyclesRegularSpeedupCiMax%
            }.
            %
            Both models use lower and upper cost bounds.
            %
            \Glspl{function} marked with \barNormValueNoBaselineSolution{} are
            those for which the naive objective function fails to produce any
            solution, and \glspl{function} marked with \barNormValueNoSolution{}
            are those where the solution produced by \gls{LLVM} is already
            optimal \wrt the model%
          }
  \labelFigure{obj-fun-refined-vs-naive-cycles-speedup-plot}
\end{figure}
%
The costs range from
\printMinCycles{
  \ObjFunRefinedVsNaiveCyclesSpeedupNaiveWLbUbCyclesAvgMin,
  \ObjFunRefinedVsNaiveCyclesSpeedupRefinedWLbUbCyclesAvgMin
} to
\printMaxCycles{
  \ObjFunRefinedVsNaiveCyclesSpeedupNaiveWLbUbCyclesAvgMax,
  \ObjFunRefinedVsNaiveCyclesSpeedupRefinedWLbUbCyclesAvgMax
} with a \gls{CV} of
\numMaxOf{
  \ObjFunRefinedVsNaiveCyclesSpeedupNaiveWLbUbCyclesCvMax,
  \ObjFunRefinedVsNaiveCyclesSpeedupRefinedWLbUbCyclesCvMax
}.
%
The \gls{GMI} is \printGMI{%
  \ObjFunRefinedVsNaiveCyclesSpeedupRefinedVsNaiveWLbUbCyclesRegularSpeedupGmean%
} with \gls{CI}~\printGMICI[round-precision=4]{%
  \ObjFunRefinedVsNaiveCyclesSpeedupRefinedVsNaiveWLbUbCyclesRegularSpeedupCiMin%
}{%
  \ObjFunRefinedVsNaiveCyclesSpeedupRefinedVsNaiveWLbUbCyclesRegularSpeedupCiMax%
}.

We see clearly that the \glsplshort{constraint model} based on the refined
\gls{objective function} yield better code quality than those based on the naive
\gls{objective function}.
%
For one \gls{function} (\cCode*{gpk\_open}), the code quality is improved by
\printZCNorm{%
  \ObjFunRefinedVsNaiveCyclesSpeedupRefinedVsNaiveWLbUbCyclesZeroCenteredSpeedupMax%
}, and for no \gls{function} is code quality degraded.
%
Hence, in terms of code quality the refined \gls{objective function} coupled
with an upper cost bound is a better design choice over naive \gls{objective
  function}.


\paragraph{Conclusions}

From the results for these experiments, we conclude:
%
\begin{enumerate*}[label=(\roman*), itemjoin={;\ }, itemjoin*={; and\ }]
  \item that the \gls{divide-then-multiply method} is superior to the
    \gls{multiply-then-divide method}
  \item that the refined implementation of the \gls{objective function} is
    superior to the naive implementation
  \item that using the refined \gls{objective function} together with an upper
    cost bound is crucial for scalability
\end{enumerate*}.


\subsection{Implied Constraints}
\labelSection{st-exp-evaluation-implied-constraints}

\NewDocumentCommand{\insertSpeedupPlot}{mm}{%
  \maxsizebox{#2\textwidth}{!}{%
    \trimBarchartPlot{\input{#1}}%
  }%
}

\NewDocumentCommand{\insertSolvTechSpeedupPlot}{O{disable}mm}{%
  \insertSpeedupPlot{%
    \expDir/solv-tech-#1-#2-pre+solving-time-speedup.plot%
  }{#3}
}

\newsavebox{\solvTechPlot}
\newlength{\solvTechPlotW}
\newlength{\solvTechSubfigW}
\NewDocumentCommand{\mkSolvTechSubfigure}{O{disable}moomO{}O{}O{}}{%
  \IfStrEq{#1}{disable}{\def\argUC{Disable}}{}%
  \IfStrEq{#1}{enable}{\def\argUC{Enable}}{}%
  \IfStrEq{#1}{best}{\def\argUC{Best}}{}%
  \edef\statsPartName{%
    SolvTech%
    \argUC%
    #5%
    PrePlusSolvingTimeSpeedupPrePlusSolvingTimeRegularSpeedup%
  }%
  \edef\gmi{\csname\statsPartName Gmean\endcsname}%
  \edef\cimin{\csname\statsPartName CiMin\endcsname}%
  \edef\cimax{\csname\statsPartName CiMax\endcsname}%
  \savebox{\solvTechPlot}{\insertSolvTechSpeedupPlot[#1]{#2}{.49}}%
  \settowidth{\solvTechPlotW}{\usebox{\solvTechPlot}}%
  \IfValueTF{#4}
            {%
              \pgfmathsetlength{\solvTechSubfigW}%
                               {max(\solvTechPlotW, #4)}%
            }%
            {\setlength{\solvTechSubfigW}{\solvTechPlotW}}%
  \begin{minipage}[t]{\solvTechSubfigW}%
    \captionsetup[sub]{justification=centerlast}%
    \IfValueTF{#4}{\captionsetup[sub]{width=#4}}{}%
    \centering%
    \usebox{\solvTechPlot}
    \subcaption{%
                 \IfValueTF{#3}{#3}{\refEquation{#2}}.
                 %
                 GMI:~\printGMI[#6]{\gmi},
                 CI:~\printGMICI[#7]{\cimin}[#8]{\cimax}%
                 \labelFigure{solv-tech-#1-#2}%
                }%
    \vspace{0pt}% Force \belowcaptionspace to be applied
  \end{minipage}%
}

We first evaluate the \gls{implied.c} \glspl{constraint} individually.
%
Based on the results, we then arrange them into groups to find the best
combination of \gls{implied.c} \glspl{constraint}.
%
To curb experiment runtimes, we include all other solving techniques in the
\glsplshort{constraint model} because, without them, a very long time limit
would have to be applied to discover the impact on solving time.


\paragraph{Impact of a Single Constraint}

We evaluate the impact of each \gls{implied.c} \gls{constraint}
(\refEquationRange{impl-cons-defs-dominate-defs}{impl-cons-fix-fall-throughs})
by comparing the solving times exhibited by two versions of the \gls{constraint
  model}:
%
\begin{modelList}
  \item \labelModel{impl-ex-wo-single-con}
    one without a particular \gls{implied.c} \gls{constraint}
  \item \labelModel{impl-ex-w-single-con}
    one with all \glspl{constraint}
\end{modelList}.

Because the \gls{pattern set} used in these experiments contains only
\glspl{pattern} that do not \gls{span.b} any \glspl{block}, we expect
\refEquationList{impl-cons-defs-in-spanned-blocks, impl-cons-spanned-input} to
have no impact on solving time.
%
For the rest, we expect some \glspl{constraint} to lead to an overall solving
time improvement while others may degrade solving time.
%
As described in \refChapter{constraint-programming}, this is because some
\glspl{constraint} may be too expensive to execute compared with the amount of
\gls{propagation} they provide.

\RefFigure{single-impl-con-vs-no-con-solving-time-plots} shows the normalized
solving times (including \gls{presolving} time) for the two \glspl{constraint
  model} described above in the first experiment, with \glsshort{constraint
  model}~\refModel{impl-ex-wo-single-con} as \gls{baseline} and
\glsshort{constraint model}~\refModel{impl-ex-w-single-con} as \gls{subject}.
%
\NewDocumentCommand{\mkFigureCaption}{s}{%
  \def\capToCText{%
    Set of plots for evaluating each implied constraint's impact on solving
    time%
  }%
  \def\capText{%
    Normalized solving times (incl.\ presolving time) for two constraint models:
    %
    \begin{inlinelist}[itemjoin={, }, itemjoin*={, and}]
      \item one without a particular implied constraint (baseline)
      \item one with all constraints (subject)
    \end{inlinelist}%
  }%
  \IfBooleanTF{#1}{%
    \caption*{\capText}%
  }{%
    \caption[\capToCText]{\capText}%
  }%
}
%
\begin{figure}
  \centering

  \mkSolvTechSubfigure{impl-cons-defs-dominate-defs}%
                      {ImplConsDefsDominateDefs}%
  \hfill%
  \mkSolvTechSubfigure{impl-cons-defs-in-spanned-blocks}%
                      {ImplConsDefsInSpannedBlocks}

  \vspace{\betweensubfigures}

  \mkSolvTechSubfigure{impl-cons-identical-entry-blocks}%
                      {ImplConsIdenticalEntryBlocks}%
  \hfill%
  \mkSolvTechSubfigure{impl-cons-defs-dominate-entry-blocks}%
                      {ImplConsDefsDominateEntryBlocks}%
                      []%
                      [round-precision=3]

  \vspace{\betweensubfigures}

  \mkSolvTechSubfigure{impl-cons-place-phi-ops-same-as-def-edges}%
                      {ImplConsPlacePhiOpsSameAsDefEdges}%
  \hfill%
  \mkSolvTechSubfigure{impl-cons-no-span-uses}%
                      {ImplConsNoSpanUses}

  \vspace{\betweensubfigures}

  \mkSolvTechSubfigure{impl-cons-no-span-defs}%
                      {ImplConsNoSpanDefs}%
  \hfill%
  \mkSolvTechSubfigure{impl-cons-no-span-use-defs}%
                      {ImplConsNoSpanUseDefs}

  \vspace{\betweensubfigures}

  \mkSolvTechSubfigure{impl-cons-spanned-input}%
                      {ImplConsSpannedInput}%
  \hfill%
  \mkSolvTechSubfigure{impl-cons-used-data-must-be-available}%
                      {ImplConsUsedDataMustBeAvailable}

  \mkFigureCaption
  \labelFigure{single-impl-con-vs-no-con-solving-time-plots}
\end{figure}
%
\let\oldthefigure\thefigure
\renewcommand{\thefigure}{\oldthefigure{} (cont.)}
%
\begin{figure}
  \ContinuedFloat
  \centering

  \mkSolvTechSubfigure{impl-cons-exterior-data-must-be-available}%
                      {ImplConsExteriorDataMustBeAvailable}%
  \hfill%
  \mkSolvTechSubfigure{impl-cons-locs-of-uses}%
                      {ImplConsLocsOfUses}

  \vspace{\betweensubfigures}

  \mkSolvTechSubfigure{impl-cons-locs-of-defs}%
                      {ImplConsLocsOfDefs}%
  \hfill%
  \mkSolvTechSubfigure{impl-cons-fix-fall-throughs}%
                      {ImplConsFixFallThroughs}

  \mkFigureCaption*
\end{figure}
%
\renewcommand{\thefigure}{\oldthefigure}
%
The solving times range from
\printMinSolvingTime{
  \SolvTechDisableImplConsDefsDominateDefsPrePlusSolvingTimeSpeedupAllPrePlusSolvingTimeAvgMin,
  \SolvTechDisableImplConsDefsDominateDefsPrePlusSolvingTimeSpeedupPrePlusSolvingTimeAvgMin,
  \SolvTechDisableImplConsDefsInSpannedBlocksPrePlusSolvingTimeSpeedupPrePlusSolvingTimeAvgMin,
  \SolvTechDisableImplConsIdenticalEntryBlocksPrePlusSolvingTimeSpeedupPrePlusSolvingTimeAvgMin,
  \SolvTechDisableImplConsDefsDominateEntryBlocksPrePlusSolvingTimeSpeedupPrePlusSolvingTimeAvgMin,
  \SolvTechDisableImplConsPlacePhiOpsSameAsDefEdgesPrePlusSolvingTimeSpeedupPrePlusSolvingTimeAvgMin,
  \SolvTechDisableImplConsNoSpanUsesPrePlusSolvingTimeSpeedupPrePlusSolvingTimeAvgMin,
  \SolvTechDisableImplConsNoSpanDefsPrePlusSolvingTimeSpeedupPrePlusSolvingTimeAvgMin,
  \SolvTechDisableImplConsNoSpanUseDefsPrePlusSolvingTimeSpeedupPrePlusSolvingTimeAvgMin,
  \SolvTechDisableImplConsSpannedInputPrePlusSolvingTimeSpeedupPrePlusSolvingTimeAvgMin,
  \SolvTechDisableImplConsUsedDataMustBeAvailablePrePlusSolvingTimeSpeedupPrePlusSolvingTimeAvgMin,
  \SolvTechDisableImplConsExteriorDataMustBeAvailablePrePlusSolvingTimeSpeedupPrePlusSolvingTimeAvgMin,
  \SolvTechDisableImplConsLocsOfUsesPrePlusSolvingTimeSpeedupPrePlusSolvingTimeAvgMin,
  \SolvTechDisableImplConsLocsOfDefsPrePlusSolvingTimeSpeedupPrePlusSolvingTimeAvgMin,
  \SolvTechDisableImplConsFixFallThroughsPrePlusSolvingTimeSpeedupPrePlusSolvingTimeAvgMin
} to
\printMaxSolvingTime{
  \SolvTechDisableImplConsDefsDominateDefsPrePlusSolvingTimeSpeedupAllPrePlusSolvingTimeAvgMax,
  \SolvTechDisableImplConsDefsDominateDefsPrePlusSolvingTimeSpeedupPrePlusSolvingTimeAvgMax,
  \SolvTechDisableImplConsDefsInSpannedBlocksPrePlusSolvingTimeSpeedupPrePlusSolvingTimeAvgMax,
  \SolvTechDisableImplConsIdenticalEntryBlocksPrePlusSolvingTimeSpeedupPrePlusSolvingTimeAvgMax,
  \SolvTechDisableImplConsDefsDominateEntryBlocksPrePlusSolvingTimeSpeedupPrePlusSolvingTimeAvgMax,
  \SolvTechDisableImplConsPlacePhiOpsSameAsDefEdgesPrePlusSolvingTimeSpeedupPrePlusSolvingTimeAvgMax,
  \SolvTechDisableImplConsNoSpanUsesPrePlusSolvingTimeSpeedupPrePlusSolvingTimeAvgMax,
  \SolvTechDisableImplConsNoSpanDefsPrePlusSolvingTimeSpeedupPrePlusSolvingTimeAvgMax,
  \SolvTechDisableImplConsNoSpanUseDefsPrePlusSolvingTimeSpeedupPrePlusSolvingTimeAvgMax,
  \SolvTechDisableImplConsSpannedInputPrePlusSolvingTimeSpeedupPrePlusSolvingTimeAvgMax,
  \SolvTechDisableImplConsUsedDataMustBeAvailablePrePlusSolvingTimeSpeedupPrePlusSolvingTimeAvgMax,
  \SolvTechDisableImplConsExteriorDataMustBeAvailablePrePlusSolvingTimeSpeedupPrePlusSolvingTimeAvgMax,
  \SolvTechDisableImplConsLocsOfUsesPrePlusSolvingTimeSpeedupPrePlusSolvingTimeAvgMax,
  \SolvTechDisableImplConsLocsOfDefsPrePlusSolvingTimeSpeedupPrePlusSolvingTimeAvgMax,
  \SolvTechDisableImplConsFixFallThroughsPrePlusSolvingTimeSpeedupPrePlusSolvingTimeAvgMax
} with a \gls{CV} of
\numMaxOf{
  \SolvTechDisableImplConsDefsDominateDefsPrePlusSolvingTimeSpeedupAllPrePlusSolvingTimeCvMax,
  \SolvTechDisableImplConsDefsDominateDefsPrePlusSolvingTimeSpeedupPrePlusSolvingTimeCvMax,
  \SolvTechDisableImplConsDefsInSpannedBlocksPrePlusSolvingTimeSpeedupPrePlusSolvingTimeCvMax,
  \SolvTechDisableImplConsIdenticalEntryBlocksPrePlusSolvingTimeSpeedupPrePlusSolvingTimeCvMax,
  \SolvTechDisableImplConsDefsDominateEntryBlocksPrePlusSolvingTimeSpeedupPrePlusSolvingTimeCvMax,
  \SolvTechDisableImplConsPlacePhiOpsSameAsDefEdgesPrePlusSolvingTimeSpeedupPrePlusSolvingTimeCvMax,
  \SolvTechDisableImplConsNoSpanUsesPrePlusSolvingTimeSpeedupPrePlusSolvingTimeCvMax,
  \SolvTechDisableImplConsNoSpanDefsPrePlusSolvingTimeSpeedupPrePlusSolvingTimeCvMax,
  \SolvTechDisableImplConsNoSpanUseDefsPrePlusSolvingTimeSpeedupPrePlusSolvingTimeCvMax,
  \SolvTechDisableImplConsSpannedInputPrePlusSolvingTimeSpeedupPrePlusSolvingTimeCvMax,
  \SolvTechDisableImplConsUsedDataMustBeAvailablePrePlusSolvingTimeSpeedupPrePlusSolvingTimeCvMax,
  \SolvTechDisableImplConsExteriorDataMustBeAvailablePrePlusSolvingTimeSpeedupPrePlusSolvingTimeCvMax,
  \SolvTechDisableImplConsLocsOfUsesPrePlusSolvingTimeSpeedupPrePlusSolvingTimeCvMax,
  \SolvTechDisableImplConsLocsOfDefsPrePlusSolvingTimeSpeedupPrePlusSolvingTimeCvMax,
  \SolvTechDisableImplConsFixFallThroughsPrePlusSolvingTimeSpeedupPrePlusSolvingTimeCvMax
}.
%
The \glspl{GMI} and \glspl{CI} are given in
\refFigure{single-impl-con-vs-no-con-solving-time-plots}.

We see that \refEquationList{impl-cons-defs-dominate-defs,
  impl-cons-defs-dominate-entry-blocks} lead to an overall solving time
improvement, and that \refEquationList{impl-cons-defs-in-spanned-blocks,
  impl-cons-place-phi-ops-same-as-def-edges, impl-cons-spanned-input} have no
impact on solving time.
%
For \refEquationList{impl-cons-identical-entry-blocks, impl-cons-no-span-uses,
  impl-cons-no-span-defs, impl-cons-no-span-use-defs,
  impl-cons-used-data-must-be-available,
  impl-cons-exterior-data-must-be-available, impl-cons-locs-of-uses,
  impl-cons-locs-of-defs, impl-cons-fix-fall-throughs}, the results are
inconclusive.


\paragraph{Impact of Groups of Constraints}

We find the best combination of \gls{implied.c} \glspl{constraint} by arranging
them into groups and evaluating the impact of each such group.
%
Because a full evaluation would require us to test every combination of
\glspl{constraint} -- which would result in an unreasonably large number of
experiments -- we limit ourselves to only comparing the solving times exhibited
by three versions of the \glsshort{constraint model}:
%
\begin{modelList}
  \item \labelModel{impl-ex-no-cons}
    one with no \gls{implied.c} \glspl{constraint}
  \item \labelModel{impl-ex-only-good-cons}
    one with only those who individually lead to an overall solving time
    improvement
  \item \labelModel{impl-ex-all-cons}
    one with all \glspl{constraint}
\end{modelList}.
%
We expect \glsplshort{constraint model}~\refModel{impl-ex-only-good-cons}
and~\refModel{impl-ex-all-cons} to all perform better than
\glsshort{constraint model}~\refModel{impl-ex-no-cons}.
%
No hypothesis is attempted regarding the relative performance between
\glsplshort{constraint model}~\refModel{impl-ex-only-good-cons}
and~\refModel{impl-ex-all-cons}.

\RefFigure{diff-impl-cons-comb-solving-time-plot} shows the normalized solving
times (including \gls{presolving} time) for three of the \glspl{constraint
  model} described above, with \glsshort{constraint
  model}~\refModel{impl-ex-no-cons} as \gls{baseline} and \glsplshort{constraint
  model}~\refModel{impl-ex-only-good-cons} and~\refModel{impl-ex-all-cons} as
\glspl{subject}.
%
\begin{figure}
  \centering%

  \mkSolvTechSubfigure[enable]%
                      {only-good-implied-cons}%
                      [%
                        Model \refModel{impl-ex-no-cons} \versus
                        \refModel{impl-ex-only-good-cons}%
                      ]%
                      {OnlyGoodImpliedCons}%
  \hfill%
  \mkSolvTechSubfigure{all-implied-cons}%
                      [%
                        Model \refModel{impl-ex-no-cons} \versus
                        \refModel{impl-ex-all-cons}%
                      ]%
                      {AllImpliedCons}

  \caption[%
            Plot for evaluating the impact on solving time made by different
            combinations of implied constraints%
          ]%
          {%
            Normalized solving times (incl.\ presolving time) for three
            constraint models:
            %
            \begin{modelList}
              \item one with no implied constraints (baseline)
              \item one with only those who individually lead to an overall
                solving time improvement
                (\refEquationList{impl-cons-defs-dominate-defs,
                  impl-cons-defs-dominate-entry-blocks}; subject)
              \item one with all \glspl{constraint} (subject)
            \end{modelList}%
          }
  \labelFigure{diff-impl-cons-comb-solving-time-plot}
\end{figure}
%
The solving times range from
\printMinSolvingTime{
  \SolvTechEnableOnlyGoodImpliedConsPrePlusSolvingTimeSpeedupNonePrePlusSolvingTimeAvgMin,
  \SolvTechEnableOnlyGoodImpliedConsPrePlusSolvingTimeSpeedupPrePlusSolvingTimeAvgMin,
  \SolvTechDisableAllImpliedConsPrePlusSolvingTimeSpeedupPrePlusSolvingTimeAvgMin
} to
\printMaxSolvingTime{
  \SolvTechEnableOnlyGoodImpliedConsPrePlusSolvingTimeSpeedupNonePrePlusSolvingTimeAvgMax,
  \SolvTechEnableOnlyGoodImpliedConsPrePlusSolvingTimeSpeedupPrePlusSolvingTimeAvgMax,
  \SolvTechDisableAllImpliedConsPrePlusSolvingTimeSpeedupPrePlusSolvingTimeAvgMax
} with a \gls{CV} of
\numMaxOf{
  \SolvTechEnableOnlyGoodImpliedConsPrePlusSolvingTimeSpeedupNonePrePlusSolvingTimeCvMax,
  \SolvTechEnableOnlyGoodImpliedConsPrePlusSolvingTimeSpeedupPrePlusSolvingTimeCvMax,
  \SolvTechDisableAllImpliedConsPrePlusSolvingTimeSpeedupPrePlusSolvingTimeCvMax
}.

To begin with, we observe that the \gls{GMI} for \glsshort{constraint
  model}~\refModel{impl-ex-only-good-cons} over \glsshort{constraint
  model}~\refModel{impl-ex-no-cons} is \printGMI{%
  \SolvTechEnableOnlyGoodImpliedConsPrePlusSolvingTimeSpeedupPrePlusSolvingTimeRegularSpeedupGmean%
} with \gls{CI}~\printGMICI{%
  \SolvTechEnableOnlyGoodImpliedConsPrePlusSolvingTimeSpeedupPrePlusSolvingTimeRegularSpeedupCiMin%
}{%
  \SolvTechEnableOnlyGoodImpliedConsPrePlusSolvingTimeSpeedupPrePlusSolvingTimeRegularSpeedupCiMax%
} (\refFigure{solv-tech-enable-only-good-implied-cons}).
%
Hence combining \refEquationList{impl-cons-defs-dominate-defs,
  impl-cons-defs-dominate-entry-blocks} yields a greater overall solving time
improvement than when picked individually (up
to~\printZCNorm{%
  \SolvTechEnableOnlyGoodImpliedConsPrePlusSolvingTimeSpeedupPrePlusSolvingTimeZeroCenteredSpeedupMax%
}).
%
However, we note that this combination yields a non-negligible solving time
degradation for one \gls{function} (down
to~\printZCNorm{%
  \SolvTechEnableOnlyGoodImpliedConsPrePlusSolvingTimeSpeedupPrePlusSolvingTimeZeroCenteredSpeedupMin%
}).

In comparison, we observe that the \gls{GMI} for \glsshort{constraint
  model}~\refModel{impl-ex-all-cons} over \glsshort{constraint
  model}~\refModel{impl-ex-no-cons} is \printGMI{%
  \SolvTechDisableAllImpliedConsPrePlusSolvingTimeSpeedupPrePlusSolvingTimeRegularSpeedupGmean%
} with \gls{CI}~\printGMICI{%
  \SolvTechDisableAllImpliedConsPrePlusSolvingTimeSpeedupPrePlusSolvingTimeRegularSpeedupCiMin%
}{%
  \SolvTechDisableAllImpliedConsPrePlusSolvingTimeSpeedupPrePlusSolvingTimeRegularSpeedupCiMax%
} (\refFigure{solv-tech-disable-all-implied-cons}), which is a greater
improvement compared to \glsshort{constraint
  model}~\refModel{impl-ex-only-good-cons}.
%
Although the maximum improvement is less than for \glsshort{constraint
  model}~\refModel{impl-ex-only-good-cons} (up to
\printZCNorm{%
  \SolvTechDisableAllImpliedConsPrePlusSolvingTimeSpeedupPrePlusSolvingTimeZeroCenteredSpeedupMax%
}), this combination yields no considerable degradation for any \gls{function}
(down to at
most~\printZCNorm{%
  \SolvTechDisableAllImpliedConsPrePlusSolvingTimeSpeedupPrePlusSolvingTimeZeroCenteredSpeedupMin%
}).
%
Hence it is most beneficial to include all \gls{implied.c} \glspl{constraint} in
the \glsshort{constraint model}.


\paragraph{Conclusions}

From the results for these experiments, we conclude:
%
\begin{enumerate*}[label=(\roman*), itemjoin={;\ }, itemjoin*={; but\ }]
  \item that some of the \gls{implied.c} \glspl{constraint} have individually a
    positive impact on solving time
  \item it is most beneficial to include all \gls{implied.c} \glspl{constraint}
    in the \glsshort{constraint model}
\end{enumerate*}.


\subsection{Symmetry and Dominance Breaking Constraints}
\labelSection{st-exp-evaluation-dom-constraints}

Like in \refSection{st-exp-evaluation-implied-constraints}, we first evaluate
the \glsshort{symmetry breaking.c} and \gls{dominance breaking.c}
\glspl{constraint} individually.
%
Based on the results, we then arrange them into groups to find the best
combination of \glsshort{symmetry breaking.c} and \gls{dominance breaking.c}
\glspl{constraint}.
%
To curb experiment runtimes, we include all other solving techniques in the
\glsplshort{constraint model} because, without them, a very long time limit
would have to be applied to discover the impact on solving time.


\paragraph{Impact of a Single Constraint}

We evaluate the impact of each \glsshort{symmetry breaking.c} and \gls{dominance
  breaking.c} \gls{constraint}
(\refEquationRange{dom-cons-locs-of-states}{dom-cons-kill-match-selection}) by
comparing the solving times exhibited by two versions of the \gls{constraint
  model}:
%
\begin{modelList}
  \item \labelModel{dom-ex-wo-single-con}
    one without a particular \glsshort{symmetry breaking.c} or \gls{dominance
      breaking.c} \gls{constraint}
  \item \labelModel{dom-ex-w-single-con}
    one with all \glspl{constraint}
\end{modelList}.
%
For the same reason as with the \gls{implied.c} \glspl{constraint}, we expect
some \glspl{constraint} to lead to an overall solving time improvement while
others may degrade solving time.

\RefFigure{single-dom-con-vs-no-con-solving-time-plots} shows the normalized
solving times (including \gls{presolving} time) for two \glspl{constraint model}
described above in the first experiment, with \glsshort{constraint
  model}~\refModel{dom-ex-wo-single-con} as \gls{baseline} and
\glsshort{constraint model}~\refModel{dom-ex-w-single-con} as \gls{subject}.
%
\begin{figure}
  \centering

  \mkSolvTechSubfigure{dom-cons-locs-of-states}%
                      {DomConsLocsOfStates}%
  \hfill%
  \mkSolvTechSubfigure{dom-cons-operands-of-non-selected-matches}%
                      {DomConsOperandsOfNonSelectedMatches}

  \vspace{\betweensubfigures}

  \mkSolvTechSubfigure{impl-cons-input-operands-not-taking-min-value}%
                      {ImplConsInputOperandsNotTakingMinValue}%
  \hfill%
  \mkSolvTechSubfigure{dom-cons-interch-data-chains}%
                      {DomConsInterchDataChains}

  \vspace{\betweensubfigures}

  \mkSolvTechSubfigure{dom-cons-null-copy-match-selection}%
                      {DomConsNullCopyMatchSelection}%
  \hfill%
  \mkSolvTechSubfigure{dom-cons-kill-match-selection}%
                      {DomConsKillMatchSelection}

  \caption[%
            Set of plots for evaluating each symmetry or dominance breaking
            constraint's impact on solving time%
          ]%
          {%
            Normalized solving times (incl.\ presolving time) for two
            constraint models:
            \begin{inlinelist}[itemjoin={, }, itemjoin*={, and}]
              \item one without a particular symmetry or dominance breaking
                constraint (baseline)
              \item one with all constraints (subject)
            \end{inlinelist}%
          }
  \labelFigure{single-dom-con-vs-no-con-solving-time-plots}
\end{figure}
%
The solving times range from
\printMinSolvingTime{
  \SolvTechDisableDomConsLocsOfStatesPrePlusSolvingTimeSpeedupAllPrePlusSolvingTimeAvgMin,
  \SolvTechDisableDomConsLocsOfStatesPrePlusSolvingTimeSpeedupPrePlusSolvingTimeAvgMin,
  \SolvTechDisableDomConsOperandsOfNonSelectedMatchesPrePlusSolvingTimeSpeedupPrePlusSolvingTimeAvgMin,
  \SolvTechDisableImplConsInputOperandsNotTakingMinValuePrePlusSolvingTimeSpeedupPrePlusSolvingTimeAvgMin,
  \SolvTechDisableDomConsInterchDataChainsPrePlusSolvingTimeSpeedupPrePlusSolvingTimeAvgMin,
  \SolvTechDisableDomConsNullCopyMatchSelectionPrePlusSolvingTimeSpeedupPrePlusSolvingTimeAvgMin,
  \SolvTechDisableDomConsKillMatchSelectionPrePlusSolvingTimeSpeedupPrePlusSolvingTimeAvgMin
} to
\printMaxSolvingTime{
  \SolvTechDisableDomConsLocsOfStatesPrePlusSolvingTimeSpeedupAllPrePlusSolvingTimeAvgMax,
  \SolvTechDisableDomConsLocsOfStatesPrePlusSolvingTimeSpeedupPrePlusSolvingTimeAvgMax,
  \SolvTechDisableDomConsOperandsOfNonSelectedMatchesPrePlusSolvingTimeSpeedupPrePlusSolvingTimeAvgMax,
  \SolvTechDisableImplConsInputOperandsNotTakingMinValuePrePlusSolvingTimeSpeedupPrePlusSolvingTimeAvgMax,
  \SolvTechDisableDomConsInterchDataChainsPrePlusSolvingTimeSpeedupPrePlusSolvingTimeAvgMax,
  \SolvTechDisableDomConsNullCopyMatchSelectionPrePlusSolvingTimeSpeedupPrePlusSolvingTimeAvgMax,
  \SolvTechDisableDomConsKillMatchSelectionPrePlusSolvingTimeSpeedupPrePlusSolvingTimeAvgMax
} with a \gls{CV} of
\numMaxOf{
  \SolvTechDisableDomConsLocsOfStatesPrePlusSolvingTimeSpeedupAllPrePlusSolvingTimeCvMax,
  \SolvTechDisableDomConsLocsOfStatesPrePlusSolvingTimeSpeedupPrePlusSolvingTimeCvMax,
  \SolvTechDisableDomConsOperandsOfNonSelectedMatchesPrePlusSolvingTimeSpeedupPrePlusSolvingTimeCvMax,
  \SolvTechDisableImplConsInputOperandsNotTakingMinValuePrePlusSolvingTimeSpeedupPrePlusSolvingTimeCvMax,
  \SolvTechDisableDomConsInterchDataChainsPrePlusSolvingTimeSpeedupPrePlusSolvingTimeCvMax,
  \SolvTechDisableDomConsNullCopyMatchSelectionPrePlusSolvingTimeSpeedupPrePlusSolvingTimeCvMax,
  \SolvTechDisableDomConsKillMatchSelectionPrePlusSolvingTimeSpeedupPrePlusSolvingTimeCvMax
}.
%
The \glspl{GMI} and \glspl{CI} are given in
\refFigure{single-dom-con-vs-no-con-solving-time-plots}.

We observe that \refEquation{dom-cons-interch-data-chains} leads to an overall
solving time improvement and that
\refEquation{impl-cons-input-operands-not-taking-min-value} has no impact on
solving time.
%
For \refEquationList{dom-cons-locs-of-states,
  dom-cons-operands-of-non-selected-matches, dom-cons-null-copy-match-selection,
  dom-cons-kill-match-selection}, the results are inconclusive.


\paragraph{Impact of Groups of Constraints}

We find the best combination of \glsshort{symmetry breaking.c} and
\gls{dominance breaking.c} \glspl{constraint} by arranging them into groups and
evaluating the impact of each such group.
%
Again, for practicality we limit ourselves to only comparing the solving times
exhibited by three versions of the \glsshort{constraint model}:
%
\begin{modelList}
  \item \labelModel{dom-ex-no-cons}
    one with no \glsshort{symmetry breaking.c} or \gls{dominance breaking.c}
    \glspl{constraint}
  \item \labelModel{dom-ex-only-good-cons}
    one with only those who individually lead to an overall solving time
    improvement
  \item \labelModel{dom-ex-all-cons}
    one with all \glspl{constraint}
\end{modelList}.
%
We expect \glsplshort{constraint model}~\refModel{dom-ex-only-good-cons}
and~\refModel{dom-ex-all-cons} to all perform better than \glsshort{constraint
  model}~\refModel{dom-ex-no-cons}.
%
No hypothesis is attempted regarding the relative performance between
\glsplshort{constraint model}~\refModel{dom-ex-only-good-cons}
and~\refModel{dom-ex-all-cons}.

\RefFigure{diff-dom-cons-comb-solving-time-plot} shows the normalized solving
times (including \gls{presolving} time) for the three \glspl{constraint model}
described, with \glsshort{constraint model}~\refModel{dom-ex-no-cons} as
\gls{baseline} and \glsplshort{constraint
  model}~\refModel{dom-ex-only-good-cons} and~\refModel{dom-ex-all-cons} as
\glspl{subject}.
%
\begin{figure}
  \centering%

  \mkSolvTechSubfigure[enable]%
                      {only-good-dom-cons}%
                      [%
                        Model \refModel{dom-ex-no-cons} \versus
                        \refModel{dom-ex-only-good-cons}%
                      ]%
                      {OnlyGoodDomCons}%
  \hfill%
  \mkSolvTechSubfigure{all-dom-cons}%
                      [%
                        Model \refModel{dom-ex-no-cons} \versus
                        \refModel{dom-ex-all-cons}%
                      ]%
                      {AllDomCons}

  \caption[%
            Plot for evaluating the impact on solving time made by different
            combinations of symmetry and dominance breaking constraints%
          ]%
          {%
            Normalized solving times (incl.\ presolving time) for three
            constraint models:
            %
            \begin{modelList}
              \item one with no symmetry or dominance breaking constraints
                (baseline)
              \item one with only those who individually lead to an overall
                solving time improvement
                (\refEquation{dom-cons-interch-data-chains}; subject)
              \item one with all \glspl{constraint} (subject)
            \end{modelList}%
          }
  \labelFigure{diff-dom-cons-comb-solving-time-plot}
\end{figure}
%
The solving times range from
\printMinSolvingTime{
  \SolvTechEnableOnlyGoodDomConsPrePlusSolvingTimeSpeedupNonePrePlusSolvingTimeAvgMin,
  \SolvTechEnableOnlyGoodDomConsPrePlusSolvingTimeSpeedupPrePlusSolvingTimeAvgMin,
  \SolvTechDisableAllDomConsPrePlusSolvingTimeSpeedupPrePlusSolvingTimeAvgMin
} to
\printMaxSolvingTime{
  \SolvTechEnableOnlyGoodDomConsPrePlusSolvingTimeSpeedupNonePrePlusSolvingTimeAvgMax,
  \SolvTechEnableOnlyGoodDomConsPrePlusSolvingTimeSpeedupPrePlusSolvingTimeAvgMax,
  \SolvTechDisableAllDomConsPrePlusSolvingTimeSpeedupPrePlusSolvingTimeAvgMax
} with a \gls{CV} of
\numMaxOf{
  \SolvTechEnableOnlyGoodDomConsPrePlusSolvingTimeSpeedupNonePrePlusSolvingTimeCvMax,
  \SolvTechEnableOnlyGoodDomConsPrePlusSolvingTimeSpeedupPrePlusSolvingTimeCvMax,
  \SolvTechDisableAllDomConsPrePlusSolvingTimeSpeedupPrePlusSolvingTimeCvMax
}.

To begin with, we observe that the \gls{GMI} for \glsshort{constraint
  model}~\refModel{dom-ex-only-good-cons} over \glsshort{constraint
  model}~\refModel{dom-ex-no-cons} is \printGMI{%
  \SolvTechEnableOnlyGoodDomConsPrePlusSolvingTimeSpeedupPrePlusSolvingTimeRegularSpeedupGmean%
} with \gls{CI}~\printGMICI{%
  \SolvTechEnableOnlyGoodDomConsPrePlusSolvingTimeSpeedupPrePlusSolvingTimeRegularSpeedupCiMin%
}{%
  \SolvTechEnableOnlyGoodDomConsPrePlusSolvingTimeSpeedupPrePlusSolvingTimeRegularSpeedupCiMax%
} (\refFigure{solv-tech-enable-only-good-dom-cons}).
%
This means that combining only the \glsshort{symmetry breaking.c} or
\gls{dominance breaking.c} \glspl{constraint} that individually have a positive
impact on solving time is not sufficient.
%
Although solving time is improved considerably for one \gls{function} (up to
\printZCNorm{%
  \SolvTechEnableOnlyGoodDomConsPrePlusSolvingTimeSpeedupPrePlusSolvingTimeZeroCenteredSpeedupMax%
}), this combination also degrades solving time for several \glspl{function}
(down
to~\printZCNorm{%
  \SolvTechEnableOnlyGoodDomConsPrePlusSolvingTimeSpeedupPrePlusSolvingTimeZeroCenteredSpeedupMin%
}).

In comparison, we observe that the \gls{GMI} for \glsshort{constraint
  model}~\refModel{dom-ex-all-cons} over \glsshort{constraint
  model}~\refModel{dom-ex-no-cons} is \printGMI{%
  \SolvTechDisableAllDomConsPrePlusSolvingTimeSpeedupPrePlusSolvingTimeRegularSpeedupGmean%
} with \gls{CI}~\printGMICI{%
  \SolvTechDisableAllDomConsPrePlusSolvingTimeSpeedupPrePlusSolvingTimeRegularSpeedupCiMin%
}{%
  \SolvTechDisableAllDomConsPrePlusSolvingTimeSpeedupPrePlusSolvingTimeRegularSpeedupCiMax%
} (\refFigure{solv-tech-disable-all-dom-cons}), which is a statistically
significant, positive impact.
%
In addition, the maximum improvement is greater than for \glsshort{constraint
  model}~\refModel{dom-ex-only-good-cons} (up to
\printZCNorm{%
  \SolvTechDisableAllDomConsPrePlusSolvingTimeSpeedupPrePlusSolvingTimeZeroCenteredSpeedupMax%
}), and this combination yields no considerable degradation for any
\gls{function} (down to at
most~\printZCNorm{%
  \SolvTechDisableAllDomConsPrePlusSolvingTimeSpeedupPrePlusSolvingTimeZeroCenteredSpeedupMin%
}).
%
Hence it is most beneficial to include all such \glspl{constraint} in the
\glsshort{constraint model}.


\paragraph{Conclusions}

From the results for these experiments, we conclude:
%
\begin{enumerate*}[label=(\roman*), itemjoin={;\ }, itemjoin*={; but\ }]
  \item that some of the \glsshort{symmetry breaking.c} and \gls{dominance
    breaking.c} \glspl{constraint} have individually a positive impact on
    solving time
  \item it is most beneficial to include all \glsshort{symmetry breaking.c} and
    \gls{dominance breaking.c} \glspl{constraint} in the \glsshort{constraint
      model}
\end{enumerate*}.


\subsection{Presolving}

Like in \refSectionList{st-exp-evaluation-implied-constraints,
  st-exp-evaluation-dom-constraints}, we first evaluate the \gls{presolving}
techniques individually.
%
Based on the results, we then arrange them into groups to find the best
combination of \gls{presolving} techniques.
%
To curb experiment runtimes, we include all other solving techniques in the
\glsplshort{constraint model} because, without them, a very long time limit
would have to be applied to discover the impact on solving time.


\paragraph{Impact of a Single Technique}

We evaluate the impact of each \gls{presolving} technique
(\refSection{st-pre-dom-matches},
\refEquationRange{illegal-matches-uncovered-ops}{redun-non-null-copy-matches},
and \refSection{st-canonical-locations}) by comparing the solving times
exhibited by two versions of the \gls{constraint model}:
%
\begin{modelList}
  \item \labelModel{pre-ex-wo-single-tech}
    one without a particular \gls{presolving} technique
  \item \labelModel{pre-ex-w-single-tech}
    one with all techniques
\end{modelList}.
%
For the same reason as with the \gls{implied.c}, \gls{symmetry breaking.c},
\gls{dominance breaking.c} \glspl{constraint}, we expect some techniques to lead
to an overall solving time improvement while others may degrade solving time.

\RefFigure{single-presolving-vs-no-presolving-solving-time-plots} shows the
normalized solving times (including \gls{presolving} time) for the two
\glspl{constraint model} described above in the first experiment, with
\glsshort{constraint model}~\refModel{pre-ex-wo-single-tech} as \gls{baseline}
and \glsshort{constraint model}~\refModel{pre-ex-w-single-tech} as
\gls{subject}.
%
\begin{figure}
  \centering

  \mkSolvTechSubfigure{dom-matches}%
                      [Dominated matches]%
                      [3.65cm]%
                      {DomMatches}%
  \hfill%
  \mkSolvTechSubfigure{illegal-matches-uncovered-ops}%
                      {IllegalMatchesUncoveredOps}

  \vspace{\betweensubfigures}

  \mkSolvTechSubfigure{illegal-matches-undefined-data}%
                      {IllegalMatchesUndefinedData}%
  \hfill%
  \mkSolvTechSubfigure{illegal-matches-kills}%
                      {IllegalMatchesKills}%

  \vspace{\betweensubfigures}

  \mkSolvTechSubfigure{illegal-matches-def-locs}%
                      {IllegalMatchesDefLocs}%
  \hfill%
  \mkSolvTechSubfigure{illegal-matches-use-locs}%
                      {IllegalMatchesUseLocs}

  \vspace{\betweensubfigures}

  \mkSolvTechSubfigure{redun-kills}%
                      {RedunKills}%
  \hfill%
  \mkSolvTechSubfigure{redun-non-null-copy-matches}%
                      {RedunNonNullCopyMatches}

  \vspace{\betweensubfigures}

  \mkSolvTechSubfigure{canonical-locs}%
                      [Canonical locations]%
                      [\linewidth]%
                      {CanonicalLocs}

  \caption[%
            Set of plots for evaluating each presolving technique's impact on
            solving time%
          ]%
          {%
            Normalized solving times (incl.\ presolving time) for two
            constraint models:
            %
            \begin{inlinelist}[itemjoin={, }, itemjoin*={, and}]
              \item one without a particular presolving technique (baseline)
              \item one with all techniques (subject)
            \end{inlinelist}%
          }
  \labelFigure{single-presolving-vs-no-presolving-solving-time-plots}
\end{figure}
%
The solving times range from
\printMinSolvingTime{
  \SolvTechDisableDomMatchesPrePlusSolvingTimeSpeedupAllPrePlusSolvingTimeAvgMin,
  \SolvTechDisableDomMatchesPrePlusSolvingTimeSpeedupPrePlusSolvingTimeAvgMin,
  \SolvTechDisableIllegalMatchesUncoveredOpsPrePlusSolvingTimeSpeedupPrePlusSolvingTimeAvgMin,
  \SolvTechDisableIllegalMatchesUndefinedDataPrePlusSolvingTimeSpeedupPrePlusSolvingTimeAvgMin,
  \SolvTechDisableIllegalMatchesKillsPrePlusSolvingTimeSpeedupPrePlusSolvingTimeAvgMin,
  \SolvTechDisableIllegalMatchesDefLocsPrePlusSolvingTimeSpeedupPrePlusSolvingTimeAvgMin,
  \SolvTechDisableIllegalMatchesUseLocsPrePlusSolvingTimeSpeedupPrePlusSolvingTimeAvgMin,
  \SolvTechDisableRedunKillsPrePlusSolvingTimeSpeedupPrePlusSolvingTimeAvgMin,
  \SolvTechDisableRedunNonNullCopyMatchesPrePlusSolvingTimeSpeedupPrePlusSolvingTimeAvgMin,
  \SolvTechDisableCanonicalLocsPrePlusSolvingTimeSpeedupPrePlusSolvingTimeAvgMin
} to
\printMaxSolvingTime{
  \SolvTechDisableDomMatchesPrePlusSolvingTimeSpeedupAllPrePlusSolvingTimeAvgMax,
  \SolvTechDisableDomMatchesPrePlusSolvingTimeSpeedupPrePlusSolvingTimeAvgMax,
  \SolvTechDisableIllegalMatchesUncoveredOpsPrePlusSolvingTimeSpeedupPrePlusSolvingTimeAvgMax,
  \SolvTechDisableIllegalMatchesUndefinedDataPrePlusSolvingTimeSpeedupPrePlusSolvingTimeAvgMax,
  \SolvTechDisableIllegalMatchesKillsPrePlusSolvingTimeSpeedupPrePlusSolvingTimeAvgMax,
  \SolvTechDisableIllegalMatchesDefLocsPrePlusSolvingTimeSpeedupPrePlusSolvingTimeAvgMax,
  \SolvTechDisableIllegalMatchesUseLocsPrePlusSolvingTimeSpeedupPrePlusSolvingTimeAvgMax,
  \SolvTechDisableRedunKillsPrePlusSolvingTimeSpeedupPrePlusSolvingTimeAvgMax,
  \SolvTechDisableRedunNonNullCopyMatchesPrePlusSolvingTimeSpeedupPrePlusSolvingTimeAvgMax,
  \SolvTechDisableCanonicalLocsPrePlusSolvingTimeSpeedupPrePlusSolvingTimeAvgMax
} with a \gls{CV} of
\numMaxOf{
  \SolvTechDisableDomMatchesPrePlusSolvingTimeSpeedupAllPrePlusSolvingTimeCvMax,
  \SolvTechDisableDomMatchesPrePlusSolvingTimeSpeedupPrePlusSolvingTimeCvMax,
  \SolvTechDisableIllegalMatchesUncoveredOpsPrePlusSolvingTimeSpeedupPrePlusSolvingTimeCvMax,
  \SolvTechDisableIllegalMatchesUndefinedDataPrePlusSolvingTimeSpeedupPrePlusSolvingTimeCvMax,
  \SolvTechDisableIllegalMatchesKillsPrePlusSolvingTimeSpeedupPrePlusSolvingTimeCvMax,
  \SolvTechDisableIllegalMatchesDefLocsPrePlusSolvingTimeSpeedupPrePlusSolvingTimeCvMax,
  \SolvTechDisableIllegalMatchesUseLocsPrePlusSolvingTimeSpeedupPrePlusSolvingTimeCvMax,
  \SolvTechDisableRedunKillsPrePlusSolvingTimeSpeedupPrePlusSolvingTimeCvMax,
  \SolvTechDisableRedunNonNullCopyMatchesPrePlusSolvingTimeSpeedupPrePlusSolvingTimeCvMax,
  \SolvTechDisableCanonicalLocsPrePlusSolvingTimeSpeedupPrePlusSolvingTimeCvMax
}.
%
The \glspl{GMI} and \glspl{CI} are given in
\refFigure{single-presolving-vs-no-presolving-solving-time-plots}.

We observe that \refEquationList{redun-kills, redun-non-null-copy-matches}, and
\gls{canonical.l} \glspl{location} lead to an overall solving time improvement,
that \refEquationList{illegal-matches-def-locs, illegal-matches-use-locs} lead
to an overall solving time degradation, and that
\refEquation{illegal-matches-uncovered-ops} has no impact on solving time.
%
For \gls{dominate.m}[d] \glspl{match} and
\refEquationList{illegal-matches-undefined-data, illegal-matches-kills}, the
results are inconclusive.

\RefEquationList{illegal-matches-def-locs, illegal-matches-use-locs} degrade
solving time in this experiment because they are expensive to compute and only
identify \gls{redundant.m} \glspl{match} that appear for \glspl{target machine}
with irregular \glspl{instruction set} (meaning the \glspl{instruction} access
different sets of \glspl{register}).
%
Since \gls{Hexagon} has a regular \gls{instruction set}, these \gls{presolving}
techniques identify too few \gls{redundant.m} \glspl{match} to offset the cost
in finding them.


\paragraph{Impact of Groups of Techniques}

We find the best combination of \gls{presolving} techniques by arranging them
into groups and evaluating the impact of each such group.
%
Again, for practicality we limit ourselves to only comparing the solving times
exhibited by four versions of the \glsshort{constraint model}:
%
\begin{modelList}
  \item \labelModel{pre-ex-no-techs}
    one with no \gls{presolving} techniques
\gls{implied.c} \glspl{constraint}
  \item \labelModel{pre-ex-only-good-techs}
    one with only those who individually lead to an overall solving time
    improvement
  \item \labelModel{pre-ex-no-bad-techs}
    one without those who individually lead to an overall solving time
    degradation
  \item \labelModel{pre-ex-all-techs}
    one with all techniques
\end{modelList}.
%
We expect \glsplshort{constraint model}~\refModel{pre-ex-only-good-techs},
\refModel{pre-ex-no-bad-techs}, and~\refModel{pre-ex-all-techs} to all perform
better than \glsshort{constraint model}~\refModel{pre-ex-no-techs}.
%
This is because many of these \gls{presolving} techniques require computations
that are expensive to execute but whose result can be shared among the
techniques, allowing this cost to be amortized when the techniques are executed
in unison.
%
No hypothesis is attempted regarding the relative performance between
\glsplshort{constraint model}~\refModel{pre-ex-only-good-techs},
\refModel{pre-ex-no-bad-techs}, and~\refModel{pre-ex-all-techs}.

\RefFigure{diff-presolving-comb-solving-time-plot} shows the normalized solving
times (including \gls{presolving} time) for the four \glspl{constraint model}
described above in the second experiment, with \glsshort{constraint
  model}~\refModel{pre-ex-no-techs} as \gls{baseline} and \glsplshort{constraint
  model}~\refModel{pre-ex-only-good-techs}, \refModel{pre-ex-no-bad-techs},
and~\refModel{pre-ex-all-techs} as \glspl{subject}.
%
\begin{figure}
  \centering%

  \mkSolvTechSubfigure[enable]%
                      {only-good-presolving}%
                      [%
                        Model \refModel{pre-ex-no-techs} \versus
                        \refModel{pre-ex-only-good-techs}%
                      ]%
                      {OnlyGoodPresolving}%
  \hfill%
  \mkSolvTechSubfigure{bad-presolving}%
                      [%
                        Model \refModel{pre-ex-no-techs} \versus
                        \refModel{pre-ex-no-bad-techs}%
                      ]%
                      {BadPresolving}

  \vspace{\betweensubfigures}

  \mkSolvTechSubfigure{all-presolving}%
                      [%
                        Model \refModel{pre-ex-no-techs} \versus
                        \refModel{pre-ex-all-techs}%
                      ]%
                      {AllPresolving}

  \caption[%
            Plot for evaluating the impact on solving time made by different
            combinations of presolving techniques%
          ]%
          {%
            Normalized solving times (incl.\ presolving time) for four
            constraint models:
            %
            \begin{modelList}
              \item one with no presolving techniques (baseline)
              \item one with only those who individually lead to an overall
                solving time improvement (\refEquationList{redun-kills,
                  redun-non-null-copy-matches}, and \gls{canonical.l}
                \glspl{location}; subject)
              \item one without those who individually lead to an overall
                solving time degradation
                (\refEquationList{illegal-matches-def-locs,
                  illegal-matches-use-locs}; subject)
              \item one with all techniques (subject)
            \end{modelList}%
          }
  \labelFigure{diff-presolving-comb-solving-time-plot}
\end{figure}
%
The solving times range from
\printMinSolvingTime{
  \SolvTechEnableOnlyGoodPresolvingPrePlusSolvingTimeSpeedupNonePrePlusSolvingTimeAvgMin,
  \SolvTechEnableOnlyGoodPresolvingPrePlusSolvingTimeSpeedupPrePlusSolvingTimeAvgMin,
  \SolvTechDisableBadPresolvingPrePlusSolvingTimeSpeedupPrePlusSolvingTimeAvgMin,
  \SolvTechDisableAllPresolvingPrePlusSolvingTimeSpeedupPrePlusSolvingTimeAvgMin
} to
\printMaxSolvingTime{
  \SolvTechEnableOnlyGoodPresolvingPrePlusSolvingTimeSpeedupNonePrePlusSolvingTimeAvgMax,
  \SolvTechEnableOnlyGoodPresolvingPrePlusSolvingTimeSpeedupPrePlusSolvingTimeAvgMax,
  \SolvTechDisableBadPresolvingPrePlusSolvingTimeSpeedupPrePlusSolvingTimeAvgMax,
  \SolvTechDisableAllPresolvingPrePlusSolvingTimeSpeedupPrePlusSolvingTimeAvgMax
} with a \gls{CV} of
\numMaxOf{
  \SolvTechEnableOnlyGoodPresolvingPrePlusSolvingTimeSpeedupNonePrePlusSolvingTimeCvMax,
  \SolvTechEnableOnlyGoodPresolvingPrePlusSolvingTimeSpeedupPrePlusSolvingTimeCvMax,
  \SolvTechDisableBadPresolvingPrePlusSolvingTimeSpeedupPrePlusSolvingTimeCvMax,
  \SolvTechDisableAllPresolvingPrePlusSolvingTimeSpeedupPrePlusSolvingTimeCvMax
}.

To begin with, we observe that the \gls{GMI} for \glsshort{constraint
  model}~\refModel{pre-ex-only-good-techs} over \glsshort{constraint
  model}~\refModel{pre-ex-no-techs} is \printGMI{%
  \SolvTechEnableOnlyGoodPresolvingPrePlusSolvingTimeSpeedupPrePlusSolvingTimeRegularSpeedupGmean%
} with \gls{CI}~\printGMICI{%
  \SolvTechEnableOnlyGoodPresolvingPrePlusSolvingTimeSpeedupPrePlusSolvingTimeRegularSpeedupCiMin%
}{%
  \SolvTechEnableOnlyGoodPresolvingPrePlusSolvingTimeSpeedupPrePlusSolvingTimeRegularSpeedupCiMax%
} (\refFigure{solv-tech-enable-only-good-presolving}), which is a statistically
significant, positive impact.
%
But although the improvement is considerable for some \glspl{function} (up to
\printZCNorm{%
  \SolvTechEnableOnlyGoodPresolvingPrePlusSolvingTimeSpeedupPrePlusSolvingTimeZeroCenteredSpeedupMax%
}), we note that this combination yields a considerable degradation for others
(down
to~\printZCNorm{%
  \SolvTechEnableOnlyGoodPresolvingPrePlusSolvingTimeSpeedupPrePlusSolvingTimeZeroCenteredSpeedupMin%
}).

In comparison, we observe that the \gls{GMI} for \glsshort{constraint
  model}~\refModel{pre-ex-no-bad-techs} over \glsshort{constraint
  model}~\refModel{pre-ex-no-techs} is \printGMI{%
  \SolvTechDisableBadPresolvingPrePlusSolvingTimeSpeedupPrePlusSolvingTimeRegularSpeedupGmean%
} with \gls{CI}~\printGMICI{%
  \SolvTechDisableBadPresolvingPrePlusSolvingTimeSpeedupPrePlusSolvingTimeRegularSpeedupCiMin%
}{%
  \SolvTechDisableBadPresolvingPrePlusSolvingTimeSpeedupPrePlusSolvingTimeRegularSpeedupCiMax%
} (\refFigure{solv-tech-disable-bad-presolving}), which is a greater solving
time improvement compared to \glsshort{constraint
  model}~\refModel{pre-ex-only-good-techs}.
%
In addition, the maximum improvement is greater than for \glsshort{constraint
  model}~\refModel{pre-ex-only-good-techs} (up to
\printZCNorm{%
  \SolvTechDisableBadPresolvingPrePlusSolvingTimeSpeedupPrePlusSolvingTimeZeroCenteredSpeedupMax%
}), and this combination yields no considerable degradation for any
\gls{function} (down to at
most~\printZCNorm{%
  \SolvTechDisableBadPresolvingPrePlusSolvingTimeSpeedupPrePlusSolvingTimeZeroCenteredSpeedupMin%
}).

Lastly, we observe that the \gls{GMI} for \glsshort{constraint
  model}~\refModel{pre-ex-all-techs} over \glsshort{constraint
  model}~\refModel{pre-ex-no-techs} is \printGMI{%
  \SolvTechDisableAllPresolvingPrePlusSolvingTimeSpeedupPrePlusSolvingTimeRegularSpeedupGmean%
} with \gls{CI}~\printGMICI{%
  \SolvTechDisableAllPresolvingPrePlusSolvingTimeSpeedupPrePlusSolvingTimeRegularSpeedupCiMin%
}{%
  \SolvTechDisableAllPresolvingPrePlusSolvingTimeSpeedupPrePlusSolvingTimeRegularSpeedupCiMax%
} (\refFigure{solv-tech-disable-all-presolving}), which is still better than
\glsshort{constraint model}~\refModel{pre-ex-only-good-techs} but worse than
\glsshort{constraint model}~\refModel{pre-ex-no-bad-techs}.
%
In addition, the maximum improvement is less than for \glsshort{constraint
  model}~\refModel{pre-ex-no-bad-techs} (up to at
most~\printZCNorm{%
  \SolvTechDisableAllPresolvingPrePlusSolvingTimeSpeedupPrePlusSolvingTimeZeroCenteredSpeedupMax%
}).
%
Hence, in this experiment it is most beneficial to exclude
\refEquationList{illegal-matches-def-locs, illegal-matches-use-locs} from the
\glsshort{constraint model}.
%
However, for \glspl{target machine} with irregular \glspl{instruction set} it
may be beneficial to keep them in the \glsshort{constraint model}.


\paragraph{Conclusions}

From the results for these experiments, we conclude:
%
\begin{enumerate*}[label=(\roman*), itemjoin={;\ }, itemjoin*={; and\ }]
  \item that some of the \gls{presolving} techniques have a positive impact on
    solving time while others have a negative impact
  \item that it is most beneficial to exclude the techniques that have a
    negative impact from the \glsshort{constraint model}
\end{enumerate*}.


\subsection{Impact of All Solving Techniques}

We now evaluate the solving-time impact made by different collections of
solving techniques by comparing the solving times exhibited by four versions of
the \gls{constraint model}:
%
\begin{modelList}
  \item \labelModel{st-ex-no-techs}
    one with no solving techniques
  \item \labelModel{st-ex-only-good-techs}
    one with only those who individually lead to an overall solving time
    improvement
  \item \labelModel{st-ex-no-bad-techs}
    one with only those who individually lead to an overall solving time
    degradation
  \item \labelModel{st-ex-all-techs}
    one with all techniques
\end{modelList}.
%
We expect \glsplshort{constraint model}~\refModel{st-ex-only-good-techs},
\refModel{st-ex-no-bad-techs}, and~\refModel{st-ex-all-techs} to all perform
better than \glsshort{constraint model}~\refModel{st-ex-no-techs}.
%
No hypothesis is attempted regarding the relative performance between
\glsplshort{constraint model}~\refModel{st-ex-only-good-techs},
\refModel{st-ex-no-bad-techs}, and~\refModel{st-ex-all-techs}.

\RefFigure{diff-solv-tech-comb-solving-time-plot} shows the normalized solving
times (including \gls{presolving} time) for the four \glspl{constraint model}
described above, with \glsshort{constraint model}~\refModel{st-ex-no-techs} as
\gls{baseline} and \glsplshort{constraint
  model}~\refModel{st-ex-only-good-techs}, \refModel{st-ex-no-bad-techs},
and~\refModel{st-ex-all-techs} as \glspl{subject}.
%
\begin{figure}
  \centering%

  \mkSolvTechSubfigure[enable]%
                      {only-all-good}%
                      [%
                        Model \refModel{st-ex-no-techs} \versus
                        \refModel{st-ex-only-good-techs}%
                      ]%
                      {OnlyAllGood}%
  \hfill%
  \mkSolvTechSubfigure{all-bad}%
                      [%
                        Model \refModel{st-ex-no-techs} \versus
                        \refModel{st-ex-no-bad-techs}%
                      ]%
                      {AllBad}

  \vspace{\betweensubfigures}

  \mkSolvTechSubfigure{all}%
                      [%
                        Model \refModel{st-ex-no-techs} \versus
                        \refModel{st-ex-all-techs}%
                      ]%
                      {All}

  \caption[%
            Plot for evaluating the impact on solving time made by different
            combinations of all solving techniques%
          ]%
          {%
            Normalized solving times (incl.\ presolving time) for four
            constraint models:
            %
            \begin{modelList}
              \item one with no solving techniques (baseline)
              \item one with only those who individually lead to an overall
                solving time improvement (subject)
                (\refEquationList{impl-cons-defs-dominate-defs,
                  impl-cons-defs-dominate-entry-blocks,
                  dom-cons-interch-data-chains, redun-kills,
                  redun-non-null-copy-matches}, and \gls{canonical.l}
                \glspl{location}; subject)
              \item one without those who individually lead to an overall
                solving time degradation
                (\refEquationList{illegal-matches-def-locs,
                  illegal-matches-use-locs}; subject)
              \item one with all techniques (subject)
            \end{modelList}.
            %
            \Glspl{function} marked with \barNormValueNoBaselineSolution{} are
            those for which model~\refModel{st-ex-no-techs} fails to produce any
            solution%
          }
  \labelFigure{diff-solv-tech-comb-solving-time-plot}
\end{figure}
%
The solving times range from
\printMinSolvingTime{
  \SolvTechEnableOnlyAllGoodPrePlusSolvingTimeSpeedupNonePrePlusSolvingTimeAvgMin,
  \SolvTechEnableOnlyAllGoodPrePlusSolvingTimeSpeedupPrePlusSolvingTimeAvgMin,
  \SolvTechDisableAllBadPrePlusSolvingTimeSpeedupPrePlusSolvingTimeAvgMin,
  \SolvTechDisableAllPrePlusSolvingTimeSpeedupPrePlusSolvingTimeAvgMin
} to
\printMaxSolvingTime{
  \SolvTechEnableOnlyAllGoodPrePlusSolvingTimeSpeedupNonePrePlusSolvingTimeAvgMax,
  \SolvTechEnableOnlyAllGoodPrePlusSolvingTimeSpeedupPrePlusSolvingTimeAvgMax,
  \SolvTechDisableAllBadPrePlusSolvingTimeSpeedupPrePlusSolvingTimeAvgMax,
  \SolvTechDisableAllPrePlusSolvingTimeSpeedupPrePlusSolvingTimeAvgMax
} with a \gls{CV} of
\numMaxOf{
  \SolvTechEnableOnlyAllGoodPrePlusSolvingTimeSpeedupNonePrePlusSolvingTimeCvMax,
  \SolvTechEnableOnlyAllGoodPrePlusSolvingTimeSpeedupPrePlusSolvingTimeCvMax,
  \SolvTechDisableAllBadPrePlusSolvingTimeSpeedupPrePlusSolvingTimeCvMax,
  \SolvTechDisableAllPrePlusSolvingTimeSpeedupPrePlusSolvingTimeCvMax
}.
%
The \glspl{GMI} and \glspl{CI} are given in
\refFigure{diff-solv-tech-comb-solving-time-plot}.

To begin with, we see clearly that all \glsplshort{constraint model}
significantly improve solving time over \glsshort{constraint
  model}~\refModel{st-ex-no-techs}.
%
For several \glspl{function}, the improvement is considerable (up
to~\printMaxZCNorm{%
  \SolvTechEnableOnlyAllGoodPrePlusSolvingTimeSpeedupPrePlusSolvingTimeZeroCenteredSpeedupMax,
  \SolvTechDisableAllBadPrePlusSolvingTimeSpeedupPrePlusSolvingTimeZeroCenteredSpeedupMax,
  \SolvTechDisableAllPrePlusSolvingTimeSpeedupPrePlusSolvingTimeZeroCenteredSpeedupMax
}), and no combination yields considerable degradation for any \gls{function}
(down
to at most~\printMinSpeedup{%
  \SolvTechEnableOnlyAllGoodPrePlusSolvingTimeSpeedupPrePlusSolvingTimeZeroCenteredSpeedupMin,
  \SolvTechDisableAllBadPrePlusSolvingTimeSpeedupPrePlusSolvingTimeZeroCenteredSpeedupMin,
  \SolvTechDisableAllPrePlusSolvingTimeSpeedupPrePlusSolvingTimeZeroCenteredSpeedupMin
}).
%
In fact, for two \glspl{function} \glsshort{constraint
  model}~\refModel{st-ex-no-techs} is not even able to produce a \gls{solution}
within the time limit.
%
Hence the solving techniques are crucial for scalability.

Next, we observe that \glsshort{constraint model}~\refModel{st-ex-no-bad-techs}
yields the largest \gls{GMI} over \glsshort{constraint
  model}~\refModel{st-ex-no-techs} (\printGMI{%
  \SolvTechDisableAllBadPrePlusSolvingTimeSpeedupPrePlusSolvingTimeRegularSpeedupGmean%
} with \gls{CI}~\printGMICI{%
  \SolvTechDisableAllBadPrePlusSolvingTimeSpeedupPrePlusSolvingTimeRegularSpeedupCiMin%
}{%
  \SolvTechDisableAllBadPrePlusSolvingTimeSpeedupPrePlusSolvingTimeRegularSpeedupCiMax%
}), closely followed by \glsshort{constraint model}~\refModel{st-ex-all-techs}
(\printGMI{%
  \SolvTechDisableAllPrePlusSolvingTimeSpeedupPrePlusSolvingTimeRegularSpeedupGmean%
} with \gls{CI}~\printGMICI{%
  \SolvTechDisableAllPrePlusSolvingTimeSpeedupPrePlusSolvingTimeRegularSpeedupCiMin%
}{%
  \SolvTechDisableAllPrePlusSolvingTimeSpeedupPrePlusSolvingTimeRegularSpeedupCiMax%
}),
which in turn considerably outperforms \glsshort{constraint
  model}~\refModel{st-ex-only-good-techs}
(\printGMI{%
  \SolvTechEnableOnlyAllGoodPrePlusSolvingTimeSpeedupPrePlusSolvingTimeRegularSpeedupGmean%
} with \gls{CI}~\printGMICI{%
  \SolvTechEnableOnlyAllGoodPrePlusSolvingTimeSpeedupPrePlusSolvingTimeRegularSpeedupCiMin%
}{%
  \SolvTechEnableOnlyAllGoodPrePlusSolvingTimeSpeedupPrePlusSolvingTimeRegularSpeedupCiMax%
}).
%
We also note that the maximum improvement for \glsshort{constraint
  model}~\refModel{st-ex-no-bad-techs} is greater than for \glsshort{constraint
  model}~\refModel{st-ex-all-techs} (up to \printZCNorm{%
  \SolvTechDisableAllBadPrePlusSolvingTimeSpeedupPrePlusSolvingTimeZeroCenteredSpeedupMax%
} \versus up to \printZCNorm{%
  \SolvTechDisableAllPrePlusSolvingTimeSpeedupPrePlusSolvingTimeZeroCenteredSpeedupMax
}).
%
Hence, only picking solving techniques that have a statistically significant,
positive effect on solving time when evaluated individually is too conservative.
%
Also, rejecting the \gls{presolving} techniques that have an overall negative
effect on solving time when evaluated individually yields, in this experiment,
the \gls{constraint model} with best performance.
%
Keep in mind, however, that for \glspl{target machine} with irregular
\glspl{instruction set} it may be beneficial to keep them in the
\glsshort{constraint model}.


\paragraph{Conclusions}

From the results for these experiments, we conclude:
%
\begin{enumerate*}[label=(\roman*), itemjoin={;\ }, itemjoin*={; and\ }]
  \item that the solving techniques introduced in this chapter are crucial for
    scalability
  \item that picking only solving techniques that have a statistically
    significant, positive effect on solving time when evaluated individually is
    too conservative
  \item that rejecting \refEquationList{illegal-matches-def-locs,
    illegal-matches-use-locs} gives the \gls{constraint model} with best
    performance for \glspl{target machine} with regular \glspl{instruction set}
\end{enumerate*}.


\section{Summary}
\labelSection{st-summary}

In this chapter, we have introduced a wide range of techniques for improving
solving of the \gls{constraint model} introduced in the previous chapter.
%
Through experimental evaluation, the techniques were demonstrated to be crucial
for scalability; for one \gls{function}, the solving time was improved
by~\printMaxZCNorm{%
  \SolvTechEnableOnlyAllGoodPrePlusSolvingTimeSpeedupPrePlusSolvingTimeZeroCenteredSpeedupMax,
  %  \SolvTechDisableAllBadPrePlusSolvingTimeSpeedupPrePlusSolvingTimeZeroCenteredSpeedupMax,
  % This measurement is not included here as we decide to keep all presolving
  % techniques
  \SolvTechDisableAllPrePlusSolvingTimeSpeedupPrePlusSolvingTimeZeroCenteredSpeedupMax
}.
%
Rejecting \gls{presolving} techniques \refEquationList{illegal-matches-def-locs,
  illegal-matches-use-locs} increased the improvement further to \printZCNorm{%
  \SolvTechDisableAllBadPrePlusSolvingTimeSpeedupPrePlusSolvingTimeZeroCenteredSpeedupMax%
} since the \gls{target machine} is fairly regular (meaning its
\glspl{instruction} can access the same set of \glspl{register}).
%
If the \gls{target machine} has an irregular \gls{instruction set}, however,
then it may be worthwhile to keep these \gls{presolving} techniques.
