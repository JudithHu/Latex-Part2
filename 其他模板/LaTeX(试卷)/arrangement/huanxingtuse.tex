\section{环形涂色问题}
环形涂色问题又称为多边形的涂色问题,在一般的题型中,可将题意抽象为环形涂色问题,该问题的一般化为:用$m(m\ge 3)$种不同颜色给$n$边形$A_1,A_2,A_3\dots A_n$各顶点涂色,且相邻顶点不同色,则不同的涂色方案有$a_n$种.
\newtheorem{theorem}{定理}[section]
\begin{theorem}
设环形涂色的方案数为 ,则$a_n$的递推公式为:$$\Bigg\{\begin{aligned} &a_n=m(m-1)^{n-1}-a_{n-1}\\&a_3=m(m-1)(m-2)\end{aligned}$$
\end{theorem}
\begin{proof}
如图所示:在$A_1$处有$m$种涂色方案,在$A_2,A_3\dots A_{n-1}$处
有$m-1$种涂色方案,此时考虑$A_n$也有$m-1$种涂色方案在此情况下,
有两种情况:
\begin{enumerate}[1)]
\item$A_n$与$A_1$同色,此时相当于$A_n$与$A_1$重合,这时问题转化为$m$种不同颜色给$n-1$边形涂色,即为$a_{n-1}$种涂色方案;
\item $A_n$与$A_1$不同色,此时问题就转化为用$m$种不同颜色给$n$边形的各顶点涂色,且相邻顶点不同色,即此时的情况就是$a_n$。根据分类原理可知$m(m-1)^{n-1}=a_n+a_{n-1}$ ,且满足初始条件:$a_3=m(m-1)(m-2)$
即递推公式为:$$\Bigg\{\begin{aligned} &a_n=m(m-1)^{n-1}-a_{n-1}\\&a_3=m(m-1)(m-2)\end{aligned}$$
\end{enumerate} 
\end{proof}
\begin{theorem}
设环形涂色的方案数为$a_n$,则$a_n$的通项公式为 $a_n=(m-1)^{n}+(-1)^n(m-1)$
\end{theorem}
\begin{proof}
根据定理一的递推公式,有:
$$\begin{aligned}a_n=&m(m-1)^{n-1}-a_{n-1}\\=&(m-1+1)(m-1)^{n-1}-a_{n-1}\\=&\dfrac{(m-1)^{n}}{(m-1)^{n-1}}-a_{n-1}\end{aligned}$$
所以$$\begin{aligned}a_n-(m-1)^n=&-[a_{n-1}-(m-1)^{n-1}]\\=&[m(m-1)(m-2)-(m-1)^3](-1)^{n-3}\\=&(m-1)(-1)^n\end{aligned}$$
所以$$a_n=(m-1)^n+(-1)^n(m-1)$$
\end{proof}