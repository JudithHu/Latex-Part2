\section{利用导数的概念解题}
\subsection{导数的定义}
\wz{若函数$f(x)$的在$x_0$附近有定义,当自变量$x$在$x_0$处取得一个增量$ \triangle x $时$ (\triangle x\text{充分小}) $,因变量$ y $也随之取得增量$ \triangle y~\left(\triangle y=f(x_0+\triangle x)-f(x_0)\right). $若$ \lim\limits_{\triangle x \to 0}\dfrac{\triangle y}{\triangle x} $存在,则称$f(x)$在$x_0$处可导,此极限值称为$ f(x) $在点$x_0$处的导数(或变化率),记作$ f'(x_0) $或$ \left.y'~\right|_{x=x_0} $或$\left.\dfrac{ dy}{dx }~\right|_{x_0}$,即$ f'(x_0)= \lim\limits_{\triangle x \to 0}\dfrac{f(x)-f(x_0)}{x-x_0}$.}
\subsection{常用函数的导数和基本运算}
\subsubsection{常用函数的导数}
\begin{center}
\begin{tabular}{|c|c|}
\hline
原函数&导数\\
\hline
$y=C~(C\text{为常数})$&$y'=0$\\
\hline
$y=x^n~(n\in\mathbf{Q^*})$&$y'=nx^{n-1}$\\
\hline
$y=\sin x$&$y'=\cos x$\\
\hline
$y=\cos x$&$y'=-\sin x$\\
\hline
$y=e^x$&$y'=e^x$\\
\hline
$y=\ln x$&\Gape[9pt]{$y'=\dfrac{1}{x}$}\\
\hline
\end{tabular}
\end{center}
\subsubsection{四则运算}
\begin{enumerate}[1)]
\item $ \left(f(x)\pm g(x)\right)'=f'(x)\pm g'(x) $;
\item $\left(f(x)g(x)\right)'=f'(x)g(x)+f(x)g'(x)$;
\item $\left(\dfrac{f(x)}{g(x)}\right)'=\dfrac{f'(x)g(x)-f(x)g'(x)}{\left[g(x)\right]^2}$
\end{enumerate}
\subsubsection{复合函数导数}
$ y=f\left[u(x)\right] $的导函数为$ y'_x=y'_u\bm{\cdot}u'_x~(\text{其中}~y'_x~\text{表示}~y~\text{关于}~x~\text{的导数}) $.
\begin{proof}
将$ y=f\left[u(x)\right] $分拆成$ \Bigg\{\begin{aligned}
y=f(u)\\
u=u(x).
\end{aligned} $.根据导数的定义:\begin{equation*}
\begin{aligned}
y'_x&=\lim \limits_{\Delta x \to 0}\dfrac{\Delta y}{\Delta x}=\lim \limits_{\Delta x \to 0}\dfrac{\Delta y}{\Delta u}\bm{\cdot}\dfrac{\Delta u}{\Delta x}\\
&=\lim \limits_{\Delta x \to 0}\dfrac{\Delta y}{\Delta u}\bm{\cdot}\lim \limits_{\Delta x \to 0}\dfrac{\Delta u}{\Delta x}\\
&=y'_u\bm{\cdot}u'_x
\end{aligned}
\end{equation*}
\end{proof}
\subsection{练习}
\begin{questions}

\question 已知函数$f(x)=\begin{dcases}
\,\dfrac{1}{2}(x^2+1),&(x\le 1)\\
\,\dfrac{1}{2}(x+1),&(x>1)
\end{dcases}$判断$f(x)$在$ x=1 $处是否可导?
\qs $f(x)=(x-a)(x-b)(x-c)$,~则$ \dfrac{a^2}{f'(a)}+\dfrac{b^2}{f'(b)}+\dfrac{c^2}{f'(c)}= $\xx
\onech{$ 1 $}{$ -1 $}{$ a+b+c $}{$ ab+bc+ca $}
\qs $f(x)$与$g(x)$是定义在$ \mathbf{R} $上的两个可导函数,若$ f(x),~g(x) $满足$ f'(x)=g'(x) $,则\xx
\twoch{$ f(x)=g(x) $}{$ f(x)-g(x) $为常数函数}{$ f(x)=g(x)=0 $}{$ f(x)+g(x) $为常数函数}
\qs 设$ f(x)=x(x-1)(x-2)\cdots(x-1000) $,则$ f'(0)= $\tk.
\qs 下列说法正确的是\tk.\\
\ding{192} $ f'(x_0) $和$ f'(x) $都称为$ f(x)$ 的导数,它们有相同的意义;\\
\ding{193} $ f'(x_0) $是$f(x_0)$的导数;\\
\ding{194} $ f'(x_0) $是$f'(x)$在$ x=x_0 $时的函数值.
\qs 函数$ y=(\sin x^2)^3 $的导数是\tk.
\qs 函数$f(x)=\dfrac{e^x}{x^2+ax+a}$的导数是\tk.
\end{questions}