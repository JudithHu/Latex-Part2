 \documentclass{BHCexam}	

\begin{document}
\biaoti{北京高考分项练习——导数}
\fubiaoti{}
\maketitle
\begin{questions}
	\qs (2018理)设函数$f(x)=\left[ax^2-\left(4a+1\right)x+4a+3\right]e^x$.
	\begin{parts}
		\part 若曲线$y=f(x)$在点$ \left(1,f(1)\right) $处的切线与$x$轴平行,求$ a $;
		\part 若$f(x)$在$ x=2 $处取得极小值,求$ a $的范围.
	\end{parts}
\kongbai
\qs (2018文) 设函数$f(x)=\left[ax^2-\left(3a+1\right)x+3a+2\right]e^x$.
\begin{parts}
	\part 若曲线$y=f(x)$在点$ \left(2,f(2)\right) $处的切线斜率为$ 0 $,求$ a $;
	\part 若$f(x)$在$ x=1 $处取得极小值,求$ a $的取值范围.
\end{parts}
\kongbai
\qs (2017文理)已知函数$f(x)=e^x\cos x-x$.
\begin{parts}
\part 求曲线$y=f(x)$在点$ \left(0,f\left(0\right)\right) $处的切线方程;
\part 求函数$f(x)$在区间$ \left[0,\dfrac{\pi}{2}\right] $上的最大值和最小值.
\end{parts}
\kongbai
\question
(2016理)设函数$f(x)=x{{e}^{a-x}}+bx$,曲线$y=f(x)$在点$(2,f(2))$处的切线方程为$y=(e-1)x+4$.
\begin{parts}
\part[5]求a,b的值;
\part[8]求$f\left( x\right) $的单调区间.
\end{parts}
\kongbai
%\kongbai
\question
(2016文)设函数$f\left( x\right) =x^3+ax^2+bx+c$.
\begin{parts}
\part[3]求曲线$y=f\left(x\right)$在点$\left(0,f(0)\right)$处的切线方程;
\part[5]设$a=b=4$,若函数$y=f\left(x\right)$有三个不同零点,求$c$的取值范围;
\part[5]求证:$a^2-3b>0$是$y=f\left(x\right)$有三个不同零点的必要而不充分条件.
\end{parts}

\kongbai
\question
(2015理)已知函数$f(x)=\ln\dfrac{1+x}{1-x}$.
\begin{parts}
\part[3]求曲线$y=f(x)$在点$\left(0,f(0)\right)$处的切线方程;
\part[5]求证:当$x\in(0,1)$时,$f(x)>2\left( x+\dfrac{x^3}{3}\right) $;
\part[5]设实数$k$使得$f(x)>k\left( x+\dfrac{x^3}{3}\right)$对$x\in(0,1)$恒成立,求$k$的最大值.
\end{parts}
\kongbai
\question
(2015文)设函数$f(x)=\dfrac{{{x}^{2}}}{2}-k\ln x,k>0$.
\begin{parts}
\part[5]求$f(x)$的单调区间和极值;
\part[8]证明:若$f(x)$存在零点,则$f(x)$在区间$\left(1,\sqrt{e}\right]$上仅有一个零点.
\end{parts}
\kongbai
\question
(2014理)已知函数$f(x)=x\cos x-\sin x,x\in\left[0,\dfrac{\pi}{2}\right]$.
\begin{parts}
\part[5]求证:$f(x)\le0$;
\part[8]若$a$<$\dfrac{\sin x}{x}<b$在$\left(0,\dfrac{\pi}{2}\right)$上恒成立,求$a$的最大值和$b$的最小值.
\end{parts}
\kongbai
\question
(2014文)已知函数$f(x)=2{{x}^{3}}-3x$.
\begin{parts}
\part[3]求$f(x)$在区间$\left[-2,1\right]$上的最大值;
\part[5]若过点$P(1,t)$存在$3$条直线与曲线$y=f(x)$相切,求$t$的取值范围;
\part[5]问过点$A(-1,2),\,B(2,10),\,C(0,2)$分别存在几条直线与曲线$y=f(x)$相切\text{?}(只需写出结论)
\end{parts}
\kongbai
\question
(2013理)设$l$为曲线$C$:~$y=\dfrac{\ln x}{x}$在点$\left(1,0\right)$处的切线.
\begin{parts}
\part[5]求$l$的方程;
\part[8]证明:除切点$(1,0)$之外,曲线$C$在直线$l$的下Z方.
\end{parts}
\kongbai
\question
(2013文)已知函数$f(x)=x^2+x\sin x+\cos x$.
\begin{parts}
\part[5]若曲线$y=f(x)$在点$(a,f(a))$处与直线$y=b$相切,求$a$与$b$的值;
\part[8]若曲线$y=f(x)$与直线$y=b$有两个不同的交点,求$b$的取值范围.
\end{parts}
\kongbai
\question
(2012理)已知函数$f(x)=ax^2+1~(a>0),\ g(x)=x^3+bx$.
\begin{parts}
\part[5] 若曲线$y=f(x)$与曲线$y=g(x)$在它们的交点$(1,c)$处具有公共切线,求$a,\,b$的值;
\part[8] 当$a^2=4b$时,求函数$f(x)+g(x)$的单调区间,并求其在区间$(-\infty,-1]$上的最大值.
\end{parts}
\kongbai
\question
(2012文)已知函数$f(x)=ax^2+1(a>0)$,$g(x)=x^3+bx$.
\begin{parts}
\part[5]若曲线$y=f(x)$与曲线$y=g(x)$在它们的交点$(1,c)$处具有公共切线,求$a,\,b$的值;
\part[8]当$a=3,\,b=-9$时,若函数$f(x)+g(x)$在区间$\left[k,2\right]$上的最大值为$28$,求$k$的取值范围.
\end{parts}
\kongbai
\question
(2011理)已知函数$f(x)=(x-k)^2e^\tfrac{x}{k}$.
\begin{parts}
\part[5]求$f(x)$的单调区间;
\part[8]若对于任意的$x\in(0,+\infty)$,都有$f(x)\le\dfrac{1}{e}$,求$k$的取值范围
\end{parts}
\kongbai
\question
(2011文)已知函数$f(x)=(x-k)e^x$.
\begin{parts}
\part[5]求$f(x)$的单调区间;
\part[8]求$f(x)$在区间$\left[0,1\right] $上的最小值.
\end{parts}
\kongbai
\question
(2010理)已知函数$f(x)=\ln\left(1+x\right)-x+\dfrac{k}{2}x^2~\left(k\ge0\right)$
\begin{parts}
\part[5]当$k=2$时,求曲线$y=f(x)$在点$(1,f(1))$处的切线方程;
\part[8]求$f(x)$的单调区间.
\end{parts}
\kongbai
\question
(2010文)设定函数$f(x)=\dfrac{a}{3}x^3+bx^2+cx+d~(a>0)$,且方程$f'(x)-9x=0$的两个根分别为$1,\,4$.
\begin{parts}
\part[5]当$a=3$且曲线$y=f(x)$过原点时,求$f(x)$的解析式;
\part[8]若$f(x)$在$(-\infty,+\infty)$无极值点,求$a$的取值范围.
\end{parts}
\kongbai
\question
(2013新课标理)已知函数$f(x)=e^x-\ln (x+m)$.
\begin{parts}                    
\part 设$x=0$是$f(x)$的极值点,求$m$,并讨论$f(x)$的单调性;
\part 当$m\le 2$时,证明$f(x)>0$
\end{parts}
\kongbai
\question
(2012新课标理)已知函数$f(x)$满足$f(x)=f'(1)e^{x-1}-f(0)x+\dfrac{1}{2}x^2$.
\begin{parts}
\part 求$f(x)$的解析式及单调区间;
\part 若$f(x)\ge \dfrac{1}{2}x^2+ax+b$,求$(a+1)b$的最大值.
\end{parts}
\kongbai
\question
(2014新课标理)设函数$f(x)=ae^x\ln x+\dfrac{be^{x-1}}{x}$,曲线$y=f(x)$在点$(1,f(x))$处的切线方程为$y=e(x-1)+2$.
\begin{parts}
\part 求$a,\,b$;
\part 证明:$f(x)>1$.
\end{parts}
\kongbai
\question
设函数$f(x)=x^2+ax+b$,$g(x)=e^x(cx+d)$,若曲线$y=f(x)$和曲线$y=g(x)$都过点$P~(0,2)$,且在点$P$处有相同的切线$y=4x+2$.
\begin{parts}
\part 求$a,b,c,d$的值;
\part 若$x\ge-2$时,$f(x)\le kg(x)$,求$k$的取值范围.
\end{parts}
\kongbai
\question
已知函数$f(x)=(x-2)e^x+a(x-1)^2$.
\begin{parts}
\part 讨论$f(x)$的单调性;
\part 若$f(x)$有两个零点,求$a$的取值范围.
\end{parts}
\kongbai
\qs 已知函数$f(x)=(x+1)\ln x-a(x-1)$.
\begin{parts}
\part 当$a=4$时,求曲线$y=f(x)$在$(1,f(1))$处的切线方程;
\part 若当$x\in (1,+\infty)$时,$f(x)>0$,求$a$的取值范围.
\end{parts}
\newpage
\qs
已知函数$f(x)=x^3+3ax^2+3x+1$.
\begin{parts}
\part 当$a=-\sqrt{2}$时,讨论$f(x)$的单调性;
\part 若$x \in \left[2,+\infty\right)$时,$f(x)\ge0$,求$a$的取值范围.
\end{parts}
\kongbai
\qs
设函数$f(x)=e^x-ax-2$.
\begin{parts}
\part 求$ f(x)  $的单调区间;
\part 若$ a=1 $,$ k $为整数,且当$ x>0 $时,$ (x-k)f'(x)+x+1>0 $,求$ k $的最大值.
\end{parts}
\newpage
\qs 已知函数$f(x)=x-\ln (x+a)$的最小值$ 0 ,~$其中$ a>0. $
\begin{parts}
\part 求$ a $的值;
\part 若对任意的$ x\in \left[0,+\infty \right) $,有$ f(x)\le kx^2 $成立,求实数$ k $的最小值.
\end{parts} 
\kongbai
\qs 已知函数$f(x)=\dfrac{ 1}{ 2}x^2-ax+(a-1)\ln x,~a>1.$讨论函数$f(x)$的单调性.
\newpage
\qs 设函数$f(x)=\dfrac{1}{3}x^3-\dfrac{a}{2}x^2+1,~$其中$ a>0,~ $若过点$ (0,2) $可作曲线$ y=f(x) $的三条不同切线,求$ a $的取值范围.
\kongbai
\qs 设函数$f(x)=1-e^{-x}$,设当$ x\ge0 $时,$f(x)\le\dfrac{x}{ax+1}$ ,求$ a $的取值范围.
\newpage
\qs 已知函数$f(x)=\dfrac{1}{2}ax^2-(2a+1)x+2\ln x~(a\inR),~g(x)=x^2-2x$,若对任意的$ x_1\in\left(0,2\right] $,均存在$ x_2\in\left(0,2\right] $使得$ f(x_1)<g(x_2) $,求$ a $的取值范围.
\end{questions}
\end{document}