\documentclass{BHCexam}
\usepackage{enumerate}
\usepackage{setspace}
\begin{document}
\biaoti{函数}
\fubiaoti{}
\maketitle
\setcounter{tocdepth}{2}
\tableofcontents
\newpage
\section{定义域}

\begin{enumerate}[1)]
\item 分式分母不能为零;
\item 偶次方根的被开方数大于或等于零;
\item 对数的真数大于零;
\item 指数和对数底数大于零且不等于$ 1 $;
\item 零次或负次指数次幂的底数不为零;
\item 正切函数$ \tan x  $的定义域为$ \left\{x\left| x\inR ,~\text{且}x \ne k\pi+\dfrac{\pi}{2},k \inZ \right.\right\} $
\end{enumerate}
%\subsection*{练习:}
\begin{questions}
\qs 函数$f(x)=\sqrt{2^x-1}$的定义域是\xx
\onech{$ \left[0,+\infty\right)$}{$ \left[1,+\infty\right)$}{$ \left(-\infty,0\right]$}{$ \left(-\infty,1\right]$}
\qs 函数$f(x)=\dfrac{1}{\sqrt{\left(\log_2x\right)^2-1}}$的定义域为\xx
\onech{$ \left(0,\dfrac{1}{2}\right)$}{$ \left(2,+\infty\right)$}{$ \left(0,\dfrac{1}{2}\right)\bigcup\left(2,+\infty\right)$}{$ \left(0,\dfrac{1}{2}\right]\bigcup\left[2,+\infty\right)$}
\qs 函数$ y=\lg \left(1-\dfrac{1}{x}\right) $的定义域为\xx
\onech{$ \left\{x\left|~x<0\right.\right\}$}{$ \left\{x\left|~x>1\right.\right\}$}{$ \left\{x\left|~0<x<1\right.\right\}$}{$ \left\{x\left|~x<0\text{或}x>1\right.\right\}$}
\qs 函数$ y=\dfrac{1}{\log_2\left(x-2\right)} $的定义域为\xx
\onech{$ \left(-\infty,2\right)$}{$ \left(2,+\infty\right)$}{$ \left(2,3\right)\bigcup\left(3,+\infty\right)$}{$ \left(2,4\right)\bigcup\left(4,+\infty\right)$}
\qs 若$f(x)=\dfrac{1}{\sqrt{\log_{\frac{1}{2}}\left(2x+1\right)}}$,则$f(x)$的定义域为\xx
\onech{$ \left(-\dfrac{1}{2},0\right)$}{$ \left(-\dfrac{1}{2},+\infty\right)$}{$ \left(-\dfrac{1}{2},+\infty\right)$}{$ \left(0,+\infty\right)$}
\qs 设函数$f(x)=\lg \dfrac{2+x}{2-x}$,则$ f\left(\dfrac{x}{2}\right)+f\left(\dfrac{2}{x}\right) $的定义域为\xx
\onech{$\left(-4,0\right)\bigcup \left(0,4\right) $}{$\left(-4,-1\right)\bigcup \left(1,4\right) $}{$ \left(-2,-1\right)\bigcup \left(1,2\right)$}{$ \left(-4,-2\right)\bigcup \left(2,4\right)$}
\question
已知函数$f(x)$的定义域为$(-1,0)$,则函数$f(2x+1)$的定义域为\xx
\onech{$(-1,1)$}{$\left(-1,-\dfrac{1}{2}\right)$}{$(-1,0)$}{$\left(\dfrac{1}{2},1\right)$}
\question
已知函数$f(2x+1)$的定义域为$\left(-2,\dfrac{1}{2}\right)$,则函数$f(x)$的定义域为\xx
\onech{$ \left(-\dfrac{3}{2},\dfrac{1}{4}\right)$}{$ \left(-1,\dfrac{3}{2}\right)$}{$ \left(-3,2\right)$}{$ \left(-3,3\right)$}
\qs 下列函数中,其定义域和值域分别与函数$y=10^{\lg x}$的定义域和值域相同的是\xx
\onech{$y=x$}{$y=\lg x$}{$y=2^x$}{$y=\dfrac{1}{\sqrt{x}}$}
\end{questions}
\newpage 



\section{单调性}
\subsection{单调性的判定方法}
\begin{enumerate}[1)]
\item 定义法:对于任意的$ x_1,x_2\in D $,且$ x_1<x_2 $,若$ f(x_1)<f(x_2) $成立,则称$ f(x) $为增函数;若$ f(x_1)>f(x_2) $成立,则称$ f(x) $为减函数.
\item 导数法:设函数$f(x)$在定义域内可导,则:
\begin{enumerate}
\item $ f'(x) >0\Rightarrow f(x)$单调递增,$ f(x) $单调递增$ \Rightarrow f'(x)\ge 0 $;
\item $ f'(x) <0\Rightarrow f(x)$单调递减,$ f(x) $单调递减$ \Rightarrow f'(x)\le 0 $;	
\end{enumerate}
\item \CJKunderdot{分段函数}单调性:分段函数单调递增(递减)意味着每个分段的区间上函数单调递增(递减)并且在分段点处函数值的大小关系也满足\CJKunderdot{递增(递减)} 
\item 对于定义在$ D $上的函数$f(x)$,设$\forall x_1,~x_2 \in D,~x_1\ne x_2$,则有:
\begin{enumerate}
\item $\dfrac{f(x_1)-f(x_2)}{x_1-x_2}>0\Leftrightarrow f(x)$是$ D $上的单调递增函数;
\item  $\dfrac{f(x_1)-f(x_2)}{x_1-x_2}<0\Leftrightarrow f(x)$是$ D $上的单调递减函数;
\end{enumerate}
\item 复合函数单调性判定:同增异减
\end{enumerate}
\subsection*{求单调区间的方法}
\ding{192} 定义法\qquad \ding{193}导数法\qquad \ding{194}图象法
\newpage 
\subsection*{练习}
\begin{questions}

\qs 已知函数$f(x)=\ln \left(1+x\right)-\ln \left(1-x\right)$,则$f(x)$是\xx
\fourch{奇函数,且在$ (0,1) $上是增函数}{奇函数,且在$ (0,1) $上是减函数}{偶函数,且在$ (0,1) $上是增函数}{偶函数,且在$ (0,1) $上是减函数}
\qs 设$f(x)=\Bigg\{\begin{aligned}
&a^x,&x<0\\
&(a-3)x+4a,&x\ge 0
\end{aligned}$对任意的$ x_1\ne x_2 $都有$ \dfrac{f(x_1)-f(x_2)}{x_1-x_2}<0 $成立,则$ a $的取值范围是\xx
\onech{$ \left(0,\dfrac{1}{4}\right]$}{$ \left(0,1\right)$}{$ \left[\dfrac{1}{4},1\right)$}{$ \left(0,3\right)$}
\qs 函数$f(x)=\Bigg\{\begin{aligned}
&2x^2-8ax+3,&x\le 1,\\
&\log_ax,&x>1.
\end{aligned}$在$\mathbf{R}$上单调,则$ a $的取值范围是\xx
\onech{$ \left(0,\dfrac{1}{2}\right]$}{$ \left[\dfrac{1}{2},1\right)$}{$ \left[\dfrac{1}{2},\dfrac{5}{8}\right]$}{$ \left[\dfrac{5}{8},1\right)$}
\qs 设$ x_1,x_2,x_3 $均为实数,且$ \left(\dfrac{1}{3}\right)^{x_1}=\log_2\left(x_1+1\right),~\left(\dfrac{1}{3}\right)^{x_2}=\log_3x_2, ~\left(\dfrac{1}{3}\right)^{x_3}=\log_2x_3$,则\xx
\onech{$ x_1<x_3<x_2$}{$ x_3<x_2<x_1$}{$ x_3<x_1<x_2$}{$ x_2<x_1<x_3$}
\qs 若函数$f(x)=\Bigg\{\begin{aligned}
&\log_2x,&x>0\\
&\log_{\frac{1}{2}}(-x),&x<0.
\end{aligned}$若$ f(a)>f(-a) $,则实数$ a $的取值范围是\xx
\twoch{$ \left(-1,0\right)\bigcup\left(0,1\right)$}{$ \left(-\infty,-1\right)\bigcup\left(1,+\infty\right)$}{$ \left(-1,0\right)\bigcup\left(1,+\infty\right)$}{$ \left(-\infty,-1\right)\bigcup\left(0,1\right)$}
\qs 已知函数$f(x)=\sin x+3x~(x\in (-1,1))$,如果$ f(1-a)<-f(1-a^2) $,则实数$ a $的取值范围是\xx
\twoch{$ \left(1,\sqrt{2}\right)$}{$ \left(-\infty,-2\right)\bigcup\left(1,+\infty\right)$}{$ \left(-\infty,-2\right)$}{$ \left(1,+\infty\right)$}
\qs 若“$ x>1 $”是“不等式$ 2^xa-x> $成立”的必要而不充分条件,则实数$ a $的取值范围是\xx
\onech{$ a>3$}{$ a<3$}{$ a>4$}{$ a<4$}

\qs 设$ a>0, $且$ a\ne 1,~ $“函数$y=log_ax$在$ \left(0,+\infty\right) $上是减函数”是“函数$ y=(2-a)x^3 $在$ \mathbf{R} $上是增函数”的\xx
\twoch{充分而不必要条件}{必要而不充分条件}{充分必要条件}{既不充分也不必要条件}
\qs 若函数$f(x)$的定义域为$ R $,则“$ \forall x\inR,f(x+1)>f(x) $”是“函数$ f(x) $是增函数”的\xx
\twoch{充分而不必要条件}{必要而不充分条件}{充分必要条件}{既不充分也不必要条件}
\qs 如果函数$y=f(x)$在定义域内存在区间$ \left[a,~b\right] $,使$f(x)$在$ \left[a,~b\right] $上的值域为$ \left[2a,~2b\right] $,那么称$f(x)$为“倍增函数”.若函数$ f(x)=\ln (e^x+m) $为“倍增函数”,则$ m $的取值范围是\xx
\onech{$ \left(-\dfrac{1}{4},+\infty\right) $}{$ \left(-\dfrac{1}{2},0\right) $}{$ \left(-1,0\right) $}{$ \left(-\dfrac{1}{4},0\right) $}
\qs 设$f(x),\ g(x)$都是单调函数,有如下四个命题:\\
\ding{192}若$f(x)$单调递增,$g(x)$单调递增,则$f(x)-g(x)$单调递增;\\
\ding{193}若$f(x)$单调递增,$g(x)$单调递减,则$f(x)-g(x)$单调递增;\\
\ding{194}若$f(x)$单调递减,$g(x)$单调递增,则$f(x)-g(x)$单调递减;\\
\ding{195}若$f(x)$单调递减,$g(x)$单调递减,则$f(x)-g(x)$单调递减;\\
其中,正确的命题是\xx
\onech{\ding{192}\ding{194}}{\ding{192}\ding{195}}{\ding{193}\ding{194}}{\ding{193}\ding{195}}
\qs 设函数$f(x)=\Bigg\{\begin{aligned}
&x^3-3x,&x\le a \\
&-2x,&x>a.
\end{aligned}$\\
\ding{192} 若$ a=0 $,则$f(x)$的最大值为\tk;\\
\ding{193} 若$f(x)$无最大值,则实数$ a $的取值范围是\tk.
\qs 已知函数$f(x)$,对于实数$ t $,若存在$ a>0,~b>0$,满足$ \forall x\in [t-a,t+b] $,使得$ \left|f(x)-f(t)\right|\le 2,~ $则记$ a+b $的最大值为$ H(t) .$
\begin{parts}
\part 当$ f(x)=2x $时,$ H(0)= $\tk;
\part 当$ f(x)=x^2 $且$ t\in[1,2] $时,函数$ H(t) $的值域为\tk.
\end{parts}
\qs 已知函数$f(x)=m(x-2m)(x+m+3),~g(x)=2^x-2.$若同时满足条件:\\
\ding{192} $ \forall x \in \mathbf{R} ,~f(x)<0~\text{或}~g(x)<0;$\\
\ding{193} $ \exists x \in (-\infty,-4),~f(x)g(x)<0, $\\
则$ m $的取值范围是\tk.
\end{questions}
\newpage


\section{奇偶对称}
\subsection{奇偶性的判断}
\begin{enumerate}[1)]
\item 如果函数$f(x)$的定义域不关于原点对称,则$f(x)$是\CJKunderdot{非奇非偶函数};
\item 如果函数$f(x)$的定义域关于原点对称且满足$ f(x)=f(-x) $,则$f(x)$是偶函数;
\item 如果函数$f(x)$的定义域关于原点对称且满足$ f(x)=-f(-x) $,则$f(x)$是奇函数,如果定义域包含$ x=0 $,则必有$ f(0)=0 $;
\end{enumerate}
%此处需要修改-2018.01.06



\subsection{奇偶性的运算}
奇函数左右对应中会有负号,偶函数没有负号,此处的规律可以参考“负负得正”.{\kaishu (以下假设奇偶函数都不恒为$ 0 $)}
\begin{enumerate}[1)]
\item 奇$\pm$奇=奇;\  偶$\pm $偶=偶;
\  奇$\pm$偶=非奇非偶 
\item 奇$\times(\div)$奇=偶;\ 偶$\times(\div)$偶=偶;\  奇$\times(\div)$偶=奇.
\item 当复合函数的内外两层函数都具有奇偶性时,有偶即偶,两奇为奇.
\end{enumerate}
\subsection{奇偶性常见类型}
\begin{enumerate}[1)]
\item 若对于任意$ x,~y\inR $,有$f(x+y)=f(x)+f(y)$,则函数$f(x)$为奇函数;
\item $x^n~(n\text{为奇数})$是奇函数,$x^n~(n\text{为偶数})$是偶函数;
\item $ \sin kx $是奇函数,$ \cos kx $是偶函数;
\item $ a^x-a^{-x} $是奇函数,$ a^x+a^{-x} $是偶函数;
\item $ \log_a\dfrac{b+cx}{b-cx} ~(a\ge 0\text{且}a\ne1)$是奇函数,$ \log_a\left(\sqrt{1+b^2x^2}\pm bx\right) $是奇函数;
\item $ \left|x+a\right|-\left|x-a\right| $是奇函数;$ \left|x+a\right|+\left|x-a\right| $是奇函数
\end{enumerate}
\subsection{奇偶性的单调性}
\begin{enumerate}[1)]
\item 如果$f(x)$是奇函数,则$f(x)$在关于原点对称的区间上单调性一致;
\item 如果$f(x)$是偶函数,则$f(x)$在关于原点对称的区间上单调性相反.
\end{enumerate}
\newpage
\subsection*{练习}
\begin{questions}
\qs 如果$f(x)$是定义在$\mathbf{R}$上的奇函数,那么下列函数中一定是偶函数的是\xx
\onech{$ x+f(x)$}{$ xf(x)$}{$ x^2+f(x)$}{$ x^2f(x)$}
\qs 设奇函数$f(x)$在$ \left(0,+\infty\right) $上增函数且$ f(1)=0 $,则不等式$ \dfrac{f(x)-f(-x)}{x}<0 $的解集为\xx
\twoch{$ \left(-1,0\right)\bigcup \left(1,+\infty\right)$}{$ \left(-\infty,-1\right)\bigcup \left(0,1\right)$}{$ \left(-\infty,-1\right)\bigcup \left(1,+\infty\right)$}{$ \left(-1,0\right)\bigcup \left(0,1\right)$}

\qs 奇函数$f(x)$的定义域为$ \mathbf{R} ,~$若$ f(x+2) $为偶函数,且$ f(1)=1,~ $则 $f(8)+f(9)=$\xx
\onech{$ -2 $}{$ -1 $}{$ 0 $}{$ 1 $}
\qs 已知函数$g(x)=f(x)-x$是偶函数,且$ f(3)=4 $,则$ f(-3)= $\xx
\onech{$-4$}{$-2$}{$0$}{$4$}
\qs 已知$f(x)=x^5+ax^3+bx-8$,且$f(-2)=10$,那么$f(2)=$\xx
\onech{$ -26$}{$ -18$}{$ -10$}{$ 10$}
\qs 已知定义在$ \mathbf{R} $上的偶函数$f(x)$和奇函数$g(x)$满足$ f(x)-g(x)=x^3+x^2+1 $,则$ f(1)+g(1)= $\xx
\onech{$ -3$}{$ -1$}{$ 1$}{$ 3$}
\qs 若定义在$\mathbf{R}$上的偶函数$f(x)$和奇函数$g(x)$满足$ f(x)+g(x)=e^x $,则$ g(x)= $\xx
\twoch{$ e^x-e^{-x}$}{$ \dfrac{1}{2}\left(e^x+e^{-x}\right)$}{$ \dfrac{1}{2}\left(e^{-x}-e^x\right)$}{$ \dfrac{1}{2}\left(e^x-e^{-x}\right)$}

\qs 已知定义域为$\mathbf{R} $的函数$f(x)$在$ \left(8,+\infty\right) $上为减函数,且函数$y=f(x+8)$为偶函数,则\xx
\onech{$ f(6)>f(7)$}{$ f(6)>f(9)$}{$ f(7)>f(9)$}{$ f(7)>f(10)$}
\question
设函数$f(x),g(x)$的定义域都为$\mathbf{R}$,且$f(x)$是奇函数,$g(x)$是偶函数,则下列结论正确的是\xx
\twoch{$f(x)g(x)$是偶函数}{$\abs{f(x)}g(x)$是奇函数}{$f(x)\abs{g(x)}$是奇函数}{$\abs{f(x)g(x)}$是奇函数}
\question
设函数$f(x),g(x)$的定义域都为$\mathbf{R}$,且$f(x)$是奇函数,$g(x)$是偶函数,则下列结论正确的是\xx
\twoch{$f(x)+\abs{g(x)}$是偶函数}{$f(x)-\abs{g(x)}$是奇函数}{$\abs{f(x)}+g(x)$是偶函数}{$\abs{f(x)}-g(x)$是奇函数}
\qs 已知函数$f(x)=\ln \left(\sqrt{1+9x^2}-3x\right)+1$,则$ f(\lg2)+f\left(\lg\dfrac{1}{2}\right) $等于\xx
\onech{$-1$}{$0$}{$1$}{$2$}
\qs 已知函数$f(x)$是定义在$ \mathbf{R} $上的偶函数,且在区间$ \left[0,+\infty\right) $上单调递增,若实数$ a $满足$ f(\log_2a) +f(\log_\frac{1}{2}a)\le 2f(1)$,则$ a $的取值范围是\xx
\onech{$ \left[1,2\right]$}{$ \left(0,\dfrac{1}{2}\right]$}{$ \left[\dfrac{1}{2},2\right]$}{$ \left(0,2\right]$}

\qs 已知$f(x)$是定义在$ \mathbf{R} $上的奇函数,当$ x\ge 0 $时,$f(x)=x^2-3x$,则函数$ g(x)=f(x)-x+3 $的零点的集合为\xx
\twoch{$ \left\{1,3\right\}$}{$\left\{-3,-1,1,3\right\} $}{$\left\{2-\sqrt{7},1,3\right\} $}{$\left\{-2-\sqrt{7},1,3\right\} $}


\qs 已知函数$f(x)$是定义域为$ \mathbf{R} $上的偶函数,当$ x\le0 $时,$f(x)=\left(x+1\right)^3\bm{e}^{x+1}$.那么函数$f(x)$的极值点的个数是\xx
\onech{$ 5$}{$ 4$}{$ 3$}{$ 2$}
\qs 若$f(x)=x\ln (x+\sqrt{a+x^2})$为偶函数,则$ a= $\tk.
\qs 若函数$f(x)=\ln (e^{3x}+1)+ax$为偶函数,则$ a= $\tk.
\qs 已知函数$f(x)=x\left(e^x+ae^{-x}\right)$是偶函数,则实数$ a= $\tk.
\qs 已知$y=f(x)+x^2$是奇函数,且$f(1)=1$,若$g(x)=f(x)+2$,则$ g(-1)= $\tk.

\qs 若$f(x)$是定义在 $\mathbf{R} $上的奇函数,当$ x\le0 $时,$f(x)=2x^2-x$,则$f(1)=$\tk.
\qs 设函数$f(x)$在$ \left(-\infty,+\infty\right) $内有定义,下列函数:\\
\ding{192} $ y=-\left|f(x)\right| $\qquad\ding{193} $ y=xf(x^2) $;\\
\ding{194} $ y=-f(-x) $\qquad \ding{195} $ y=f(x)-f(-x) $.\\
中必为奇函数的有\tk.(要求填写正确答案的序号)
\qs 已知函数$f(x)=e^{-\abs{x}}+\cos \pi x$,给出下来命题:\\
\ding{192} $f(x)$的最大值为$2$;\\
\ding{193} $f(x)$在$ \left(-10,10\right) $内的零点之和为$ 0 $;\\
\ding{194} $f(x)$的任何一个极大值都大于$ 1 $.\\
其中,所有正确的命题的序号是\tk.
\qs 已知偶函数$f(x)$在$ \left[0,+\infty\right) $单调递减,$ f(2)=0 $,若$f(x-1)>0$,则$ x $的取值范围是\tk.
\end{questions}
\newpage


\section{周期性}
\subsection{常用周期性模型}
\begin{enumerate}[1)]
\item 若$ f(x+a)+f(x)=C $,其中$ C $为常数,则函数$f(x)$的周期为$ T=2\left|a\right|$;
\item 若$ f(x+a)f(x)=C $,其中$ C $为常数且$ C\ne 0 $,则函数$f(x)$的周期为$ T=2\left|a\right|$;
\item 若$f(x)$满足$ f(x+2a)=f(x+a)-f(x) $,则$f(x)$的周期为$ T=6\left|a\right| $;
\end{enumerate}
\subsection{对称性和周期性}
\begin{enumerate}[1)]
\item $f(x)$关于直线$ x=a $对称$ \Leftrightarrow f(x)=f(2a-x)\Leftrightarrow f(x+a)=f(a-x) $.
\item $f(x)$关于点$ (a,0) $对称$ \Leftrightarrow f(x)=-f(2a-x)\Leftrightarrow f(x+a)=-f(a-x) $.
\item $f(x)$关于点$ (a,b) $对称$ \Leftrightarrow f(x)+f(2a-x)=2b$.
\item 如果$f(x)$关于$ x=a $和$ x=b ~(a>b)$对称,则$ T=2(a-b) .$
\item 如果$f(x)$关于$ (a,0) $和点$ (b,0)~ (a>b)$对称,则$ T=2(a-b) .$
\item 如果$f(x)$关于$ (a,0) $和直线$ x=b$对称,则$ T=4\left|a-b\right| .$
\end{enumerate}
\subsection*{学霸总结}
\begin{spacing}{2}
若$ f(A)=f(B) $且$ A-B $为常数,则$ f(x) $是以$ \left|A-B\right| $为周期的函数;若$ f(A)=f(B) $且$ A+B $为常数,则$ f(x) $关于直线$ x=\dfrac{A+B}{2} $对称;若$ f(A)=-f(B) $且$ A-B $为常数,则$ f(x) $是以$ 2\left|A-B\right| $为周期的函数;若$ f(A)=-f(B) $且$ A+B $为常数,则$f(x)$关于点$ \left(\dfrac{A+B}{2},0\right) $中心对称.
\end{spacing}
\subsection*{练习}
\begin{questions}

\qs 设函数$f(x)$是$ \left(-\infty,+\infty\right) $上的奇函数,$ f(x+2)=-f(x) ,~$当$ 0\le x\le 1 $时,$f(x)=2x$,~则$f(2015)=$\\\mbox{\hspace{1em}}\hfill\xx
\onech{$ -1$}{$ -2$}{$ 1$}{$ 2$}
\qs 定义在$ \mathbf{R} $上的函数$ y=f(x) $在区间$ \left(-\infty,2\right) $上是增函数,且$ y=f(x+2) $的图象关于$ x=1 $对称,则\xx
\onech{$ f(1)<f(5)$}{$ f(1)>f(5)$}{$ f(1)=f(5)$}{$ f(0)=f(5)$}
\qs 设函数$y=f(x)~(x\inR)$的图象关于直线$ x=0 $及直线$ x=1 $对称,且$ x\in \left[0,1\right] $时,$f(x)=x^2$,则$ f\left(-\dfrac{3}{2}\right) =$\xx
\onech{$ \dfrac{1}{2}$}{$ \dfrac{1}{4}$}{$ \dfrac{3}{4}$}{$ \dfrac{9}{4}$}
\qs $f(x)$的定义域为$\mathbf{R}$,若$f(x+1)$与$f(x-1)$都是奇函数,则\xx
\twoch{$f(x)$是偶函数}{$f(x)$是奇函数}{$f(x)=f(x+2)$}{$f(x+3)$是奇函数} 
\qs $f(x)$为定义在$ \mathbf{R} $上的函数,$ f(10+x)=f(10-x) ,~f(20+x)=-f(20-x)$,则$f(x)$是\xx
\twoch{周期为$20 $的奇函数}{周期为$20 $的偶函数}{周期为$40 $的奇函数}{周期为$40 $的偶函数} 

\qs 下列函数中,对于任意$ x\in \mathbf{R} ,~$同时满足$f(x-\pi)=f(x)$的函数是\xx
\twoch{$ f(x)=\sin x $}{$ f(x)=\sin x\cos x $}{$ f(x)=\cos x $}{$ f(x)=\cos^2x-\sin^2x $}
\qs 已知函数$y=f(x)$的周期为$ 2 $,当$ x\in\left[-1,1\right] $时,$ f(x)=x^2 $,那么函数$y=f(x)$的图象与函数$ y=\abs{\lg x} $的图象的交点的个数为\xx
\onech{$ 10$个}{$ 9$个}{$8 $个}{$1 $个} 
\qs $f(x)$是定义在$\mathbf{R}$上的以$ 3 $为周期的偶函数,且$ f(2)=0 $,则方程$ f(x)=0 $在区间$ \left(0,6\right) $内的解的个数的最小值是\xx
\onech{$ 5$}{$ 4$}{$ 3$}{$ 2$}
\qs 函数$ y=\dfrac{1}{1-x} $的图象与函数$ y=\sin\left(2\pi x\right)~(-2\le x\le 4) $的图象所有交点的横坐标之和等于\xx
\onech{$ 2$}{$ 4$}{$ 6$}{$ 8$}  
\qs 已知$f(x)$是$\mathbf{R}$上最小正周期为$ 2 $的周期函数,且当$ 0\le x<2 $时,$f(x)=x^3-x$,则函数$ y=f(x) $的图象在区间$ \left[0,6\right] $上与$x$轴的交点的个数为\xx
\onech{$ 6$}{$ 7$}{$ 8$}{$ 9$}
\qs 已知函数$f(x)~(x\in \mathbf{R})$满足$f(-x)=2-f(x)$,若函数$y=\dfrac{x+1}{x}$与$y=f(x)$图象的交点为$\left(x_1,y_1\right),\left(x_2,y_2\right),\cdots$,\\$\left(x_m,y_m\right)$,则$\sum\limits_{i=1}^{m}(x_i+y_i)=$\xx
\onech{$0$}{$m$}{$2m$}{$4m$}
\qs 已知函数$f(x)=\dfrac{ \sin x}{x^2+1 }$,下列命题:\\
\ding{192} 函数$f(x)$的图象关于原点对称;\\
\ding{193} 函数$f(x)$是周期函数;\\
\ding{194} 当$ x=\dfrac{\pi}{2} $时,函数$f(x)$取最大值;\\
\ding{195} 函数$f(x)$的图象与函数$ y=\dfrac{1}{x} $的图象没有公共点.\\
其中正确的命题的序号是:\\
\onech{\ding{192}\ding{194}}{\ding{193}\ding{194}}{\ding{192}\ding{195}}{\ding{193}\ding{195}}
\qs 定义在$\mathbf{R}$上的函数$f(x)$满足$ f(0)=0,f(x)+f(1-x)=2,f\left(\dfrac{x}{5}\right)=\dfrac{1}{2} f(x)$,且当$ 0\le x_1<x_2\le1 $时,$ f(x_1)\le f(x_2) $,且$ f\left(\dfrac{1}{5}\right)=\tk,f\left(\dfrac{1}{2017}\right)=\tk $
\end{questions}
\newpage


\section{图象变换}
\subsection{性质}
%\subsubsection*{平移}
%左加右减:
\textbf{平移:}$\Bigg\{\begin{aligned}
y=f(x)\xrightarrow{\text{左移}a\text{个单位}}y=f(x+a)\\
y=f(x)\xrightarrow{\text{右移}b\text{个单位}}y=f(x-b)
\end{aligned}$\qquad
%上加下减:
$\Bigg\{\begin{aligned}
y=f(x)\xrightarrow{\text{上移}c\text{个单位}}y=f(x)+c\\
y=f(x)\xrightarrow{\text{下移}d\text{个单位}}y=f(x)-d
\end{aligned}$\par 
%\subsubsection*{对称}
\textbf{对称:}
$\Bigg\{\begin{aligned}
y=f(x)\xrightarrow{\text{关于}x\text{轴对称}}y=-f(x)\\
y=f(x)\xrightarrow{\text{关于}y\text{轴对称}}y=f(-x)
\end{aligned}$\par
%\subsubsection*{翻折}
\textbf{翻折:}
$\Bigg\{\begin{aligned}
y=f(x)\xrightarrow{\text{留上翻下}}y=\abs{f(x)}\\
y=f(x)\xrightarrow{\text{去左留右}}y=f\left(\abs{x}\right)
\end{aligned}$\par 
%\subsubsection*{缩放}
\textbf{缩放:}
$\begin{dcases}
y=f(x)\xrightarrow{\text{纵坐标伸缩为原来的}\ k\ \text{倍}}y=kf(x)\\
y=f(x)\xrightarrow{\text{横坐标伸缩为原来的}\tfrac{1}{e}\text{倍}}y=f(ex)
\end{dcases}$
 
\subsection*{练习}
\begin{questions}
\qs 函数$f(x)$的图象向右平移$ 1 $个单位长度,所得图象与$ y=e^x $关于$ y $轴对称,则$f(x)=$\\\mbox{\hspace{1ex}}\hfill\xx
\onech{$ e^{x+1} $}{$e^{x-1}$}{$ e^{-x+1} $}{$ e^{-x-1} $}
\qs 设函数$y=f(x)$的图象与$ y=2^{x+a} $的图象关于直线$ y=-x $对称,且$ f(-2)+f(-4)=1 $,则$ a= $\xx
\onech{$ -1$}{$ 1$}{$ 2$}{$ 4$}
\qs 函数$ y=-e^x $的图象\xx
\twoch{与$ y=e^x $的图象关于$y$轴对称}{与$ y=e^x $的图象关于坐标原点对称}{与$ y=e^{-x} $的图象关于$y$轴对称}{与$ y=e^{-x} $的图象关于坐标原点对称}
\qs 为了得到$ y=\lg \dfrac{x+3}{10} $的图象,只需把函数$ y=\lg x $的图象上所有的点\xx
\twoch{向左平移$ 3 $个单位,再向上平移$ 1 $个单位}{向右平移$ 3 $个单位,再向上平移$ 1 $个单位} {向左平移$ 3 $个单位,再向下平移$ 1 $个单位}{向右平移$ 3 $个单位,再向下平移$ 1 $个单位}
\qs 若函数$ f(x)=a^x+b-1 ~(a>0\text{且}a\ne1)$的图象经过第二、三、四象限,则一定有\xx
\twoch{$ 0<a<1\text{且}b>0$}{$ a>1\text{且}b>0$}{$ 0<a<1\text{且}b<0$}{$ a>1\text{且}b<0$}
\qs 已知函数$f(x)=\Bigg\{\begin{aligned}
&\abs{\log_2\abs{x-1}},&x\ne1,\\
&0,&x=1
\end{aligned}$
\begin{parts}
\part 写出函数$f(x)$的单调区间;
\part 若关于$ x $的方程$ \left[f(x)\right]^2+bf(x)+c=0 $有$ 7 $个解,求$ b,~c $满足的条件.
\end{parts}
\end{questions}
\newpage


\section{分段函数}
\begin{questions}
\qs 已知函数$f(x)=\Bigg\{\begin{aligned}
&\abs{\lg x},&0<x\le10 \\
&-\dfrac{1}{2}x+6,&x>10.
\end{aligned}$若$ a,b,c $互不相等,且$ f(a)=f(b)=f(c) $,则$ abc $的取值范围是\xx
\twoch{$ \left(1,10\right)$}{$ \left(5,6\right)$}{$ \left(10,12\right)$}{$ \left(20,24\right)$}
\qs 已知函数$f(x)=\Bigg\{\begin{aligned}
&\left|\log_4x\right|,&0<x\le 4,\\&x^2-10x+25,&x>4.
\end{aligned}$若$ a,~b,~c,~d $是互不相同的正数,且$ f(a)=f(b)=f(c)=f(d),~ $则$ abcd $的取值范围是\xx
\onech{$ \left(24,25\right) $}{$ \left(18,24\right) $}{$ \left(21,24\right) $}{$ \left(18,25\right) $}
\qs 设定义在$\mathbf{R}$上的函数$f(x)=\Bigg\{\begin{aligned}
&\abs{\lg\abs{x-1}},&x\ne1,\\
&0,&x=1.
\end{aligned}$则关于$ x $的方程$ f^2(x)+bf(x)+c=0 $有$ 7 $个不同的实数解的充要条件是\xx
\twoch{$ b<0\text{且}c>0$}{$ b>0\text{且}c>0$}{$b<0\text{且}c=0 $}{$ b\ge0\text{且}c=0$}
\question
已知函数$f(x)=\begin{dcases}
-x^2+2x,&x\le0,\\
\ln(x+1),&x>0.
\end{dcases}$若$\abs{f(x)}\ge ax$,则$a$的取值范围是\xx
\onech{$\left(-\infty,0\right]$}{$\left(-\infty,1\right]$}{$\left[-2,-1\right]$}{$\left[-2,0\right]$}
\qs $f(x)=\Bigg\{\begin{aligned}
&\left(x-a\right)^2,&x\le0,\\
&x+\dfrac{1}{x}+a,&x>0
\end{aligned}$若$f(0)$是$f(x)$的最小值,则$ a $的取值范围是\xx
\twoch{$ \left[-1,2\right]$}{$ \left[-1,0\right]$}{$ \left[1,2\right]$}{$ \left[0,2\right]$}
\qs 已知函数$f(x)=\Bigg\{\begin{aligned}
&x-1,&x\le2;\\
&2+\log_ax,&x>2
\end{aligned}\left(a>0\text{且}a\ne 1\right)$的最大值为$1$,则实数$ a $的取值范围是\xx
\onech{$ \left[\dfrac{1}{2},1\right)$}{$ \left(0,1\right)$}{$ \left(0,\dfrac{1}{2}\right]$}{$ \left(1,+\infty\right)$}
\qs 已知函数$f(x)=\Bigg\{\begin{aligned}
&-x^2+4x,&x\le 4,\\
&\log_2x,&x>4.
\end{aligned}$若$y=f(x)$在区间$ \left(a,a+1\right) $上单调递增,则实数$ a $的取值范围是\xx
\twoch{$ \left(-\infty,1\right] $}{$ \left[1,4\right] $} {$ \left[4,+\infty\right) $}{$ \left(-\infty,1\right]\bigcup\left[4,+\infty\right) $}
\qs 已知函数$f(x)=\Bigg\{\begin{aligned}
&\dfrac{2}{x}&x\ge2\\
&(x-1)^3&x<2.
\end{aligned}$若关于$ x $的方程$ f(x)=k $有两个不同的实根,则实数$ k $的取值范围是\tk.
\qs
设函数$f(x)=\Bigg\{\begin{aligned}
&~2^x-a& x<1;\\
&~4(x-a)(x-2a)& x\ge 1.
\end{aligned}$\\
\ding{192}~若$a=1$,则$ f(x) $的最小值为\tk;\\
\ding{193}~若$ f(x) $恰有$ 2 $个零点,则实数$ a $的取值范围是\tk.
\qs 设函数$f(x)=\begin{dcases}
x^3-3x,&x\le a, \\
-2x,&x>a.
\end{dcases}$\\
\ding{192} 若$ a=0 $,则$f(x)$的最大值为\tk;\\
\ding{193} 若$f(x)$无最大值,则实数$ a $的取值范围是\tk.
\qs 关于$ x $的方程$ g(x)=t(t\inR) $的实数根的个数记为$ f(t) $,若$g(x)=\ln x$,则$f(t)=$\tk;若$g(x)=\Bigg\{\begin{aligned}
&x,&x\le0;\\
&-x^2+2ax+a,&x\ge0.
\end{aligned}(a\inR)$,存在$ t $使得$ f(t+2)>f(t) $成立,则$ a $的取值范围是\tk.
\qs 已知函数$f(x)=\Bigg\{\begin{aligned}
&(x-2a)(a-x),&x\le 1,\\&\sqrt{x}+a-1,&x>1.
\end{aligned}$
\begin{parts}
\part 若$ a=0,~x\in\left[0,4\right],~ $则$f(x)$的值域为\tk;
\part 若$f(x)$恰有三个零点,则实数$ a $的取值范围是\tk.
\end{parts}
\qs 已知函数$f(x)=\Bigg\{\begin{aligned}
&1-x^2,&x\ge 0,\\
&\cos \pi x,&x<0.
\end{aligned}$若关于$ x $的方程$ f(x+a)=0 $在$ (0,+\infty) $内有唯一实根,则实数$ a $的最小值是\tk.

\end{questions}
\end{document}