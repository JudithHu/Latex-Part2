\documentclass[marginline,noindent,answers,adobefonts]{BHCexam}	
\begin{document}
\begin{questions}
\qs 已知函数$f(x)=\dfrac{ \sin x}{x^2+1 }$,下列命题:\\
\ding{192} 函数$f(x)$的图象关于原点对称;\\
\ding{193} 函数$f(x)$是周期函数;\\
\ding{194} 当$ x=\dfrac{\pi}{2} $时,函数$f(x)$取最大值;\\
\ding{195} 函数$f(x)$的图象与函数$ y=\dfrac{1}{x} $的图象没有公共点.\\
其中正确的命题的序号是:\\
\onech{\ding{192}\ding{194}}{\ding{193}\ding{194}}{\ding{192}\ding{195}}{\ding{193}\ding{195}}
\qs 设$ f(x)=\Bigg\{\begin{aligned}
x^3,x<a,\\x^2,x\ge a.
\end{aligned} $若存在实数$ b $,使得函数$ g(x)=f(x)-b $有两个零点,则$ b $的取值范围是\tk. 
\qs 已知函数$f(x)$是定义域为$ \mathbf{R} $的偶函数,当$ x\le 0 $时,$ f(x)=(x+1)^3e^{x+1} .$那么函数$ f(x) $的极值点的个数是\xx
\onech{5}{4}{3}{2}
\qs 如图,点$A,~B$在函数$ y=log_2x+2 $的图象上,点$C$在函数$y=log_2x$的图象上,若$ \triangle ABC$为等边三角形,且直线$ BC\sslash y$轴,设点$A$的坐标为$ (m,n) $,则$ m= $\xx
\vspace{-1em}
\begin{center}
\begin{tikzpicture}[domain=0.1:4,scale=0.6]
\clip (-0.5,-1) rectangle (5,5);
\tikzmath{
\i=sqrt(3);
\a=log2(\i);
\b=\i+sqrt(3);
\c=log2(\b);
\d=\c+2;
}
%    \draw[very thin,color=gray] (0.1,-1.1) grid (3.9,3.9);
    \draw[->] (-0.2,0) -- (4.2,0) node[right] {$x$};
    \draw[->] (0,-1.2) -- (0,4.2) node[above] {\small$f(x)$};
    \draw plot(\x,{log2(\x)+2});%node[above]{$f(x)=log_2x+2$}; 
    \draw plot(\x,{log2(\x)});%node[above]{$f(x)=log_2x$}; 
\coordinate [label=left:\small$A$] (A) at($(\i,\a+2)$);
\coordinate [label=above:\small$B$] (B) at($(\b,\d)$);
\coordinate [label=below:\small$C$] (C) at($(\b,\c)$);
\draw(A)--(B)--(C)--cycle;
\end{tikzpicture}
\end{center}
\vspace{-2em}
\onech{2}{3}{$ \sqrt{2} $}{$ \sqrt{3} $}
 \qs 若“$x>1$”是不等式“$2^x>a-x$成立”的必要而不充分条件,则实数$ a $的取值范围是\xx
\onech{$ a>3 $}{$ a<3 $}{$ a>4 $}{$ a<4 $}
\qs 下列函数中,对于任意$ x\in \mathbf{R} ,~$同时满足$f(x-\pi)=f(x)$的函数是\xx
\twoch{$ f(x)=\sin x $}{$ f(x)=\sin x\cos x $}{$ f(x)=\cos x $}{$ f(x)=\cos^2x-\sin^2x $}
\qs 设$ a>0, $且$ a\ne 1,~ $“函数$y=log_ax$在$ \left(0,+\infty\right) $上是减函数”是“函数$ y=(2-a)x^3 $在$ \mathbf{R} $上是增函数”的\xx
\twoch{充分而不必要条件}{必要而不充分条件}{充分必要条件}{既不充分也不必要条件}
\qs 一次猜奖游戏中,1,2,3,4四扇门里摆放了$ a,~b,~c,~d,~ $四件奖品(每扇门内仅放一件).甲同学说:$1$号门里是$ b,~ $$3$号门里是$ c;~ $乙同学说:$2$号门里是$ b,~ $$3$号门里是$ d;~ $丙同学说:$4$号门里是$ b,~ $$2$号门里是$ c;~ $丁同学说:$4$号门里是$ a,~ $$3$号门里是$ c;~ $,如果他们每个人都猜对了一半,那么$4$号门里是\xx
\onech{$a$}{$b$}{$c$}{$d$}
\qs 已知函数$f(x)=\Bigg\{\begin{aligned}
&\left|log_4x\right|,&0<x\le 4,\\&x^2-10x+25,&x>4.
\end{aligned}$若$ a,~b,~c,~d $是互不相同的正数,且$ f(a)=f(b)=f(c)=f(d),~ $则$ abcd $的取值范围是\xx
\onech{$ \left(24,25\right) $}{$ \left(18,24\right) $}{$ \left(21,24\right) $}{$ \left(18,25\right) $}
\qs 现有10支队伍比赛,规定:比赛采取单循环比赛制,每支队伍与其他9支队伍各比赛一场,每场比赛中,胜方得2分,负方得0分,平局双方各得1分.下面关于这10支队伍得分的叙述正确的是\xx
\twoch{可能有两支队伍得分都是18分}{各队得分总和为180分}{各支队伍中最高得分不少于10分}{得偶数分的队伍必有偶数个} 
\qs 在平面直角坐标系$xOy$中,动点$ P(x,y) $到两坐标轴的距离之和等于它到定点$ (1,1) $的距离,记点$ P $的轨迹为$ C $,给出下面四个结论:\\
\ding{192} 曲线$ C $关于原点对称;\\
\ding{193} 曲线$ C $关于$ y=x $对称;\\
\ding{194} 点$ (-a^2,1)(a\inR) $在曲线$ C $上;\\
\ding{195} 在第一象限内,曲线$ C $与$x$轴的非负半轴、$y$轴的非负半轴围成的封闭图形的面积小于$ \dfrac{1}{2}. $\\
其中所有的正确结论的序号是\tk.
\qs 已知函数$ f(x)=e^x-e^{-x} ,~$下列命题正确的有\tk.(写出所有正确命题的编号)\\
\ding{192}$f(x)$是奇函数;\\
\ding{193}$f(x)$在$ \mathbf{R} $上是单调递增函数;\\
\ding{194}方程$ f(x)=x^2+2x $有且仅有$ 1 $个实数根;\\
\ding{195}如果对于任意$ x\in (0,+\infty),~ $都有$ f(x)>kx,~ $那么$ k $的最大值为2.
\qs 已知函数$f(x)=\Bigg\{\begin{aligned}
&(x-2a)(a-x),&x\le 1,\\&\sqrt{x}+a-1,&x>1.
\end{aligned}$\\
\begin{parts}
\part 若$ a=0,~x\in\left[0,4\right],~ $则$f(x)$的值域为\tk;
\part 若$f(x)$恰有三个零点,则实数$ a $的取值范围是\tk.
\end{parts}

\qs 已知函数$f(x)=\cos x-2^x-2^{-x}-b,~(b \inR)$.
\begin{parts}
\part 当$ b=0 $时,函数$f(x)$的零点个数为\tk; 
\part 若函数$f(x)$有两个不同的零点,则$ b $的取值范围是\tk.
\end{parts}
\newpage
\qs 已知函数$f(x)$,对于实数$ t $,若存在$ a>0,~b>0$,满足$ \forall x\in [t-a,t+b] $,使得$ \left|f(x)-f(t)\right|\le 2,~ $则记$ a+b $的最大值为$ H(t) .$
\begin{parts}
\part 当$ f(x)=2x $时,$ H(0)= $\tk;
\part 当$ f(x)=x^2 $且$ t\in[1,2] $时,函数$ H(t) $的值域为\tk.
\end{parts} 
\qs 关于$ x $的方程$ g(x)=t(t\inR) $的实数根的个数记为$ f(t) $,若$g(x)=\ln x$,则$f(t)=$\tk;若$g(x)=\Bigg\{\begin{aligned}
&x,&x\le0;\\
&-x^2+2ax+a,&x\ge0.
\end{aligned}(a\inR)$,存在$ t $使得$ f(t+2)>f(t) $成立,则$ a $的取值范围是\tk.
\qs 已知函数$f(x)=\Bigg\{\begin{aligned}
&(x-2a)(a-x),&x\le 1,\\&\sqrt{x}+a-1,&x>1.
\end{aligned}$\\
\begin{parts}
\part 若$ a=0,~x\in\left[0,4\right],~ $则$f(x)$的值域为\tk;
\part 若$f(x)$恰有三个零点,则实数$ a $的取值范围是\tk.
\end{parts}
\end{questions}
\end{document}