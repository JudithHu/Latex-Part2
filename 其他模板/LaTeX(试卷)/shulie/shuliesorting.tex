\documentclass{BHCexam}
%\usepackage[colorlinks,linkcolor=black]{hyperref}
\begin{document}
\setcounter{tocdepth}{2}
\biaoti{数列知识点汇总}
\fubiaoti{}
\maketitle
\tableofcontents
\newpage 
\section{数列基本概念}
按照一定的顺序排列的数叫做\CJKunderdot{数列},数列中的每一个数叫做数列的\CJKunderdot{项}.排在第一位的数称作数列的\CJKunderdot{首项},排在第二位的称为数列的第$ 2 $项$ \cdots\cdots $排在第$ n $位的称为这个数列的第$ n $项.数列的一般形式为\[a_1,~a_2,~a_3,~\cdots a_n,~\cdots, \]简记为$ \{a_n\} $.项数有限的数列叫做有穷数列,项数无限的数列叫做无穷数列.
\subsection{数列与函数}
在函数的意义下,数列是定义域为正整数集$ \mathbf{N^*} $(或它的有限子集$ \left\{1,2,3,\cdots,n\right\} $)的特殊函数,数列的通项公式就是相应的函数解析式,即$ a_n=f(n)~(n\in\mathbf{N^*}) $.
\subsection{通项公式}
\subsubsection{通项公式}
如果数列$\{a_n\}$的第$ n $项与序号$ n $之间的关系可以用一个式子来表示,那么这个式子叫做这个数列的\CJKunderdot{通项公式}. 
\subsubsection{递推公式}
如果已知数列$\{a_n\}$的第一项(或前几项),且从第二项(或某一项)开始任何一项$ a_n $与它的前一项$ a_{n-1} $(或前几项)间的关系可以用一个式子来表示,那么这个式子叫做数列$\{a_n\}$的递推公式.

\subsection{数列的分类}
类比函数的性质及其分类,对数列进行恰当的分类可以更深刻的理解和认识数列.
\begin{enumerate}[1)]
\item 根据项数是有限还是无限分类:\begin{enumerate}[i)]
\item 有穷数列:项数有限的数列;
\item 无穷数列:项数无限的数列;
\end{enumerate}
\item 根据项的增减规律分类:\begin{enumerate}[i)]
\item 递增数列:从第二项起每一项都大于它的前一项;
\item 递减数列:从第二项起每一项都小于它的前一项;
\end{enumerate}
递增数列和递减数列统称单调数列;
\item 根据任何一项的绝对值是否都小于某一个正数(常数)来分类:
\begin{enumerate}[i)]
\item 有界数列:$ \forall x\in\mathbf{N^*},\abs{a_n}\le M~(M\text{为常数}) $;
\item 无界数列:$ \forall M\in\mathbf{R^+} ,\exists x\in\mathbf{N^*},\text{使得}\abs{a_n}>M.$
\end{enumerate}
\end{enumerate}
\subsection{数列通项求法}
\subsubsection{数列的前$ n $项和与通项公式的关系}
\begin{enumerate}[1)]
\item $ S_n=a_1+a_2+a_3+\cdots+a_n $;
\item $ a_n=\Bigg\{\begin{aligned}
&S_1&\left(n=1\right)\\
&S_n-S_{n-1}&\left(n\ge2\right)
\end{aligned} $
\end{enumerate}
注意:一定要验证$ n=1 $的情况.
\subsubsection{利用递推关系通项}
已知数列$\{a_n\}$的递推关系求通项时,通常用累加法、累乘法和构造法求解.
\begin{enumerate}
\item 形如$ a_n=a_{n-1}+m~ (n\ge 2,n\in\mathbf{N^*})$时,构造等差数列求解,形如$ a_n=xa_{n-1}+y~(n\ge2,n\in\mathbf{N^*}) $时,构造等比数列求解;
\item 形如$ a_n=a_{n-1} +f(n)~(n\ge 2,n\in\mathbf{N^*})$时,用累加法;
\item 形如$ \dfrac{a_n}{a_{n-1}}=f(n) ~(n\ge 2,n\in\mathbf{N^*})$时,用累乘法求解.
\end{enumerate}
\section{等差数列}
\subsection{基本性质} 
\subsubsection{定义}
一般地,如果一个数列从第二项开始,每一项都与前一项的差为一个常数,那么这个数列就叫做等差数列,这个常数叫做这个数列的公差,通常用字母$ d $表示.\\
{\kaishu 注:目前大部分等差数列考题都可以通过转化为$ a_1 $和$ d $求出.}
\subsubsection{通项公式}
如果等差数列$\{a_n\}$的首项为$ a_1 $,公差为$ d $,那么使用\CJKunderdot{累加法}可得它的通项公式是$ a_n=a_1+(n-1)d ,n\in\mathbf{N^*}$
\begin{proof}
由给定条件可得:\begin{equation*}
\begin{aligned}
 a_2-a_1& =d\\
a_3-a_2&=d\\
\vdots&\\
a_n-a_{n-1}&=d.
\end{aligned}
\end{equation*}
等号两边累加可得:~$ a_n-a_1=(n-1)d .$即:~$$a_n=a_1+(n-1)d$$
\end{proof}
\subsubsection{等差中项}
\begin{enumerate}[1)]
\item 如果$ A=\dfrac{a+b}{2} ,$则称$ A $为$ a $和$ b $的等差中项(考试常用);
\item 等差数列中,等间隔的三项$a_{n-p},~a_n,~a_{n+p} (n,p\in\mathbf{N^*}~\text{且}~n<p) $满足:$ 2a_n=a_{n-p}+a_{n+p} $;
\item 在等差数列$ \left\{a_n\right\} $中,若有$ k+l=m+n \left(k,l,m,n\in\mathbf{N^*}\right)$,则有$ a_k+a_l=a_m+a_n $.
\end{enumerate}
\subsubsection{前$ n $项和公式}
设等差数列$\{a_n\}$的公差为$ d $,则其前$ n $项和$ S_n=\dfrac{n\left(a_1+a_n\right)}{2} $或$ S_n=na_1+\dfrac{n(n-1)}{2}d $.
\begin{proof}
在等差数列中,根据性质$ a_k+a_l=a_m+a_n~(k+l=m+n) $可得$$ a_1+a_n=a_2+a_{n-1}=\cdots=a_k+a_{n-k+1} ~\left(k\le\dfrac{n}{2}\right)$$
\begin{equation*}
\begin{aligned}
S_n&=a_1+a_2+a_3+\cdots+a_n\\
 &=\left(a_1+a_n\right)+\left(a_2+a_{n-1}\right)+\cdots+\left(a_k+a_{n-k+1}\right)\\
&=\dfrac{n(a_1+a_n)}{2}\\
&=\dfrac{n(a_1+a_1+(n-1)d}{2}=na_1+\dfrac{n(n-1)}{2}d.
\end{aligned}
\end{equation*}
\end{proof}
\subsection{性质扩充}
\subsubsection{等差数列的常用性质}
\begin{enumerate}[(1)]
\item 通项公式的推广:$ a_n=a_m+\left(n-m\right)d \left(n,m\in\mathbf{N^*}\right)$;
\item 若$\{a_n\}$是等差数列,公差为$ d $,则$\{a_{2n}\}$ 也是等差数列,公差为$ 2d $;
\item 若$\{a_n\},~\{b_n\}$是等差数列,则$ \left\{pa_n+qb_n\right\}~(p,q\text{是常数}) $也是等差数列;
\item 若$\{a_n\}$是等差数列,公差为$ d $,则$ a_k,~a_{k+m},~ a_{k+2m},~a_{k+3m},\cdots\left(k,m\in\mathbf{N^*}\right)$组成公差为$ md $的等差数列.
\end{enumerate}
\subsubsection{与和有关的性质}
\begin{enumerate}[(1)]
\item 若$\{a_n\}$是等差数列,则$ \dfrac{S_n}{n} $也是等差数列,其首项与$ \{a_n\} $的首项相同,公差是$\{a_n\}$的公差的$ \dfrac{1}{2} $;
\item 若$ S_m,S_{2m},S_{3m} $分别是$\{a_n\}$的前$ m $项,前$ 2m $项,前$ 3m $项的和,则$ S_m,~S_{2m}-S_m,~S_{3m}-S_{2m} $成等差数列;
\item 关于非零等差数列奇数项和与偶数项和的性质\begin{enumerate}[i)]
\item 若项数为$ 2n $,则$ S_{\text{偶}}-S_{\text{奇}}=nd ,~\dfrac{S_{\text{偶}}}{S_{\text{奇}}}=\dfrac{a_n}{a_{n+1}}$.
\item 若项数为$ 2n-1 $,则$  S_{\text{偶}}=(n-1)a_n,~S_{\text{奇}}-S_{\text{偶}}=a_n,~\dfrac{S_{\text{奇}}}{S_{\text{偶}}}=\dfrac{n}{n-1}$.
\item 若两个等差数列$\{a_n\}$、$\{b_n\}$的前$ n $项和分别为$ S_n\text{、}~T_n,~ $则$ \dfrac{a_n}{b_n}=\dfrac{S_{2n-1}}{T_{2n-1}} $.
\end{enumerate}
\end{enumerate}
\subsubsection{等差数列前$ n $项和的最值问题}
\begin{enumerate}[1)]
\item 二次函数法:当公差$d\ne0$时,将$ S_n $看作关于$ n $的二次函数,运用配方法,借助函数的单调性及数形结合,使问题得解;
\item 通项公式法:求使$ a_n\ge0 \left(\text{或}a_n\le0\right)$成立的最大$ n $值即可得到$ S_n $的最大(或最小)值;
\item 不等式法:借助$ S_n $最大时,有$\Bigg\{\begin{aligned}
S_n\ge S_{n-1},\\
S_n\ge S_{n+1}.
\end{aligned}~(n\ge2,n\in\mathbf{N^*})$,解此不等式组确定$ n $的范围,进而确定$ n $的值和对应$ S_n $的值.
\end{enumerate}
\section{等比数列}
\subsection{基本性质}
\subsubsection{定义}
一般地,如果一个数列从第二项起,每一项与它的前一项的比等于一个常数,那么这个数列就叫做等比数列.这个常数叫做这个数列的公比,通常用字母$ q $表示.
\subsubsection{通项公式}
如果等比数列$\{a_n\}$的首项为$a_1$,公比为$ q $,则它的通项公式为$ a_n=a_1q^{n-1}~(q\ne0). $
\begin{proof}
已知等比数列$\{a_n\}$中,有$\dfrac{a_n}{a_{n-1}}=q~(q\ne0)$.\\
则有$$ \begin{aligned}
\dfrac{a_2}{a_1}=&q\\
\dfrac{a_3}{a_2}=&q\\
\vdots&\\
\dfrac{a_n}{a_{n-1}}=&q 
\end{aligned}
$$
左右两侧累乘即得到:$\dfrac{a_n}{a_1}=q^{n-1}$
即:$$ a_n=a_1q^{n-1} $$
\end{proof}
\subsubsection{等比中项}
\begin{enumerate}[(1)]
\item 如果三个数$ a,G,b $成等比数列,则$ G $叫做$ a $和$ b $的等比中项,且$ \dfrac{G}{a}=\dfrac{b}{G} $,即$ G^2=ab $;
\item 等比数列$ \{a_n\} $中,等间隔的三项$ a_{n-s},~a_n,~a_{n+s}~(s\in\mathbf{N^*},\text{且} s<n ) $有$ a_{n-s}a_{n+s}=a_n^2 $;
\item 等比数列$ \{a_n\} $中,若$ m+n=p+q $,则$ a_m\bm{\cdot}a_n=a_p\bm{\cdot}a_q $.
\end{enumerate}
\subsubsection{前$ n $项和}
$S_n=\Bigg\{\begin{aligned}
&na_1&\left(q=1\right)\\
&\dfrac{a_1\left(1-q^n\right)}{1-q}&\left(q\ne1\right)
\end{aligned}$
\begin{proof}
给定等比数列$\{a_n\}$.\\
\ding{192}~当$ q=1$时,有$ a_1=a_2=\cdots=a_n $,
$ S_n=a_1+a_2+\cdots+a_n=na_1. $\\
\ding{193}~当$ q\ne1 $时,有:\begin{equation}\label{db1}
\begin{aligned}
S_n=&a_1+a_2+\cdots+a_n \\
=&a_1+a_1q+a_1q^2+\cdots+a_1q^{n-1};\end{aligned}
\end{equation}
两边同时乘以公比$q$,有:\begin{equation}\label{db2}
qS_n=a_1q+a_1q^2+\cdots+a_1q^{n-1}+a_1q^n
\end{equation}
(\ref{db1})-(\ref{db2})得到:\begin{equation*}
\begin{aligned}
S_n-qS_n=&\left(a_1+a_1q+a_1q^2+\cdots+a_1q^{n-1}\right)-\left(a_1q+a_1q^2+a_1q^3+\cdots+a_1q^{n-1}+a_1q^n\right)\\
=&a_1+\left(a_1q-a_1q\right)+\left(a_1q^2-a_1q^2\right)+\cdots+\left(a_1q^{n-1}-a_1q^{n-1}\right)-a_1q^n\\
=&a_1-a_1q^n
\end{aligned}
\end{equation*}
化简得:$ S_n=\dfrac{a_1(1-q^n)}{1-q}~(q\ne1) $
\end{proof}
\subsubsection{等比数列的性质}
已知等比数列$\{a_n\}$的前$ n $项和为$S_n$.
\begin{enumerate}[(1)]
\item 数列$\{c\bm{\cdot}a_n\}~\left(c\ne0\right),~\left\{\abs{a_n}\right\},~\left\{a_n\bm{\cdot}b_n\right\}~\left(\left\{b_n\right\}\text{是等比数列}\right),~\left\{a^2_n\right\},~\left\{\dfrac{1}{a_n}\right\}$等也是等比数列;
\item 数列$ a_m,a_{m+k},a_{m+2k},a_{m+3k},\cdots $仍是等比数列;
\item $ a_1a_n=a_2a_{n-1}=\cdots=a_ma_{n-m+1} $;
\item 当数列$\{a_n\}$的公比$ q\ne-1$(或$ q=-1\text{且}m\text{为奇数} $)时,数列$ S_m,~S_{2m}-S_m,~S_{3m}-S_{2m} ,\cdots$是等比数列;
\item 当$ n $是偶数时,$ S_{\text{偶}}=S_{\text{奇}}\bm{\cdot}q  $;\\
当$ n $是奇数时,$ S_{\text{奇}}=S_{\text{偶}}\bm{\cdot}q. $
\end{enumerate}
\section{数列求和相关问题}
\subsection{求前$ n $项和的方法}
\begin{enumerate}
\item 公式法\begin{enumerate}
\item 等差数列的前$ n $项和公式:$S_n=\dfrac{n(a_1+a_n)}{2}=na_1+\dfrac{n(n-1)}{2}d$.
\item 等比数列的前$ n $项和公式:$S_n=\Bigg\{\begin{aligned}
&na_1&\left(q=1\right)\\
&\dfrac{a_1\left(1-q^n\right)}{1-q}&\left(q\ne1\right)
\end{aligned}$
\end{enumerate}
\item 分组求和:把一个数列分成几个可以直接求和的数列;
\item 拆项相消:有时把一个数列的通项公式分成两项差的形式,相加过程中消去中间项,只剩下有限项再求和;
\item 错位相减:适用于一个等差数列和一个等比数列对应项相乘构成的数列求和;
\item 倒序相加:把数列正着写和倒着写再相加,例如等差数列前$ n $项和公式的推导方法.
\end{enumerate}
\subsection{裂项相消法}
\begin{enumerate}
\item 对于裂项后明显有能够相消的项的一类数列,在求和时常用“裂项相消法”,分式数列的求和多用此法;
\item 利用裂项相消法求和时,应注意抵消后并不一定只剩下第一项和最后一项,也可能有前面两相和最后两项,有些情况下,裂项时需要调整前面的系数,使裂开的两项之差和系数之积与原通项相等.
\item 常用的拆项公式:\begin{enumerate}
\item $ \dfrac{1}{n(n+1)}=\dfrac{1}{n}-\dfrac{1}{n+1}; $
\item $ \dfrac{1}{n(n+d)}=\dfrac{1}{d}\left(\dfrac{1}{n}-\dfrac{1}{n+d}\right) $;
\item $\dfrac{1}{\sqrt{n}+\sqrt{n+1}}=\sqrt{n+1}-\sqrt{n}$;
\item $\dfrac{1}{n(n+1)(n+2)}=\dfrac{1}{2}\left[\dfrac{1}{n(n+1)}-\dfrac{1}{(n+1)(n+2)}\right]$
\item 若数列$\{a_n\}$为等差数列,公差为$ d (d\ne0)$,则$ \dfrac{1}{a_n\bm{\cdot}a_{n+1}}=\dfrac{1}{d}\left(\dfrac{1}{a_n}-\dfrac{1}{a_{n+1}}\right). $
\end{enumerate}
\end{enumerate}
\begin{proof}
对于分式数列,通常会考虑裂项相消法进行消项,对于$ \dfrac{1}{n(n+d)} $式数列,可以使用待定系数法得到展开式,
假设:$$ \dfrac{1}{n(n+d)}=\dfrac{k}{n}-\dfrac{k}{n+d} ~(k\text{为待定系数})$$
右边通分有$$\dfrac{1}{n(n+d)}= \dfrac{k}{n}-\dfrac{k}{n+d}=\dfrac{kd}{n(n+d)} $$
即有$ kd=1 $,算得$ k=\dfrac{1}{d} $.即得证
\end{proof}
\subsection{错位相减法}
\begin{enumerate}
\item 一般地,如果数列$\{a_n\}$是等差数列,数列$\{b_n\}$是等比数列,求数列$ \left\{a_n\bm{\cdot}b_n\right\} $的前$ n $项和时,可以采用错位相减法.
\item 应用等比数列求和公式时,必须注意公比$ q\ne1 $这一前提条件,如果不能确定公比$ q $是否为$ 1 $,应分两种情况进行讨论.
\end{enumerate}
\begin{proof}
设数列$ \left\{a_n\right\} $为等差数列,公差为$ d $,数列$ \left\{b_n\right\} $为等比数列,公比为$ q ~(q\ne1)$,数列$ \left\{c_n\right\}$满足$ c_n=a_n\bm{\cdot}b_n  $,则数列$ \left\{c_n\right\} $有:\begin{equation*}
\begin{aligned}
S_n=&c_1+c_2+c_3+\cdots+c_n\\
=&a_1b_1+a_2b_2+a_3b_3+\cdots+a_nb_n\\
qS_n=&a_1b_1q+a_2b_2q+\cdots+a_nb_nq\\
=&a_1b_2+a_2b_3+a_3b_4+\cdots+a_nb_{n+1}\\
S_n-qS_n=&a_1b_1+b_2\left(a_2-a_1\right)+b_3\left(a_3-a_2\right)+\cdots++b_n\left(a_n-a_{n-1}\right)-a_nb_{n+1}\\
=&a_1b_1+db_2+db_3+\cdots+db_n-a_nb_{n+1}\\
=&a_1b_1+d\left(b_2+b_3+\cdots+b_n\right)-a_nb_{n+1}\\
=&a_1b_1-a_nb_{n+1}+\dfrac{b_2\left(1-q^{n-1}\right)}{1-q}d.
\end{aligned}
\end{equation*}
故而有:$$S_n=\dfrac{a_1b_1-a_nb_{n+1}+\dfrac{b_2\left(1-q^{n-1}\right)}{1-q}d}{1-q}$$
\end{proof}
\newpage 
\section{练习}
\begin{questions}
 
\question
设等差数列$\left\{a_n\right\}$的前$n$项和为$S_n$,$S_{m-1}=-2,S_m=0,S_{m+1}=3$,则$m=$\xx
\onech{3}{4}{5}{6}
\qs
已知$\left\{a_n\right\}$是公差为1的等差数列,$S_n$为$\left\{a_n\right\}$的前$n$项和,若$S_8=4S_4$,则$a_{10}=$\xx
\onech{$\dfrac{17}{2}$}{$\dfrac{19}{2}$}{$10$}{$12$}
\qs 设$S_n$是等差数列$\left\{a_n\right\}$的前$n$项和,若$a_1+a_3+a_5=3$,则$S_5=$\xx
\onech{$5$}{$7$}{$9$}{$11$}
\qs 等比数列$\{a_n\}$满足$a_1=3$,$a_1+a_3+a_5=21$,则$ a_3+a_5+a_7= $\xx
\onech{21}{42}{63}{84}
\qs 已知等比数列$\left\{a_n\right\}$满足$a_1=\dfrac{1}{4}$,$a_3a_5=4(a_4-1)$,则$a_2=$\xx
\onech{$2$}{$1$}{$\dfrac{1}{2}$}{$\dfrac{1}{8}$}
\qs 若数列$\{a_n\}$满足$a_{n+1}=2a_n\left(a_n\ne 0, n\in\mathbf{N^*}\right)$,且$a_2$与$a_4$的等差中项是$ 5 $,则$ a_1+a_2+\cdots +a_n$等于\xx
\onech{$ 2^n$}{$ 2^n-1$}{$ 2^{n-1}$}{$ 2^{n-1}-1$}
\qs
已知等差数列$ \left\{a_n\right\} $的前$ 9 $项和为$27$,$a_{10}=8$,求$a_{100}=$\xx
\onech{100}{99}{98}{97}
\qs 已知数列$\{a_n\}$满足$ a_1+a_2+\cdots+a_n=2a_2 \left(n=1,2,3,\cdots\right)$,则\xx
\onech{$ a_1<0$}{$ a_1>0$}{$ a_1\ne a_2$}{$ a_2=0$}
\qs
设$ \left\{a_n\right\} $是公比为$q$的等比数列,则“$ q>1$”是“$ \left\{a_n\right\} $为递增数列”的\xx
\twoch{充分且不必要条件}{必要且不充分条件}{充分必要条件}{既不充分也不必要条件}
\qs
下面是关于公差$d>0$的等差数列$ \left\{a_n\right\} $的四个命题:\\
$p_1$:数列$ \left\{a_n\right\} $是递增数列;\qquad
\phantom{p}$p_2$:数列$ \left\{na_n\right\} $是递增数列;\\
$p_3$:数列$ \left\{\dfrac{a_n}{n}\right\} $是递增数列;\qquad
$p_4$:数列$\left\{a_n+3nd\right\}$是递增数列.\\
其中的真命题为\xx
\onech{$p_1,p_2$}{$p_3,p_4$}{$p_2,p_3$}{$p_1,p_4$}
\qs 已知各项都为正数的等比数列$\{a_n\}$,$a_1a_2a_3=5,~a_7a_8a_9=10,~  $则$ a_4a_5a_6= $\xx
\onech{$ 5\sqrt{2} $}{$ 7 $}{$ 6 $}{$ 4\sqrt{2} $}
\qs 已知数列$\{a_n\}$的前$n$项和为$S_n$,$a_1=1$,$S_n=2a_{n+1}$,~则$S_n=$\xx
\onech{$2^{n-1}$}{$ \left(~\dfrac{3}{2}~\right)^{n-1} $}{$ \left(~\dfrac{2}{3}~\right)^{n-1} $}{$ \dfrac{1}{2^{n-1}} $}
\qs
在等比数列$\{a_n\}$中,$a_1=1$,公比$ |q|\ne 1 $.若$ a_m=a_1a_2a_3a_4a_5 $,则$ m= $\xx
\onech{9}{10}{11}{12}

\qs 
设$ \left\{a_n\right\} $是等差数列,下列结论中正确的是\xx
\twoch{若$a_1+a_2>0$,则$a_2+a_3>0$}{若$a_2+a_3>0$,则$a_1+a_2<0$}{若$0<a_1<a_2$,则$ a_2>\sqrt{a_1a_3} $}{若$a_1<0$,则$(a_2-a_1)(a_2-a_3)>0$}

\qs 数列$ \left\{a_n\right\} $满足$ a_{n+1}+(-1)^na_n=2n-1 $,则$\{a_n\}$的前$ 60 $项和为\xx
\onech{3690}{3660}{1845}{1830}
\qs 设$S_n$是等差数列$\left\{a_n\right\}$的前n项和,若$\dfrac{a_5}{a_3}=\dfrac{5}{9}$,~则$ \dfrac{S_9}{S_5} $=\xx
\onech{1}{$ -1 $}{$ 2 $}{$ \dfrac{1}{2} $}

\qs 已知某等差数列共有10项,其奇数项之和为15,偶数项之和为30,则其公差为\xx
\onech{2}{3}{4}{5}
\qs 在各项均不为零的等差数列$\left\{a_n\right\}$中,若$ a_{n+1}+a^2_n+a_{n-1}=0(n\ge 2),~$则$ S_{2n-1}-4n= $\xx
\onech{-2}{0}{1}{2}
\qs 等差数列$\{a_n\}$的前$ n $ 项和为$S_n$,~已知$ a_{m-1}+a_{m+1}-a_m^2=0,~S_{2m-1}=38,~ $则$ m= $\xx
\onech{38}{20}{10}{9}
\qs 已知数列$\{a_n\}$为等比数列,下面结论中正确的是\xx
\twoch{$ a_1+a_3\ge 2a_2 $}{$ a^2_1+a^2_3\ge 2a^2_2 $}{$ \text{若}a_1=a_3,~\text{则}a_1=a_2 $}{$ \text{若}a_3>a_1,~\text{则}a_4>a_2 $}
\qs 若等比数列$\{a_n\}$满足$ a_na_{n+1}=16^n,~ $则公比$ q= $\xx
\onech{2}{4}{8}{16}
\qs 设等比数列$\{a_n\}$的前$n$项和为$S_n$,若$ S_2=3,~S_4=15,~ $则$ S_6= $\xx
\onech{31}{32}{63}{64}
\qs 已知数列$\{a_n\}$是首项为1的等比数列,$S_n$是$\{a_n\}$的前$ n $项和,且$ 9S_3=S_6 ,~$则数列$ \left\{\dfrac{1}{a_n}\right\} $的前$ 5 $项和为\xx
\onech{$ \dfrac{15}{8}\text{或}~5 $}{$ \dfrac{31}{16}\text{或}~5 $}{$ \dfrac{31}{16} $}{$ \dfrac{15}{8} $}
\qs 若等差数列$\left\{a_n\right\}$满足$ a_7+a_8+a_9>0$,$ a_7+a_{10}<0 $,则当$n=$\tk 时$\left\{a_n\right\}$的前$n$项和最大.  
\qs 在等比数列$\{a_n\}$中,$a_1=\dfrac{1}{2},~a_4=-4$,则公比$ q=\tk $;$ |a_1|+|a_2|+\cdots+|a_n|= $\tk. 
\qs 设等比数列$\{a_n\}$  满足$ a_1+a_3=10 $,$a_2+a_4=5$,则$ a_1a_2\cdots a_n $的最大值为\tk.
\qs 若数列$\{a_n\}$的前$n$项和$S_n=\dfrac{2}{3}a_n+\dfrac{1}{3}$,则数列$\{a_n\}$的通项公式是$a_n=$\tk. 
\qs 若等比数列$\{a_n\}$满足$ a_2+a_4=20,~a_3+a_5=40,~$则公比$ q= $\tk;~前$ n $项和$S_n=$\tk.
\qs 已知等比数列$\{a_n\}$为递增数列,且$ a_5^2=a_{10},~2(a_n+a_{n+2})=5a_{n+1},~ $则数列数列$\{a_n\}$的通项公式$ a_n= $\tk.
\qs 设等比数列$\{a_n\}$的公比为$ q $,前$n$项和为$S_n$,若$ S_{n+1},~S_n,~S_{n+2} $成等差数列,则$ q $的值为\tk.
\question
设$S_n$是数列$\{a_n\}$的前$n$项和,且$a_1=-1$,$a_{n+1}=S_nS_{n+1}$,则$S_n=$\tk.
\qs 已知数列$\{a_n\}$的前$n$项和为$S_n,a_n\ne0\left(n\in\mathbf{N^*}\right),~a_na_{n+1}=S_n$.则$ a_3-a_1 =$\tk.
\qs 设数列$\left\{a_n\right\}$,~$\left\{b_n\right\}$都是等差数列,若$ a_1+b_1=7,~a_3+b_3=21,~ $则$ a_5+b_5= $\tk.
\qs 若数列$\{a_n\}$满足$a_1=-2$,且对于任意的$ m,n \in\mathbf{N^*}$,都有$ a_{m+n}=a_m\bm{\cdot}a_n $,则$ a_3=\tk $;数列$\{a_n\}$的前$ 10 $项和$ S_{10}=\tk. $
\qs 已知数列$\{a_n\}$的前$n$项和为$S_n$,满足$a_1=1,a_2=-2$,且$a_{n+1}=a_{n}+a_{n+2},n\in\mathbf{N^*}$,则$ a_5= $\tk;数列$\{a_n\}$的前$ 2016 $项的和为\tk.
\clearpage
\question
已知数列$\{a_n\}$满足$a_1=1,a_{n+1}=3a_n+1$,其中$n\in \mathbf{N^*}.$
\begin{parts}
\part 证明$\{a_n+\dfrac{1}{2}\}$是等比数列,并求$\{a_n\}$的通项公式;
\part 证明$\dfrac{1}{a_1}+\dfrac{1}{a_2}\cdots\dfrac{1}{a_n}<\dfrac{3}{2}$.
\end{parts}
\kongbai
\qs 数列$\{a_n\}$是各项都为正数的等比数列,$ a_{11}=8 ,~$设$ b_n=\log_2a_n,~ $且$ b_4 =17.$
\begin{parts}
\part 求证:数列$\{b_n\}$是以$ -2 $为公差的等差数列;
\part 设数列$\{b_n\}$的前$n$项和为$ S_n $,~求$S_n$的最大值. 
\end{parts}
\kongbai 
\question
已知各项都为正数的数列$\left\{a_n\right\}$满足$a_1=1,a_n^2-(2a_{n+1}-1)a_n-2a_{n+1}=0$.
\begin{parts}
\part 求$a_2,a_3$;
\part 求$\left\{a_n\right\}$的通项公式.
\end{parts}
\kb
\qs 等差数列$\{a_n\}$的前$n$项和为$S_n$,已知$a_1=10,~$$a_2$为整数,且$S_n\le S_4.$
\begin{parts}
\part 求$\{a_n\}$的通项公式;
\part 设$ b_n=\dfrac{1}{a_na_{n+1}},~ $求数列$ \{b_n\} $的前$n$项和$ T_n~. $
\end{parts}
\kongbai
\qs 已知数列$\{a_n\}$是等差数列,且$a_1=2,~a_1+a_2+a_3=12.$
\begin{parts}
\part 求数列$\{a_n\}$的通项公式;
\part 令$b_n=a_n3^n~(x\in \mathbf{R})$,求数列$\{b_n\}$前$n$项和的公式.
\end{parts}
\kongbai
\qs 已知正项数列$ \{b_n\} $的前$n$项和$ B_n=\dfrac{1}{4}(b_n+1)^2,~$求$\{b_n\}$的通项公式.
\kongbai
\question
已知数列$\left\{a_n\right\}$是公差为3的等差数列,数列${b_n}$满足$b_1=1$,$b_2=\dfrac{1}{3}$,$a_nb_{n+1}+b_{n+1}=nb_n$.
\begin{parts}
\part 求$\left\{a_n\right\}$的通项公式;
\part 求$\{b_n\}$的前$n$项和.
\end{parts}
\newpage
\qs 数列$\left\{a_n\right\}$满足$a_1=1$,$a_2=2$,$a_{n+2}=2a_{n+1}-a_n+2$.
\begin{parts}
\part 设$b_n=a_{n+1}-a_n$,证明$\{b_n\}$是等差数列;
\part 求数列$\left\{a_n\right\}$的通项公式.
\end{parts}
\kongbai
\qs 已知等差数列$\left\{a_n\right\}$的公差不为零,$a_1=25$,且$a_1,a_{11},a_{13}$成等比数列.
\begin{parts}
\part 求$\left\{a_n\right\}$的通项公式;
\part 求$a_1+a_4+a_7+\cdots+a_{3n-2}$.
\end{parts}
\newpage
\qs
已知等比数列$\left\{a_n\right\}$的首项$a_1=2$,前$ n $项和$ S_n $,且$ a_2 $是$ 3S_2-4 $与$ 2-\dfrac{5}{2}S_1 $的等差中项.
\begin{parts}
\part 求数列$ \left\{a_n\right\} $的通项公式;
\part 设$ {b_n}=(n+1)a_n $,$ T_n $是数列$ {b_n} $的前$ n $项和,$ n\in\mathbf{N^*} $,求$ T_n $.
\end{parts}
\kongbai
\qs 已知等差数列$\{a_n\}$满足$ a_1+a_2=10 $,~$a_4-a_3=2$.
\begin{parts}
\part 求$\left\{a_n\right\}$的通项公式;
\part  设等比数列$\{b_n\}$满足$ b_2=a_3,~b_3=a_7 $,问:$ b_6 $与数列$\{a_n\}$的第几项相等?
\end{parts}
\newpage
\qs 已知等差数列$\{a_n\}$满足$a_1=3,~a_4=12$,数列$ \left\{b_n\right\} $满足$ b_1=4,~b_4=20 $,且$\{b_n-a_n\}  $是等比数列.
\begin{parts}
\part 求数列$ \{a_n\} $和$ \{b_n\} $的通项公式;
\part 求数列$ \{b_n\} $的前$ n $项和. 
\end{parts}
\kongbai
\qs 等差数列$\{a_n\}$中,$a_3+a_4=4,~a_5+a_7=6$.
\begin{parts}
\part 求$\{a_n\}$的通项公式;
\part 设$ b_n=[a_n] $,求数列$\{b_n\}$的前$ 10 $项和,其中$ [x] $表示不超过$ x $的最大整数,如$ [0.9]=0,~[2.6]=2 .$
\end{parts}
\newpage
\qs 已知数列$\{a_n\}$的前$ n $项和$ S_n=1+\lambda a_n $,其中$ \lambda \ne 0 $.
\begin{parts}
\part 证明$\{a_n\}$是等比数列,并求其通项公式;
\part 若$ S_5=\dfrac{31}{32} $,求$ \lambda $.
\end{parts}
\kongbai
\question
$S_n$为数列$\{a_n\}$的前$n$项和,已知$a_n>0$,$a_n^2+2a_n=4S_n+3$,其中$n\in \mathbf{N}^*$.
\begin{parts}
\part 求$\{a_n\}$的通项公式;
\part 设$b_n=\dfrac{1}{a_na_{n+1}}$,求数列$\{b_n\}$的前$n$项和.
\end{parts}
\kongbai
\qs 已知$ \left\{a_n\right\} $是递增的等差数列,$a_2$,$a_4$是方程$ x^2-5x+6=0 $的根.
\begin{parts}
\part 求$ \left\{a_n\right\}  $的通项公式;
\part 求数列$ \left\{\dfrac{a_n}{2^n}\right\} $的前$ n $项和.
\end{parts}
\kongbai
\qs 已知等差数列$\{a_n\}$的前$ n $项和$ S_n $满足$ S_3=0,~S_5=-5 $.
\begin{parts}
\part 求$ \left\{a_n\right\} $的通项公式;
\part 求数列$ \dfrac{1}{a_{2n-1}a_{2n+1}} $的前$ n $项和 
\end{parts}
\kongbai
\qs 等比数列$\{a_n\}$的各项均为正数,且$ 2a_1+3a_2=1,~a^2_3=9a_2a_6 $.
\begin{parts}
\part 求数列$\{a_n\}$的通项公式;
\part 设$b_n=\log_3a_1+\log_3a_2+\cdots+\log_3a_n  $,求数列$\left\{ \dfrac{1}{b_n} \right\} $的前$n$项和.
\end{parts}
\kongbai
\qs 已知等差数列$\{a_n\}$和等比数列$\{b_n\}$满足$ a_1=b_1=1,~a_2+a_4=10,~b_2b_4=a_5 .$
\begin{parts}
\part 求$\{a_n\}$的通项公式;
\part 求和:$ b_1+b_3+b_5+\cdots+b_{2n-1} .$
\end{parts}
\end{questions}
\end{document}