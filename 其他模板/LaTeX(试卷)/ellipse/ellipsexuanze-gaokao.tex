\documentclass{BHCexam}
\begin{document}

\biaoti{圆锥曲线选填}
\fubiaoti{}
\maketitle
\section{直线与圆}
\begin{questions}
\qs 已知点$ M(a,b) $在圆$ O:x^2+y^2=1 $外,则直线$ ax+by=1 $与圆$ O $的位置关系是\xx
\onech{相切}{相交}{相离}{不确定} 
\qs 已知圆$ C:x^2+y^2-4x=0 ,l$为过点$ P(3,0) $的直线,则\xx
\fourch{$ l$与$ C $相交}{$ l$与$ C $相切}{$ l$与$ C $相离}{以上三个选项均有可能}
\qs 过点$\left(1,0\right)$且与直线$ x-2y-2=0 $平行的直线方程是\xx
\twoch{$ x-2y-1=0$}{$ x-2y+1=0$}{$ 2x+y-2=0$}{$ x+2y-1=0$}
\qs 已知直线$ 3x+4y-3=0 $与直线$ 6x+my+14=0 $平行,则它们之间的距离是\xx
\onech{$ \dfrac{17}{10}$}{$ \dfrac{17}{5}$}{$ 8$}{$ 2$}
\qs 直线$ ax+by+c=0 $同时要经过第一、第二、第四象限,则$ a,b,c $要满足\xx
\twoch{$ ab>0,\; bc<0$}{$ ab>0,\; bc>0$}{$ ab<0,\; bc>0$}{$ ab<0,\; bc<0$}
\qs 若动点$ P_1\left(x_1,y_1 \right),P_2\left(x_2,y_2\right) $分别在直线$ l_1:x-y-5=0 ,\ l_2:x-y-15=0$上移动,则$ P_1P_2 $的中点$ P $到原点的距离的最小值是\xx
\onech{$ \dfrac{5\sqrt{2}}{2}$}{$ 5\sqrt{2}$}{$ \dfrac{15\sqrt{2}}{2}$}{$ 15\sqrt{2}$}
\qs 已知$ a\ne 0 $,直$ ax+(b+2)y+4=0 $与直线$ ax+(b-2)y-3=0 $互相垂直,则$ ab $的最大值为\xx
\onech{$ 0$}{$ 2$}{$ 4$}{$ \sqrt{2}$}
\qs 与直线$ x-y-4=0 $和圆$ x^2+y^2+2x-2y=0 $都相切的半径最小的圆的方程是\xx
\twoch{$ \left(x+1\right)^2+\left(y+1\right)^2=2$}{$ \left(x+1\right)^2+\left(y+1\right)^2=4$}{$ \left(x-1\right)^2+\left(y+1\right)^2=2$}{$\left(x-1\right)^2+\left(y+1\right)^2=4 $}
\qs 已知直线通过点$ M\left(-3,4\right) $,被直线$ l:x-y+3=0 $反射,反射光线通过点$ N\left(2,6\right) $,则反射光线所在直线的方程是\tk.
\qs 设直线$ l:~3x+4y+a=0,~ $圆$ C:(x-2)^2+y^2=2 $,若在圆$ C $上存在两点$ P,~Q,~ $在直线$ l $上存在一点$ M $,使得$ \angle PMQ =90^{\circ}$,则$ a $的取值范围是\xx
\twoch{$ \left[-18,6\right]$}{$ \left[6-5\sqrt{2},6+5\sqrt{2}\right]$}{$ \left[-16,4\right]$}{$ \left[-6-5\sqrt{2},-6+5\sqrt{2}\right]$}
\qs 已知圆$ C_1:\left(x-2\right)^2+\left(y-3\right)^2=1 $,圆$ C_2:\left(x-3\right)^2+\left(y-4\right)^2=9 $,$ M,\ N $分别是圆$ C_1,\ C_2 $上的动点,$ P $为$x$轴上的动点,则$ \abs{PM}+\abs{PN} $的最小值为\xx
\onech{$ 5\sqrt{2}-4$}{$ \sqrt{17}-1$}{$ 6-2\sqrt{2}$}{$ \sqrt{17}$}
\qs 已知圆$ C:(x-3)^2+(y-4)^2=1 $和两点$ A(-m,0),B(m,0)~(m>0) $,若圆上存在点$ P $,使得$ \angle APB=90^{\circ} $,则$ m $的最大值为\xx
\onech{$7$}{$6$}{$5$}{$4$}
\qs 设$ m,n\inR $,若直线$ (m+1)x+(n-1)y-2=0 $与$ \left(x-1\right)^2+\left(y-1\right)^2=1 $相切,则$ m+n $的取值范围是\xx
\twoch{$ \left[1-\sqrt{3},1+\sqrt{3}\right]$}{$ \left(-\infty,1-\sqrt{3}\right]\cup\left[1+\sqrt{3},+\infty\right)$}{$ \left[2-2\sqrt{2},2+2\sqrt{2}\right]$}{$ \left(-\infty,2-2\sqrt{2}\right]\cup\left[2+2\sqrt{2},+\infty\right)$}
\qs 若直线$ l:\dfrac{x}{a}+\dfrac{y}{b}=1\left(a>0,\; b>0\right) $经过点$ \left(1,2\right) $,则直线$ l $在$x$轴和$y$轴上的截距之和的最小值是\tk.
\qs 已知点$ A(1,0),B(3,0) $,若直线$ y=kx+1 $上存在点$ P $,满足$ PM\bot PB $,则$ k $的取值范围是\tk.

\end{questions}
\section{圆锥曲线}

\begin{questions}
\qs 已知$ P(5,2) ,\ F_1(-6,0),\ F_2(6,0)$三点,那么以$ F_1,\ F_2 $为焦点且过点$ P $的椭圆的短轴长为\xx
\onech{$ 3$}{$ 6$}{$ 9$}{$ 12$}
\qs 已知$ F_1(-1,0) ,\ F_2(1,0)$是椭圆$ C $的两个焦点,过$ F_2 $且垂直于$x$轴的直线交$ C $于$ A,\ B $两点,且$ \abs{AB}=3 $,则$ C $的方程是\xx
\twochx{$\dfrac{x^2}{2}+y^2=1 $}{$ \dfrac{x^2}{3}+\dfrac{y^2}{2}=1$}{$ \dfrac{x^2}{4}+\dfrac{y^2}{3}=1$}{$ \dfrac{x^2}{5}+\dfrac{y^2}{4}=1$}
\qs 设$ P,Q $分别为圆$ x^2+\left(y-6\right)^2 =2$和椭圆$ \dfrac{x^2}{10} +y^2=1$上的点,则$ P,Q $两点间的最大距离是\xx
\onech{$ 5\sqrt{2}$}{$ \sqrt{46}+\sqrt{2}$}{$ 7+\sqrt{2}$}{$ 6\sqrt{2}$}
\qs 设$ P $是双曲线$\dfrac{x^2}{a^2}-\dfrac{y^2}{9}=1$上一点,双曲线的一条渐近线方程为$ 3x-2y=0 ,~F_1,~F_2$分别是双曲线的左、右焦点,若$ \abs{PF_1}=3 $,则$ \abs{PF_2} =$\xx
\onech{$ 1\text{或}5$}{$ 6$}{$ 7$}{$ 9$}


\question
已知$O$为坐标原点,$F$是椭圆$C$:$\dfrac{x^2}{a^2}+\dfrac{y^2}{b^2}=1~(a>b>0)$的左焦点,$A$,$B$分别是$C$的左、右顶点.~$P$为$C$上一点,且$PF\bm{\bot} x$轴,且过点$A$的直线$l$与线段$PF$交于点$M$,与$y$轴交于点$E$,若直线$BM$经过$OE$的中点,则$C$的离心率为\xx 
\onech{$\dfrac13$}{$\dfrac12$}{$\dfrac23$}{$\dfrac34$}
\question
已知方程$\dfrac{x^2}{m^2+n}-\dfrac{y^2}{3m^2-n}=1$表示双曲线,且该双曲线两焦点间的距离为4,则$n$的取值范围是\xx
\onech{$(-1,3)$}{$\left(-1,\sqrt{3}\right)$}{$(0,3)$}{$\left(0,\sqrt{3}\right)$}
\qs 已知椭圆$ \dfrac{x^2}{16} +\dfrac{y^2}{9}=1$的左、右焦点分别为$ F_1,\ F_2 ,~$点$ P $在椭圆上,若$ P,\ F_1,\ F_2 $是一个直角三角形的三个顶点,则点$P $到$x$轴的距离为\xx
\onech{$ \dfrac{9}{5}$}{$ 3$}{$ \dfrac{9\sqrt{7}}{7}$}{$ \dfrac{9}{4}$}

\qs 若实数$ k $满足$ 0<k<9 $,则曲线$ \dfrac{x^2}{25}-\dfrac{y^2}{9-k} =1$与曲线$ \dfrac{x^2}{25-k} -\dfrac{y^2}{9}=1$的\xx
\onech{离心率相等}{虚半轴长相等}{实半轴长相等}{焦距相等}
\qs 已知椭圆$C_1:\dfrac{x^2}{m}+y^2=1(m>1)$与双曲线$ C_2:\dfrac{x^2}{n}-y^2=1~(n>0) $的焦点重合,$ e_1,e_2 $分别为$ C_1,\ C_2 $的离心率,则\xx
\onech{$ m>n\text{且}e_1e_2>1$}{$m>n\text{且}e_1e_2<1 $}{$m<n\text{且}e_1e_2>1 $}{$m<n\text{且}e_1e_2<1 $}


\question
已知$M\left(x_0,y_0\right)$是双曲线$C:\dfrac{x^2}{2}-y^2=1$上的一点,$F_1,F_2$是$C$的两个焦点.~~若$\vv{MF_1}\cdot\vv{MF_2}<0$,则$y_0$的取值范围是\xx
\onechx{$\left(-\dfrac{\sqrt{3}}{3},\dfrac{\sqrt{3}}{3}\right)$}{$\left(-\dfrac{\sqrt{3}}{6},\dfrac{\sqrt{3}}{6}\right)$}{$\left(-\dfrac{2\sqrt{2}}{3},\dfrac{2\sqrt{2}}{3}\right)$}{$\left(-\dfrac{2\sqrt{3}}{3},\dfrac{2\sqrt{3}}{3}\right)$}
\question
已知椭圆$E$:$\dfrac{x^2}{a^2}+\dfrac{y^2}{b^2}=1~(a>b>0)$的右焦点为$F(3,0)$,过点$F$的直线交$E$于$A,B$两点,若$AB$的中点坐标为$(1,-1)$,则$E$的方程为\xx
\onechx{$\dfrac{x^2}{45}+\dfrac{y^2}{36}=1$}{$\dfrac{x^2}{36}+\dfrac{y^2}{27}=1$}{$\dfrac{x^2}{27}+\dfrac{y^2}{18}=1$}{$\dfrac{x^2}{18}+\dfrac{y^2}{9}=1$}

\qs 椭圆$ C:\dfrac{x^2}{4}+\dfrac{y^2}{3}=1 $的左右顶点分别为$ A_1,~A_2 $,点$ P $在$ C $上且直线$ PA_2 $斜率的取值范围是$ \left[-2,-1\right] $,那么直线$ PA_1 $的斜率的取值范围是\xx
\onechx{$ \left[\dfrac{1}{2},\dfrac{3}{4}\right]$}{$\left[\dfrac{3}{8},\dfrac{3}{4}\right] $}{$ \left[\dfrac{1}{2},1 \right]$}{$ \left[\dfrac{3}{4},1\right]$}
\qs 椭圆$ \dfrac{x^2}{12}+\dfrac{y^2}{3}=1 $的焦点为$ F_1 $,点$ P $在椭圆上,如果线段$ PF_1 $的中点$ M $在$y$轴上,那么点$ M $的纵坐标是\xx
\onech{$ \pm \dfrac{\sqrt{3}}{4}$}{$ \pm \dfrac{\sqrt{3}}{2}$}{$ \pm \dfrac{\sqrt{2}}{2}$}{$ \pm\dfrac{3}{4}$}
\qs
已知椭圆$C$:$\dfrac{x^2}{a^2}+\dfrac{y^2}{b^2}=1~(a>b>0)$的左,右焦点分别为$F_1$,$F_2$,离心率为$\dfrac{\sqrt{3}}{3}$,过$F_2$的直线$l$交$C$于$A,B$两点,若$\triangle AF_1B$的周长为$4\sqrt{3}$,则$C$的方程为\xx
\onechx{$\dfrac{x^2}{3}+\dfrac{y^2}{2}=1$}{$\dfrac{x^2}{3}+y^2=1$}{$\dfrac{x^2}{12}+\dfrac{y^2}{8}=1$}{$\dfrac{x^2}{12}+\dfrac{y^2}{4}=1$}
\qs
已知椭圆$C$:$\dfrac{x^2}{4}+\dfrac{y^2}{3}=1$的左,右焦点分别为$F_1,F_2$,椭圆$C$上的点$A$满足$AF_2\bot F_1F_2$,若点$P$是椭圆$C$上的动点,则$\vv{F_1P}\cdot\vv{F_2A}$的最大值为\xx  
\onech{$\dfrac{\sqrt{3}}{2}$}{$\dfrac{3\sqrt{3}}{2}$}{$\dfrac{9}{4}$}{$\dfrac{15}{4}$}
\qs 已知动点$ P(x,y) $在椭圆$ C:\dfrac{x^2}{25}+\dfrac{y^2}{16}=1 $上,$ F $为椭圆$ C $的右焦点,若点$ M $满足$ \left|\vv{MF}\right|=1 $且$\vv{MP}\cdot\vv{MF}=0  $,则$ \left|\vv{PM}\right| $的最小值为\xx
\onech{$ \sqrt{3} $}{3}{$ \dfrac{12}{5} $}{1}
\qs 设点$ M(x_0,1),~ $若在圆$ O: x^2+y^2=1 $上存在点$ N $,使得$ \angle OMN=45^{\circ} $,则$ x_0 $的取值范围是\xx
\onech{$ \left[-1,1\right]$}{$ \left[-\dfrac{1}{2},\dfrac{1}{2}\right]$}{$\left[-\sqrt{2},\sqrt{2}\right] $}{$ \left[-\dfrac{\sqrt{2}}{2},\dfrac{\sqrt{2}}{2}\right]$}
\qs 已知$ F $为双曲线$ C:x^2-my^2=3m~(m>0)$的一个焦点,则点$ F $到$ C $的一条渐近线的距离为\xx
\onech{$ \sqrt{3}$}{$ 3$}{$ \sqrt{3}m$}{$ 3m$}
\qs 已知$ A,B $为双曲线$ E $的左、右顶点,点$ M $在$ E $上,$ \triangle ABM$为等腰三角形,且顶角为$ 120^{\circ} $,则$ E $的离心率为\xx
\onech{$ \sqrt{5}$}{$ 2$}{$ \sqrt{3}$}{$ \sqrt{2}$}
\qs 已知双曲线$ \dfrac{x^2}{4}-\dfrac{y^2}{b^2}=1(b>0) $,以原点为圆心,双曲线的实半轴长为半径长的圆与双曲线的两条渐近线相交于$ A,B,C,D $四点,四边形$ ABCD $的面积为$ 2b $,则双曲线的方程为\xx
\onechx{$ \dfrac{x^2}{4}-\dfrac{3y^2}{4}=1$}{$ \dfrac{x^2}{4}-\dfrac{4y^2}{3}=1$}{$\dfrac{x^2}{4}-\dfrac{y^2}{4} =1$}{$ \dfrac{x^2}{4}-\dfrac{y^2}{12}=1$}
\qs 已知点$ Q\left(2\sqrt{2},0\right) $及抛物线$ x^2=4y $上一动点$ P\left(x,y\right) $,则$ y+\abs{PQ} $的最小值为\xx
\onech{$ \dfrac{1}{2}$}{$ 1$}{$ 2$}{$ 3$}
\qs 已知抛物线$ x^2=4y $上有一条长为$ 6 $的动弦$ AB $,则$ AB $的中点到$x$轴的最短距离为\xx
\onech{$ \dfrac{3}{4}$}{$ \dfrac{3}{2}$}{$ 1$}{$ 2$} 
\qs 若直线$ y=x+b $与曲线$ y=3-\sqrt{4x-x^2} $有公共点,则$ b $的取值范围是\xx
\onech{$ \left[1-2\sqrt{2},1+2\sqrt{2}\right]$}{$ \left[1-\sqrt{2},3\right]$}{$ \left[-1,1+2\sqrt{2}\right]$}{$ \left[1-2\sqrt{2},3\right]$}
\qs 若抛物线$ C:y^2=2px~\left(p>0\right) $的焦点为$ F $,直线$ l $过$ F $且与$ C $交于$ A,\ B $两点,若$ \abs{AF}=3\abs{BF} $,则$ l $的方程为\xx
\twochx{$ y=x-1\ \text{或}\ y=-x+1$}{$ y=\dfrac{\sqrt{3}}{3}\left(x-1\right)\ \text{或}\ y=-\dfrac{\sqrt{3}}{3}\left(x-1\right)$}{$y=\sqrt{3}\left(x-1\right)\ \text{或}\ y=-\sqrt{3}\left(x-1\right) $}{$y=\dfrac{\sqrt{2}}{2}\left(x-1\right)\ \text{或}\ y=-\dfrac{\sqrt{2}}{2}\left(x-1\right) $}
\qs 设$ F_1,F_2   $是椭圆$E:\dfrac{x^2}{a^2}+\dfrac{y^2}{b^2}=1~(a>b>0)$的左,右焦点,$ P $为直线$ x=\dfrac{3a}{2} $上一点,$ \triangle F_1PF_2 $是底角为$ 30^{\circ} $的等腰三角形,则$ E $的离心率为\xx
\onech{$\dfrac{1}{2}$}{$\dfrac{2}{3}$}{$\dfrac{3}{4}$}{$\dfrac{4}{5}$}


\qs 设抛物线$ C:y^2=2px~(p>0) $的焦点为$ F $,点$ M $在$ C $上,$ \abs{MF}=5 $,若以$ MF $为直径的圆过点$ \left(0,2\right) $,则$ C $的方程为\xx
\twoch{$ y^2=4x$或$ y^2=8x $}{$ y^2=2x$或$ y^2=8x $}{$ y^2=4x$或$ y^2=16x $}{$ y^2=2x$或$ y^2=16x $}
\qs 设$ F $为抛物线$ C:y^2=3x $的焦点,过$ F $且倾斜角为$ 30^{\circ} $的直线交$ C $于$ A,B $两点,$ O $为坐标原点,则$ \bm{\triangle}OAB $的面积为\xx
\onech{$ \dfrac{3\sqrt{3}}{4}$}{$ \dfrac{9\sqrt{3}}{8}$}{$ \dfrac{63}{32}$}{$ \dfrac{9}{4}$}
\qs 已知圆$ C:(x-3)^2+(y-4)^2=1 $和两点$ A(-m,0),B(m,0)~(m>0) $,若圆上存在点$ P $,使得$ \angle APB=90^{\circ} $,则$ m $的最大值为\xx
\onech{7}{6}{5}{4}
\qs 设双曲线$ C $的中心为点$ O $,若有且只有一对相交于点$ O $、所成角为$ 60^{\circ} $的直线$ A_1B_1 $和$ A_2B_2 $,使得$ \abs{A_1B_1}=\abs{A_2B_2} $,其中$ A_1,~B_1 $和$ A_2,~B_2 $分别是这对直线与双曲线$ C $的交点,则该双曲线的离心率的取值范围是\xx
\onechx{$ \left(\dfrac{2\sqrt{3}}{3},2\right]$}{$ \left[\dfrac{2\sqrt{3}}{3},2\right)$}{$\left(\dfrac{2\sqrt{3}}{3},+\infty\right) $}{$ \left[\dfrac{2\sqrt{3}}{3},+\infty\right)$}
%\newpage 
\qs 在平面直角坐标系$xOy$中,直线$ l $过抛物线$y^2=4x$的焦点$ F $,且与抛物线相交于$ A,B $两点,其中点$ A $在$x$轴上方.若直线$ l $的倾斜角为$ 60^{\circ} $,则$ \triangle OAF $的面积为\tk.
\qs 过点$ M(1,1) $作斜率为$ -\dfrac{1}{2} $的直线与椭圆$C$:$\dfrac{x^2}{a^2}+\dfrac{y^2}{b^2}=1~(a>b>0)$相交于$ A,B $两点,若$ M $是线段$ AB $的中点,则椭圆$C$的离心率是\tk.
\qs 若椭圆$C$:$\dfrac{x^2}{a^2}+\dfrac{y^2}{b^2}=1$的焦点在$x$轴上,过点$ \left(1,\dfrac{1}{2}\right) $的切线,切点分别为$ A,\ B $,直线$ AB $恰好经过椭圆的右焦点和上顶点,则椭圆方程是\tk. 

%spacing
\begin{spacing}{1.5}
\qs 椭圆$\dfrac{x^2}{a^2}+\dfrac{y^2}{b^2}=1~(a>b>0)$的左、右顶点
分别是$ A,B $,左、右焦点分别是$ F_1,\ F_2 $.若$ \abs{AF_1},\abs{F_1F_2},\abs{F_1B} $成等比数列,则次椭圆的离心率为\tk.\qs 设$ F_1,F_2 $分别是椭圆$ E:x^2+\dfrac{y^2}{b^2}=1 (0<b<1)$的左、右焦点,过点$ F_1 $的直线交椭圆$ E $与$ A,\ B $两点,若$ \abs{AF_1}=3\abs{BF_1},\ AF_2\bot x\text{轴} $,则椭圆$ E $的方程为\tk.

\qs 椭圆$ \dfrac{x^2}{9}+\dfrac{y^2}{2}=1 $的焦点为$ F_1,~F_2 $,点$ P $在椭圆上,若$ \abs{PF_1}=4 $,则$ \abs{PF_2} =\tk,~\angle F_1PF_2$的大小为\tk.
\end{spacing}
\vspace*{-0.3em}
%endspacing

\qs 曲线$ C $是平面内与两个定点$ F_1(-1,0) $和$ F_2(1,0) $的距离的积等于常数$ a^2 (a>1)$的点的轨迹,给出以下三个结论:\\
\ding{192}曲线$ C $过坐标原点;\qquad
\ding{193}曲线$ C $关于原点对称;\\
\ding{194}若点$ P $在曲线$ C $上,则$                                                                                                                                                                                                                              \triangle F_1PF_2 $的面积大于$ \dfrac{1}{2}a^2 $.\\
其中,所有正确的结论的序号是\tk.
\qs 已知双曲线$\dfrac{x^2}{a^2}-\dfrac{y^2}{b^2}=1$的离心率为$ 2 $,焦点与椭圆$ \dfrac{x^2}{25}+\dfrac{y^2}{9}=1 $的焦点相同,那么双曲线的焦点坐标为\tk;渐近线方程为\tk.
\qs 设双曲线$ C $经过点$ \left(2,2\right) $,且与$ \dfrac{y^2}{4}-x^2=1 $具有相同渐近线,则$ C $的方程为\tk;渐近线方程为\tk. 
\qs 已知动点$ P $到定点$ A(-2,0) $与点$ B(2,0)$的斜率之积为$ -\dfrac{1}{4} $,点$ P 		$的轨迹方程为\tk.
\qs 双曲线$\dfrac{x^2}{a^2}-\dfrac{y^2}{b^2}=1~(a>0,b>0)$的渐近线为正方形$ OABC $的边$ OA,OC $所在的直线,点$ B $为该双曲线的焦点,若正方形$ OABC $的边长为$ 2 $,则$ a= $\tk.
\qs 若双曲线$ M $上存在四个点$ A,B,C,D $,使得四边形$ ABCD $是正方形,则双曲线$ M $的离心率的取值范围是\tk.
\qs 设$ F_1,F_2 $分别为椭圆$ \dfrac{x^2}{3}+y^2=1 $的左、右焦点,点$ A,~B $在椭圆上,若$ \vv{F_1A}=5\vv{F_2B} $,则点$ A $的坐标是\tk.
\qs 设直线$ x-3y+m=0 ~(m\ne0)$与$ \dfrac{x^2}{a^2}-\dfrac{y^2}{b^2}=1~(a>0,b>0)$的两条渐近线分别交于$ A,B .$若点$ P(m,0) $满足$ \abs{PA}=\abs{PB} $,则该双曲线的离心率是\tk.
\qs 已知$ P $是椭圆$C$:$\dfrac{x^2}{25}+\dfrac{y^2}{16}=1$上的一点,$ M,N $分别是圆$ \left(x+3\right)^2+y^2=1 $和圆$ \left(x-3\right)^2+y^2=4 $上的点,则$ \abs{PM}+\abs{PN} $的最小值为\tk.
\qs 设抛物线$ C:y^2=4x $的焦点为$ F $,$ M $为抛物线$ C $上一点,$ N(2,2) $,则$ \abs{MN}+\abs{MF} $的取值范围是\tk.
\end{questions}
\end{document}