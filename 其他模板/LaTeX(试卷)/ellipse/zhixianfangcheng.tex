\documentclass{BHCexam}
%\usepackage{booktabs}
%\usepackage{colortbl}
%\usepackage{multirow}
\begin{document}
\section{交点和距离问题} 
\subsection{直线交点}
设两条直线的方程为$ l_1:A_1x+B_1y+C_1=0,l_2:A_2x+B_2y+C_2=0 $,两条直线是否有交点,就看这两条直线所组成的方程组$\begin{dcases}
A_1x+B_1y+C_1=0\\A_2x+B_2y+C_2=0
\end{dcases}$是否有为唯一解.
\begin{enumerate}[1)]
\item 若方程组无解,则直线$ l_1,l_2 $平行;反之,也成立;
\item 若方程组有无穷多个解,则直线$ l_1,l_2 $重合;反之,也成立;
\item 当有交点时$\left( A_1B_2-A_2B_1\ne0 \right)$,方程组的解就是交点坐标.
\end{enumerate}
\subsection{距离公式}
\subsubsection{两点间距离公式}
平面内两点$ P_1(x_1,y_1) ,\ P_2(x_2,y_2)$间的距离公式为$ \abs{P_1P_2}=\sqrt{\left(x_1-x_2\right)^2+\left(y_1-y_2\right)^2} $\par 
{\kaishu 若$ P_1,\ P_2 $在直线$ y=kx+b $上,则可根据$ y_1=kx_1+b,\ y_2=kx_2+b $代入两点间的距离公式得到距离公式:\[d=\sqrt{\left(1+k^2\right)\left(x_1-x_2\right)^2}=\sqrt{\left(1+k^2\right)\left[\left(x_1+x_2\right)^2-4x_1x_2\right]}\]
如果用$ y $置换$ x $,则有$ d=\sqrt{\left(1+\dfrac{1}{k^2}\right)(y_1-y_2)^2} $}

\subsubsection{点到直线的距离}
点$ P(x_0,y_0) $到直线$ Ax+By+C=0 (A,B\text{不同时为零})$的距离为$ d=\dfrac{\abs{Ax_0+By_0+C}}{\sqrt{A^2+B^2}} $.
\subsubsection{平行直线间的距离}
两条平行直线$l_1: Ax+By+C_1=0 $和$l_2: Ax+By+C_2=0 (C_1\ne C_2)$间的距离为$ d=\dfrac{\abs{C_1-C_2}}{\sqrt{A^2+B^2}} $\par
{\kaishu 可以在直线$ l_1 $上取点$ P(x_0,y_0) $,代入点到直线的公式中得到证明}
  

\section{对称性问题}
\subsection{求点关于点对称的点}
求点$ P(x_0,y_0) $关于点$ A(a,b) $的对称点$ P' $的问题,主要根据$ A $是线段$ PP' $的中点来解决.
设点$ P'(x_1,y_1) $,则由中点性质有:
$\begin{dcases}
\dfrac{x_0+x_1}{2}=a,\\
\dfrac{y_0+y_1}{2}=b.
\end{dcases}$解得$ P'(2a-x_0,2b-y_0) $.
\subsection{求点关于直线对称的点}
设点$ P(x_0,y_0) $关于直线$ y=kx+b $的对称点为$ P'(x',y') $,则由:
$\begin{dcases}
\dfrac{y'-y_0}{x'-x_0}\bm{\cdot}k=-1,\\
\dfrac{y'+y_0}{2}=k\bm{\cdot}\dfrac{x'+x_0}{2}+b.
\end{dcases}
$可求出$ x',\ y' $.\par
几种特殊位置的对称:
\begin{center}
%指定列宽度,指定居中
\begin{tabular}{c|p{3cm}<{\centering}|p{3cm}<{\centering}}
%\toprule
\hline 
\rowcolor[rgb]{.8,.9,.9}
\textbf{点}&\textbf{对称轴}&\textbf{对称点坐标}\\
\hline 
\multirow{6}{*}{$P(a,b)$}&$x$轴&$(a,-b)$\\
\cline{2-3}
&$y$轴&$(-a,b)$\\
\cline{2-3}
&$y=x$&$(b,a)$\\
\cline{2-3}
&$y=-x$&$(-b,-a)$\\
\cline{2-3}
&$x=m(m\ne0)$&$(2m-a,b)$\\
\cline{2-3}
&$y=n(n\ne0)$&$(a,2n-b)$\\
%\bottomrule
\hline 
\end{tabular}
\end{center}

\subsection{求直线关于点对称的直线}
求直线$ l $关于点$ M(m,n) $对称直线$ l' $的问题,主要依据$ l' $上任一点$ T(x,y) $关于$ M(x,y) $的对称点$ T'(2m-x,2n-y) $在$ l $上来求解.
\subsection{直线关于直线对称的直线}
曲线$ f(x,y)=0 $关于直线$ y=kx+b $的对称曲线求法:\par 
设曲线$ f(x,y)=0 $上任意一点为$ P(x_0,y_0) $,$ P $点关于直线$ y=kx+b $的对称点为$ P'(x,y) $,则$ P $与$ P' $的坐标满足
\[\begin{dcases}
\dfrac{y-y_0}{x-x_0}\bm{\cdot}k=-1,\\
\dfrac{y+y_0}{2}=k\bm{\cdot}\dfrac{x+x_0}{2}+b.
\end{dcases}\]
从而解出$ x_0,y_0 $,代入已知曲线$ f(x,y)=0 $,应有$ f(x_0,y_0) =0$.利用坐标代换法就可求出曲线$ f(x,y)=0 $关于直线$ y=kx+b $对称的曲线方程.\par 
{\kaishu 曲线$ f(x,y)=0 $关于直线$ x+y+c=0 $的对称方程为$ f(-y-c,-x-c)=0 $,关于直线$ x-y+c=0 $的对称曲线的方程为$ f(y-c,x+c)=0 $.\par
}
\section{轨迹}
轨迹问题基本步骤:(直译法)
\begin{enumerate}[1)]
\item \textbf{建系设点}\par 
 建立适当坐标系,用有序数对$ (x,y) $表示曲线上任意一点$ M $的坐标;
\item \textbf{列式}\par 
写出适合条件$ p $的点$ M $的集合$ P=\left\{M\left|p(M)\right.\right\} $;
\item \textbf{代换}\par 
用坐标表示条件$ p\left(M\right) $,列出方程$ f(x,y)=0 $;
\item \textbf{化简}\par 
把方程$ f(x,y)=0 $化简为最简形式;
\item \textbf{查漏除杂}\par 
验证方程表示的曲线是否为已知曲线,重点检查方程表示的曲线是否有多余的点,或者曲线上是否有遗漏的点.
\end{enumerate}
\subsection{定义法}
若动点运动的几何条件恰好与某圆锥曲线的定义吻合,可直接根据定义建立动点的轨迹方程.用定义法可以先确定曲线的类型与方程的具体结构式,再用待定系数法求之.
\subsection{直译法}
直接将动点满足的几何等量关系“翻译”成动点坐标所满足的关系式,得方程$ f(x,y)=0 $,即为所求动点的轨迹方程.
\subsection{相关点法}
若所求轨迹上的动点$ P(x,y) $与另一个已知曲线$ C:F(x,y)=0 $上的动点$ Q(x_1,y_1) $存在某种联系,可把点$ Q $的坐标用点$ P $的坐标表示出来,然后代入已知曲线$ C $的方程$ F(x,y)=0 $,化简即得所求轨迹方程.
\subsection{参数法}
如果动点$ P(x,y) $的坐标之间的关系不易找到,可先考虑将$ x,\ y $用一个或几个参数来表示,消去参数得轨迹方程,此法称为参数法.
\subsection{交轨法}
在求动点的轨迹问题时,有时会出现求两动曲线交点的轨迹问题,这类问题常常通过解方程组得出交点(含参数)的坐标,再消去参数得出所求轨迹的方程,该方程经常与参数法并用.
\end{document}