\documentclass{BHCexam}	
\begin{document}
\biaoti{北京高考分项练习——圆锥曲线}
\fubiaoti{}
\maketitle
 \begin{questions}
\question (2018理) 已知抛物线$C:y^2=2px$经过点$P(1,2)$.过点$Q(0,1)$的直线$l$与抛物线$C$有两个不同的交点$A,~B$,且直线$PA$交$y$轴于$M$,直线$PB$交$y$轴于$N$.
\begin{parts}
\part 求直线$l$的斜率的取值范围;
\part 设$O$为原点,$\vv{QM}=\lambda\vv{QO},~\vv{QN}=\mu\vv{QO}$,求证:$\dfrac{1}{\lambda}+\dfrac{1}{\mu}$为定值.
\end{parts}
\kongbai
\qs (2018文)
已知椭圆$M:\dfrac{x^2}{a^2}+\dfrac{y^2}{b^2}=1~(a>b>0)$的离心率为$\dfrac{\sqrt{6}}{3}$,焦距为$2\sqrt{2}$.斜率为$k$的直线$l$与椭圆$M$有两个不同的交点$A,~B$.
\begin{parts}
\part 求椭圆$M$的方程;
\part 若$k=1$,求$\abs{AB}$的最大值;
\part 设$P(-2,0)$.直线$PA$与椭圆$M$的另一个交点为$C$,直线$PB$与椭圆$M$的另一个交点为$D$,若$C,~D$和点$Q\left(-\dfrac{7}{4},\dfrac{1}{2}\right)$共线,求$k$.
\newpage
\end{parts}
\qs (2017理)已知抛物线$ C:y^2=2px $过点$ P\left(1,1\right) $,过点$ \left(0,\dfrac{1}{2}\right) $作直线$ l $与抛物线$ C $交于不同的两点$ M,N $,过点$ M $作$x$轴的垂线分别与直线$ OP,\ ON $交于点$ A,\ B $,其中$ O $为原点.
\begin{parts}
\part 求抛物线$ C $的方程,并求其焦点坐标和准线方程; 
\part 求证:$ A $为线段$ BM $的中点.
\end{parts}

\kongbai 
\qs (2017文)已知椭圆$C$的两个顶点分别为$ A(-2,0),B(2,0) $.焦点在$x$轴上,离心率为$ \dfrac{\sqrt{3}}{2} $.
\begin{parts}
\part 求椭圆$C$的方程;
\part 点$ D $为$x$轴上一点,过$ D $作$x$轴的垂线交椭圆$C$于不同的两点$ M,N $,过$ D $作$ AM $的垂线交$ BN $于点$ E $.求证:$ \bm{\triangle} BDE $与$ \bm{\triangle} BDN $的面积之比为$ 4:5 $.
\end{parts}
\newpage 
\question
(2016理)已知椭圆$C$:$\dfrac{x^2}{a^2}+\dfrac{y^2}{b^2}=1$的离心率为$\dfrac{\sqrt{3}}{2}$,$A(a,0)$,~$B(0,b)$,~$O(0,0)$,~$\triangle OAB$的面积为1.
\begin{parts}
\part 求椭圆的方程;
\part 设$P$是椭圆$C$上一点,直线$PA$与$y$轴交与点$M$,直线$PB$与$x$轴交与点$N$,求证:$\abs{AN}\cdot\abs{BM}$为定值.
\end{parts}
\kongbai
\question
(2016文)已知椭圆$C$:$\dfrac{x^2}{a^2}+\dfrac{y^2}{b^2}=1(a>b>0)$过点$A(2,0)$,$B(0,1)$两点.
\begin{parts}
\part 求椭圆$C$的方程及离心率;
\part 设$P$为第三象限内一点且在椭圆$C$上,直线$PA$与$y$轴交于点$M$,直线$PB$与$x$轴交于点$N$,求证:四边形$ABNM$的面积为定值.
\end{parts}
\newpage
\question
(2015理)已知椭圆$C$:$\dfrac{x^2}{a^2}+\dfrac{y^2}{b^2}=1(a>b>0)$的离心率为$\dfrac{\sqrt{2}}{2}$,点$P(0,1)$和点$A(m,n)(m\ne 0)$都在椭圆$C$上,直线$PA$交$x$轴于点$M$.
\begin{parts}
\part 求椭圆$C$的方程,并求点$M$的坐标(用$m$,$n$表示);
\part 设$O$为原点,点$B$与点$A$关于$x$轴对称,直线$PB$交$x$轴于点$N$.问:$y$轴上是否存在点$Q$,使得$\angle OQM=\angle ONQ~$\textbf{?}若存在,求点$Q$的坐标;若不存在,说明理由.
\end{parts}
\kongbai
\question
(2015文)已知椭圆$C$:$x^2+3y^2=3$,过点$D(1,0)$且不过点$E(2,1)$的直线与椭圆$C$交于$A$,$B$两点,直线$AE$与直线$x=3$交于点$M$.
\begin{parts}
\part 求椭圆$C$的离心率;
\part 若$AB$垂直于$x$轴,求直线$BM$的斜率;
\part 试判断直线$BM$与直线$DE$的位置关系,并说明理由
\end{parts}
\newpage
\question
(2014理)已知椭圆$C$:$x^2+2y^2=4$.
\begin{parts}
\part 求椭圆$C$的离心率;
\part 设$O$为原点,若点$A$在椭圆$C$上,点$B$在直线$y=2$上,且$OA\bot OB$ ,求直线$AB$与圆$x^2+y^2=2$的位置关系,并证明你的结论.
\end{parts}
\kongbai
\question
(2014文)已知椭圆$C$:$x^2+2y^2=4$.
\begin{parts}
\part 求椭圆$C$的离心率;
\part 设$O$为原点,若点$A$在直线$y=2$上,点$B$在椭圆$C$上,且$OA\bot OB$,求线段$AB$长度的最小值.
\end{parts}
\newpage

\question
(2013理)已知$A$、$B$、$C$是椭圆$W$:$\dfrac{x^2}{4}+y^2=1$上的三个点,$O$是坐标原点.
\begin{parts}
\part 当点$B$是$W$的右顶点,且四边形$OABC$为菱形时,求此菱形的面积;
\part 当点$B$不是$W$的顶点时,判断四边形$OABC$是否可能为菱形,并说明理由.
\end{parts}
\kongbai
\question
(2013文)直线$y=kx+m~(m\ne 0) $与椭圆$W$:$\dfrac{x^2}{4}+y^2=1$相交于$A$,$C$两点,$O$是坐标原点.
\begin{parts}
\part 当点$B$的坐标为$(0,1)$,且四边形$OABC$为菱形时,求$AC$的长;
\part 当点$B$在$W$上且不是$W$的顶点时,证明四边形$OABC$不可能为菱形.
\end{parts}
\newpage
\question

(2012理)已知曲线$C$:$(5-m)x^2+(m-2)y^2=8~\left(m \in \mathrm{R}\right)$.
\begin{parts}
\part 若曲线$C$是焦点在$x$轴上的椭圆,求$m$的取值范围;
\part 设$m=4$,曲线$C$与$y$轴的交点为$A$,$B$(点$A$位于点$B$的上方),直线$y=kx+4$与曲线$C$交于不同的两个点$M,\ N$,直线$y=1$与直线$BM$交于点$G$,求证:$A$,$G$,$N$三点共线.
\end{parts}
\kongbai
\question
(2012文)已知椭圆$C$:$\dfrac{x^2}{a^2}+\dfrac{y^2}{b^2}=1(a>b>0)$的一个顶点为$A(2,0)$,离心率为$\dfrac{\sqrt{2}}{2}$,直线$y=k(x-1)$与椭圆$C$交于不同的两点$M$,$N$.
\begin{parts}
\part 求椭圆$C$的方程
\part 当$\triangle AMN$的面积为$\dfrac{\sqrt{10}}{3}$时,求$k$的值. 
\end{parts}
\newpage

\question
(2011理)已知椭圆$G$:$\dfrac{x^2}{4}+y^2=1$.过点$(m,0)$作圆$x^2+y^2=1$的切线$I$交椭圆$G$于$A$,$B$两点.
\begin{parts}
\part 求椭圆$G$的焦点坐标和离心率;
\part 将$\abs{AB}$表示为$m$的函数,并求$\abs{AB}$的最大值.
\end{parts}
\kongbai
\question
(2011文)已知椭圆$C$:$\dfrac{x^2}{a^2}+\dfrac{y^2}{b^2}=1$的离心率为$\dfrac{\sqrt{6}}{3}$,右焦点为$(2\sqrt{2},0)$,斜率为1的直线 $l$与椭圆$G$交于$A$,$B$两点,以$AB$为底边作等腰三角形,顶点为$P(-3,2)$.
\begin{parts}
\part 求椭圆$G$的方程;
\part 求$\triangle PAB$的面积.
\end{parts}
\newpage
\question
(2010理)在平面直角坐标系$xOy$中,点$B$与点$A(-1,1)$关于原点$O$对称,$P$是动点,且直线$AP$与$BP$的斜率之积等于$-\dfrac{1}{3}$.
\begin{parts}
\part 求动点$P$的轨迹方程;
\part 设直线$AP$和$BP$分别与直线$x=3$交于点$M$,$N$,问:是否存在点$P$使得$\triangle PAB$与$\triangle PMN$的面积相等\textbf{?}~若存在,求出$P$的坐标;若不存在,说明理由.
\end{parts}
\kongbai
\question
(2010文)已知椭圆$C$的左、右焦点分别是$(-\sqrt{2},0)$,$(\sqrt{2},0)$,离心率是$\dfrac{\sqrt{6}}{3}$,直线$y=t$与椭圆$C$交于不同的两点$M$、$N$,以线段$MN$为直径做圆$P$,圆心为$P$.
\begin{parts}
\part 求椭圆$C$的方程;
\part 若圆$P$与$x$轴相切,求圆心$P$的坐标;
\part 设$Q(x,y)$是圆$P$上的动点,当$t$变化时,求$y$的最大值.
\end{parts}
\newpage
\question
设$F_1,F_2$分别是椭圆$C$:$\dfrac{x^2}{a^2}+\dfrac{y^2}{b^2}=1(a>b>0)$的左右焦点,$M$是$C$上一点且$MF_2$与$x$轴垂直,直线$MF_1$与$C$的另一个交点为$N$.
\begin{parts}
\part 若直线$MN$的斜率为$\dfrac{3}{4}$,求$C$的离心率;
\part 若直线$MN$在$y$轴上的截距为2,且$\abs{MN}=5\abs{F_1N}$,求$a,b$.
\end{parts}
\kongbai
\qs
已知椭圆$C$:$\dfrac{x^2}{a^2}+\dfrac{y^2}{b^2}=1~(a>b>0)$的离心率为$\dfrac{\sqrt{2}}{2}$,点$(2,\sqrt{2})$在$C$上.
\begin{parts}
\part 求$C$的方程;
\part 直线$l$不经过原点$O$,且不平行于坐标轴,$l$与$C$有两个交点$A,B$,线段$AB$中点为$M$,证明:直线$OM$的斜率与直线$l$的斜率乘积为定值.
\end{parts}
\newpage
\qs 
已知椭圆$C$:$9x^2+y^2=m^2~(m>0)$,直线$l$不过原点$O$且不平行于坐标轴,$l$与$C$有两个交点$A,B$,线段$AB$的中点为$M$.
\begin{parts}
\part 证明:直线$OM$的斜率与$l$的斜率的乘积为定值;
\part 若$l$过点$\left(\dfrac{m}{3},m\right)$,延长线段$OM$与$C$交于点$P$,四边形$OAPB$能否为平行四边形?若能,求此时$l$的斜率;若不能,说明理由
\end{parts}
\kongbai
\qs 
在平面直角坐标系$xOy$中,过椭圆$M$:$\dfrac{x^2}{a^2}+\dfrac{y^2}{b^2}=1~(a>b>0)$的右焦点的直线$x+y-\sqrt{3}=0$交$M$于$A,B$两点,$P$为$AB$的中点,且$OP$的斜率为$\dfrac{1}{2}$.
\begin{parts}
\part 求$M$的方程;
\part $C,D$为$M$上的两点,若四边形$ABCD$的对角线$CD\bot AB$,求四边形$ABCD$面积的最大值.
\end{parts}	
\newpage
\qs 已知圆$M:(x+1)^2+y^2=1$,圆$N:(x-1)^2+y^2=9$,动圆$P$与圆$M$外切并与圆$N$内切,圆心$P$的轨迹为曲线$C$.
\begin{parts}
\part 求$C$的方程;
\part $l$是与圆$P$,圆$M$都相切的一条直线,$l$与曲线$C$交于$A,B$两点,当圆$P$的半径最长时,求$|AB|$.
\end{parts}
\kongbai
\qs 
在平面直角坐标系$xOy$中,已知点$A(0,-1)$,$B$点在直$y=-3$上,$M$点满足$\vv{MB}\sslash\vv{OA}$,$\vv{MA}\cdot\vv{AB}=\vv{MB}\cdot\vv{BA}$,$M$点的轨迹为曲线$C$.
\begin{parts}
\part 求$C$的方程;
\part $P$为$C$上动点,$l$为$C$在$P$点处的切线,求$O$点到$l$距离的最小值.
\end{parts}
\newpage
\qs 已知椭圆$C$:$\dfrac{x^2}{a^2}+\dfrac{y^2}{b^2}=1~(a>b>0)$的一个焦点为$ (\sqrt{5},0) $,离心率为$ \dfrac{\sqrt{5}}{3} $.
\begin{parts}
\part 求椭圆$ C $的标准方程;
\part 若动点$ P(x_0,y_0) $为椭圆外一点,且点$ P $到椭圆$C$的两条切线相互垂直,求点$ P $的轨迹方程.
\end{parts}
\kongbai 
\qs 如图所示,在平面直角坐标系$ xOy $中,已知椭圆$ \dfrac{x^2}{3}+y^2=1. $斜率为$ k (k>0)$且不过原点的直线$ l $交椭圆$ C $于$A,~B  $两点,线段$ AB $的中点为$ E $,射线$ OE $交椭圆于点$ G $,交直线$ x=-3 $于点$ D(-3,m) $,若$\abs{OG}^2=\abs{OD}\bm{\cdot}\abs{OE} $,求证:直线$ l $过定点.


\end{questions}
\end{document}