 \begin{Meeting}[Preseason]
      {Programming Chassis Suitable to Test Localization}  
      {August 19-25}
      {20 hours}
      {Nicolas, Zachary}
      {
        \TaskInfo{First Iteration Mecanum Drive Module}
           {aug19: programming chassis first draft}
           {First attempt at lightweight chassis, worked well but could be made more compact}
        \TaskInfo{Second Iteration Mecanum Drive and Integration into Chassis}
           {aug19:programming chassis second draft}
           {Second attempt is more compact and stronger}
      }
 
%%%%%%%%%%%%%%%%%%%%%%%%%%%%%%%%%%%%%%%%%%%%%%%%%%%%%%%%%%%%
% meeting summary, or meeting goal
% high level description of the goal of the meeting, in a paragraph following the command
\MeetingSummary
 
The goal of this week is to develop new technology for the season. We
focus on Mecanum wheels, which we have not used for a long time. Our
immediate goal is design a platform to learn to program encoder
wheels. We also want to gain experience in using bear motors, namely
motors without internal gear boxes.

%%%%%%%%%%%%%%%%%%%%%%%%%%%%%%%%%%%%%%%%%%%%%%%%%%%%%%%%%%%%
% NEW TASK: First Iteration Mecanum Drive Module
%%%%%%%%%%%%%%%%%%%%%%%%%%%%%%%%%%%%%%%%%%%%%%%%%%%%%%%%%%%%

 %1 Strategy; 2 Design; 3 Build; 4 STEM; 5 Software; 6 Team
\Task{2}[3]
 
%%%%%%%%%%%%%%%%%%%%%%%%%%%%%%%%%%%%%%%%%%%%%%%%%%%%%%%%%%%%
\Section{Goals}
\begin{itemize}
  \item Design a mecanum chassis to use for testing localization and autonomous driving.
  \item Use the chassis to validate (or invalidate) new design ideas (bare motor drivetrain).
  \item Low cost.
 \end{itemize}

\Section{Design Process}

First, we plan components to use for the drive train. We do so by
first considering our design goals for this robot in order of
importance, then assessing how we can best accomplish these
goals. Often, one design choice can satisfy many factors
simultaneously.

\begin{DescriptionTable}{Factors}{Solutions}%
    {Design goals for the programming chassis}{table:aug19:goals}
  %
  \TableEntryTextItem{Testing New Designs}
    {
      \item Incorporate odometry wheels (for position tracking)
      \item Prototype use of motors without gearboxes (With external reduction)
      \item Test mecanum wheels 
    } \\ \hline
  %
  \TableEntryTextItem{Low Cost}
    {
      \item Use motors without gearboxes: this will allow us to use
        our classic Neverest 20 motors (which we decommissioned due to
        their fragile gearboxes).
      \item Design with mostly plywood, EuroBoard, and 3d printed parts.
      \item Use Nexus mecanum wheels (already on hand).
      \item Use EMS22Q Bourns encoder for odometry wheels (least
        expensive compatible encoder that satisfies the design
        constraints).  } \\ \hline
  %
  \TableEntryTextItem{Analogous to Typical Competition Robots}
  { 
    \item Make the robot lightweight, so we can add weight to match
      any future robot?s weight for testing
    \item Use Mecanum wheels (we already have test tank chassis, and
      are looking to experiment with mecanum) }
  %
\end{DescriptionTable}


%%%%%%%%%%%%%%%%%%%%%%%%%%%%%%%%%%%%%%%%%%%%%%%%%%%%%%%%%%%%
\Section{CAD and Build}

A complete chassis requires 4 identical wheel modules, which contain a
mecanum wheel and its motor. The CAD model is shown in
\FigureRef{aug19:first cad}. We CNCed the parts as well as 3D printed
the large pulley. The result is shown in \FigureRef{aug19:first
  build}.

\ExplainedPictFigure{src/aug19/first-cad.jpg}[0.4]%
  {CAD model of mecanum wheel module (first iteration)}{aug19:first cad}
  {
  \begin{compactitem}
    \item Nexus mecanum wheel
    \item Single belt reduction from bare motor to wheel
    \item Adjustable tensioner pulley
    \item EuroBoard side plates
    \item Churro standoffs
    \item Extremely compact
    \end{compactitem}
  }
  
\PictFigure{src/aug19/first-build.jpg}[0.4]%
  {Prototype of mecanum wheel module (first iteration)}{aug19:first build}%
  [\Callout{-8, 4}{Unsupported Idle Pulley}{-0.5, -0.5}]         

\begin{DescriptionTable*}{Works}{Need Improvement}%
  {Conclusion after first build}{table:aug19:improvement}
  %
  \TableEntryItemItem{
    \item Wheel runs smoothly
    \item Press fit bearings in wheel work flawlessly
    \item Motor standoffs work well
    \item EuroBoard is a fantastic prototyping material - cuts easily on the CNC 
  } {
    \item Cantilevered idler bearing deforms the EuroBoard under load - %
      needs support from both sides
    \item EuroBoard is not very strong - not suitable for competition %
      robot drivetrain, but works for light 
  }
\end{DescriptionTable*}


%%%%%%%%%%%%%%%%%%%%%%%%%%%%%%%%%%%%%%%%%%%%%%%%%%%%%%%%%%%%
\Section{Conclusion}

The module looks promising, and has already successfully demonstrated
the effectiveness of using EuroBoard as a prototyping material, though
we should avoid using it structurally on a competition robot. The
idler pulley needs to be redesigned with support on either side, and
we can likely make the entire module even more compact by using a
slightly shorter belt!


With these small modifications, the module is ready to be used on the
programming chassis. We now need to design the chassis itself, as well
as mounting points for all the sensors.


%%%%%%%%%%%%%%%%%%%%%%%%%%%%%%%%%%%%%%%%%%%%%%%%%%%%%%%%%%%%
% NEW TASK Second Iteration Mecanum Drive and Integration into Chassis
%%%%%%%%%%%%%%%%%%%%%%%%%%%%%%%%%%%%%%%%%%%%%%%%%%%%%%%%%%%%

\Task[\TaskRef{aug19: programming chassis first draft}]{2}[3]

\Section{Goals}
\begin{itemize}
  \item Suggested improvements from \TaskRef{aug19: programming chassis first draft}.
  \item Design odometry wheel modules.
  \item Design complete chassis.
\end{itemize}

\newpage

\Section{Design}

Using the feedback from \TaskRef{aug19: programming chassis first
  draft}, we redesigned the CAD model for the wheel module, shown in
\FigureRef{aug19:second cad}. We reused an odometry design, shown in
\FigureRef{aug19:odometry cad}. The full chassis consists of 4 wheel
modules and 3 odometry modules. The Chassis CAD is shown in
\FigureRef{aug19:chassi cad}.

We CNCed the parts as well as 3D printed the large pulley. The result
is shown in \FigureRef{aug19:first build}.

\ExplainedPictFigure{src/aug19/second-cad.jpg}[0.4]%
  {CAD model of mecanum wheel module (second iteration)}{aug19:second cad}
  {
    Improvements:
    \begin{compactitem}
      \item Idler Bearing supported from both sides
      \item Shorter plate layout
      \item Slightly smaller pulley on the wheel to avoid scraping on the mat
    \end{compactitem}
  }
  
\ExplainedPictFigure{src/aug19/encoder-cad.jpg}[0.4]%
  {CAD model of odometry wheel}{aug19:odometry cad}
  {
    Features:
    \begin{compactitem}
      \item 38mm omniwheel
      \item 1024 ppr direct mounted encoder
      \item Shielding to protect encoder
      \item Spring-loaded against the mat for improved reliability
      \item Accurate mounting holes
    \end{compactitem}
  }

\ExplainedPictFigure{src/aug19/chassi-cad.jpg}[0.4]%
  {CAD of entire Chassis}{aug19:chassi cad}
  {
    Features:
    \begin{compactitem}
      \item Lightweight simple chassis
      \item Fast Mecanum wheel base
      \item 3 odometry omniwheels
      \item 2 light sensors facing the mat
      \item Plywood base - easy to manufacture
    \end{compactitem}
  }

  \PictFigure{src/aug19/build-pict.jpg}[0.7]%
    {Building of full chassis (second iteration)}{aug19:second build}


\end{Meeting}

 

