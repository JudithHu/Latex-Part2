%      ********   FIG.TEX    ************************
%      (c)  R. A. Adams and R. B. Israel
%           Dept. of Mathematics
%           The University of British Columbia
%           Vancouver, B.C., Canada V6T 1Y4
%
%           September, 1992.      Version 1.22
%      **********************************************
%
% being macros suitable for the inclusion of figures created by
% "MG" into TeX documents.
%
% In vertical mode use:
%
% \cfig{<filename>}{<caption text>}
%
% to include a centred figure whose PostScript description and
% associated TeX labels, both produced by "MG", are in the
% files <filename>.PS and <filename>.LBL respectively.
% The first line of the .LBL file is the MG version number.  The
% second and third lines are the width and height of the figure
% (in points). The macro reads them there to provide
% enough vertical space.
%
% example:
%
% \cfig{myplot}{Figure 3}
%
% Similarly, two side-by-side graphs can be inserted by
%
% \twofigs{<filename1>}{<caption1 text>}{<filename2>}{<caption2 text>}
%
% A graph on the right side of some text can be produced by
% \rfig{<text on the left>}{<filename>}{<caption text>}
%
%
% Labels on figures will print in the default TeX families
% (typically 10 point Computer Modern  or whatever families
% are installed in your version of TeX.)
% If you want to change this use, for example,
%
%      \def\setlabelsize{\ninepoint}
%
% within your TeX document. To reinstate the defaults use
%
%      \def\setlabelsize{}
%
% macros \ds, \ss, \ts defined below can also be used to control
% size of labels
%
% \ninepoint and \eightpoint are defined hereunder for CM fonts.
% If you want any other size or family  you will have to define
% your own.
%
\font\ninerm=cmr9   \font\sixrm=cmr6   \font\eightrm=cmr8
\font\ninei=cmmi9   \font\sixi=cmmi6   \font\eighti=cmmi8
\font\ninesy=cmsy9  \font\sixsy=cmsy6  \font\eightsy=cmsy8
\font\ninebf=cmbx9  \font\sixbf=cmbx6  \font\eightbf=cmbx8
\font\nineit=cmti9                     \font\eightit=cmti8
\font\ninesl=cmsl9                     \font\eightsl=cmsl8
\font\ninett=cmtt9                     \font\eighttt=cmtt8
%
\def\ninepoint{\def\rm{\fam0\ninerm}% switch to 9-point type
 \textfont0=\ninerm \scriptfont0=\sixrm \scriptscriptfont0=\fiverm
 \textfont1=\ninei \scriptfont1=\sixi \scriptscriptfont1=\fivei
 \textfont2=\ninesy \scriptfont2=\sixsy \scriptscriptfont2=\fivesy
 \textfont3=\tenex \scriptfont3=\tenex \scriptscriptfont3=\tenex
 \textfont\itfam=\nineit \def\it{\fam\itfam\nineit}%
 \textfont\slfam=\ninesl \def\sl{\fam\slfam\ninesl}%
 \textfont\ttfam=\ninett \def\tt{\fam\ttfam\ninett}%
 \textfont\bffam=\ninebf \scriptfont\bffam=\sixbf
 \scriptscriptfont\bffam=\fivebf \def\bf{\fam\bffam\ninebf}%
 \normalbaselineskip=11pt
 \setbox\strutbox=\hbox{\vrule height8pt depth3pt width0pt}%
 \let\sc=\sevenrm \let\big=\ninebig \normalbaselines\rm}
%
\def\eightpoint{\def\rm{\fam0\eightrm}% switch to 8-point type
 \textfont0=\eightrm \scriptfont0=\sixrm \scriptscriptfont0=\fiverm
 \textfont1=\eighti \scriptfont1=\sixi \scriptscriptfont1=\fivei
 \textfont2=\eightsy \scriptfont2=\sixsy \scriptscriptfont2=\fivesy
 \textfont3=\tenex \scriptfont3=\tenex \scriptscriptfont3=\tenex
 \textfont\itfam=\eightit \def\it{\fam\itfam\eightit}%
 \textfont\slfam=\eightsl \def\sl{\fam\slfam\eightsl}%
 \textfont\ttfam=\eighttt \def\tt{\fam\ttfam\eighttt}%
 \textfont\bffam=\eightbf \scriptfont\bffam=\sixbf
 \scriptscriptfont\bffam=\fivebf \def\bf{\fam\bffam\eightbf}%
 \normalbaselineskip=9pt
 \setbox\strutbox=\hbox{\vrule height7pt depth2pt width0pt}%
 \let\sc=\sixrm \let\big=\eightbig \normalbaselines\rm}
%
% The next five \TeX\ macros can be used in labels to simplify
% some size choices.  (If they interfere with other macros in
% your system, comment them out.)
%
  \def\ds{\displaystyle}             % size of math in displays
  \def\ss{\scriptstyle}              % size of subscripts in displays
  \def\ts{\textstyle}                % size of math in text
  \def\frac#1#2{\ds {#1\over#2}}     % fractions in displaystyle
  \def\sfrac#1#2{{\ts{#1\over#2}}}   % fractions in textstyle
%
% use the current tex font family for labels
\def\setlabelsize{}
%
% by default, boxes will not be drawn around figures
\newif\ifboxfigures
\boxfiguresfalse
% you can declare \boxfigurestrue in your TeX source if you want
% the boxes drawn
\newdimen\boxrulewidth \newdimen\boxborderwidth
\boxrulewidth=1pt \boxborderwidth=2pt
% boxes drawn around figures will have rules 1 point wide
% surrounding a 2 point wide border by default.
% You can change these measurements by redefining \boxrulewidth
% and \boxborderwidth in your TeX source.
%
% macro for drawing boxes
%
\def\boxit#1{\vtop{\hrule height\boxrulewidth%
\hbox{\vrule width\boxrulewidth\kern\boxborderwidth%
\vbox{\kern\boxborderwidth#1\kern\boxborderwidth}\kern\boxborderwidth%
\vrule width\boxrulewidth}%
\hrule height\boxrulewidth}}
%
% Now for the definitions and main macro for figure inclusion.
%
\newcount\mgversion
\newcount\pswidth
\newcount\psheight
\newcount\justx \newcount\justy
\global\justx=0 \global\justy=0
\newcount\vpos \newtoks\LABEL
\newread\labelfile \newcount\xcoord \newcount\ycoord
\newif\ifdoit
\newbox\labox
%
\def\figfiginsert#1{\par%  #1=filename
\openin\labelfile=pics/#1.lbl
\global\read\labelfile to\mgversion\message{#1}
\global\read\labelfile to\pswidth
\global\read\labelfile to\psheight
\ifboxfigures\boxit{\fi\vbox to\psheight pt{\vfill
%%%
%%% NOTE: The next uncommented line transfers the PostScript file
%%%       for the desired figure to the output file.  The version
%%%       here is for the Arbortext DVILASER/PS driver.
%%%
%\special{ps: plotfile #1.ps}% version for DVIPS
\special{PSfile =pics/#1.ps}% version for DVIPS
%\epsfbox{#1.ps}
%%%
%%% If you are using Personal TeX's PTILaser/PS driver, or using
%%% this file to incorporate MG graphs into TeX documents on a
%%% NEXT with the NEXT DVIPS driver, comment out the \special line
%%% above (by inserting a percent % at the left margin on that
%%% line, and uncomment (by removing the percent % at the left
%%% margin) the appropriate \special line below.
%%%
%\special{ps: #1.ps}%         version for PTILaser/PS
%\special{psfile=#1.ps}%      version for NeXTTeX
%%%
%%% If you will be using this file to incorporate MG graphs
%%% into TeX documents on a Macintosh, using Blue Sky Research's
%%% Textures driver, instead of either of the \special commands
%%% above, use (uncomment) the following PAIR of \specials.
%%%
%\special{postscript grestore newpath gsave}% two line version for
%\special{postscriptfile #1.ps}%              the TEXTURES driver
%%%
%%% For other drivers you're on your own to find the right form
%%% for the \special.  Read Section 9.1 in the MG Users Manual, and
%%% the documentation for your driver.
%%%
\vskip-\psheight pt\setlabelsize%
\hbox to\pswidth pt{\hss}%
\parindent=0pt\offinterlineskip
\vpos=0%
\loop\global\read\labelfile to\xcoord
\ifnum \xcoord < -999 \doitfalse\else\doittrue\fi
\ifdoit \global\read\labelfile to\ycoord
\global\read\labelfile to\justx
\global\read\labelfile to\justy
\global\read\labelfile
to\LABEL\global\setbox\labox=\hbox{\LABEL\hskip-0.3em}%
\advance\vpos by-\ycoord
\vskip-\vpos pt \vpos=\ycoord
\hbox to\pswidth pt{\hskip\xcoord pt%
\hbox to 0pt{\ifnum\justx>0\hss\fi%
\vbox to0pt{%
\ifnum\justy<2\vss\fi%
\copy\labox\kern0pt%
\ifnum\justy>0\vss\fi}%
\ifnum\justx<2\hss\fi}%
\hss}%
\repeat%
\advance\vpos by-\psheight%
\vskip-\vpos pt%
}\ifboxfigures}\fi\closein\labelfile}
%
% define a font for figure captions in the following macros
% one of the next two lines should be commented out
%\font\figfont=psmhlvb at10pt  % e.g. Arbortext's mapped helvetica bold
\def\figfont{\bf}              % boldface from your default family.
%
% You may wish to modify the following high-level figure inclusion macros
% or create new ones of your own.
%
% Macro for centred figure, captioned below.
%
\def\cfig#1#2{\par\smallskip
\openin\labelfile=pics/#1.lbl
\ifeof\labelfile\immediate\write16{Can't find #1.lbl; I quit!}\end\fi
\closein\labelfile
\vbox{\centerline{\figfiginsert{#1}}\smallskip
   \centerline{\figfont#2}}\smallskip}
%
% Macro for text on left side of page and figure on right side.
% Figure is captioned below.
%
\newdimen\textsize
\long\def\rfig#1#2#3{\par\smallskip
\openin\labelfile=pics/#2.lbl
\ifeof\labelfile\immediate\write16{Can't find #2.lbl; I quit!}\end\fi
\global\read\labelfile to\mgversion
\global\read\labelfile to\pswidth
\closein\labelfile
\textsize=\hsize
\advance\textsize by-\pswidth pt
\advance\textsize by-1pc
\hbox to \hsize{\vtop{\hsize=\textsize{#1}\vfil}
\hfill\vtop{\hsize=\pswidth pt\vskip0pt
\figfiginsert{#2}\smallskip
\hbox to \hsize{\hfil\figfont#3\hfil}\smallskip}}}
%
% Macro for two side-by-side figures, captioned below.
%
\newcount\rightfigwidth \newcount\leftfigwidth
\def\twofigs#1#2#3#4{\par\smallskip
\openin\labelfile=pics/#1.lbl
\ifeof\labelfile\immediate\write16{Can't find #1.lbl; I quit!}\end\fi
\global\read\labelfile to\mgversion
\global\read\labelfile to\leftfigwidth
\closein\labelfile
\openin\labelfile=pics/#3.lbl
\ifeof\labelfile\immediate\write16{Can't find #3.lbl; I quit!}\end\fi
\global\read\labelfile to\mgversion
\global\read\labelfile to\rightfigwidth
\closein\labelfile
\vbox{\hbox to \hsize{\hfill\figfiginsert{#1}\hfill\hfill\figfiginsert{#3}\hfill}
\smallskip
\hbox to \hsize{\hfill\hbox to\leftfigwidth pt{\hfil\figfont#2\hfil}
\hfill\hfill\hbox to\rightfigwidth pt{\hfil\figfont#4\hfil}\hfill}}\smallskip}
%
%
