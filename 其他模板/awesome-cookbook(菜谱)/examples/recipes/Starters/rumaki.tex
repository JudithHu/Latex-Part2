% This will start the recipe and print the header, under the hood it is just a new section.
% Options:
%		- style: will print the header in a different style (style1, style2, style3, style4)
%		- startleft: will start the recipe on a left page (eg. to get the recipe and fullpage
%						picture near echother. To enable set this to true (no captial)
%
% Since this is just a section everything else is optional and can be in any order you like.
% You may also include LaTeX code yourself, eg. if you want a graph, table, picture, .. go nuts,
% none of the given "sections" (environments or commands) are mandatory.
% Note: some of the values can be set to a default in your main .tex file above the \begin{docuemnt}
% environment. See cookbook.tex for an example.
\recipe[style=style2]{Rumaki}

% The icons given general information about the recipe, the icons will always be in the same
% order. If you do not provide a value for the icon (eg. time, servings, energy, cost, source or
% urlsource the icon will not be displayed.
% Note: the text/currency values will be set by default, you do not have to define them in every recipe
% but could overwrite the default value if so desired.
% Options:
%		- servings: number of servings for this recipe
%		- servingstext: the text under the servings
%		- time: the total cooktime for the recipe
%		- timeunit: the text under the time
%		- cost: the cost for the recipe
%		- currency: the currency symbole (eg. $, £, €, ...)
%		- source: the source of the recipe
%		- urlsource: the source of the recipe (displayed as a link, provide the full url)
%		- sourcetext: the text under the source
\info[servings=24,
		time = 40, 
		energy = 40 (each),
		cost =  25,
		urlsource = http://allrecipes.com/recipe/235095/easy-rumaki-with-pineapple/]{}

% The ingredients for the recipe, there are 3 different environments.
% \begin{ingredients}	: will put the ingredients in a column, right, next to the text below.
% \begin{ingredientsh}	: will put the ingredients horizontal, automatically in 3 columns, above the text below.
% \begin{ingredientsc}	: will put the ingredients horizontal, in the user defined columns
%							(optionally with a caption), above the text below.
%						 To define a column use the \begin{column}[title] environment (see other recipes)
\begin{ingredients}
	\ingredient{24}{}{pinapple (cubes)}
	\ingredient{24}{}{chestnut (slices)}
	\ingredient{8}{}{bacon slices (thick)}
	\ingredient{120}{ml}{sesame-ginger salad dressing}
	\ingredient{150}{ml}{white wine (dry)}
	\ingredient{100}{g}{green onions}
\end{ingredients}

% This environment declares the preparation steps and has no options.
\begin{preparation}
	\step Preheat oven to 375 degrees F (190 degrees C). Line the bottom section of a broiler pan with aluminum foil, top with the broiler rack, and spray rack with cooking spray.
	
	\step Place a water chestnut slice atop each pineapple cube; wrap each with 1 bacon slice, securing with a toothpick. Arrange wrapped pineapple on the prepared broiler rack. 
	
	\step Bake in the preheated oven for 7 minutes; turn and continue baking until bacon is almost crisp, about 8 more minutes. Brush rumaki with sesame-ginger dressing and continue baking until bacon is crisp, about 5 more minutes. Garnish rumaki with green onion. 
\end{preparation}

% This environment will let you define alternatives to the recipe (eg. "use pasata instead of fresh tomatoes). 
% You can refere to the preparation and alternative section steps by using \refstep{<number} and 
% \refaltstep{<number>}.
%\begin{alternatives}
%	\alternative Alternative 1
%	\alternative Alternative 2
%\end{alternatives}

% This environment declares the notes for the recipe and has no options. You can refere to the preparation and
% alternative section steps by using \refstep{<number} and \refaltstep{<number>}.
\begin{notes}
	\note{This recipe uses \texttt{style2} header and the \texttt{ingredient} environment. It also shows the cost.}
	\note{There is also a picture, using \texttt{recipefigure} with the \texttt{wide} style.}
\end{notes}

% Include pictures there are different options available. You only have to define the image names (including 
% the extension). LaTeX will search for the images in the user defined directory (see cookbook.tex).
% Options: 
%		- style : the type of image
%					+ background: put the image in the center of the page with 50% opacity.
%						(be carefull, this command should be place before the text otherwise the image will
%						be lay over the text instead of vice versa)
%					+ fullpage: will put the image in the center of a new page without footer (page number),
%						the image is not enlarged.
%					+ wide: the image will be scaled to a % of the \textheight and igonre margins. You can define
%						the desire height with the height = option.
%					+ crop: the image will be scale to the set width and cropped to the set height (first enlarged,
%						then cropped). You can set the width and height with: width = and height = options
%						respectively.
%					+ fullpage4 : puts 4 images on a new page without footer. You can specify the 3 other pictures
%						with the fig2 =, fig3 =, fig = options.
\recipefigure[style = wide, height = 0.5]{rumaki.jpg}