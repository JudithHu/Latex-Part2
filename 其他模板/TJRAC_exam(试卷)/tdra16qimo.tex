\documentclass[twocolumn,UTF8]{ctexart}
\usepackage[mtphrb]{mtpro2}%,amssymb
\ctexset{section = {name = {,、\hspace*{-5mm}},number = \chinese{section},format = {\biaosong\zihao{-4}\raggedright}}}
\usepackage{titlesec}%titlesec宏包调整section与正文间距
\titlespacing*{\section} {0pt}{9pt}{4pt}
%===============================================================================%
\setCJKmainfont{Adobe Song Std L}%中文默认字体:adobe 宋体
%\renewcommand{\songti}{\CJKfontspec{Adobe Song Std L}}
\renewcommand{\heiti}{\CJKfontspec{Adobe Heiti Std R}}%adobe 黑体
%\renewcommand{\heiti}{\CJKfontspec{Hiragino Sans GB}}%冬青黑体简体中文
\renewcommand{\fangsong}{\CJKfontspec{Adobe Fangsong Std R}}%adobe 仿宋
\renewcommand{\kaishu}{\CJKfontspec{Adobe Kaiti Std R}}%adobe 楷体
\setCJKfamilyfont{huawenxingkai}{华文行楷} \newcommand*{\xingkai}{\CJKfamily{huawenxingkai}}
\newcommand*{\zhongsong}{\CJKfamily{STZhongsong}}
%\newcommand{\zhongsong}{\CJKfontspec{方正中宋简体}}%方正中宋
\newcommand{\biaosong}{\CJKfontspec{方正小标宋简体}}%方正粗宋简体
%===============================================================================%
\usepackage{zref-user,zref-lastpage}%使用zref宏包,引用数字标签值和LastPage标签,感谢qingkuan大神指导
\usepackage{bigstrut}
\usepackage{enumerate}
\usepackage{amsmath,bm}
\everymath{\displaystyle}
\newcommand\dif{\mathop{}\!\mathrm{d}}
\DeclareMathOperator{\grad}{grad}
\usepackage{scalerel} %\scaleobj{1.5}{} 缩放公式大小
\usepackage{CJKnumb}%中文小写数字
\usepackage[paperwidth=42cm,paperheight=29.7cm,top=4.4cm,bottom=2.5cm,left=4.5cm,right=1cm]{geometry}
\usepackage{fancyhdr}\pagestyle{fancy}
\renewcommand{\headrulewidth}{0pt}
\renewcommand{\footrulewidth}{0pt}
%%%%%%%%%%%%%%%%%%%%%%%%%%%%%%%%%%%%%%%%%%%%%%%%%%%%%%%%%%%%%%%%%%%%%%%
\usepackage{tikz}
\usepackage{fancybox}
\fancyput(1.50cm,-25.3cm){\tikz \draw[solid,line width=2pt](0,0) rectangle (1cm+\textwidth,1.2cm+\textheight);}    
 %solid,dashed%pdfmanual.pdf---p167
%%%%%%%%%%%%%%%%%%%%%%%%%%%%%%%%%%%%%%%%%%%%%%%%%%%%%%%%%%%%%%%%%%%%%%%

\usepackage{array,multirow}
%%%%%%%%%%%%%%%%%%%%%%%%%%%%%%%%%%%选择题%%%%%%%%%%%%%%%%%%%%%%%%%%%%%%%%%%%%%%%%
%选项单行
\newcommand{\xo}[4]{\makebox[100pt][l]{(A) #1} \hfill
                    \makebox[100pt][l]{(B) #2} \hfill
                    \makebox[100pt][l]{(C) #3} \hfill
                    \makebox[100pt][l]{(D) #4}}
%选项分两行。
\newcommand{\xab}[2]{\makebox[100pt][l]{(A) #1} \hfill
                     \makebox[220pt][l]{(B) #2}}
\newcommand{\xcd}[2]{\makebox[100pt][l]{(C) #1} \hfill
                     \makebox[220pt][l]{(D) #2}}
%选项分四行.
\newcommand{\xa}[1]{(A) #1}
\newcommand{\xb}[1]{(B) #1}
\newcommand{\xc}[1]{(C) #1}
\newcommand{\xd}[1]{(D) #1}
%%%%%%%%%%%%%%%%%%%%%%%%%%%%%%%%%%%%%%%%%%%%%%%%%%%%%%%%%%%%%%%%%%%%%%%%
\textwidth=34.6cm        %文本的宽度

\begin{document}\zihao{-4}
\setlength{\columnseprule}{0pt}
\renewcommand\arraystretch{1.5}
\fancyhead[LO,LE]{\zihao{4}\vspace*{-18mm}\hspace{-4mm}{\heiti 系别}\underline{\hspace{1.5cm}}{\heiti 专业}\underline{\hspace{3.5cm}}\hspace{1cm}\underline{\hspace{1.2cm}}{\heiti 班}}
\fancyhead[CO,CE]{\vspace*{-18mm}{\setlength{\unitlength}{4mm}\begin{picture}(15,0)\put(-3,2.5){\zihao{-2}天津大学仁爱学院专用纸}\end{picture}}\\\zihao{4}{\heiti 年级}\underline{\hspace{2cm}}{\heiti 学号}\underline{\hspace{4cm}}{\heiti 姓名}\underline{\hspace{32mm}}}
\fancyhead[RO,RE]{\vspace*{-18mm}\zihao{4}({\bf A}) {\heiti 卷}\quad 第\;\thepage\;页\quad\; 共\;\,\zpageref{LastPage}\; 页\hspace*{1.25cm}}
\cfoot{雷电法王杨永信}  
%%%%%%%%%%%%%%%%%%%%%%%%%%%%%%%%%%%%%%%%%%%%%%%%%%%%%%%%%%%%%%%%%%%%%%%%%%
\begin{center}\vspace*{-4mm}
{\zihao{4}\heiti 2016$\sim$2017学年第二学期期末考试试卷}\\[5mm]
{\zihao{4}\heiti《高等数学1B1》\;(共\zpageref{LastPage}页\quad A{\heiti 卷})}\\[2mm]
%输出"绝密"字样      
%{\heiti 绝密$\bigstar$启用前\\[-13.5mm]%缩短"绝密"字样与总计分表之间的距离
({\zihao{-4} 考试时间: 2017年6月28日})\\
\begin{tabular}{|c|*{7}{m{2.9em}<{\centering}|}c|}\hline
题号       &一&二&三& 四&五& 六&成绩&\makebox[5em]{核分人}\\\hline
得分       &  &  &  &  &  &   &    &\multirow{2}*{}      \\\cline{1-8}
评分人签字 &  &  &  &  &  &   &    &                      \\\hline
\end{tabular}\\[5mm]
\end{center}
%%%%%%%%%%%%%%%%%%%%%%%%%%%%%%%%%一、填空题%%%%%%%%%%%%%%%%%%%%%%%%%%%%%%%%%%%%%%%
\section{填空题\songti{(共9分, 每小题3分)请将正确答案填在题中的横线上.}}
 \begin{enumerate}
\item $z=\ln(x+y^2)$, 则$\dif z=$ \underline{\hspace{2.5cm}}.\\[-0.3cm]
\item 已知曲线 $L:\,y=x^2$ 从 $(1,1)$ 到 $(-3,9)$, 计算第二类曲线积分 $\int_Lx\dif x+y\dif y=$ \underline{\hspace{1.3cm}}.\\[-0.3cm]
\item 函数 $u=x^2y+yz^2$ 在点 $M_0(2,-1,1)$ 处的梯度 $\grad u\Big|_{M_0}=$ \underline{\hspace{3.5cm}}.\\[-0.6cm]
\end{enumerate}
%%%%%%%%%%%%%%%%%%%%%%%%%%%%%%%%%二、单项选择题%%%%%%%%%%%%%%%%%%%%%%%%%%%%%%%%%%%%
\section{单项选择题\songti{(共9分, 每小题3分)请将正确答案的代号填在题中的括号内.}}
\begin{enumerate}
\item 二元函数 $z=f(x\cos y)$ 具有一阶连续偏导数, 其对 $x$ 的偏导数 $\frac{\partial z}{\partial x}=$ $(\hspace{1.0cm})$\\[3mm]
\xo{$\sin y\, f$;}{$\cos y\,f$;}{$\sin y\,f'$;}{$\cos y\,f'$.}\\[-2mm]
\item 下列级数收敛是 $(\hspace{1.0cm})$\\[3mm]
\xo{$\sum_{n=1}^\infty\frac{1}{n+1}$;}{$\sum_{n=1}^\infty\left(\frac{3}{2}\right)^n$;}{$\sum_{n=1}^\infty(-1)^{n-1}\frac{1}{2^{n-1}}$;}{$\sum_{n=1}^\infty\frac{1}{\sqrt[n]{n}}$.}\\[-2mm]
\item 设 $I_1=\iint_D\sqrt{x^2+y^2}\dif\sigma$, $I_2=\iint_D(x^2+y^2)\dif\sigma$, 其中 $D=\big\{(x,y)\big|x^2+y^2\leqslant1\big\}$, 则 $(\hspace{1.0cm})$\\[2mm]
\xab{$I_2<I_1<\pi$;}{$I_1<\pi<I_2$;}\\[3mm]
\xcd{$I_1<I_2<\pi$;}{$I_2<I_1<\pi$.}
\end{enumerate}
%%%%%%%%%%%%%%%%%%%%%%%%%%%%%%%%三、解下列各题%%%%%%%%%%%%%%%%%%%%%%%%%%%%%%%%%%%%%
\newpage\section{解下列各题\songti{(本题满分24分, 每小题6分)}}
\begin{enumerate}
\item 设 $z=f(x,y)$ 由方程 $x^3y^3+3z-\ln z=0$ 所确定, 求 $\frac{\partial z}{\partial x}$,$\frac{\partial z}{\partial y}$.\\[3.5cm]
\item 求曲面 $x^3-3y^2-3z=1$ 在 $P_0(1,1,-1)$ 处的切平面方程与法线方程.\\[3.5cm]
\item 已知平面 $\pi$ 过点 $M_1(2,0,0)$,\,$M_2(2,1,2)$,\,$M_3(1,3,2)$, 求平面 $\pi$ 的方程.\\[3.5cm]
\item 求二元函数 $f(x,y)=x^3+y^3-3xy+2$ 的极值.\\[3.5cm]
\end{enumerate}
%%%%%%%%%%%%%%%%%%%%%%%%%%%%%%%%四、解下列各题%%%%%%%%%%%%%%%%%%%%%%%%%%%%%%%%%%%%%
\newpage\section{解下列各题\songti{(本题满分35分, 每小题7分)}}
\begin{enumerate}
\item 计算二重积分 $\iint_D3x\dif x\dif y$, 其中 $D$ 是由 $y=\frac{1}{x}$,\,$y=x$,\,$x=3$  所围成的有界闭区域\\[5.5cm]
\item 计算第一类曲线积分 $\int_Lx\dif s$, 其中曲线 $L:\,y=x^2\;\big(0\leqslant x\leqslant\sqrt{2}\big)$.\\[5.5cm]
\item 计算第一类曲面积分 $\iint_\Sigma(x^2+y^2)\dif S$, 其中 $\Sigma$ 为锥面 $z=\sqrt{x^2+y^2}$ 被平面 $z=0$ \\
和 $z=2$ 截得的有限部分.
\end{enumerate}
\newpage%\vspace*{4cm}
\begin{enumerate}\setcounter{enumi}{3}
\item 计算三重积分 $\iiint_\Omega2z\dif v$, 其中 $\Omega$ 是抛物面 $z=x^2+y^2$ 与球面 $z=\sqrt{2-x^2-y^2}$ 所围成的空间.\\[9cm]
\item 计算第二类曲面积分 $\oiint_\Sigma xz\dif y\dif z+y\dif z\dif x+z\dif x\dif y$, 其中 $\Sigma$ 为球面 $x^2+y^2+z^2=4$ 的外侧.
\end{enumerate}
%%%%%%%%%%%%%%%%%%%%%%%%%%%%%%五、解下列各题%%%%%%%%%%%%%%%%%%%%%%%%%%%%%%%%%%%%%%%
\newpage\section{解下列各题\songti{(本题满分18分, 每小题6分)}}
\begin{enumerate}
\item 判断级数 $\sum_{n=1}^{\infty}(-1)^{n-1}\ln\Big(1+\scaleobj{0.8}{\frac{3}{n}}\Big)$ 的敛散性, 若收敛指出是绝对收敛还是条件收敛.\\[9cm]
\item 求幂级数 $\sum_{n=1}^{\infty}\frac{1}{n}\left(\frac{x}{4}\right)^n$ 的收敛域及和函数.
\newpage
\item 将 $f(x)=\frac{1}{4-x}$ 展开成 $x-1$ 的幂级数,  并写出其收敛域.\\[8cm]
\end{enumerate}
%%%%%%%%%%%%%%%%%%%%%%%%%%%%%%%%%六、证明题%%%%%%%%%%%%%%%%%%%%%%%%%%%%%%%%%%%%%%%
%\newpage
\section{计算题\songti{(本题5分)}}
计算第二类曲线积分 $\oint_L\frac{x\dif y-y\dif x}{x^2+y^2}$, 其中 $L$ 为不过原点的任意一条闭合曲线.












\end{document}
