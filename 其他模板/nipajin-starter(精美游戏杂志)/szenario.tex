%========================================================================
% NIP'AJIN Autorenpaket v1.1, (C) Markus Leupold-Löwenthal
%========================================================================
% Dieses Werk untersteht folgender Lizenz:
% Namensnennung–Weitergabe unter gleichen Bedingungen 3.0 Österreich
% (CC BY-SA 3.0) http://creativecommons.org/licenses/by-sa/3.0/at/
%========================================================================

\begin{center}
\noindent\emph{Dieses Szenario benutzt das Rollenspielsystem \logonipajin~aus dem Anhang (\seite{Regeln}).} 
\end{center}

\banner{Spielleiterinformationen}{Spielleiterinformationen}

\begin{multicols}{2}

\noindent
Der hier vorgeschlagene Aufbau eines Szenarios hat sich bei mir bewährt, ist aber natürlich nur eine Empfehlung~\ldots

\mysection{Überblick}

Hier kommt eine kurze Einführung hin, worum es in dem Szenario geht. Es sollen keine Details verraten, sondern dem SL geholfen werden, den Inhalt als (nicht) leitenswert zu beurteilen.

\lipsum[1]

\mysection{Setting}\zlabel{Setting}

Hier erklärt das Szenario das Umfeld, in dem es sich bewegt. Allgemeine Geographie, Geschichte usw. sind hier gut aufgehoben. Alles, was ein SL über den Prolog hinaus wissen muss, kann hier hinein. \zitat{Charakterwissen} sollte jedoch großteils schon im Prolog (\seite{Prolog}) den Spielern nahegebracht worden sein. 

\lipsum[2-4]

\mysection{Setting-Regeln}\zlabel{SettingRegeln}

Wenn es besondere Regeln für dein Setting gibt (\zB~Wahnsinn-Regeln für ein Horrorabenteuer), sind die besser hier als im settingunabhängigen Regelanhang aufgehoben.

\lipsum[18] 

\mysection{Verlauf des Szenarios}

Hier passt der rote Faden des Szenarios gut hin, oder bei einem Sandbox-Abenteuer eben die Information, dass es den Faden nicht gibt.

\lipsum[5-9]
 
\mysection{Einstieg ins Szenario}

Wie werden die Charaktere mit dem Szenario zu Beginn konfrontiert und was sind die ersten, wahrscheinlichen Handlungsmöglichkeiten? 

\lipsum[10-11]

\mysection{Spieltechnisches}

Einzelne Ortbeschreibungen, NSCs, Monster und Spielwerte passen hier gut hin, wenn sie über allgemeine Informationen vom Abschnitt \zitat{Setting} (\seite{Setting}) hinaus gehen.

\lipsum[12-15]

\mysection{Ende gut, alles gut?}

Mögliche Enden des Szenarios oder ein Epilog, soweit vorhanden, gehören hier hin. Zudem passt hier noch gut ein Ausblick hin, wie ein SL das Szenario bei Gefallen selbst weiter gestalten könnte.

\lipsum[16-17]

\end{multicols}

\newpage