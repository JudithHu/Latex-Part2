
\setcounter{chapter}{-1}
\chapter{引言}

\section{经典力学的发展}

17世纪由Galileo和Newton奠基。1687年发表的《自然哲学之数学原理》标志着经典力学体系的建立,即Newton力学。
\begin{itemize}
    \item {\heiti 核心概念}:质点、力、加速度;
    \item {\heiti 基本原理}:Newton三大定律;
    \item {\heiti 数学方法}:三维实空间几何、矢量代数。
\end{itemize}

18至19世纪,d'Alembert、Euler、Lagrange、Hamilton、Jacobi、Gauss、Poisson等建立力学的后Newton形式——分析力学。
\begin{itemize}
    \item {\heiti 核心概念}:能量、作用量;
    \item {\heiti 基本原理}:Hamilton原理(最小作用量原理);
    \item {\heiti 数学方法}:抽象空间、数学分析;
    \item {\heiti Lagrange形式}:广义坐标的引入使力学具备描述其它非力学体系的潜力。
    \item {\heiti Hamilton形式或正则形式}:力学体系内在对称性的揭示开始支配物理学思维,从而完成从力学到相对论和量子力学的升华。
\end{itemize}

分析力学给出了动量、角动量和能量的普适定义,使力学成为整个物理学的原型(量子力学、统计物理和量子场论的理论基石),为物理理论的统一奠定基础。

分析力学当代的发展结果:非线性物理与混沌、随机性、宿命论因果律破产。

\section{经典力学的适用范围}

{\heiti 宏观物体}在{\heiti 弱引力场}中的{\heiti 低速}运动。

\begin{itemize}
    \item {\heiti 高速运动}:狭义相对论
    \item {\heiti 强引力场}:广义相对论
    \item {\heiti 微观体系}:量子力学
\end{itemize}

% \section{课程的价值取向}

% 经典力学:通向现代物理宏伟圣殿的阶梯,非线性前沿科学成长的摇篮。

% 总结先贤理论遗产精华,领悟物理学的基本原理、通用语言和理论方法。%——醉翁之意不在酒:着眼于整个物理学。

% \begin{quote}
    % \kai ……科学知识使人们能制造许多产品,做许多事业。……科学的另一个价值是提供智能与思辩的享受。……如果我们社会进步的最终目标正是为了让各种人能享受它想做的事,那么科学家们思辩求知的享受就和其它事具有同等的重要性了。
%   \begin{flushright}
        % —— R.P. Feynman,《科学的价值》
    % \end{flushright}
% \end{quote}