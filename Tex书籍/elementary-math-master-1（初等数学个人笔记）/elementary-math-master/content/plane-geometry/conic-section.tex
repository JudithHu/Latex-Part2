
\section{圆锥曲线}
\label{sec:conic-section}

\subsection{椭圆的方程}
\label{sec:ellipse-equation}

椭圆是生活中常见的图形,它是由圆拉伸而来,中心在原点的单位圆的参数方程是
\[
  \begin{cases}
    x = \cos{\theta} \\
    y = \sin{\theta}
  \end{cases}
\]
于是在伸缩变换$L(a,b)(a>0,b>0)$下,这方程就成为
\[
  \begin{cases}
    x = a \cos{\theta} \\
    y = b \sin{\theta}
  \end{cases}
\]
这就是椭圆的参数方程,消去参数即得椭圆的普通方程
\[ \frac{x^2}{a^2} + \frac{y^2}{b^2} = 1 \]
从方程来看,显然椭圆具有轴对称和中心对称性,而原点是它的对称中心,称为\emph{椭圆的中心}.两个坐标轴是这椭圆的对称轴。

\begin{example}
  我们来证明,平面上,到两个定点的距离之和为恒定值(大于两定点之间的距离)的动点轨迹是一个椭圆。

  设两个定点为$F_1(-c,0)$和$F_2(c,0)$,动点$P(x,y)$满足$|PF_1|+|PF_2|=2a(a>c)$,那么有
  \[ \sqrt{(x+c)^2+y^2} + \sqrt{(x-c)^2+y^2} = 2a \]
  令
  \[ \sqrt{(x+c)^2+y^2} = a + t, \  \sqrt{(x-c)^2+y^2} = a - t \]
  这里$t$是与动点$P$有关的量,上两式平方之后相减得
  \[ t = \frac{c}{a}x \]
  将它代入以上两式中任意一式得
  \[ \frac{x^2}{a^2} + \frac{y^2}{a^2-c^2} = 1 \]
  因为$a>c>0$,所以可令$b^2=a^2-c^2(b>0)$(接下来会看到,$b$是有几何意义的),就得
  \[ \frac{x^2}{a^2} + \frac{y^2}{b^2} = 1 \]
  这就是一个椭圆.
\end{example}

  显然,任意一个椭圆,也有唯一确定的两个定点$F_1$和$F_2$和距离和$2a$,使得椭圆上任一点到这两个定点的距离和恒为$2a$,称这两个定点为椭圆的\emph{焦点},两个焦点之间的距离$2c$称为\emph{焦距}.

  在上面已经得出了动点$P$到两个焦点的距离(称为\emph{焦半径})的表达式
  \[
  \begin{cases}
    |PF_1| = a + \frac{c}{a} x \\
    |PF_2| = a - \frac{c}{a} x \\
  \end{cases}
  \]

  显然,椭圆的形状由伸缩变换$L(a,b)$确定,实际上由比例$\frac{b}{a}$所唯一确定,定义椭圆的\emph{离心率}为
  \[ e = \frac{c}{a} \]
  它与伸缩比例的关系是
  \[ e = \sqrt{1-\left( \frac{b}{a} \right)^2} \]
  显然,椭圆离心率满足$0<e<1$,离心率越小,则比例$\frac{b}{a}$越接近1,椭圆的形状就越接近圆形,反之,离心越大,比例$\frac{b}{a}$越接近0,椭圆就越扁.

  引入了离心率后,焦半径公式就可以写成
  \[
  \begin{cases}
    |PF_1| = a + e x \\
    |PF_2| = a - e x \\
  \end{cases}
\]
式中$x$的范围是$|x| \leqslant a$,所以
\[ a-c \leqslant |PF| \leqslant a+c \]
这里$F$是左右焦点中任意一个。

从椭圆的方程可以看出,椭圆曲线刚好被矩形$|x| \leqslant a, |y| \leqslant b$所框住,并且矩形四边的中点刚好在椭圆上,对边中点相连形成两条十字交叉的腰线,因为这两条腰线也是椭圆的对称轴,所以把较长的那条腰线称为椭圆的\emph{长轴},而较短的那条腰线称为椭圆的\emph{短轴}.习惯上把长轴的一半记为$a$,而把短轴记为$b$,也就是说,在椭圆方程中限定$a>b$,而竖着的椭圆的方程则写为
\[ \frac{x^2}{b^2} + \frac{y^2}{a^2} = 1 \]
所以从椭圆上来看,哪一项的分母大,哪一项对应的坐标轴就是长轴,而另一项对应的坐标轴就是短轴.

\begin{example}
  作为焦半径公式的一个应用,我们来考虑如下问题:椭圆上的点对两个焦点的最大张角问题。

  由余弦定理可得
  \begin{align*}
    \cos{\angle{F_1PF_2}} & = \frac{|PF_1|^2+|PF_2|^2-|F_1F_2|^2}{2|PF_1| \cdot |PF_2|} \\
                          & = \frac{(a+t)^2+(a-t)^2-4c^2}{2(a+t)(a-t)} \\
                          & = \frac{a^2+t^2-2c^2}{a^2-t^2} \\
    & = \frac{2b^2}{a^2-t^2} - 1 
  \end{align*}
  因为$t=ex$,由$|x| \leqslant a$ 得$|t| \leqslant c$,所以得
  \[ 1-2e^2 \leqslant \cos{\angle{F_1PF_2}} \leqslant 1 \]
  并且左边的最小值在$t=0$时取到,这时点$P$正是短轴的端点,所以结论就是,椭圆上的动点在短轴端点位置时,它对两个焦点的张角最大。
\end{example}
  

  因为这个椭圆是由单位圆通过伸缩变换$L(a,b)$而来,所以它的面积是
  \[ S = \pi a b \]

\subsection{切线与光学性质}
\label{sec:tangent-line-and-light-perproties}



\begin{theorem}
  椭圆$\dfrac{x^2}{a^2}+\dfrac{y^2}{b^2}=1(a>0,b>0)$上任一点$P(x_0,y_0)$处的切线方程是$\dfrac{x_0x}{a^2}+\dfrac{y_0y}{b^2}=1$.
\end{theorem}

\begin{proof}[证明]
在这直线上任取一点 $T(x_T,y_T)$,有:
\begin{equation}
\left(\frac{x_0^2}{a^2}+\frac{y_0^2}{b^2}\right)+\left(\frac{x_T^2}{a^2}+\frac{y_T^2}{b^2}\right) \geqslant 2\left(\frac{x_0x_T}{a^2}+\frac{y_0y_T}{b^2}\right)=2
\end{equation}
所以得到:
\begin{equation}
\frac{x_T^2}{a^2}+\frac{y_T^2}{b^2} \geqslant 1
\end{equation}
这表明直线\ref{eq:tangent}上除点$P$外任何一点都在椭圆外,与椭圆只有$P$一个交点,所以它理所当然就是点$P$处的切线方程。
\end{proof}

\begin{proof}[证明二]
  设$P(x_0,y_0)$处对应椭圆离心角$\theta_0$,将椭圆参数方程
\[ 
\begin{cases}
x & = a \cos{\theta} \\
y & = b \sin{\theta}
\end{cases}
\]
代入直线方程
\[ \frac{x_0x}{a^2} + \frac{y_0y}{b^2} = 1 \]
同时用$x_0=a\cos{\theta_0},y_0=b\sin{\theta_0}$替换掉$x_0$和$y_0$得
\[ \cos{\theta_0}\cos{\theta}+\sin{\theta_0}\sin{\theta}=1 \]
即
\[ \cos{(\theta-\theta_0)} = 1 \]
显然,只有$\theta$与$\theta_0$相差$\pi$的偶数倍,才能成为椭圆与直线的交点,而这些交点实际上都与$P(x_0,y_0)$是同一点,于是即证。
\end{proof}

\begin{theorem}
  双曲线$\dfrac{x^2}{a^2}-\dfrac{y^2}{b^2}=1(a>0,b>0)$上任一点$P(x_0,y_0)$处的切线方程是$\dfrac{x_0x}{a^2}-\dfrac{y_0y}{b^2}=1$.
\end{theorem}

\begin{proof}[证明]
  在直线$\dfrac{x_0x}{a^2}-\dfrac{y_0y}{b^2}=1$上任取一点$Q(x_T,y_T)$,有
  \begin{equation}
    \label{eq:4hdks85hdks018}
   \frac{x_0^2}{a^2}-\frac{y_0^2}{b^2}=1, \  \frac{x_0x_T}{a^2}-\frac{y_0y_T}{b^2}=1 
  \end{equation}
  我们将证明
  \begin{equation}
    \label{eq:hss8wsk13hcfsud0wurg}
    \frac{x_T^2}{a^2}-\frac{y_T^2}{b^2} \leqslant 1  
  \end{equation}
  且等号仅在$y_T=y_0$时成立,这样一来,点$P(x_0,y_0)$就是直线与双曲线的唯一公共点,即为切线。

  记
  \[ r=\frac{x_0}{a}+\frac{y_0}{b}, \  s=\frac{x_0}{a}-\frac{y_0}{b} \]
  以及
  \[ u=\frac{x_T}{a}+\frac{y_T}{b}, \  v=\frac{x_T}{a}-\frac{y_T}{b} \]
  那么\autoref{eq:4hdks85hdks018}即为$rs=1$以及$\dfrac{rv+su}{2}=1$,而要证明的\autoref{eq:hss8wsk13hcfsud0wurg}即是$uv\leqslant 1$.

  由$4=(rv+su)^2=r^2v^2+s^2u^2+2rsuv \geqslant 4rsuv= 4uv$即证得$uv \leqslant 1$,定理得证。
\end{proof}

网友kuing提出,椭圆和双曲线有统一的类似的漂亮写法,因为
\begin{align*}
  (a^2+b^2)(u^2+v^2) & =  (au+bv)^2+(av-bu)^2 \geqslant (au+bv)^2 \\
  (a^2-b^2)(u^2-v^2) & =  (au-bv)^2-(av-bu)^2 \leqslant (au-bv)^2 
\end{align*}
所以
\begin{align*}
  \left( \frac{x_0^2}{a^2}+\frac{y_0^2}{b^2} \right) \left( \frac{x_T^2}{a^2}+\frac{y_T^2}{b^2} \right) & \geqslant  \left( \frac{x_0x_T}{a^2}+\frac{y_0y_T}{b^2} \right)^2 \\
  \left( \frac{x_0^2}{a^2}-\frac{y_0^2}{b^2} \right) \left( \frac{x_T^2}{a^2}-\frac{y_T^2}{b^2} \right) & \leqslant  \left( \frac{x_0x_T}{a^2}-\frac{y_0y_T}{b^2} \right)^2 
\end{align*}
如此证法将更为漂亮。

\begin{theorem}
  抛物线$y^2=2px(p>0)$上任一点$P(x_0,y_0)$处的切线方程是$y_0y=p(x+x_0)$.
\end{theorem}

\begin{proof}[证明]
  在直线$y_0y=p(x+x_0)$上任取一点$Q(x_T,y_T)$,则
  \[ y_T^2+y_0^2 \geqslant 2y_Ty_0=2p(x_T+x_0) \]
  而$y_0^2=2px_0$,所以得$y_T^2 \geqslant 2px_T$,且等号仅在$y_T=y_0$时成立,这表明点$P$是该直线与抛物线的唯一公共点,即为切线。
\end{proof}

%%% Local Variables:
%%% mode: latex
%%% TeX-master: "../../elementary-math-note"
%%% End:
