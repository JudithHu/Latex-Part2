%This chapter was modified on 4/2/97.
%\setcounter{chapter}{6}
\chapter[Sums of Random Variables]{Sums of Independent Random Variables}\label{chp 7}  

\section{Sums of Discrete Random Variables}\label{sec 7.1} 


In this chapter we turn to the important question of determining the
distribution of a sum of independent random variables in terms of the
distributions of the individual constituents.  In this section we consider only
sums of discrete random variables, reserving the case of continuous random
variables for the next section.

We consider here only random variables whose values are integers.  Their distribution
functions are then defined on these integers.  We
shall find it convenient to assume here that these distribution functions are
defined for \emx {all} integers, by defining them to be~0 where they are not
otherwise defined.

\subsection*{Convolutions}

\par
Suppose $X$ and $Y$ are two independent discrete random variables with distribution functions 
$m_1(x)$ and $m_2(x)$.  Let $Z = X+Y$.  We would like to determine the distribution 
function $m_3(x)$ of $Z$.  To do this, it is enough to determine the probability that $Z$ 
takes on the value $z$, where $z$ is an arbitrary integer.  Suppose that $X = k$, where $k$
is some integer.  Then $Z = z$ if and only if $Y = z-k$.  So the event $Z = z$ is the union
of the pairwise disjoint events
$$(X = k)\  \mbox{and\ }(Y = z-k)\ ,$$
where $k$ runs over the integers.  Since these events are pairwise disjoint, we have
$$P(Z = z) = \sum_{k = -\infty}^\infty P(X = k)\cdot P(Y = z - k)\ .$$
Thus, we have found the distribution function of the random variable $Z$.  This
leads to the following definition.

\begin{definition}\label{defn 7.1}
Let $X$ and $Y$ be two independent integer-valued random variables, with
distribution functions $m_1(x)$~and~$m_2(x)$ respectively.  Then the {\em
convolution}\index{convolution} of
$m_1(x)$~and~$m_2(x)$ is the distribution function $m_3 = m_1*m_2$ given by
$$ m_3(j) = \sum_k m_1(k) \cdot m_2(j - k)\ ,$$
for $j = \ldots,\ -2,\ -1,\ 0,\ 1,\ 2,\ \ldots$.  The function $m_3(x)$ is the distribution
function of the random variable $Z = X + Y$.  
\end{definition}

\par
It is easy to see that the convolution operation is commutative, and it is straightforward to
show that it is also associative.
\par
Now let $S_n = X_1 + X_2 +\cdots+ X_n$ be the sum of $n$
independent random variables of an independent trials process with common distribution
function $m$ defined on the integers.  Then the distribution function of $S_1$ is $m$.  We can
write
$$
S_n = S_{n - 1} + X_n\ .
$$
Thus, since we know the distribution function of $X_n$ is $m$, we can find the distribution
function of $S_n$ by induction.

\begin{example}
A die is rolled twice.  Let $X_1$ and $X_2$ be the outcomes, and let $S_2 = X_1
+ X_2$ be the sum of these outcomes.  Then $X_1$ and $X_2$ have the common
distribution function:
$$
m = \pmatrix{
1 & 2 & 3 & 4 & 5 & 6 \cr
1/6 & 1/6 & 1/6 & 1/6 & 1/6 & 1/6\cr}.
$$
The distribution function of $S_2$ is then the convolution of this distribution with itself. 
Thus,

\begin{eqnarray*}
P(S_2 = 2) &=& m(1)m(1) \\
           &=& \frac 16 \cdot \frac 16 = \frac 1{36}\ , \\
P(S_2 = 3) &=& m(1)m(2) + m(2)m(1) \\
           &=& \frac 16 \cdot \frac 16 + \frac 16 \cdot \frac 16 = \frac 2{36}\ ,\\
P(S_2 = 4) &=& m(1)m(3) + m(2)m(2) + m(3)m(1) \\
           &=& \frac 16 \cdot \frac 16 + \frac 16 \cdot \frac 16 + \frac 16
\cdot \frac 16 = \frac 3{36}\ .\\
\end{eqnarray*}
Continuing in this way we would find $P(S_2 = 5) = 4/36$, $P(S_2 = 6) = 5/36$,
$P(S_2 = 7) = 6/36$, $P(S_2 = 8) = 5/36$, $P(S_2 = 9) = 4/36$, $P(S_2 = 10) =
3/36$, $P(S_2 = 11) = 2/36$, and $P(S_2 = 12) = 1/36$.

The distribution for $S_3$ would then be the convolution of the distribution for $S_2$
with the distribution for $X_3$.  Thus
\begin{eqnarray*}
P(S_3 = 3) &=& P(S_2 = 2)P(X_3 = 1) \\
           &=& \frac 1{36} \cdot \frac 16 = \frac 1{216}\ , \\
P(S_3 = 4) &=& P(S_2 = 3)P(X_3 = 1) + P(S_2 = 2)P(X_3 = 2) \\
           &=& \frac 2{36} \cdot \frac 16 + \frac 1{36} \cdot \frac 16 = \frac
3{216}\ ,\\
\end{eqnarray*}
and so forth.
\par
This is clearly a tedious job, and a program should be written to carry out this calculation. 
To do this we first write a program to form the convolution of two densities $p$~and~$q$ and
return the density $r$.  We can then write a program to find
the density for the sum $S_n$ of $n$ independent random variables with a common
density $p$, at least in the case that the random variables have a finite number of possible
values.
\putfig{3.5truein}{PSfig7-1}{Density of $S_n$ for rolling a die $n$ times.}{fig 7.1} 
\par
Running this program for the example of
rolling a die $n$ times for~$n = 10,\ 20,\ 30$ results in the distributions shown in
Figure~\ref{fig 7.1}.  We see that, as in the case of Bernoulli trials, the distributions
become bell-shaped.  We shall discuss in Chapter~\ref{chp 9} a very general theorem called
the {\em Central Limit Theorem} that will explain this phenomenon.
\end{example}

\begin{example}
A well-known method for evaluating a bridge\index{bridge} hand is: an ace is assigned a value
of~4, a king~3, a queen~2, and a jack~1.  All other cards are assigned a value
of~0.  The \emx {point count}\index{point count} of the hand is then the sum of the values of the
cards in the hand.  (It is actually more complicated than this, taking into
account voids in suits, and so forth, but we consider here this simplified form
of the point count.)  If a card is dealt at random to a player, then the point
count for this card has distribution
$$
p_X = \pmatrix{
0 & 1 & 2 & 3 & 4 \cr
36/52 & 4/52 & 4/52 & 4/52 & 4/52\cr}.
$$

Let us regard the total hand of 13~cards as 13 independent trials with this
common distribution.  (Again this is not quite correct because we assume here
that we are always choosing a card from a full deck.)  Then the distribution for the
point count $C$ for the hand can be found from the program {\bf
NFoldConvolution}\index{NFoldConvolution (program)} by using the distribution for a single
card and choosing
$n = 13$.  A player with a point count of~13 or more is said to have an  \emx {opening bid.} 
The probability of having an opening bid is then
$$
P(C \geq 13)\ .
$$

Since we have the distribution of $C$, it is easy to compute this probability.  Doing this we
find that
$$
P(C \geq 13) = .2845\ ,
$$
so that about one in four hands should be an opening bid according to this
simplified model.  A more realistic discussion of this problem can be found in
Epstein,\index{EPSTEIN, R.}  \emx {The Theory of Gambling and Statistical Logic.}\footnote{R.~A.
Epstein,  \emx {The Theory of Gambling and Statistical Logic,} rev.~ed.\ (New
York: Academic Press, 1977).}
\end{example}
\par
For certain special distributions it is possible to find an expression for the
distribution that results from convoluting the distribution with itself $n$~times.
\par
The convolution of two binomial distributions,\index{convolution!of binomial distributions} one with
parameters
$m$ and
$p$ and the other with parameters $n$ and $p$, is a binomial distribution with parameters $(m+n)$
and
$p$.  This fact follows easily from a consideration of the experiment which consists of first
tossing a coin $m$ times, and then tossing it $n$ more times.
\par
The convolution of $k$ geometric distributions\index{convolution!of geometric distributions} with
common parameter
$p$ is a negative binomial distribution with parameters $p$ and $k$.  This can be seen by
considering the experiment which consists of tossing a coin until the $k$th head appears.


\exercises 
\begin{LJSItem}


\i\label{exer 7.1.1} A die is rolled three times.  Find the probability that the sum of
the outcomes is
\begin{enumerate}
\item greater than 9.

\item an odd number.
\end{enumerate}

\i\label{exer 7.1.2} The price of a stock on a given trading day changes according to the
distribution
$$
p_X = \pmatrix{
-1 & 0 & 1 & 2 \cr
1/4 & 1/2 & 1/8 & 1/8\cr}.$$
Find the distribution for the change in stock price after two (independent) trading
days.

\i\label{exer 7.1.3} Let $X_1$ and $X_2$ be independent random variables with common
distribution
$$
p_X = \pmatrix{
0 & 1 & 2 \cr
1/8 & 3/8 & 1/2\cr}.$$
Find the distribution of the sum $X_1 + X_2$.

\i\label{exer 7.1.4} In one play of a certain game you win an amount $X$ with distribution
$$
p_X = \pmatrix{
1 & 2 & 3 \cr
1/4 & 1/4 & 1/2\cr}.$$
Using the program {\bf NFoldConvolution} find the distribution for your total winnings
after ten (independent) plays.  Plot this distribution.

\i\label{exer 7.1.5} Consider the following two experiments: the first has outcome $X$ taking
on the values~0,~1, and~2 with equal probabilities; the second results in an
(independent) outcome $Y$ taking on the value~3 with probability 1/4 and~4
with probability 3/4.  Find the distribution of
\begin{enumerate}
\item $Y + X$.

\item $Y - X$.
\end{enumerate}

\i\label{exer 7.1.6} People arrive at a queue according to the following scheme: During each 
minute of time either 0~or~1 person arrives.  The probability that 1 person
arrives is $p$ and that no person arrives is $q = 1 - p$.  Let $C_r$ be the
number of customers arriving in the first $r$~minutes.  Consider a Bernoulli
trials process with a success if a person arrives in a unit time and failure
if no person arrives in a unit time.  Let $T_r$ be the number of failures
before the $r$th success.
\begin{enumerate}
\item What is the distribution for $T_r$?

\item What is the distribution for $C_r$?

\item Find the mean and variance for the number of customers arriving in the
first $r$ minutes.
\end{enumerate}

\i\label{exer 7.1.7} 
\begin{enumerate}
\item A die is rolled three times with outcomes $X_1$,~$X_2$, and~$X_3$.  Let
$Y_3$ be the maximum of the values obtained.  Show that
$$
P(Y_3 \leq j) = P(X_1 \leq j)^3\ .
$$
Use this to find the distribution of~$Y_3$.  Does $Y_3$ have a bell-shaped distribution?
\item
Now let $Y_n$ be the maximum value when $n$ dice are rolled.  Find the distribution of
$Y_n$.  Is this distribution bell-shaped for large values of $n$?
\end{enumerate}

\i\label{exer 7.1.10} A baseball player is to play in the World Series.  Based upon his season
play, you estimate that if he comes to bat four times in a game the number of
hits he will get has a distribution
$$
p_X = \pmatrix{
0 & 1 & 2 & 3 & 4 \cr
.4 & .2 & .2 & .1 & .1\cr}.$$

Assume that the player comes to bat four times in each game of the series. 
\begin{enumerate}
\item
Let $X$ denote the number of hits that he gets in a series.  Using the program {\bf
NFoldConvolution}, find the distribution of $X$ for each of the possible series lengths:
four-game, five-game, six-game, seven-game. 

\item
Using one of the distribution found in part (a), find the probability that his batting average
exceeds .400 in a four-game series.  (The batting average is the number of hits divided by the
number of times at bat.)

\item Given the distribution $p_X$, what is his long-term batting average?
\end{enumerate}

\i\label{exer 7.1.11} Prove that you cannot load two dice in such a way that
the probabilities for any sum from 2~to~12 are the same.  (Be sure to consider
the case where one or more sides turn up with probability zero.)

\i\label{exer 7.1.12} (L\'evy\footnote{See M. Krasner and B. Ranulae, ``Sur une Propriet\'e
des Polynomes de la Division du Circle"; and the following note by J.
Hadamard, in  \emx {C.\ R.\ Acad.\ Sci.,} vol.~204 (1937), pp.~397--399.}) Assume
that $n$ is an integer, not prime.  Show that you can find two distributions
$a$~and~$b$ on the nonnegative integers such that the convolution of
$a$~and~$b$ is the equiprobable distribution on the set 0, 1, 2, \dots, $n - 1$.  If
$n$ is prime this is not possible, but the proof is not so easy.  (Assume that
neither $a$~nor~$b$ is concentrated at 0.)

\i\label{exer 7.1.13} Assume that you are playing craps with dice that are loaded in the
following way: faces two, three, four, and five all come up with the same
probability $(1/6) + r$.  Faces one and six come up with probability $(1/6) - 2r$,
with $0 < r < .02$.  Write a computer program to find the probability of
winning at craps with these dice, and using your program find which values
of~$r$ make craps a favorable game for the player with these dice.
\end{LJSItem}




\choice{}{\section{Sums of Continuous Random Variables}\label{sec 7.2}
In this section we consider the continuous version of the problem posed in the
previous section: How are sums of independent random variables distributed?

\subsection*{Convolutions}
\begin{definition} Let $X$ and $Y$ be two continuous random variables with density
functions $f(x)$ and $g(y)$, respectively.  Assume that both $f(x)$ and $g(y)$ are defined for
all real numbers.  Then the  \emx {convolution}\index{convolution} $f*g$ of
$f$~and~$g$ is the function given by
\begin{eqnarray*} (f*g)(z) &=& \int_{-\infty}^{+\infty} f(z - y) g(y)\,dy \\
         &=& \int_{-\infty}^{+\infty} g(z - x) f(x)\, dx\ .
\end{eqnarray*} 
\end{definition} 

This definition is analogous to the definition, given in Section~\ref{sec 7.1}, of the
convolution of two distribution functions.  Thus it should not be surprising that if $X$ and
$Y$ are independent, then the density of their sum is the convolution of their densities. 
This fact is stated as a theorem below, and its proof is left as an exercise (see
Exercise~\ref{exer 7.2.0.5}).  

\begin{theorem} Let $X$ and $Y$ be two independent random variables with density functions
$f_X(x)$ and $f_Y(y)$ defined for all~$x$.  Then the sum $Z = X + Y$ is a random variable with
density function $f_Z(z)$, where $f_Z$ is the convolution of $f_X$~and~$f_Y$.
\end{theorem}
\par
To get a better understanding of this important result, we will look at some examples.
\pagebreak[4]
\subsection*{Sum of Two Independent Uniform Random Variables}
\begin{example}\label{exam 7.6}
Suppose we choose independently two numbers at random from the interval
$[0,1]$ with uniform probability density\index{convolution!of uniform densities}.  What is the
density of their sum?

Let $X$ and $Y$ be random variables describing our choices and $Z = X + Y$
their sum.  Then we have

$$
f_X(x) = f_Y(x) = \left \{ \begin{array}{ll}
                               1 & \;\mbox{if $0 \leq x \leq 1$,} \\
                               0 & \;\mbox{otherwise;}
                  \end{array}
         \right. 
$$
and the density function for the sum is given by
$$
f_Z(z) = \int_{-\infty}^{+\infty} f_X(z - y) f_Y(y)\,dy\ .
$$
Since $f_Y(y) = 1$ if $0 \leq y \leq 1$ and 0 otherwise, this becomes
$$
f_Z(z) = \int_0^1 f_X(z - y)\,dy\ .
$$
Now the integrand is 0 unless $0 \leq z - y \leq 1$ (i.e., unless $z - 1 \leq y
\leq z$) and then it is~1.  So if $0 \leq z \leq 1$, we have
$$
f_Z(z) = \int_0^z \, dy = z\ ,
$$
while if $1 < z \leq 2$, we have
$$
f_Z(z) = \int_{z - 1}^1\, dy = 2 - z\ ,
$$
and if $z < 0$ or $z > 2$ we have $f_Z(z) = 0$ (see Figure~\ref{fig 7.5}).  Hence,
$$
f_Z(z) = \left \{ \begin{array}{ll}
                               z,          & \;\mbox{if $0 \leq z \leq 1,$} \\
                               2-z,        & \;\mbox{if $1 < z \leq 2,$} \\
                               0,          & \;\mbox{otherwise.}
\end{array}
\right. 
$$
\putfig{3.5truein}{PSfig7-5}{Convolution of two uniform densities.}{fig 7.5} %4.5truein
Note that this result agrees with that of Example~\ref{exam 2.1.4.5}.
\end{example}


\subsection*{Sum of Two Independent Exponential Random Variables}
\begin{example}\label{exam 7.7}
Suppose we choose two numbers at random from the interval $[0,\infty)$ with
an  \emx {exponential} density\index{convolution!of exponential densities} with parameter $\lambda$. 
What is the density of their sum?

Let $X$, $Y$, and $Z = X + Y$ denote the relevant random variables, and
$f_X$,~$f_Y$, and~$f_Z$ their densities.  Then
$$
f_X(x) = f_Y(x) = \left \{ \begin{array}{ll}
           \lambda e^{-\lambda x}, & \;\mbox{if $x \geq 0$},\\
                                 0, & \;\mbox{otherwise;}
                  \end{array}
         \right. 
$$
and so, if $z > 0$,
\begin{eqnarray*}
f_Z(z) &=& \int_{-\infty}^{+\infty} f_X(z - y) f_Y(y)\, dy \\
       &=& \int_0^z \lambda e^{-\lambda(z - y)} \lambda e^{-\lambda y}\, dy \\
       &=& \int_0^z \lambda^2 e^{-\lambda z}\, dy \\
       &=& \lambda^2 z e^{-\lambda z},\\
\end{eqnarray*}
while if $z < 0$, $f_Z(z) = 0$ (see Figure~\ref{fig 7.6}).  Hence,
\putfig{3.5truein}{PSfig7-6}
{Convolution of two exponential densities with $\lambda = 1$.}{fig 7.6} 
$$
f_Z(z) = \left \{ \begin{array}{ll}
                       \lambda^2 z e^{-\lambda z}, 
                                          & \;\mbox{if $z \geq 0$},\\
                                      0,  & \;\mbox{otherwise.}
                  \end{array}
         \right. 
$$
\end{example}

\subsection*{Sum of Two Independent Normal Random Variables}
\begin{example}\label{exam 7.8}
It is an interesting and important fact that the convolution of two normal 
densities with means $\mu_1$~and~$\mu_2$ and variances $\sigma_1$~and~$\sigma_2$ is 
again a normal density, with mean $\mu_1 + \mu_2$ and variance $\sigma_1^2 + \sigma_2^2$.
We will show this in the special case that both random variables are standard normal.
The general case can be done in the same way, but the calculation is messier.  Another
way to show the general result is given in Example~\ref{exam 10.3.3}.
\par
Suppose $X$ and $Y$ are two independent random variables, each with the standard {\em
normal}\index{convolution!of normal densities} density (see Example~\ref{exam 5.16}).  
We have
$$
f_X(x) = f_Y(y) = \frac 1{\sqrt{2\pi}} e^{-x^2/2}\ ,
$$
and so
\begin{eqnarray*}
f_Z(z) &=& f_X * f_Y(z) \\
&=& \frac 1{2\pi} \int_{-\infty}^{+\infty} e^{-(z -
y)^2/2} e^{-y^2/2}\, dy \\
       &=& \frac 1{2\pi} e^{-z^2/4} \int_{-\infty}^{+\infty} e^{-(y - z/2)^2}\,
dy \\
       &=& \frac 1{2\pi} e^{-z^2/4}\sqrt {\pi} \biggl[\frac 1{\sqrt {\pi}}\int_{-\infty}^\infty
e^{-(y-z/2)^2}\,dy\ \biggr]\ .\\
\end{eqnarray*}
The expression in the brackets equals 1, since it is the integral of the normal density
function with $\mu = 0$ and $\sigma = \sqrt 2$.  So, we have
$$
f_Z(z) = \frac 1{\sqrt{4\pi}} e^{-z^2/4}\ .
$$
\end{example}

\subsection*{Sum of Two Independent Cauchy Random Variables}
\begin{example}\label{exam 7.9}
Choose two numbers at random from the interval
$(-\infty,+\infty)$ with the Cauchy\index{convolution!of Cauchy densities} density 
with parameter $a = 1$ (see Example~\ref{exam 5.20}).  Then 
$$
f_X(x) = f_Y(x) = \frac 1{\pi(1 + x^2)}\ ,
$$
and $Z = X + Y$ has density
$$
f_Z(z) = \frac 1{\pi^2} \int_{-\infty}^{+\infty} \frac {1}{1 + (z - y)^2} \frac
{1}{1 + y^2} \, dy\ .
$$
This integral requires some effort, and we give here only the result
(see Section~\ref{sec 10.3}, or Dwass\footnote{M. Dwass, ``On the Convolution of Cauchy
Distributions,"  \emx {American Mathematical Monthly,} vol.~92, no.~1, (1985),
pp.~55--57; see also R.~Nelson, letters to the Editor, ibid., p.~679.}):
$$
f_Z(z) = \frac {2}{\pi(4 + z^2)}\ .
$$
\par
Now, suppose that we ask for the density function of the \emx {average} 
$$
A = (1/2)(X + Y)
$$ 
of $X$~and~$Y$.  Then $A = (1/2)Z$.  Exercise~\ref{sec 5.2}.\ref{exer
5.2.18} shows that if $U$ and $V$ are two continuous random variables  with density functions
$f_U(x)$ and $f_V(x)$, respectively, and if $V = aU$, then 
$$
f_V(x) = \biggl(\frac 1a\biggr)f_U\biggl(\frac xa\biggr)\ .
$$
Thus, we have
$$
f_A(z) = 2f_Z(2z) = \frac 1{\pi(1 + z^2)}\ .
$$
Hence, the density function for the average of two random variables, each
having a Cauchy density, is again a random variable with a Cauchy density; this
remarkable property is a peculiarity of the Cauchy density.  One consequence of
this is if the error in a certain measurement process had a Cauchy
density and you averaged a number of measurements, the average could not be
expected to be any more accurate than any one of your individual measurements!
\end{example}

\subsection*{Rayleigh Density}\index{density function!Rayleigh}\index{Rayleigh
density}
\begin{example}\label{exam 7.10}
Suppose $X$ and $Y$ are two independent standard normal random variables. 
Now suppose we locate a point~$P$ in the $xy$-plane with coordinates $(X,Y)$ and
ask: What is the density of the square of the distance of~$P$ from the origin? 
(We have already simulated this problem in Example~\ref{exam 5.19}.)  Here, with the preceding
notation, we have
$$
f_X(x) = f_Y(x) = \frac 1{\sqrt{2\pi}} e^{-x^2/2}\ .
$$
Moreover, if $X^2$ denotes the square of $X$, then (see Theorem~\ref{thm 5.1} and the
discussion following) 

\begin{eqnarray*}
f_{X^2}(r) &=& \left \{ \begin{array}{ll}
        \frac{1}{2\sqrt r} (f_X(\sqrt r) + f_X(-\sqrt r))  & \;\mbox{if $r > 0,$} \\
                                     0                     & \;\mbox{otherwise.}
\end{array}
\right. \\
           &=& \left \{ \begin{array}{ll}
        \frac{1}{\sqrt {2 \pi r}} (e^{-r/2}) \hspace{.8in} & \;\mbox{if $r > 0,$} \\
                                       0                   & \;\mbox{otherwise.}
\end{array}
\right. \\
\end{eqnarray*}
This is a gamma density with $\lambda = 1/2$, $\beta = 1/2$ (see
Example~\ref{exam 7.7}).  Now let $R^2 = X^2 + Y^2$.  Then
\begin{eqnarray*}
f_{R^2}(r) &=& \int_{-\infty}^{+\infty} f_{X^2}(r - s) f_{Y^2}(s)\, ds \\
           &=& \frac 1{4\pi} \int_{-\infty}^{+\infty} e^{-(r - s)/2} 
{\frac{r-s}{2}}^{-1/2} e^{-s} {\frac{s}{2}}^{-1/2}\, ds\ , \\
           &=& \left \{\begin{array}{ll}
                       {\frac {1}{2}} e^{-r^2/2}, & \;\mbox{if $r \geq 0,$} \\
                               0,                 & \;\mbox{otherwise.}
                  \end{array}
         \right. 
\end{eqnarray*}
Hence, $R^2$ has a gamma density with $\lambda = 1/2$, $\beta = 1$.  We can
interpret this result as giving the density for the square of the distance
of~$P$ from the center of a target if its coordinates are normally distributed.

The density of the random variable $R$ is obtained from that of $R^2$ in the
usual way (see Theorem~\ref{thm 5.1}), and we find
$$
f_R(r) = \left \{ \begin{array}{ll}
                       \frac 12 e^{-r^2/2} \cdot 2r = re^{-r^2/2}, 
                                          & \;\mbox{if $r \geq 0,$} \\
                               0,         & \;\mbox{otherwise.}
                  \end{array}
         \right. 
$$

Physicists will recognize this as a Rayleigh density.  Our
result here agrees with our simulation in Example~\ref{exam 5.19}.
\end{example}

\subsection*{Chi-Squared Density}\index{chi-squared density}\index{density function!chi-squared}
More generally, the same method shows that the sum of the squares of~$n$
independent normally distributed random variables with mean~0 and standard
deviation~1 has a gamma density with $\lambda = 1/2$ and $\beta = n/2$.  Such a
density is called a  \emx {chi-squared density} with $n$ degrees of freedom.  This
density was introduced in Chapter~\ref{chp 5}.  In Example~\ref{exam 5.20}, we
used this density to test the hypothesis that two traits were independent.  
\par
Another important use of the chi-squared density is in comparing experimental data
with a theoretical discrete distribution, to see whether the data supports the 
theoretical model.  More specifically, suppose that we have an experiment with a
finite set of outcomes.  If the set of outcomes is countable, we group them into
finitely many sets of outcomes.  We propose a theoretical distribution which we think
will model the experiment well.  We obtain some data by repeating the  experiment a
number of times.  Now we wish to check how well the theoretical distribution fits the
data.
\par
Let $X$ be the random variable which represents a theoretical outcome in the model
of the experiment, and let $m(x)$ be the distribution function of $X$.  In a manner
similar to what was done in Example~\ref{exam 5.20}, we calculate the value of the
expression
$$
V = \sum_x \frac{(o_x - n \cdot m(x))^2}{n \cdot m(x)}\ ,
$$
where the sum runs over all possible outcomes $x$, $n$ is the number of data
points, and $o_x$ denotes the number of outcomes of type $x$ observed
in the data.  Then for moderate or large values of $n$, the quantity $V$ is
approximately chi-squared distributed, with $\nu - 1$ degrees of freedom, where
$\nu$ represents the number of possible outcomes.  The proof of this is beyond the
scope of this book, but we will illustrate the reasonableness of this statement in
the next example.  If the value of $V$ is very large, when compared with the
appropriate chi-squared density function, then we would tend to reject the hypothesis
that the model is an appropriate one for the experiment at hand.  We now give an
example of this procedure.

\begin{example}
Suppose we are given a single die.  We wish to test the hypothesis that the die is
fair.  Thus, our theoretical distribution is the uniform distribution on the integers
between 1 and 6.  So, if we roll the die $n$ times, the expected number of data
points of each type is $n/6$.  Thus, if $o_i$ denotes the actual number of data
points of type $i$, for $1 \le i \le 6$, then the expression
$$V = \sum_{i = 1}^6 \frac{(o_i - n/6)^2}{n/6}$$
is approximately chi-squared distributed with 5 degrees of freedom.
\par
Now suppose that we actually roll the die 60 times and obtain the 
data in Table~\ref{table 7.1}.
\begin{table}
\centering
\begin{tabular}{|c|c|}
\hline
 Outcome & Observed Frequency \\
\hline 1 & 15\\
\hline 2 & \hspace{.08in}8 \\
\hline 3 & \hspace{.08in}7 \\
\hline 4 & \hspace{.08in}5 \\
\hline 5 & \hspace{.08in}7 \\
\hline 6 & 18 \\
\hline
\end{tabular}
\caption{Observed data.}
\label{table 7.1}
\end{table}
If we calculate $V$ for this data, we obtain the value 13.6.  The graph of the 
chi-squared density with 5 degrees of freedom is shown in Figure~\ref{fig 7.7}.  One
sees that values as large as 13.6 are rarely taken on by $V$ if the die is fair, so we
would reject the hypothesis that the die is fair.   (When using this test, a
statistician will reject the hypothesis if the data gives a value of $V$ which is
larger than 95\% of the values one would expect to obtain if the hypothesis is true.)
\putfig{3.5truein}{PSfig7-7}{Chi-squared density with 5 degrees of freedom.}{fig 7.7} 
\par
In Figure~\ref{fig 7.8}, we show the results of rolling a die 60 times, then
calculating $V$, and then repeating this experiment 1000 times.  The program that performs
these calculations is called {\bf DieTest}.\index{DieTest (program)}  We
have  superimposed the chi-squared density with 5 degrees of freedom; one can see
that the data values fit the curve fairly well, which supports the statement
that the chi-squared density is the correct one to use. 
\putfig{4.5truein}{PSfig7-8}{Rolling a fair die.}{fig 7.8} 
\end{example}

So far we have looked at several important special cases for which the
convolution integral can be evaluated explicitly.  In general, the convolution
of two continuous densities cannot be evaluated explicitly, and we must resort
to numerical methods.  Fortunately, these prove to be remarkably effective, at
least for bounded densities.

\subsection*{Independent Trials}
We now consider briefly the distribution of the sum of~$n$ independent random
variables, all having the same density function.  If $X_1$,~$X_2$, \dots,~$X_n$
are these random variables and $S_n = X_1 + X_2 +\cdots+ X_n$ is their sum, then we
will have
$$
f_{S_n}(x) = \left( f_{X_1} * f_{X_2} *\cdots* f_{X_n} \right)(x)\ ,
$$
where the right-hand side is an $n$-fold convolution.  It is possible to
calculate this density for general values of~$n$ in certain simple cases.

\begin{example}\label{exam 7.12}
Suppose the $X_i$ are uniformly distributed\index{convolution!of uniform densities} on the interval
$[0,1]$.  Then
$$
f_{X_i}(x) = \left \{ \begin{array}{ll}
                         1,         & \;\mbox{if $0 \leq x \leq 1,$} \\
                         0,         & \;\mbox{otherwise,}
                   \end{array}
          \right. 
$$
and $f_{S_n}(x)$ is given by the formula\index{USPENSKY, J. B.}\footnote{J.~B. Uspensky,
{\em Introduction to Mathematical Probability} (New York: McGraw-Hill, 1937),
p.~277.}
$$
f_{S_n}(x) = \left \{ \begin{array}{ll}
                         \frac 1{(n - 1)!} \sum_{0 \leq j \leq x} (-1)^j
 { n \choose j} (x - j)^{n - 1},     & \;\mbox{if $0 < x < n,$} \\       
                         0,          & \;\mbox{otherwise.}
                   \end{array}
          \right. 
$$
The density $f_{S_n}(x)$ for~$n = 2$,~4, 6, 8,~10 is shown in Figure~\ref{fig 7.9}.

\putfig{4.5truein}{PSfig7-9}{Convolution of $n$ uniform densities.}{fig 7.9} 


If the $X_i$ are distributed normally,\index{convolution!of standard normal densities} with mean~0
and variance~1, then (cf.~Example~\ref{exam 7.8})
$$
f_{X_i}(x) = \frac 1{\sqrt{2\pi}} e^{-x^2/2}\ ,
$$
and
$$
f_{S_n}(x) = \frac 1{\sqrt{2\pi n}} e^{-x^2/2n}\ .
$$
Here the density $f_{S_n}$ for~$n = 5$,~10, 15, 20,~25 is shown in Figure~\ref{fig 7.10}.

\putfig{4.5truein}{PSfig7-10}{Convolution of $n$ standard normal densities.}{fig 7.10} 

If the $X_i$ are all exponentially distributed,\index{convolution!of exponential densities}
with mean~$1/\lambda$, then
$$
f_{X_i}(x) = \lambda e^{-\lambda x}\ ,
$$
and
$$
f_{S_n}(x) = \frac {\lambda e^{-\lambda x}(\lambda x)^{n - 1}}{(n - 1)!}\ .
$$
In this case the density $f_{S_n}$ for~$n = 2$,~4, 6, 8,~10 is shown in
Figure~\ref{fig 7.11}.
\putfig{4.5truein}{PSfig7-11}
{Convolution of $n$ exponential densities with $\lambda = 1$.}{fig 7.11} 
\end{example}


\exercises
\begin{LJSItem}

\i\label{exer 7.2.0.5}  Let $X$ and $Y$ be independent real-valued random variables 
with density functions $f_X(x)$ and $f_Y(y)$, respectively.  Show that the density 
function of the sum $X + Y$ is the convolution of the functions $f_X(x)$ and $f_Y(y)$.
\emx {Hint}:  Let $\bar X$ be the joint random variable $(X, Y)$.  Then the joint
density function of $\bar X$ is $f_X(x)f_Y(y)$, since $X$ and $Y$ are independent.  Now
compute the probability that $X+Y \le z$, by integrating the joint density function over the
appropriate region in the plane.  This gives the cumulative distribution function of $Z$. 
Now differentiate this function with respect to $z$ to obtain the density function of $z$.

\i\label{exer 7.2.1} Let $X$ and $Y$ be independent random variables defined
on the space $\Omega$, with density functions $f_X$~and~$f_Y$, respectively. 
Suppose that $Z = X + Y$.  Find the density $f_Z$ of~$Z$ if
\begin{enumerate}
\item $$f_X(x) = f_Y(x) = \left \{ \begin{array}{ll}
                              1/2,   & \;\mbox{if $-1 \leq x \leq +1,$} \\
                              0,     & \;\mbox{otherwise.}
                  \end{array}
         \right. $$

 
\item $$f_X(x) = f_Y(x) = \left \{ \begin{array}{ll}
                              1/2,   & \;\mbox{if $3 \leq x \leq 5,$} \\
                              0,     & \;\mbox{otherwise.}
                  \end{array}
         \right. $$


\item $$f_X(x) = \left \{ \begin{array}{ll}
                              1/2,   & \;\mbox{if $-1 \leq x \leq 1,$} \\
                              0,     & \;\mbox{otherwise.}
                  \end{array}
         \right. $$
\smallskip
$$f_Y(x) = \left \{ \begin{array}{ll}
                              1/2,   & \;\mbox{if $3 \leq x \leq 5,$} \\
                              0,     & \;\mbox{otherwise.}
                  \end{array}
         \right. $$
\item What can you say about the set $E = \{\,z : f_Z(z)> 0\,\}$ in each
case?
\end{enumerate}

\i\label{exer 7.2.2} Suppose again that $Z = X + Y$.  Find $f_Z$ if
\begin{enumerate}
\item $$f_X(x) = f_Y(x) = \left \{ \begin{array}{ll} 
                    x/2, & \mbox{if $0 < x < 2,$} \\
                      0, & \mbox{otherwise}.
                  \end{array}
         \right. $$

\item $$f_X(x) = f_Y(x) = \left \{ \begin{array}{ll} 
                 (1/2)(x - 3), & \mbox{if $3 < x < 5,$} \\
                            0, & \mbox{otherwise}. 
                  \end{array}
         \right. $$

\item $$f_X(x) = \left \{ \begin{array}{ll} 
                       1/2,    & \mbox{if $0 < x < 2,$} \\
                            0, & \mbox{otherwise},
                  \end{array}
         \right. $$
\smallskip
 $$f_Y(x) = \left \{ \begin{array}{ll} 
                          x/2, & \mbox{if $0 < x < 2,$} \\
                            0, & \mbox{otherwise}.
                  \end{array}
         \right. $$
\item What can you say about the set $E = \{\,z : f_Z(z)> 0\,\}$ in each
case?
\end{enumerate}

\i\label{exer 7.2.3} Let $X$, $Y$, and $Z$ be independent random variables
with 
$$f_X(x) = f_Y(x) = f_Z(x)  = \left \{ \begin{array}{ll}
                              1,     & \mbox{if $0 < x < 1,$} \\
                              0,     & \mbox{otherwise.}
                  \end{array}
         \right. $$
\noindent Suppose that $W = X + Y + Z$.  Find $f_W$ directly, and compare your answer
with that given by the formula in Example~\ref{exam 7.12}.   \emx {Hint}:  See 
Example~\ref{exam 7.6}.

\i\label{exer 7.2.3.5} Suppose that $X$ and $Y$ are independent and $Z = X + Y$.  Find
$f_Z$ if

\begin{enumerate}
\item $$f_X(x) = \left \{ \begin{array}{ll}
                              \lambda e^{-\lambda x}, & \mbox{if $x > 0,$} \\
                               0,                     & \mbox{otherwise.}
                  \end{array}
         \right. $$
\smallskip
$$f_Y(x) = \left \{ \begin{array}{ll}
                              \mu e^{-\mu x}, & \mbox{if $x > 0,$} \\
                              0,              & \mbox{otherwise.}
                  \end{array}
         \right. $$
\item $$\ \ \ f_X(x) = \left \{ \begin{array}{ll}
                              \lambda e^{-\lambda x}, & \mbox{if $x > 0,$} \\
                              0,                      & \mbox{otherwise.}
                  \end{array}
         \right. $$

\smallskip
$$f_Y(x) = \left \{ \begin{array}{ll}
                    1, & \mbox{if $0 < x < 1,$} \\
                    0, & \mbox{otherwise.}
                  \end{array}
         \right. $$
\end{enumerate}

\i\label{exer 7.2.100} Suppose again that $Z = X + Y$.  Find $f_Z$ if
\begin{eqnarray*}
f_X(x) &=& \frac 1{\sqrt{2\pi}\sigma_1} e^{-(x - \mu_1)^2/2\sigma_1^2} \\
f_Y(x) &=& \frac 1{\sqrt{2\pi}\sigma_2} e^{-(x - \mu_2)^2/2\sigma_2^2}\ .
\end{eqnarray*}

%********The next problem is too hard.
\istar\label{exer 7.2.101} Suppose that $R^2 = X^2 + Y^2$.  Find $f_{R^2}$ and $f_R$ if
\begin{eqnarray*}
f_X(x) &=& \frac 1{\sqrt{2\pi}\sigma_1} e^{-(x - \mu_1)^2/2\sigma_1^2} \\
f_Y(x) &=& \frac 1{\sqrt{2\pi}\sigma_2} e^{-(x - \mu_2)^2/2\sigma_2^2}\ .
\end{eqnarray*}

\i\label{exer 7.2.101.5} Suppose that $R^2 = X^2 + Y^2$.  Find $f_{R^2}$ and $f_R$ if
$$
f_X(x) = f_Y(x) = \left \{ \begin{array}{ll}
                    1/2, & \mbox{if $-1 \leq x \leq 1,$} \\
                    0,   & \mbox{otherwise.}
                  \end{array}
         \right. 
$$

\i\label{exer 7.2.102} Assume that the service time for a customer at a bank is
exponentially distributed with mean service time 2~minutes.  Let $X$ be the total service
time for 10 customers.  Estimate the probability that $X > 22$ minutes.

\i\label{exer 7.2.9} Let $X_1$,~$X_2$, \dots,~$X_n$ be $n$ independent random
variables each of which has an exponential density with mean~$\mu$.  Let $M$ be
the \emx {minimum} value of the $X_j$.  Show that the density for~$M$ is
exponential with mean $\mu/n$.   \emx {Hint}:  Use cumulative distribution functions.

\i\label{exer 7.2.103} A company buys 100 lightbulbs, each of which has an exponential
lifetime of 1000 hours.  What is the expected time for the first of these bulbs to burn
out?  (See Exercise~\ref{exer 7.2.9}.)

\i\label{exer 7.2.104} An insurance company assumes that the time between claims from
each of its homeowners' policies is exponentially distributed with mean~$\mu$.  It
would like to estimate $\mu$ by averaging the times for a number of policies,
but this is not very practical since the time between claims is about
30~years.  At Galambos'\index{GALAMBOS, J.}\footnote{J. Galambos,  \emx {Introductory
Probability Theory} (New York: Marcel Dekker, 1984), p.~159.} suggestion the company puts
its customers in groups of~50 and observes the time of the first claim within
each group.  Show that this provides a practical way to estimate the value
of~$\mu$.

\i\label{exer 7.2.105} Particles are subject to collisions that cause them to split into
two parts with each part a fraction of the parent.  Suppose that this fraction is
uniformly distributed between 0~and~1.  Following a single particle through
several splittings we obtain a fraction of the original particle $Z_n = X_1
\cdot X_2 \cdot\dots\cdot X_n$ where each $X_j$ is uniformly distributed
between 0~and~1.  Show that the density for the random variable $Z_n$ is
$$
f_n(z) = \frac 1{(n - 1)!}( -\log z)^{n - 1}.
$$
 \emx {Hint}: Show that $Y_k = -\log X_k$ is exponentially distributed.  Use
this to find the density function for $S_n = Y_1 + Y_2 +\cdots+ Y_n$, and
from this the cumulative distribution and density of $Z_n = e^{-S_n}$.

\i\label{exer 7.2.106} Assume that $X_1$ and $X_2$ are independent random variables, each
having an exponential density with parameter~$\lambda$.  Show that $Z = X_1 - X_2$ has
density
$$
f_Z(z) = (1/2)\lambda e^{-\lambda |z|}\ .
$$

\i\label{exer 7.2.107} Suppose we want to test a coin for fairness.  We flip the coin $n$
times and record the number of times $X_0$ that the coin turns up tails and the
number of times $X_1 = n - X_0$ that the coin turns up heads.  Now we set
$$
Z= \sum_{i = 0}^1 \frac {(X_i - n/2)^2}{n/2}\ .
$$
Then for a fair coin $Z$ has approximately a chi-squared distribution with $2 -
1 = 1$ degree of freedom.  Verify this by computer simulation first for a fair
coin ($p~=~1/2$) and then for a biased coin ($p~=~1/3$).

\i\label{exer 7.2.108} Verify your answers in Exercise~\ref{exer 7.2.1}(a) by computer
simulation: Choose $X$ and $Y$ from $[-1,1]$ with uniform density and calculate
$Z = X + Y$.  Repeat this experiment 500 times, recording the outcomes in a bar
graph on $[-2,2]$ with 40~bars.  Does the density $f_Z$ calculated in Exercise~\ref{exer
7.2.1}(a) describe the shape of your bar graph?  Try this for Exercises~
\ref{exer 7.2.1}(b)~and~Exercise~\ref{exer 7.2.1}(c), too.

\i\label{exer 7.2.109} Verify your answers to Exercise~\ref{exer 7.2.2} by computer
simulation.

\i\label{exer 7.2.110} Verify your answer to Exercise~\ref{exer 7.2.3} by computer
simulation.

\i\label{exer 7.2.18} The  \emx {support} of a function $f(x)$ is defined to be the set
$$
\{x\ :\ f(x) > 0\}\ .$$
Suppose that $X$ and $Y$ are two continuous random variables with
density functions $f_X(x)$ and $f_Y(y)$, respectively, and suppose that the supports of these
density functions are the intervals $[a, b]$ and $[c, d]$, respectively.  Find the support of the
density function of the random variable $X+Y$.

\i\label{exer 7.2.111} Let $X_1$,~$X_2$, \dots,~$X_n$ be a sequence of independent random
variables, all having a common density function $f_X$ with support $[a,b]$ (see
Exercise~\ref{exer 7.2.18}).  Let $S_n = X_1 + X_2 +\cdots+ X_n$, with density
function $f_{S_n}$.  Show that the support of~$f_{S_n}$ is the interval
$[na,nb]$.   \emx {Hint}: Write $f_{S_n} = f_{S_{n - 1}} * f_X$.  Now use
Exercise~\ref{exer 7.2.18} to establish the desired result by induction.

\i\label{exer 7.2.112} Let $X_1$,~$X_2$, \dots,~$X_n$ be a sequence of independent random
variables, all having a common density function $f_X$.  Let $A = S_n/n$ be
their average.  Find $f_A$ if
\begin{enumerate}
\item $f_X(x) = (1/\sqrt{2\pi}) e^{-x^2/2}$ (normal density).

\item $f_X(x) = e^{-x}$ (exponential density).
\par
\noindent \emx {Hint}: Write $f_A(x)$ in terms of $f_{S_n}(x)$.
\end{enumerate}

\end{LJSItem}}

