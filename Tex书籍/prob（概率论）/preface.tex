% Modified on 4/2/97.
Probability theory began in seventeenth century France when the two great
French 
mathematicians, Blaise Pascal and Pierre de Fermat, corresponded over two
problems from 
games of chance. Problems like those Pascal and Fermat solved continued to
influence such 
early researchers as Huygens, Bernoulli, and DeMoivre in establishing a
mathematical theory of 
probability. Today, probability theory is a well-established branch of
mathematics that finds 
applications in every area of scholarly activity from music to physics, and in
daily experience 
from weather prediction to predicting the risks of new medical treatments.

This text is designed for an introductory probability course taken by
sophomores, juniors, and
seniors in mathematics, the physical and social sciences, engineering, and
computer
science. It  presents a thorough treatment of probability ideas and techniques
necessary for a firm 
understanding of the subject. The text can be used in a variety of course
lengths, levels, and areas 
of emphasis.

For use in a standard one-term course, in which both discrete and continuous
probability is covered,
students should have taken as a prerequisite two terms of calculus, including
an introduction to
multiple integrals. In order to cover Chapter~\ref{chp 11}, which contains
material on Markov
chains,  some knowledge of matrix theory is necessary.

The text can also be used in a discrete probability course. The material has
been organized in 
such a way that the discrete and continuous probability discussions are
presented in a separate, 
but parallel, manner. This organization dispels an overly rigorous or formal
view of probability 
and offers some strong pedagogical value in that the discrete discussions can
sometimes serve to 
motivate the more abstract continuous probability discussions. For use in a
discrete probability 
course, students should have taken one term of calculus as a prerequisite.

Very little computing background is assumed or necessary in order to obtain
full benefits from the use of the computing material and examples in the text. 
All of the
programs that are used in the text have been written in each of
the languages TrueBASIC, Maple, and Mathematica.   

This book is distributed on the Web as part of the Chance Project,
which is devoted to providing materials for beginning
courses in probability
and statistics.  The computer programs, solutions to the odd-numbered
exercises, and current errata
are also available at this site.  Instructors may obtain all of the solutions
by writing to either
of the authors, at jlsnell@dartmouth.edu and cgrinst1@swarthmore.edu.

\bigskip\centerline{\bf FEATURES}\medskip

{\it Level of rigor and emphasis:} Probability is a wonderfully intuitive and
applicable field of 
mathematics. We have tried not to spoil its beauty by presenting too much
formal mathematics. 
Rather, we have tried to develop the key ideas in a somewhat leisurely style,
to provide a variety 
of interesting applications to probability, and to show some of the
nonintuitive examples that 
make probability such a lively subject.

{\it Exercises:} There are over 600 exercises in the text providing plenty of
opportunity for practicing 
skills and developing a sound understanding of the ideas. In the exercise sets
are routine 
exercises to be done with and without the use of a computer and more
theoretical exercises to 
improve the understanding of basic concepts. More difficult exercises are
indicated by an 
asterisk.  A solution manual for all of the exercises is available to
instructors.

{\it Historical remarks:} Introductory probability is a subject in which the
fundamental ideas are still 
closely tied to those of the founders of the subject. For this reason, there
are numerous historical 
comments in the text, especially as they deal with the development of discrete
probability.

{\it Pedagogical use of computer programs:} Probability theory makes
predictions about experiments 
whose outcomes depend upon chance. Consequently, it lends itself beautifully to
the use of 
computers as a mathematical tool to simulate and analyze chance experiments.

In the text the computer is utilized in several ways. First, it provides a
laboratory where chance 
experiments can be simulated and the students can get a feeling for the variety
of such 
experiments. This use of the computer in probability has been already
beautifully illustrated by 
William Feller in the second edition of his famous text {\it An Introduction to
Probability Theory and Its 
Applications} (New York: Wiley, 1950). In the preface, Feller wrote about his
treatment of 
fluctuation in coin tossing: ``The results are so amazing and so at variance
with common intuition 
that even sophisticated colleagues doubted that coins actually misbehave as
theory predicts. The 
record of a simulated experiment is therefore included."

In addition to providing a laboratory for the student, the computer is a
powerful aid in 
understanding basic results of probability theory. For example, the graphical
illustration of the 
approximation of the standardized binomial distributions to the normal curve is
a
more convincing 
demonstration of the Central Limit Theorem than many of the formal proofs of
this fundamental 
result.

Finally, the computer allows the student to solve problems that do not lend
themselves to 
closed-form formulas such as waiting times in queues. Indeed, the introduction
of the computer 
changes the way in which we look at many problems in probability. For example,
being able to 
calculate exact binomial probabilities for experiments up to 1000 trials
changes the way we view 
the normal and Poisson approximations.

\bigskip\centerline{\bf ACKNOWLEDGMENTS }\medskip

Anyone writing a probability text today owes a great debt to William Feller,
who taught us all how to make probability come alive as a subject matter.  If
you find an example, an application, or an exercise that you really like, it
probably had its origin in Feller's classic text, {\sl An Introduction to
Probability Theory and Its Applications.}



We are indebted to many people for their help in this undertaking. The approach
to Markov Chains presented in the book was developed by John Kemeny and the
second author.  Reese Prosser was a silent co-author for the material on
continuous probability in an earlier version of this book. Mark Kernighan
contributed 40 pages of comments on the earlier
edition.  Many of
these comments were very thought-provoking; in addition, they provided a
student's perspective on the
book.   Most of the major changes in this version of the book have their genesis
in
these notes.

Fuxing Hou and Lee Nave  provided extensive help with the
typesetting and the
figures.  John Finn provided valuable pedagogical advice on the text and and the
computer programs.   Karl Knaub and Jessica Sklar are responsible for the
implementations of the
computer programs in
Mathematica and Maple.  Jessica and Gang Wang assisted with the solutions.

Finally, we thank the American Mathematical Society, and in particular
Sergei Gelfand and John Ewing,
for their interest in this
book; their help in its production; and their willingness to
make the work freely redistributable.
